\begin{frame}{La trigonométrie}
    \section{Propriété : la trigonométrie}
    Dans un triangle rectangle, pour tout angle $\alpha$ autre que l'angle droit, on a :
     $$\cos(\alpha) = \dfrac {\text{côté adjacent}}{\text{hypoténuse}}$$
    $$\sin(\alpha) = \dfrac {\text{côté opposé}}{\text{hypoténuse}}$$
    $$\tan(\alpha) = \dfrac {\text{côté opposé}}{\text{côté adjacent}}$$
\end{frame}

\begin{frame}{Moyen mnémotechnique}
    Pour retenir la propriété, on pourra penser à "CAH SOH TOA", ou "SOH CAH TOA"
     $$\text{\textcolor{red}{C}os}(\alpha) = \dfrac {\text{côté \textcolor{red}{A}djacent}}{\text{\textcolor{red}{H}ypoténuse}}$$
    $$\text{\textcolor{red}{S}in}(\alpha) = \dfrac {\text{côté \textcolor{red}{O}pposé}}{\text{\textcolor{red}{H}ypoténuse}}$$
    $$\text{\textcolor{red}{T}an}(\alpha) = \dfrac {\text{côté \textcolor{red}{O}pposé}}{\text{côté \textcolor{red}{A}djacent}}$$
\end{frame}


\begin{frame}{Application : compléter les égalités}
 \begin{minipage}{0.4\textwidth}
    \tikzmath{
    \xa=1;\ya=2;\xb=3;\yb=2;} % Pour générer un triangle rectangle à partir de deux points
\begin{tikzpicture}
    \coordinate (a) at (\xa,\ya) ;
    \coordinate (b) at (\xb,\yb) ;
    \coordinate (m) at ($ (a)!.5!(b) $);
    \draw (a)--(m);
    \coordinate (o) at ($ (m)!1.5!90:(b)$);
    \coordinate (c) at ($ (b)!2!(o)$);
    \draw (a) node [left] {E}-- node [midway,below,sloped] {} (b)node [right] {F}-- node [midway,above,sloped] {} (c) node[above] {G}-- node [midway,below,sloped] {} cycle;
    \draw[gray] ($(a)!2ex!(b)$) -- ($(a)!2ex!(b)!2ex!90:(b)$)--($(a)!2ex!(c)$); %Angle droit
    \draw pic["$\alpha$",draw=blue,fill=blue!20,angle eccentricity=1.3, angle radius=0.8cm]{angle=c--b--a};
\end{tikzpicture}
\end{minipage}
\begin{minipage}{0.4\textwidth}
    \tikzmath{
    \xa=1;\ya=2;\xb=1;\yb=4;} % Pour générer un triangle rectangle à partir de deux points
\begin{tikzpicture}
    \coordinate (a) at (\xa,\ya) ;
    \coordinate (b) at (\xb,\yb) ;
    \coordinate (m) at ($ (a)!.5!(b) $);
    \draw (a)--(m);
    \coordinate (o) at ($ (m)!1.5!90:(b)$);
    \coordinate (c) at ($ (b)!2!(o)$);
    \draw (a) node [right] {H}-- node [midway,below,sloped] {} (b)node [above] {I}-- node [midway,above,sloped] {} (c) node[left] {J}-- node [midway,below,sloped] {} cycle;
    \draw[gray] ($(a)!2ex!(b)$) -- ($(a)!2ex!(b)!2ex!90:(b)$)--($(a)!2ex!(c)$); %Angle droit
    \draw pic["$\omega$",draw=blue,fill=blue!20,angle eccentricity=1.3, angle radius=0.8cm]{angle=c--b--a};
\end{tikzpicture}
\end{minipage}
\begin{minipage}{0.15\textwidth}
    $$\cos(\alpha)=\dots$$
    $$\tan(\alpha)=\dots$$
    $$\cos(\omega)=\dots$$
    $$\sin(\omega)=\dots$$
\end{minipage}
\begin{minipage}{0.3\textwidth}
    \tikzmath{
    \xa=1;\ya=1.5;\xb=2;\yb=0;} % Pour générer un triangle rectangle à partir de deux points
\begin{tikzpicture}
    \coordinate (a) at (\xa,\ya) ;
    \coordinate (b) at (\xb,\yb) ;
    \coordinate (m) at ($ (a)!.5!(b) $);
    \draw (a)--(m);
    \coordinate (o) at ($ (m)!1.5!90:(b)$);
    \coordinate (c) at ($ (b)!2!(o)$);
    \draw (a) node [left] {L}-- node [midway,below,sloped] {} (b)node [below] {M}-- node [midway,above,sloped] {} (c) node[above] {K}-- node [midway,below,sloped] {} cycle;
    \draw[gray] ($(a)!2ex!(b)$) -- ($(a)!2ex!(b)!2ex!90:(b)$)--($(a)!2ex!(c)$); %Angle droit
    \draw pic["$\beta$",draw=blue,fill=blue!20,angle eccentricity=1.3, angle radius=0.8cm]{angle=c--b--a};
\end{tikzpicture}
\end{minipage}
\begin{minipage}{0.5\textwidth}
    \tikzmath{
    \xa=1;\ya=2;\xb=3;\yb=3.5;} % Pour générer un triangle rectangle à partir de deux points
\begin{tikzpicture}
    \coordinate (a) at (\xa,\ya) ;
    \coordinate (b) at (\xb,\yb) ;
    \coordinate (m) at ($ (a)!.5!(b) $);
    \draw (a)--(m);
    \coordinate (o) at ($ (m)!1.5!90:(b)$);
    \coordinate (c) at ($ (b)!2!(o)$);
    \draw (a) node [below] {R}-- node [midway,below,sloped] {} (b)node [right] {Q}-- node [midway,above,sloped] {} (c) node[left] {P}-- node [midway,below,sloped] {} cycle;
    \draw[gray] ($(a)!2ex!(b)$) -- ($(a)!2ex!(b)!2ex!90:(b)$)--($(a)!2ex!(c)$); %Angle droit
    \draw pic["$\gamma$",draw=blue,fill=blue!20,angle eccentricity=1.3, angle radius=0.8cm]{angle=c--b--a};
\end{tikzpicture}
\end{minipage}
\begin{minipage}{0.15\textwidth}
    $$\cos(\gamma)=\dots$$
    $$\tan(\gamma)=\dots$$
    $$\cos(\beta)=\dots$$
    $$\tan(\beta)=\dots$$
\end{minipage}
\end{frame}
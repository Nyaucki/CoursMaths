\begin{frame}{Point de vocabulaire}
    \section{Vocabulaire}
    \begin{minipage}{0.45\textwidth}
    \tikzmath{
    \xa=0;\ya=2;\xb=1.5;\yb=4;} % Pour générer un triangle rectangle à partir de deux points
\begin{tikzpicture}
    \coordinate (a) at (\xa,\ya) ;
    \coordinate (b) at (\xb,\yb) ;
    \coordinate (m) at ($ (a)!.5!(b) $);
    \draw (a)--(m);
    \coordinate (o) at ($ (m)!1.5!90:(b)$);
    \coordinate (c) at ($ (b)!2!(o)$);
    \draw (a) node [below] {A}-- node [midway,below,sloped] {côté opposé} (b)node [right] {B}-- node [midway,above,sloped] {hypothénuse} (c) node[left] {C}-- node [midway,below,sloped] {côté adjacent} cycle;
    \draw[gray] ($(a)!2ex!(b)$) -- ($(a)!2ex!(b)!2ex!90:(b)$)--($(a)!2ex!(c)$); %Angle droit
    \draw pic["$\widehat{C}$",draw=blue,fill=blue!20,angle eccentricity=1.3, angle radius=0.8cm]{angle=a--c--b};
\end{tikzpicture}
\end{minipage}
\hfil
\begin{minipage}{0.45\textwidth}
    \tikzmath{
    \xa=1;\ya=2;\xb=3;\yb=3.5;} % Pour générer un triangle rectangle à partir de deux points
\begin{tikzpicture}
    \coordinate (a) at (\xa,\ya) ;
    \coordinate (b) at (\xb,\yb) ;
    \coordinate (m) at ($ (a)!.5!(b) $);
    \draw (a)--(m);
    \coordinate (o) at ($ (m)!1.5!90:(b)$);
    \coordinate (c) at ($ (b)!2!(o)$);
    \draw (a) node [below] {A}-- node [midway,below,sloped] {côté adjacent} (b)node [right] {B}-- node [midway,above,sloped] {hypothénuse} (c) node[left] {C}-- node [midway,below,sloped] {côté opposé} cycle;
    \draw[gray] ($(a)!2ex!(b)$) -- ($(a)!2ex!(b)!2ex!90:(b)$)--($(a)!2ex!(c)$); %Angle droit
    \draw pic["$\widehat{B}$",draw=blue,fill=blue!20,angle eccentricity=1.3, angle radius=0.8cm]{angle=c--b--a};
\end{tikzpicture}
\end{minipage}
\end{frame}


\begin{frame}{Nomme les côtés des triangles par rapport  l'ange $\alpha$}
 \begin{minipage}{0.45\textwidth}
    \tikzmath{
    \xa=1;\ya=2;\xb=3;\yb=2;} % Pour générer un triangle rectangle à partir de deux points
\begin{tikzpicture}
    \coordinate (a) at (\xa,\ya) ;
    \coordinate (b) at (\xb,\yb) ;
    \coordinate (m) at ($ (a)!.5!(b) $);
    \draw (a)--(m);
    \coordinate (o) at ($ (m)!1.5!90:(b)$);
    \coordinate (c) at ($ (b)!2!(o)$);
    \draw (a) node [left] {E}-- node [midway,below,sloped] {} (b)node [right] {F}-- node [midway,above,sloped] {} (c) node[above] {G}-- node [midway,below,sloped] {} cycle;
    \draw[gray] ($(a)!2ex!(b)$) -- ($(a)!2ex!(b)!2ex!90:(b)$)--($(a)!2ex!(c)$); %Angle droit
    \draw pic["$\alpha$",draw=blue,fill=blue!20,angle eccentricity=1.3, angle radius=0.8cm]{angle=c--b--a};
\end{tikzpicture}
\end{minipage}
\begin{minipage}{0.45\textwidth}
    \tikzmath{
    \xa=1;\ya=2;\xb=1;\yb=4;} % Pour générer un triangle rectangle à partir de deux points
\begin{tikzpicture}
    \coordinate (a) at (\xa,\ya) ;
    \coordinate (b) at (\xb,\yb) ;
    \coordinate (m) at ($ (a)!.5!(b) $);
    \draw (a)--(m);
    \coordinate (o) at ($ (m)!1.5!90:(b)$);
    \coordinate (c) at ($ (b)!2!(o)$);
    \draw (a) node [right] {H}-- node [midway,below,sloped] {} (b)node [above] {I}-- node [midway,above,sloped] {} (c) node[left] {J}-- node [midway,below,sloped] {} cycle;
    \draw[gray] ($(a)!2ex!(b)$) -- ($(a)!2ex!(b)!2ex!90:(b)$)--($(a)!2ex!(c)$); %Angle droit
    \draw pic["$\alpha$",draw=blue,fill=blue!20,angle eccentricity=1.3, angle radius=0.8cm]{angle=c--b--a};
\end{tikzpicture}
\end{minipage}
\begin{minipage}{0.45\textwidth}
    \tikzmath{
    \xa=1;\ya=1.5;\xb=2;\yb=0;} % Pour générer un triangle rectangle à partir de deux points
\begin{tikzpicture}
    \coordinate (a) at (\xa,\ya) ;
    \coordinate (b) at (\xb,\yb) ;
    \coordinate (m) at ($ (a)!.5!(b) $);
    \draw (a)--(m);
    \coordinate (o) at ($ (m)!1.5!90:(b)$);
    \coordinate (c) at ($ (b)!2!(o)$);
    \draw (a) node [left] {L}-- node [midway,below,sloped] {} (b)node [below] {M}-- node [midway,above,sloped] {} (c) node[above] {K}-- node [midway,below,sloped] {} cycle;
    \draw[gray] ($(a)!2ex!(b)$) -- ($(a)!2ex!(b)!2ex!90:(b)$)--($(a)!2ex!(c)$); %Angle droit
    \draw pic["$\alpha$",draw=blue,fill=blue!20,angle eccentricity=1.3, angle radius=0.8cm]{angle=c--b--a};
\end{tikzpicture}
\end{minipage}
\begin{minipage}{0.45\textwidth}
    \tikzmath{
    \xa=1;\ya=2;\xb=3;\yb=3.5;} % Pour générer un triangle rectangle à partir de deux points
\begin{tikzpicture}
    \coordinate (a) at (\xa,\ya) ;
    \coordinate (b) at (\xb,\yb) ;
    \coordinate (m) at ($ (a)!.5!(b) $);
    \draw (a)--(m);
    \coordinate (o) at ($ (m)!1.5!90:(b)$);
    \coordinate (c) at ($ (b)!2!(o)$);
    \draw (a) node [below] {R}-- node [midway,below,sloped] {} (b)node [right] {Q}-- node [midway,above,sloped] {} (c) node[left] {P}-- node [midway,below,sloped] {} cycle;
    \draw[gray] ($(a)!2ex!(b)$) -- ($(a)!2ex!(b)!2ex!90:(b)$)--($(a)!2ex!(c)$); %Angle droit
    \draw pic["$\alpha$",draw=blue,fill=blue!20,angle eccentricity=1.3, angle radius=0.8cm]{angle=c--b--a};
\end{tikzpicture}
\end{minipage}
\end{frame}
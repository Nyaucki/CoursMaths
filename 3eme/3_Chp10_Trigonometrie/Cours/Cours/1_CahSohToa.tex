\section{Cah Soh Toa ! }

\dfnt{Vocabulaire}
{Dans un triangle rectangle, on appelle :
\begin{itemize}
    \item Le côté opposé à un angle est le côté en face de cet angle.
    \item L'hypoténuse est le côté opposé à l'angle droit.
    \item Le côté adjacent à un angle est le côté (autre que l'hypoténuse) touchant cet angle.
\end{itemize}
}

\rmq {Attention, les côtés adjacents et opposés dépendent de l'angle choisi}

\exmpl{}

\begin{minipage}{0.45\textwidth}
    \tikzmath{
    \xa=1;\ya=2;\xb=3;\yb=4;} % Pour générer un triangle rectangle à partir de deux points
\begin{tikzpicture}
    \coordinate (a) at (\xa,\ya) ;
    \coordinate (b) at (\xb,\yb) ;
    \coordinate (m) at ($ (a)!.5!(b) $);
    \draw (a)--(m);
    \coordinate (o) at ($ (m)!1.5!90:(b)$);
    \coordinate (c) at ($ (b)!2!(o)$);
    \draw (a) node [below] {A}-- node [midway,below,sloped] {côté opposé} (b)node [right] {B}-- node [midway,above,sloped] {hypothénuse} (c) node[left] {C}-- node [midway,below,sloped] {côté adjacent} cycle;
    \draw[gray] ($(a)!2ex!(b)$) -- ($(a)!2ex!(b)!2ex!90:(b)$)--($(a)!2ex!(c)$); %Angle droit
    \draw pic["$\widehat{C}$",draw=blue,fill=blue!20,angle eccentricity=1.3, angle radius=0.8cm]{angle=a--c--b};
\end{tikzpicture}
\end{minipage}
\hfil \vrule \hfil
\begin{minipage}{0.45\textwidth}
    \tikzmath{
    \xa=1;\ya=2;\xb=3;\yb=4;} % Pour générer un triangle rectangle à partir de deux points
\begin{tikzpicture}
    \coordinate (a) at (\xa,\ya) ;
    \coordinate (b) at (\xb,\yb) ;
    \coordinate (m) at ($ (a)!.5!(b) $);
    \draw (a)--(m);
    \coordinate (o) at ($ (m)!1.5!90:(b)$);
    \coordinate (c) at ($ (b)!2!(o)$);
    \draw (a) node [below] {A}-- node [midway,below,sloped] {côté adjacent} (b)node [right] {B}-- node [midway,above,sloped] {hypothénuse} (c) node[left] {C}-- node [midway,below,sloped] {côté opposé} cycle;
    \draw[gray] ($(a)!2ex!(b)$) -- ($(a)!2ex!(b)!2ex!90:(b)$)--($(a)!2ex!(c)$); %Angle droit
    \draw pic["$\widehat{B}$",draw=blue,fill=blue!20,angle eccentricity=1.3, angle radius=0.8cm]{angle=c--b--a};
\end{tikzpicture}
\end{minipage}

\prop{Cosinus Sinus et Tangente}
{Dans un triangle rectangle, pour un angle autre que l'angle droit, on a :
\begin{multicols}{3}
    \noindent $$\cos = \dfrac {\text{côté adjacent}}{\text{hypoténuse}}$$
    $$\sin = \dfrac {\text{côté opposé}}{\text{hypoténuse}}$$
    $$\tan = \dfrac {\text{côté opposé}}{\text{côté adjacent}}$$
\end{multicols}}

\exmpl{ Pour le triangle de l'exemple précédent (ABC rectangle en A): 
\begin{multicols}{3}
    \begin{itemize}
    \item $\cos (\widehat{C})=\dfrac{AC}{BC}$
    \item $\cos (\widehat{B})=\dfrac{AB}{BC}$
    \item $\sin (\widehat{C})=\dfrac{AB}{BC}$
    \item $\sin (\widehat{B})=\dfrac{AC}{BC}$
    \item $\tan (\widehat{C})=\dfrac{AB}{AC}$
    \item $\tan (\widehat{B})=\dfrac{AC}{AB}$
\end{itemize}
\end{multicols}}


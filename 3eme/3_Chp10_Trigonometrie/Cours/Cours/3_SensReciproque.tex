\section{Trouver un angle à partir de longueurs}


Il faut choisir la propriété à utiliser en fonction des informations à notre disposition et de ce que nous cherchons, conformément aux exemples ci-dessous et utiliser la fonction $\cos^{-1}$ (ou $\arccos$), $\sin^{-1}$ (ou $\arcsin$) ou $\tan^{-1}$ (ou $\arctan$) de la calculatrice. Ces fonctions sont disponibles en appuyant sur la touche shift avant d'appuyer sur la touche  $\cos$, $\sin$ ou $\tan$.


On cherche à trouver l'angle $\widehat{C}$ dans les triangles ci-dessous :

\begin{minipage}[b]{0.45\textwidth}
\begin{center}
\tikzmath{
    \xa=1;\ya=2;\xb=-0.5;\yb=3.5;} % Pour générer un triangle rectangle à partir de deux points
\begin{tikzpicture}
    \coordinate (a) at (\xa,\ya) ;
    \coordinate (b) at (\xb,\yb) ;
    \coordinate (m) at ($ (a)!.5!(b) $);
    \coordinate (o) at ($ (m)!1.5!90:(b)$);%Centre du cercle
    \coordinate (c) at ($ (b)!2!(o)$);
    \draw (a) node [right] {A}-- node [midway,below,sloped] {} (b)node [above] {B}-- node [midway, left] {10} (c) node[below] {C}-- node [midway,below right] {4} cycle;
    \draw[gray] ($(a)!2ex!(b)$) -- ($(a)!2ex!(b)!2ex!90:(b)$)--($(a)!2ex!(c)$); %Angle droit
\end{tikzpicture}
\end{center}   

Le triangle ABC est rectangle en A.

Comme on connaît le côté adjacent et l'hypoténuse, nous allons utiliser le cosinus. D'où : 
\begin{align*}
    &\cos (\widehat{C})=\dfrac{AC}{BC}\\
    \Rightarrow &\cos (\widehat{C})=\dfrac{4}{10}\\
    \Rightarrow & \widehat{C} = \cos ^{-1} \left(\dfrac{4}{10}\right)\\
    \Rightarrow & \widehat{C} \approx 66,42
\end{align*}
\end{minipage}
\hfil \vrule \hfil
\begin{minipage}[b] {0.45\textwidth}
\begin{center}
\tikzmath{
    \xa=1;\ya=2;\xb=1.5;\yb=3;} % Pour générer un triangle rectangle à partir de deux points
\begin{tikzpicture}
    \coordinate (a) at (\xa,\ya) ;
    \coordinate (b) at (\xb,\yb) ;
    \coordinate (m) at ($ (a)!.5!(b) $);
    \draw (a)--(m);
    \coordinate (o) at ($ (m)!3!90:(b)$);
    \coordinate (c) at ($ (b)!2!(o)$);
    \draw (a) node [right] {A}-- node [midway,right] {2} (b)node [above] {B}-- node [midway,above,sloped] {10} (c) node[left] {C}-- node [midway,below,sloped] {} cycle;
    \draw[gray] ($(a)!2ex!(b)$) -- ($(a)!2ex!(b)!2ex!90:(b)$)--($(a)!2ex!(c)$); %Angle droit
\end{tikzpicture} 
\end{center}
Le triangle ABC est rectangle en A. 

Comme on connaît l'hypoténuse et le côté opposé, et on cherche le côté adjacent, nous allons utiliser le sinus. D'où : 
\begin{align*}
    &\sin (\widehat{C})=\dfrac{AB}{BC}\\
    \Rightarrow &\sin (\widehat{C})=\dfrac{2}{10}\\
    \Rightarrow & \widehat{C} = \sin ^{-1} \left(\dfrac{2}{10}\right)\\
    \Rightarrow & \widehat{C} \approx 78,46
\end{align*}
\end{minipage}
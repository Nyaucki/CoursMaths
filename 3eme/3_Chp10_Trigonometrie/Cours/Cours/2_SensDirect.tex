\section{Trouver une longueur à partir d'un angle}

Il faut choisir la propriété à utiliser en fonction des informations à notre disposition et de ce que nous cherchons et utiliser la touche $\cos$, $\sin$ ou $\tan$ de la calculatrice conformément aux exemples ci-dessous.

\rmq{On veillera à ce que la calculatrice (qui fait tout le travail) soit bien en mode degré.}

On cherche à trouver la longueur AB dans les triangles ci-dessous :

\begin{minipage}[b]{0.45\textwidth}
\begin{center}
\tikzmath{
    \xa=-1;\ya=2;\xb=0;\yb=4;} % Pour générer un triangle rectangle à partir de deux points
\begin{tikzpicture}
    \coordinate (a) at (\xa,\ya) ;
    \coordinate (b) at (\xb,\yb) ;
    \coordinate (m) at ($ (a)!.5!(b) $);
    \draw (a)--(m);
    \coordinate (o) at ($ (m)!1.5!90:(b)$);
    \coordinate (c) at ($ (b)!2!(o)$);
    \draw (a) node [below] {A}-- node [midway,below,sloped] {} (b)node [right] {B}-- node [midway,above,sloped] {} (c) node[left] {C}-- node [midway,below,sloped] {4} cycle;
    \draw[gray] ($(a)!2ex!(b)$) -- ($(a)!2ex!(b)!2ex!90:(b)$)--($(a)!2ex!(c)$); %Angle droit
    \draw pic["$25$",draw=blue,fill=blue!20,angle eccentricity=1.3, angle radius=0.8cm]{angle=c--b--a};
\end{tikzpicture}
\end{center}   

Le triangle ABC est rectangle en A.

Comme on connaît le côté opposé, et on cherche le côté adjacent, nous allons utiliser la tangente. D'où : 
\begin{align*}
    &\tan (\widehat{B})=\dfrac{AC}{AB}\\
    \Rightarrow &\tan (25)=\dfrac{4}{AB}\\
    \Rightarrow & AB = \dfrac{4}{\tan(25)}\\
    \Rightarrow & AB \approx 8,58
\end{align*}
\end{minipage}
\hfil \vrule \hfil
\begin{minipage}[b] {0.45\textwidth}
\begin{center}
\tikzmath{
    \xa=-1;\ya=1;\xb=-2;\yb=3;} % Pour générer un triangle rectangle à partir de deux points
\begin{tikzpicture}
    \coordinate (a) at (\xa,\ya) ;
    \coordinate (b) at (\xb,\yb) ;
    \coordinate (m) at ($ (a)!.5!(b) $);
    \draw (a)--(m);
    \coordinate (o) at ($ (m)!1.5!90:(b)$);
    \coordinate (c) at ($ (b)!2!(o)$);
    \draw (a) node [right] {A}-- node [midway,below,sloped] {} (b)node [above] {B}-- node [midway,above,sloped] {10} (c) node[left] {C}-- node [midway,below,sloped] {} cycle;
    \draw[gray] ($(a)!2ex!(b)$) -- ($(a)!2ex!(b)!2ex!90:(b)$)--($(a)!2ex!(c)$); %Angle droit
    \draw pic["$40$",draw=blue,fill=blue!20,angle eccentricity=1.3, angle radius=0.8cm]{angle=c--b--a};
\end{tikzpicture} 
\end{center}
Le triangle ABC est rectangle en A. 

Comme on connaît l'hypoténuse', et on cherche le côté adjacent, nous allons utiliser le cosinus. D'où : 
\begin{align*}
    &\cos (\widehat{B})=\dfrac{AB}{BC}\\
    \Rightarrow &\cos (40)=\dfrac{AB}{10}\\
    \Rightarrow & AB = 10\times\cos (40) \\
    \Rightarrow & AB \approx 7,66
\end{align*}
\end{minipage}
\exo{4}{Calcul numérique}

Effectuer les calculs suivants en détaillant les étapes.

\begin{multicols}{4}
    \cnt\\ 
    $\dfrac{1}{2}+2\times\dfrac{5}{6}$

    \cnt\\
      $5^3-2$

    \cnt\\
    $4-(3-2\times 5)\times 2$

    \cnt\\
      $(9-2\times 3)^3+(5\times 4-1)$
\end{multicols}

\exo{4}{Cours}

Les figures suivantes ne sont pas à l'échelle. 

\begin{minipage}[t]{0.45\textwidth}
    \cnt Calculer $AC'$ 
        \begin{figure}[H]
        \centering
        \begin{tikzpicture}[scale=0.8]
            \node (A) at (4,1) {}; %positions A
            \node (B) at (8,2) {}; %positions B
            \node (C) at (3,4) {}; %positions C
            \node (B') at (0,0) {}; %positions B'
            \node (C') at (5,-2) {}; %positions C'
            \node [above=0.01cm of A] (A_label) {$A$}; %label A
            \node [right=0.01cm of B] (B_label) {$B$}; %label B
            \node [above=0.01cm of C] (C_label) {$C$}; %label C
            \node [left=0.01cm of B'] (B'_label) {$B'$}; %label B'
            \node [below=0.01cm of C'] (C'_label) {$C'$}; %label C'
            \draw (A.base) -- (B.base);
            \draw (A.base) -- (C.base) node [midway,right] {2};
            \draw (B.base) -- (C.base) node [midway,above] {4};
            \draw (A.base) -- (B'.base);
            \draw (A.base) -- (C'.base);
            \draw (B'.base) -- (C'.base) node [midway,above] {6} ;
            \draw [fill=black] (A) circle (0.21em) ;
            \draw [fill=black] (B) circle (0.21em) ;
            \draw [fill=black] (C) circle (0.21em) ;
            \draw [fill=black] (B') circle (0.21em) ;
            \draw [fill=black] (C') circle (0.21em) ;
        \end{tikzpicture}
        $(BC)$ et $(B'C')$ sont parallèles
    \end{figure} 
\end{minipage}
\hfill
\begin{minipage}[t]{0.45\textwidth}
    \cnt Calculer $CC'$
    \begin{figure}[H]
        \centering
        \begin{tikzpicture}[scale=1.3]
            \node (A) at (0,0) {}; %positions A
            \node (B) at (0.8,-1.6) {}; %positions B
            \node (C) at (1.6,-0.8) {}; %positions C
            \node (B') at (2,-4) {}; %positions B'
            \node (C') at (4,-2) {}; %positions C'
            \node [above=0.01cm of A] (A_label) {$A$}; %label A
            \node [left=0.01cm of B] (B_label) {$B$}; %label B
            \node [above=0.01cm of C] (C_label) {$C$}; %label C
            \node [left=0.01cm of B'] (B'_label) {$B'$}; %label B'
            \node [right=0.01cm of C'] (C'_label) {$C'$}; %label C'
            \draw (A.base) -- (B.base) node [midway,left] {1};
            \draw (A.base) -- (C.base) node [midway,above] {3};
            \draw (B.base) -- (C.base) ;
            \draw (B.base) -- (B'.base) node [midway,left] {2};
            \draw (A.base) -- (C'.base) ;
            \draw (B'.base) -- (C'.base) ;
            \draw [fill=black] (A) circle (0.14em) ;
            \draw [fill=black] (B) circle (0.14em) ;
            \draw [fill=black] (C) circle (0.14em) ;
            \draw [fill=black] (B') circle (0.14em) ;
            \draw [fill=black] (C') circle (0.14em) ;
        \end{tikzpicture}
        \\$(BC)$ et $(B'C')$ sont parallèles
    \end{figure} 
\end{minipage}
\vspace{3em}\\
\begin{minipage}[t]{0.45\textwidth}
    \cnt $(BC)$ et $(B'C')$ sont elles parallèles ?
    \begin{figure}[H]
        \centering
        \begin{tikzpicture}[scale=1.3]
            \node (A) at (0,0) {}; %positions A
            \node (B) at (0.8,-1.6) {}; %positions B
            \node (C) at (1.6,-0.8) {}; %positions C
            \node (B') at (2,-4) {}; %positions B'
            \node (C') at (4,-2) {}; %positions C'
            \node [above=0.01cm of A] (A_label) {$A$}; %label A
            \node [left=0.01cm of B] (B_label) {$B$}; %label B
            \node [above=0.01cm of C] (C_label) {$C$}; %label C
            \node [left=0.01cm of B'] (B'_label) {$B'$}; %label B'
            \node [right=0.01cm of C'] (C'_label) {$C'$}; %label C'
            \draw (A.base) -- (B.base) node [midway,left] {1,5};
            \draw (A.base) -- (C.base) node [midway,above] {6};
            \draw (B.base) -- (C.base) node [midway,right] {3};
            \draw (B.base) -- (B'.base) node [midway,left] {0,5};
            \draw (C.base) -- (C'.base) node [midway,above] {2};
            \draw (B'.base) -- (C'.base) node [midway,right] {4};
            \draw [fill=black] (A) circle (0.14em) ;
            \draw [fill=black] (B) circle (0.14em) ;
            \draw [fill=black] (C) circle (0.14em) ;
            \draw [fill=black] (B') circle (0.14em) ;
            \draw [fill=black] (C') circle (0.14em) ;
        \end{tikzpicture}
    \end{figure} 
\end{minipage}
\hfill
\begin{minipage}[t]{0.45\textwidth}
    \cnt $(BC)$ et $(B'C')$ sont elles parallèles ?
        \begin{figure}[H]
        \centering
        \begin{tikzpicture}[scale=0.8]
            \node (A) at (4,1) {}; %positions A
            \node (B) at (8,2) {}; %positions B
            \node (C) at (3,4) {}; %positions C
            \node (B') at (0,0) {}; %positions B'
            \node (C') at (5,-2) {}; %positions C'
            \node [above=0.01cm of A] (A_label) {$A$}; %label A
            \node [right=0.01cm of B] (B_label) {$B$}; %label B
            \node [above=0.01cm of C] (C_label) {$C$}; %label C
            \node [left=0.01cm of B'] (B'_label) {$B'$}; %label B'
            \node [below=0.01cm of C'] (C'_label) {$C'$}; %label C'
            \draw (A.base) -- (B.base) node [midway,below] {2};
            \draw (A.base) -- (C.base) node [midway,left] {4};
            \draw (B.base) -- (C.base) node [midway,above] {3};
            \draw (A.base) -- (B'.base) node [midway,above] {3};
            \draw (A.base) -- (C'.base) node [midway,right] {6};
            \draw (B'.base) -- (C'.base) node [midway,below] {5} ;
            \draw [fill=black] (A) circle (0.21em) ;
            \draw [fill=black] (B) circle (0.21em) ;
            \draw [fill=black] (C) circle (0.21em) ;
            \draw [fill=black] (B') circle (0.21em) ;
            \draw [fill=black] (C') circle (0.21em) ;
        \end{tikzpicture}
    \end{figure} 
\end{minipage}

\newpage

\exo{8}{Problème, calculer}

\begin{minipage}[t]{0.45\textwidth}
    Nicolas doit faire un parcours en canoë. Le parcours est représenté sur la figure de droite en trait pleins.
    
    Les pointillés servent à tracer le parcours.

    On donne les informations suivantes :
    \begin{itemize}
        \item Les points $B$, $D$ et $F$ sont alignés
        \item Les points $C$, $D$ et $E$ sont alignés
        \item Les points $A$, $B$ et $C$ sont alignés
        \item Les points $E$, $F$ et $G$ sont alignés
        \item Les points $A$, $D$ et $G$ sont alignés
    \end{itemize}
    \vspace{1em}

    \cnt Calculer la longueur $BD$.

    \cnt Montrer que les droites $(BC)$ et $(EF)$ sont parallèles.

    \cnt (/2 points) Calculer la longueur $DF$.

    \cnt (/2 points) Montrer que $FG=7,5~km$.

    \cnt Calculer la distance parcourue par Nicolas.

    \cnt Sachant que Nicolas va à 10 km/h, combien de temps (en heures et minutes) lui faudra-t-il pour finir le parcours ? 
\end{minipage}
\hfill
\begin{minipage}[t]{0.55\textwidth}
        \begin{figure}[H]
        \centering
        Toutes les longueurs sont données en kilomètres.\\ \vspace{1em}
        \begin{tikzpicture}[scale=1]
            \node (A) at (0,0) {}; %positions A
            \node (B) at (2,0) {}; %positions B
            \node (C) at (3,0) {}; %positions C
            \node (D) at (3,-3) {}; %positions D
            \node (E) at (3,-8) {}; %positions E
            \node (F) at (4.67,-8) {}; %positions F
            \node (G) at (8,-8) {}; %positions G
            \node [above=0.01cm of A] (A_label) {$A$}; %label A
            \node [above=0.01cm of B] (B_label) {$B$}; %label B
            \node [above=0.01cm of C] (C_label) {$C$}; %label C
            \node [left=0.01cm of D] (D_label) {$D$}; %label D
            \node [below=0.01cm of E] (E_label) {$E$}; %label E
            \node [below=0.01cm of F] (F_label) {$F$}; %label F
            \node [below=0.01cm of G] (G_label) {$G$}; %label G
            \draw (A.base) -- (B.base) node [midway,above] {3};
            \draw [dotted] (B.base) -- (C.base) node [midway,above] {1,5};
            \draw (2.75,0)--(2.75,-0.25) ;
            \draw (3,-0.25)--(2.75,-0.25) ;
            \draw (3,-7.75) -- (3.25,-7.75) ;
            \draw (3.25,-8) -- (3.25,-7.75) ;
            \draw  [dotted] (C.base) -- (D.base) node [midway,right] {2};
            \draw (B.base) -- (D.base);
            \draw  [dotted] (D.base) -- (E.base) node [midway,left] {6};
            \draw (D.base) -- (F.base);
            \draw [dotted] (E.base) -- (F.base);
            \draw (F.base) -- (G.base) ;
            \draw [fill=black] (A) circle (0.21em) ;
            \draw [fill=black] (B) circle (0.21em) ;
            \draw [fill=black] (C) circle (0.21em) ;
            \draw [fill=black] (D) circle (0.21em) ;
            \draw [fill=black] (E) circle (0.21em) ;
            \draw [fill=black] (G) circle (0.21em) ;
            \draw [fill=black] (F) circle (0.21em) ;
        \end{tikzpicture}
    \end{figure} 
\end{minipage}

\exo{4}{Modéliser, communiquer}

\begin{minipage}[t]{0.6\textwidth}
    Dans la série Fallout, on peut entendre la phrase suivante : "En cas d'explosion nucléaire, si le champignon atomique est plus petit que ton pouce à la verticale quand tu tends le bras, alors ça va."
    \vspace{1em}\\
    \cnt Faire un schéma représentant la situation avec uniquement des points et des segments. On représentera l'oeil par un point, et le bras, le pouce et le champignon atomique par des segments.
    \vspace{1em}\\
    On donne les longueurs suivantes (pour un adulte moyen) :

\begin{itemize}
    \item Hauteur d'un pouce : $8~cm$
    \item Longueur d'un bras : $90~cm$
    \item Hauteur d'un champignon atomique : $30~km$
\end{itemize}
\vspace{1em}
\cnt Préciser sur le schéma précédent les longueurs connues (en mètres) ainsi que les segments parallèles.
\end{minipage}
\hfill
\begin{minipage}[t]{0.35\textwidth}
    \begin{figure}[H]
        \centering
        \includegraphics[width=\textwidth]{fallout.png}
    \end{figure}
\end{minipage}
\vspace{1em}

Selon Lucy, la série dit n'importe quoi, parce qu'il faut être à plus de $250~km$ pour être en sécurité. 

\cnt (/2points) Expliquez à Lucy pourquoi la série à aussi raison à l'aide de théorèmes du cours.

\textit{Ceci est vrai, en théorie. En pratique, la zone de sécurité change en fonction du sens du vent. Donc si tu vois un champignon atomique, mieux vaut courir.}
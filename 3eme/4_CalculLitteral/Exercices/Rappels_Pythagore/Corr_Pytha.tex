\begin{multicols}{2}
    %%%%%%%%%%%%%%% Exercie 1 +12 %%%%%%%%%%%
     \exo{}{} On a :
     \begin{itemize}
         \item $ABC$ est rectangle en $B$
     \end{itemize}
     Ainsi, d'après le théorème de Pythagore :
     \begin{align*}
         &AC^2=AB^2+BC^2\\
         &13^2=AB^2+5^2\\
         &169=AB^2+25\\
         &AB^2=169-25\\
         &AB^2=144\\
        %  &AB=\sqrt{144}\\
         &AB=12
     \end{align*}
 
    %%%%%%%%%%%%%%% Exercie 2 +12 %%%%%%%%%%%
     \exo{}{} On a :
     \begin{itemize}
        \item $ABC$ est rectangle en $A$
    \end{itemize}
    Ainsi, d'après le théorème de Pythagore :
    \begin{align*}
        &BC^2=AC^2+AB^2\\
        &10^2=AC^2+6^2\\
        &100=AC^2+36\\
        &AC^2=100-36\\
        &AC^2=64\\
        % &AC=\sqrt{64}\\
        &AC=8
    \end{align*}
   \end{multicols}
% \hrule \vspace{-0.5em} %Car fin de page 
\begin{multicols}{2}
    %%%%%%%%%%%%%%% Exercie 3 +12 %%%%%%%%%%%
    \exo{}{} On a :
    \begin{multicols}{2}
        \noindent
        \begin{align*}
            AC^2&=7^2\\
            &=49
        \end{align*}
        \begin{align*}
            AB^2+BC^2&=5^2+5^2\\
            &=25+25\\
            &=50
        \end{align*}
    \end{multicols}
    Ainsi, on a :
    \begin{itemize}
        \item $AC^2\neq  AB^2+BC^2$
        \item $AC$ est le plus grand côté
    \end{itemize}
    Donc, d'après la contraposé du théorème de Pythagore, $ABC$ n'est pas rectangle.

    %%%%%%%%%%%%%%% Exercie 4 +12 %%%%%%%%%%%
    \exo{}{} On a :
    \begin{multicols}{2}
        \noindent
        \begin{align*}
            BC^2&=*17^2\\
            &=289
        \end{align*}
        \begin{align*}
            AB^2+AC^2&=15^2+8^2\\
            &=225+64\\
            &=289
        \end{align*}
    \end{multicols}
    Ainsi, on a :
    \begin{itemize}
        \item $BC^2 = AB^2+AC^2$
    \end{itemize}
    Donc, d'après la réciproque du théorème de Pythagore, $ABC$ est rectangle en $A$.
\end{multicols}
\hrule \vspace{-0.5em}%séparation

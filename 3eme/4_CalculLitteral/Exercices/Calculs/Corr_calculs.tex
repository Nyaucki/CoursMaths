\begin{multicols}{2}
    %%%%%%%%%%%% Exercie 1 +28 %%%%%%%%
    \exo{}{} On a :
    \begin{itemize}
      \item $H$, $G$ et $J$ sont alignés
      \item $H$, $E$ et $D$ sont alignés
      \item $(EG)$ et $(JD)$ sont parallèles
    \end{itemize}
    Ainsi, d'après le théorème de Thalès on a :
    \begin{align*}
      &\dfrac{HG}{HJ}=\dfrac{HE}{HD}=\dfrac{GE}{JD}&&\\
      &\dfrac{3}{HJ}=\dfrac{1}{6}=\dfrac{GE}{JD}&&\text{Remplacer les valeurs}\\
      &\dfrac{3}{HJ}=\dfrac{1}{6}=&&\text{Garder les fractions utiles}\\
      &HJ\times 1=3\times 6 &&\text{Egalité produits en croix}\\
      &HJ=18
    \end{align*}
    %%%%%%%%%%%% Exercie 2 +28%%%%%%%
    \exo{}{} On a : \vspace{-0.5em}
    \begin{multicols}{3}
        \noindent
        \begin{align*}
            \dfrac{YW}{YB}&=\dfrac{2}{2,5}\\
            &=\dfrac{4}{5}
        \end{align*}
        \begin{align*}
            \dfrac{YM}{YZ}&=\dfrac{1}{1,5}\\
            &=\dfrac{2}{3}
        \end{align*}
        \begin{align*}
            \dfrac{WM}{ZB}&=\dfrac{4}{5}\\
            &=\dfrac{1}{3}
        \end{align*}
    \end{multicols}
    \vspace{-1.75em }
    Ainsi :
    \begin{itemize}
        \item $\dfrac{YW}{YB}\neq\dfrac{YM}{YZ}\vspace{0.25em}$
        \item $W$, $Y$ et $B$ sont alignés
        \item $M$, $Y$ et $Z$ sont alignés
    \end{itemize}
    Donc, d'après la contraposé du théorème de Thalès, $(WM)$ et $(ZB)$ ne sont pas parallèles.
  \end{multicols}
\hrule \vspace{-0.5em}%séparation

\begin{multicols}{2}
    %%%%%%%%%%%% Exercie 3 +28 %%%%%%%%
    \exo{}{} On a :
    \begin{itemize}
      \item $A$, $B$ et $B'$ sont alignés
      \item $A$, $C$ et $C'$ sont alignés
      \item $(BC)$ et $(B'C')$ sont parallèles
    \end{itemize}
    Ainsi, d'après le théorème de Thalès on a :
    \begin{align*}
        &\dfrac{AB}{AB'}=\dfrac{AC}{AC'}=\dfrac{BC}{B'C'}&&\\
        &\dfrac{\dfrac{3}{4}}{AB'}=\dfrac{AC}{AC'}=\dfrac{\dfrac{1}{3}}{\dfrac{5}{2}}&&\text{Remplacer les valeurs}\\
        &\dfrac{\dfrac{3}{4}}{AB'}=\dfrac{\dfrac{1}{3}}{\dfrac{5}{2}} &&\text{Garder les fractions utiles}\\
        &\dfrac{\dfrac{3}{4}}{AB'}=\dfrac{\dfrac{1}{3}}{\dfrac{5}{2}} &&\text{Garder les fractions utiles}\\
        &AB'\times \dfrac{1}{3}=\dfrac{3}{4}\times \dfrac{5}{2}&&\text{Egalité produits en croix}\\
        &AB'\times \dfrac{1}{3}= \dfrac{15}{8}&&\\
        &AB'=\dfrac{15}{8}:\dfrac{1}{3}&&\\
        &AB'=\dfrac{45}{8}
    \end{align*}
    %%%%%%%%%%%% Exercie 4 +28%%%%%%%
    \exo{}{} On a : \vspace{-0.5em}
    \begin{multicols}{3}
        \noindent
        \begin{align*}
            \dfrac{AB}{AB'}&=\dfrac{\dfrac{2}{3}}{\dfrac{2}{5}}\\
            &=\dfrac{2}{3}:\dfrac{2}{5}\\
            &=\dfrac{2}{3}\times\dfrac{5}{2}\\
            &=\dfrac{5}{3}
        \end{align*}
        \begin{align*}
            \dfrac{AC}{AC'}&=\dfrac{\dfrac{1}{4}}{\dfrac{3}{20}}\\
            &=\dfrac{1}{4}:\dfrac{3}{20}\\
            &=\dfrac{1}{4}\times\dfrac{20}{3}\\
            &=\dfrac{5}{3}
        \end{align*}
        \begin{align*}
            \dfrac{BC}{BC'}&=\dfrac{\dfrac{1}{2}}{\dfrac{3}{10}}\\
            &=\dfrac{1}{2}:\dfrac{3}{10}\\
            &=\dfrac{1}{2}\times\dfrac{10}{3}\\
            &=\dfrac{5}{3}
        \end{align*}
    \end{multicols}
    \vspace{-1.75em }
    Ainsi :
    \begin{itemize}
        \item $\dfrac{AB}{AB'}=\dfrac{AC}{AC'}=\dfrac{BC}{BC'}$\vspace{0.25em}
        \item $W$, $Y$ et $B$ sont alignés
        \item $M$, $Y$ et $Z$ sont alignés
    \end{itemize}
    Donc, d'après la réciproque du théorème de Thalès, $(BC)$ et $(B'C')$ sont parallèles.
  \end{multicols}
%\hrule \vspace{-0.5em}%séparation

\newpage
    %%%%%%%%%%%% Exercie 5+28 %%%%%%%%
\exo{}{}

\begin{multicols}{2}
    \underline{Calcul de $BC$ :}

    \begin{itemize}
      \item $A$, $E$ et $F$ sont alignés
      \item $A$, $B$ et $C$ sont alignés
      \item $(EB)$ et $(CF)$ sont parallèles
    \end{itemize}
    Ainsi, d'après le théorème de Thalès on a :
    \begin{align*}
      &\dfrac{AB}{AC}=\dfrac{AE}{AF}=\dfrac{BE}{CF}&&\\
      &\dfrac{3}{AC}=\dfrac{2}{6}=\dfrac{BE}{CF}&&\text{Remplacer les valeurs}\\
      &\dfrac{3}{AC}=\dfrac{2}{6}&&\text{Garder les fractions utiles}\\
      &AC\times 2=3\times 6 &&\text{Egalité produits en croix}\\
      &AC\times 2=18 &&\\
      &AC=18:2&&\\
      &AC=9&&\\
      &BC=6&&
    \end{align*}
    %%%%%%%%%%%% ED %%%%%%%
    \underline{Calcul de $ED$ :}

    \begin{itemize}
      \item $F$, $E$ et $A$ sont alignés
      \item $F$, $D$ et $C$ sont alignés
      \item $(AC)$ et $(ED)$ sont parallèles
    \end{itemize}
    Ainsi, d'après le théorème de Thalès on a :
    \begin{align*}
      &\dfrac{FE}{FA}=\dfrac{FD}{FC}=\dfrac{ED}{AC}&&\\
      &\dfrac{4}{6}=\dfrac{FD}{FC}=\dfrac{ED}{9}&&\text{Remplacer les valeurs}\\
      &\dfrac{4}{6}=\dfrac{ED}{9} &&\text{Garder les fractions utiles}\\
      &ED\times 6=9\times 4&&\text{Egalité produits en croix}\\
      &ED\times 6=36 &&\\
      &ED=36:6 &&\\
      &ED=6
    \end{align*}
  \end{multicols}

%%%%%%%%%%%% Exercie 6+28 %%%%%%%%
\exo{}{}


\begin{multicols}{2}
  \underline{Calcul de $CE$ :}

  \begin{itemize}
    \item $C$, $E$ et $B$ sont alignés
    \item $C$, $D$ et $A$ sont alignés
    \item $(ED)$ et $(AB)$ sont parallèles
  \end{itemize}
  Ainsi, d'après le théorème de Thalès on a :
  \begin{align*}
    &\dfrac{CA}{CD}=\dfrac{CB}{CE}=\dfrac{AB}{ED}&&\\
    &\dfrac{3}{1}=\dfrac{CB}{CE}=\dfrac{2}{ED}&&\text{Remplacer les valeurs}\\
    &\dfrac{3}{1}=\dfrac{2}{ED}&&\text{Garder les fractions utiles}\\
    &ED\times 3=1\times 2 &&\text{Egalité produits en croix}\\
    &ED\times 3=2 &&\\
    &ED=\dfrac{2}{3} &&
  \end{align*}
  \columnbreak
  
  %%%%%%%%%%%% ED %%%%%%%
  \underline{Calcul de $CG$ :}

  \begin{itemize}
    \item $C$, $E$ et $F$ sont alignés
    \item $C$, $D$ et $G$ sont alignés
    \item $(FG)$ et $(ED)$ sont parallèles
  \end{itemize}
  Ainsi, d'après le théorème de Thalès on a :
  \begin{align*}
    &\dfrac{CD}{CG}=\dfrac{CE}{CF}=\dfrac{ED}{FG}&&\\
    &\dfrac{1}{CG}=\dfrac{CE}{CF}=\dfrac{\dfrac{2}{3}}{5}&&\text{Remplacer les valeurs}\\
    &\dfrac{1}{CG}=\dfrac{\dfrac{2}{3}}{5} &&\text{Garder les fractions utiles}\\
    &CG\times \dfrac{2}{3}=5\times 1&&\text{Egalité produits en croix}\\
    &EG=5:\dfrac{2}{3} &&\\
    &EG=7,5 &&
  \end{align*}
\end{multicols}

\newpage

%%%%%%%%%%%% Exercie 7+28 %%%%%%%%
\exo{}{}


\begin{multicols}{2}
  \underline{$(BG)$ et $(DI)$ :}

  \begin{multicols}{3}
    \noindent
    \begin{align*}
        \dfrac{AG}{AI}&=\dfrac{1}{4}\\
    \end{align*}
    \begin{align*}
        \dfrac{AB}{AD}&=\dfrac{0,75}{3}\\
        &=\dfrac{1}{4}
    \end{align*}
    \begin{align*}
        \dfrac{BG}{DI}&=\dfrac{1,5}{6}\\
        &=\dfrac{1}{4}
    \end{align*}
\end{multicols}
\vspace{-1.75em }
Ainsi :
\begin{itemize}
    \item $\dfrac{AG}{AI}=\dfrac{AC}{AE}=\dfrac{BG}{DI}$\vspace{0.25em}
    \item $A$, $G$ et $I$ sont alignés
    \item $A$, $B$ et $D$ sont alignés
\end{itemize}
Donc, d'après la réciproque du théorème de Thalès, $(BG)$ et $(DI)$ sont parallèles.
  
  %%%%%%%%%%%% ED %%%%%%%
  \underline{$(EF)$ et $(CH)$ :}

  \begin{multicols}{3}
    \noindent
    \begin{align*}
        \dfrac{AH}{AF}&=\dfrac{3}{8}\\
    \end{align*}
    \begin{align*}
        \dfrac{AC}{AE}&=\dfrac{1,5}{4}\\
        &=\dfrac{3}{8}
    \end{align*}
    \begin{align*}
        \dfrac{CH}{EF}&=\dfrac{4,5}{12}\\
        &=\dfrac{3}{8}
    \end{align*}
\end{multicols}
\vspace{-1.75em }
Ainsi :
\begin{itemize}
    \item $\dfrac{AH}{AF}=\dfrac{AC}{AE}=\dfrac{CH}{EF}$\vspace{0.25em}
    \item $A$, $H$ et $F$ sont alignés
    \item $A$, $C$ et $E$ sont alignés
\end{itemize}
Donc, d'après la réciproque du théorème de Thalès, $(CH)$ et $(EF)$ sont parallèles
\end{multicols}

%%%%%%%%%%%% Exercie 8+28 %%%%%%%%
\exo{}{}


\begin{multicols}{2}
  \underline{$(AB)$ et $(DE)$ :}

  \begin{multicols}{3}
    \noindent
    \begin{align*}
        \dfrac{CA}{CE}&=\dfrac{4}{2}\\
        &=2
    \end{align*}
    \begin{align*}
        \dfrac{CB}{CD}&=\dfrac{3}{1,5}\\
        &=2
    \end{align*}
    \begin{align*}
        \dfrac{AB}{DE}&=\dfrac{2}{1}\\
        &=2
    \end{align*}
  \end{multicols}
  \vspace{-1.75em }
  Ainsi :
  \begin{itemize}
      \item $\dfrac{CA}{CE}=\dfrac{CB}{CD}=\dfrac{AB}{DE}$\vspace{0.25em}
      \item $C$, $A$ et $E$ sont alignés
      \item $C$, $B$ et $D$ sont alignés
  \end{itemize}
  Donc, d'après la réciproque du théorème de Thalès, $(AB)$ et $(DE)$ sont parallèles.
    
  %%%%%%%%%%%% ED %%%%%%%
  \underline{$(DE)$ et $(GH)$ :}

  \begin{multicols}{3}
    \noindent
    \begin{align*}
        \dfrac{GH}{DE}&=\dfrac{3}{1}\\
        &=3
    \end{align*}
    \begin{align*}
        \dfrac{FG}{FE}&=\dfrac{6}{2}\\
        &=3
    \end{align*}
    \begin{align*}
        \dfrac{FH}{FD}&=\dfrac{4,5}{1,5}\\
        &=3
    \end{align*}
  \end{multicols}
  \vspace{-1.75em }
  Ainsi :
  \begin{itemize}
      \item $\dfrac{GH}{DE}=\dfrac{FG}{FE}=\dfrac{FH}{FD}$\vspace{0.25em}
      \item $A$, $H$ et $F$ sont alignés
      \item $A$, $C$ et $E$ sont alignés
  \end{itemize}
  Donc, d'après la réciproque du théorème de Thalès, $(GH)$ et $(DE)$ sont parallèles
\end{multicols}
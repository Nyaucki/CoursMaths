\dfnt{Points Invariants}
{Les points invariants d'une trnasformations sont ceux qui ne bougent pas avec la transformation : ils sont leurs propres image.}

\prop{Invariants de transformations}
{Les translations, symétrie axiales, symétrie centrales et les rotations conservent :
\begin{itemize}
    \item Les distances
    \item Les alignements
    \item Les angles
    \item Les droites parallèles
\end{itemize}
Les homotéties ne conservent que :
\begin{itemize}
    \item Les alignements
    \item Les angles
    \item Les droites parallèles
\end{itemize}
}

\rmq{Certaines homotétie particulières (de rapport 1 ou -1) conservent les distances.}%Pourra être donné en exo de type chercher

\prop{Parallelisme de l'image}
{L'image d'une droite par une translation, une symétrie centrale ou une homotétie est une droite parallèle à la première.\\
L'image d'une droite par une symétrie axiale ou une rotation est une droite qui n'est en générale pas parallèle à celle de départ.}

%Idem, faire chercher mes exceptions en exo.
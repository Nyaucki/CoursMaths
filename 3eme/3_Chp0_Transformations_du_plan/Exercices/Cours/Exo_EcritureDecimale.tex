\textbf{Pour les exercices \ref{Ecriture1} à \ref{Ecriture8}}, écrire les fractions données sous forme de nombres décimaux (exemple : $\dfrac{212}{100}=2,12$)

\begin{multicols}{4}
    \exo{cours}{Ecriture1}\vspace{0.5em}\\
    $\dfrac{32}{10}$

    \exo{cours}{Ecriture2}\vspace{0.5em}\\
    $\dfrac{475}{10}$

    \exo{cours}{Ecriture3}\vspace{0.5em}\\
    $\dfrac{34}{100}$

    \exo{cours}{Ecriture4}\vspace{0.5em}\\
    $\dfrac{327}{100}$
\end{multicols}

\begin{multicols}{4}
    \exo{cours}{Ecriture5}\vspace{0.5em}\\
    $\dfrac{8}{10}$

    \exo{cours}{Ecriture6}\vspace{0.5em}\\
    $\dfrac{98}{10}$

    \exo{cours}{Ecriture7}\vspace{0.5em}\\
    $\dfrac{1024}{100}$

    \exo{cours}{Ecriture8}\vspace{0.5em}\\
    $\dfrac{5}{100}$
\end{multicols}

\textbf{Pour les exercices \ref{Ecriture9} à \ref{Ecriture16}}, écrire les nombres données sous forme de fractions décimales (exemple : $2,12=\dfrac{212}{100}$)

\begin{multicols}{4}
    \exo{cours}{Ecriture9}\\
    $3,5$

    \exo{cours}{Ecriture10}\\
    $8,96$

    \exo{cours}{Ecriture11}\\
    $12,1$

    \exo{cours}{Ecriture12}\\
    $0,13$
\end{multicols}

\begin{multicols}{4}
    \exo{cours}{Ecriture13}\\
    $77,23$

    \exo{cours}{Ecriture14}\\
    $13,05$

    \exo{cours}{Ecriture15}\\
    $0,045$

    \exo{cours}{Ecriture16}\\
    $31,91$
\end{multicols}

\textbf{Pour les exercices \ref{Decompose1} à \ref{Decompose8}}, écrire les fractions données sous forme de nombres décimaux (exemple : $\dfrac{213}{100}=2+\dfrac{1}{10}+\dfrac{3}{100}$)

\begin{multicols}{4}
    \exo{cours}{Decompose1}\vspace{0.5em}\\
    $\dfrac{345}{100}$

    \exo{cours}{Decompose2}\vspace{0.5em}\\
    $\dfrac{625}{10}$

    \exo{cours}{Decompose3}\vspace{0.5em}\\
    $\dfrac{8}{10}$

    \exo{cours}{Decompose4}\vspace{0.5em}\\
    $\dfrac{947}{100}$
\end{multicols}

\begin{multicols}{4}
    \exo{cours}{Decompose5}\\
    $4,25$

    \exo{cours}{Decompose6}\\
    $6,4$

    \exo{cours}{Decompose7}\\
    $71,82$

    \exo{cours}{Decompose8}\\
    $0,324$
\end{multicols}
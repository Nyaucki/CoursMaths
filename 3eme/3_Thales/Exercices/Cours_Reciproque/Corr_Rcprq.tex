\begin{multicols}{2}
   %%%%%%%%%%%%%%% Exercie 1 +6 %%%%%%%%%%%
    \exo{}{} On a : \vspace{-0.5em}
    \begin{multicols}{3}
        \noindent
        \begin{align*}
            \dfrac{AC}{AC'}&=\dfrac{2,5}{5}\\
            &=\dfrac{1}{2}
        \end{align*}
        \begin{align*}
            \dfrac{AB}{AB'}&=\dfrac{8}{4}\\
            &=2
        \end{align*}
        \begin{align*}
            \dfrac{BC}{B'C'}&=\dfrac{6}{3}\\
            &=2
        \end{align*}
    \end{multicols}
    \vspace{-1.75em }
    Ainsi :
    \begin{itemize}
        \item $\dfrac{AC}{AC'}\neq\dfrac{AB}{AB'}$\vspace{0.25em}
        \item $A$, $B$ et $C$ sont alignés
        \item $A$, $B'$ et $C'$ sont alignés
    \end{itemize}
    Donc, d'après la contraposé du théorème de Thalès, $(BC)$ et $(B'C')$ ne sont pas parallèles.

   %%%%%%%%%%%%%%% Exercie 2 +6 %%%%%%%%%%%
    \exo{}{} On a : \vspace{-0.5em}\
    \begin{multicols}{3}
        \noindent
        \begin{align*}
            \dfrac{AC}{AC'}&=\dfrac{2}{6}\\
            &=\dfrac{1}{3}
        \end{align*}
        \begin{align*}
            \dfrac{AB}{AB'}&=\dfrac{1}{3}
        \end{align*}
        \begin{align*}
            \dfrac{BC}{B'C'}&=\dfrac{2}{4}\\
            &=\dfrac{1}{2}
        \end{align*}
    \end{multicols}
    \vspace{-1.75em }
    Ainsi :
    \begin{itemize}
        \item $\dfrac{AB}{AB'}\neq\dfrac{BC}{B'C'}$\vspace{0.25em}
        \item $A$, $B$ et $C$ sont alignés
        \item $A$, $B'$ et $C'$ sont alignés
    \end{itemize}
    Donc, d'après la contraposé du théorème de Thalès, $(BC)$ et $(B'C')$ ne sont pas parallèles.
  \end{multicols}
\hrule \vspace{-0.5em}%séparation
\begin{multicols}{2}
    %%%%%%%%%%%%%%% Exercie 3 +6 %%%%%%%%%%%
    \exo{}{} On a : \vspace{-0.5em}
    \begin{multicols}{3}
        \noindent
        \begin{align*}
            \dfrac{AC}{AC'}&=\dfrac{2}{10}\\
            &=\dfrac{1}{5}
        \end{align*}
        \begin{align*}
            \dfrac{AB}{AB'}&=\dfrac{1}{5}
        \end{align*}
        \begin{align*}
            \dfrac{BC}{B'C'}&=\dfrac{0,5}{2,5}\\
            &=\dfrac{1}{5}
        \end{align*}
    \end{multicols}
    \vspace{-1.75em }
    Ainsi :
    \begin{itemize}
        \item $\dfrac{AC}{AC'}=\dfrac{AB}{AB'}=\dfrac{BC}{B'C'}$\vspace{0.25em}
        \item $A$, $B$ et $C$ sont alignés
        \item $A$, $B'$ et $C'$ sont alignés
    \end{itemize}
    Donc, d'après la réciproque du théorème de Thalès, $(BC)$ et $(B'C')$ sont parallèles.
 
    %%%%%%%%%%%%%%% Exercie 4 +6 %%%%%%%%%%%
    \exo{}{} On a : \vspace{-0.5em}
    \begin{multicols}{3}
        \noindent
        \begin{align*}
            \dfrac{AC}{AC'}&=\dfrac{2}{4,5}\\
            &=\dfrac{4}{9}
        \end{align*}
        \begin{align*}
            \dfrac{AB}{AB'}&=\dfrac{4}{5}
        \end{align*}
        \begin{align*}
            \dfrac{BC}{B'C'}&=\dfrac{6}{8}\\
            &=\dfrac{3}{4}
        \end{align*}
    \end{multicols}
    \vspace{-1.75em }
    Ainsi :
    \begin{itemize}
        \item $\dfrac{AB}{AB'}\neq\dfrac{BC}{B'C'}$\vspace{0.25em}
        \item $A$, $B$ et $C$ sont alignés
        \item $A$, $B'$ et $C'$ sont alignés
    \end{itemize}
    Donc, d'après la contraposé du théorème de Thalès, $(BC)$ et $(B'C')$ ne sont pas parallèles.
\end{multicols}
\hrule \vspace{-0.5em}%séparation
\begin{multicols}{2}
    %%%%%%%%%%%%%%% Exercie 5 +6 %%%%%%%%%%%
    \exo{}{} On a : \vspace{-0.5em}
    \begin{multicols}{3}
        \noindent
        \begin{align*}
            \dfrac{AC}{AC'}&=\dfrac{1,5}{4,5}\\
            &=\dfrac{1}{3}
        \end{align*}
        \begin{align*}
            \dfrac{AB}{AB'}&=\dfrac{2}{6}\\
            &=\dfrac{1}{3}
        \end{align*}
        \begin{align*}
            \dfrac{BC}{B'C'}&=\dfrac{3}{9}\\
            &=\dfrac{1}{3}
        \end{align*}
    \end{multicols}
    \vspace{-1.75em }
    Ainsi :
    \begin{itemize}
        \item $\dfrac{AC}{AC'}=\dfrac{AB}{AB'}=\dfrac{BC}{B'C'}$\vspace{0.25em}
        \item $A$, $B$ et $C$ sont alignés
        \item $A$, $B'$ et $C'$ sont alignés
    \end{itemize}
    Donc, d'après la réciproque du théorème de Thalès, $(BC)$ et $(B'C')$ sont parallèles.

    %%%%%%%%%%%%%%% Exercie 6 +6 %%%%%%%%%%%
    \exo{}{} On a : \vspace{-0.5em}
    \begin{multicols}{3}
        \noindent
        \begin{align*}
            \dfrac{AC}{AC'}&=\dfrac{2}{2,5}\\
            &=\dfrac{4}{5}
        \end{align*}
        \begin{align*}
            \dfrac{AB}{AB'}&=\dfrac{1}{1,5}\\
            &=\dfrac{2}{3}
        \end{align*}
        \begin{align*}
            \dfrac{BC}{B'C'}&=\dfrac{4}{5}
        \end{align*}
    \end{multicols}
    \vspace{-1.75em }
    Ainsi :
    \begin{itemize}
        \item $\dfrac{AC}{AC'}\neq\dfrac{AB}{AB'}$\vspace{0.25em}
        \item $A$, $B$ et $C$ sont alignés
        \item $A$, $B'$ et $C'$ sont alignés
    \end{itemize}
    Donc, d'après la contraposé du théorème de Thalès, $(BC)$ et $(B'C')$ ne sont pas parallèles.
\end{multicols}
\hrule \vspace{-0.5em}

% \begin{itemize}
%     \item $\dfrac{AC}{AC'}=\dfrac{2,5}{5}=\dfrac{1}{2}$
%     \item $\dfrac{AB}{AB'}=\dfrac{8}{4}=2$
%     \item $\dfrac{BC}{B'C'}=\dfrac{6}{3}=2$
% \end{itemize}
\prop{Le théorème de Thalès}
{\begin{minipage}{0.40\textwidth}
    Soit deux triangles $ABC$ et $AB'C'$.\\
    Si on a  :
    \begin{itemize}
        \item $A$, $B$ et $B'$ sont alignés
        \item $A$, $C$ et $C'$ sont alignés
        \item ($BC$) et ($B'C'$) sont parallèles
    \end{itemize}
    \vspace{1em}
    Alors, $\dfrac{AB}{AB'}=\dfrac{AC}{AC'}=\dfrac{BC}{B'C'}$
\end{minipage}
\hfill
\begin{minipage}{0.55\textwidth}
    \begin{tikzpicture}[scale=2]
        \draw (-1,0) node[below] {$A$} -- (3,0) node[below] {$B'$} ; 
        \draw (-1,0) -- (2,2) node[right] {$C'$} ; 
        \draw (3,0) -- (2,2) ;
        \draw (1.25,1.5) node[above] {$C$} -- (2,0) node[below] {$B$} ; 
        \draw [fill=black] (-1,0) circle (0.04) ;
        \draw [fill=black] (3,0) circle (0.04) ;
        \draw [fill=black] (2,0) circle (0.04) ;
        \draw [fill=black] (2,2) circle (0.04) ;
        \draw [fill=black] (1.25,1.5) circle (0.04) ;
    \end{tikzpicture}
\end{minipage}}

\rmq{On a aussi $\dfrac{AB}{AB'}=\dfrac{AC}{AC'}=\dfrac{BC}{B'C'}$ : tant que $B'$ et $C'$ sont au même niveau dans les fractions, l'égalité est juste.}

\rmq{Les hypothèses "$A$, $B$ et $B'$ sont alignés" et "$A$, $C$ et $C'$ sont alignés" sont équivalentes à : "Les droites $(BB')$ et $(CC')$ sont sécantes en $A$". On peut donc écrire l'un ou l'autre.}

\exmpl
{Dans les deux triangles suivants, $(BC)$ est parallèle à $(B'C')$.\\
\begin{minipage}{0.45\textwidth}
    \begin{figure}[H]
        \centering
        \begin{tikzpicture}[scale=1.5]
            \draw (0,0) node[below] {$A$} -- (2,-4) node[below] {$B'$} ; 
            \draw (1,-0.5) -- (4,-2) node[right] {$C'$} node [midway,above] {8}; %CC'
            \draw (0,0) -- (1,-0.5) node [midway,above] {2}; %AC
            \draw (2,-4) -- (4,-2) node [midway,below] {6}; %B'C'
            \draw (1,-0.5) node[above] {$C$} -- (0.5,-1) node[below] {$B$} ; 
            \draw [fill=black] (0,0) circle (0.04) ; %A
            \draw [fill=black] (2,-4) circle (0.04) ; %B'
            \draw [fill=black] (0.5,-1) circle (0.04) ; %B
            \draw [fill=black] (4,-2) circle (0.04) ;%C'
            \draw [fill=black] (1,-0.5) circle (0.04) ;%C
        \end{tikzpicture}
    \end{figure} 
    \textbf{Cherchons à trouver la longueur BC :}
    \begin{itemize}
        \item $A$, $B$ et $B'$ sont alignés
        \item $A$, $C$ et $C'$ sont alignés
        \item ($BC$) et ($B'C'$) sont parallèles
    \end{itemize}
    D'après le théorème de Thalès, on a l'égalité : 
    \begin{align*}
        \dfrac{AB}{AB'}&=\dfrac{AC}{AC'}=\dfrac{BC}{B'C'}&&\\
        \dfrac{AB}{AB'}&=\dfrac{2}{2+8}=\dfrac{BC}{6}&&\text{AC'=AC+CC'}\\
        \dfrac{2}{10}&=\dfrac{BC}{6}&&\text{On retire la fraction inutile}\\
        \dfrac{2}{10}\times 6&=BC&&\text{On isole BC}\\
        \dfrac{12}{10}&=BC&&\\
        \dfrac{6}{5}&=BC&&\text{On a simplifié par 2}
    \end{align*}
    Donc, $BC=\dfrac{6}{5}$
\end{minipage}
\hfill
\begin{minipage}{0.45\textwidth}
    \begin{figure}[H]
        \centering
        \begin{tikzpicture}[scale=1.5]
            \draw (0,0) node (A)[above] {$A$} -- (2,-4) node [below] {$B'$} ; 
            \draw (0,0) node (A)[above] {$A$} -- (1,-2) node [midway,above] {2} ;
            \draw (-1,-2) -- (-2,-4) node[below] {$C'$} node [midway,above] {3} ; %CC'
            \draw (0,0) -- (-1,-2) node [midway,above] {2} ; %AC
            \draw (2,-4) -- (-2,-4) ;%C'B'
            \draw (-1,-2) node[left] {$C$} -- (1,-2) node[right] {$B$} ; 
            \draw [fill=black] (0,0) circle (0.04) ;
            \draw [fill=black] (2,-4) circle (0.04) ;
            \draw [fill=black] (1,-2) circle (0.04) ;
            \draw [fill=black] (-2,-4) circle (0.04) ;
            \draw [fill=black] (-1,-2) circle (0.04) ;
        \end{tikzpicture}
    \end{figure}
    \textbf{Cherchons à trouver la longueur AB' :}
    \begin{itemize}
        \item Les droites $(BB')$ et $(CC')$ sont sécantes en $A$
        \item ($BC$) et ($B'C'$) sont parallèles
    \end{itemize}
    D'après le théorème de Thalès, on a l'égalité : 
    \begin{align*}
        \dfrac{AB}{AB'}&=\dfrac{AC}{AC'}=\dfrac{BC}{B'C'}&&\\
        \dfrac{2}{AB'}&=\dfrac{2}{5}=\dfrac{BC}{B'C'}&&\text{AC'=AC+CC'}\\
        \dfrac{2}{AB'}&=\dfrac{2}{5}&&\text{On retire la fraction inutile}\\
        AB'\times 2&=2\times 5&&\text{Egalité des produits en croix}\\
        AB'\times 2&=10&&\\
        AB'&=5&&
    \end{align*}
    Donc, $AB'=5$ 
\end{minipage}
    }


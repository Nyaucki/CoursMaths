\exo{4}{Calcul numérique}

Effectuer les calculs suivants en détaillant les étapes.

\begin{multicols}{2} 
    \begin{align*}
      &\cnt\\
      &\dfrac{1}{2}+2\times\dfrac{5}{6}&&\\
      =&\dfrac{1}{2}+\dfrac{2\times 5}{6}&&\\
      =&\dfrac{1}{2}+\dfrac{10}{6}&&\\
      =&\dfrac{1\times 3}{2\times 3}+\dfrac{10}{6} & &\text{Même dénominateur}\\
      =&\dfrac{3}{6}+\dfrac{10}{6} & &\\
      =&\dfrac{3+10}{6} & &\\
      =&\dfrac{13}{6} & &
    \end{align*}
    \columnbreak\\
    \begin{align*}
      &\cnt\\
      &5^3-2\\
      =&5\times 5\times 5 -2\\
      =&25\times 5-2\\
      =&125-2\\
      =&123
    \end{align*}
\end{multicols}

\begin{multicols}{2}
  \begin{align*}
    &\cnt\\ 
    &4-(3-2\times 5)\times 2&&\\
    =&4-(3-10)\times 2&&\text{Produit dans parenthèse}\\
    =&4-(-7)\times 2&&\text{Continue la parenthèse}\\
    =&4-(-14)&&\text{Produit avant soustraction}\\
    =&4+14&&-\times - \rightarrow +\\
    =&18
  \end{align*}
  \columnbreak\\
  \begin{align*}
    &\cnt\\
    &(9-2\times 3)^3+(5\times 4-1)&&\\
    =&(9-6)^3+(20-1)&&\text{Produit dans parenthèse}\\
    =&(3)^3+(19)&&\text{Parenthèse}\\
    =&27+(19)&&\text{Exposant}\\
    =&46&&
  \end{align*}
\end{multicols}

\newpage

\exo{4}{Cours}

\begin{multicols}{2}
  \cnt On a :\\
  \begin{itemize}
    \item $A$, $B$ et $C$ alignés
    \item $A$, $B'$ et $C'$ alignés
    \item $(BC)$ et $(B'C')$ parallèles
  \end{itemize}
  Ainsi, d'après le théorème de Thalès on a :
  \begin{align*}
    &\dfrac{AB}{AB'}=\dfrac{AC}{AC'}=\dfrac{BC}{B'C'}&&\\
    &\dfrac{AB}{AB'}=\dfrac{2}{AC'}=\dfrac{4}{6}&&\text{Remplacer les valeurs}\\
    &\dfrac{2}{AC'}=\dfrac{4}{6}&&\text{Garder les fractions utiles}\\
    &2\times 6=AC'\times 4 &&\text{Egalité produits en croix}\\
    &12=AC'\times 4 &&\\
    &AC'=12:4&&\\
    &AC'=3&&
  \end{align*}
  \columnbreak\\
  \cnt On a :\\
  \begin{itemize}
    \item $A$, $B$ et $C$ alignés
    \item $A$, $B'$ et $C'$ alignés
    \item $(BC)$ et $(B'C')$ parallèles
  \end{itemize}
  Ainsi, d'après le théorème de Thalès on a :
  \begin{align*}
    &\dfrac{AB}{AB'}=\dfrac{AC}{AC'}=\dfrac{BC}{B'C'}&&\\
    &\dfrac{1}{3}=\dfrac{3}{AC'}=\dfrac{BC}{B'C'}&&\text{Remplacer les valeurs}\\
    &\dfrac{1}{3}=\dfrac{3}{AC'} &&\text{Garder les fractions utiles}\\
    &3\times 3=AC'\times 1&&\text{Egalité produits en croix}\\
    &AC'=9 &&\\
    &CC'+3=9&&\text{Car}AC'=AC+CC'\\
    &CC'=6&&
  \end{align*}
\end{multicols}

\begin{multicols}{2}
  \cnt\\
  \begin{itemize}
    \item $\dfrac{AC}{AC'}=\dfrac{6}{8}=\dfrac{3}{4}$
    \item $\dfrac{AB}{AB'}=\dfrac{1,5}{2}=\dfrac{3}{4}$
    \item $\dfrac{BC}{B'C'}=\dfrac{3}{4}$
  \end{itemize}
  Ainsi, on a :
  \begin{itemize}
    \item $\dfrac{AC}{AC'}=\dfrac{AB}{AB'}=\dfrac{BC}{B'C'}$
    \item $A$, $B$ et $C$ sont alignés
    \item $A$, $B'$ et $C'$ sont alignés
  \end{itemize}
  Donc, d'après la réciproque du théorème de Thalès, $(BC)$ et $(B'C')$ sont parallèles.
  \columnbreak\\
  \cnt\\
  \begin{itemize}
    \item $\dfrac{AC}{AC'}=\dfrac{4}{6}=\dfrac{2}{3}$
    \item $\dfrac{AB}{AB'}=\dfrac{2}{3}$
    \item $\dfrac{BC}{B'C'}=\dfrac{3}{5}\neq\dfrac{2}{3}$
  \end{itemize}
  Ainsi, on a :
  \begin{itemize}
    \item $\dfrac{AC}{AC'}=\dfrac{AB}{AB'}\neq\dfrac{BC}{B'C'}$
    \item $A$, $B$ et $C$ sont alignés
    \item $A$, $B'$ et $C'$ sont alignés
  \end{itemize}
  Donc, d'après la contraposé du théorème de Thalès, $(BC)$ et $(B'C')$ ne sont pas parallèles.
\end{multicols}

\exo{8}{Problème, calculer}

\begin{multicols}{2}
  \cnt Le triangle $BCD$ est rectangle en $B$. Ainsi, d'après le théorème de Pythagore, on a :
  \begin{align*}
    BD^2&=BC^2+CD^2\\
    &=1,5^2+2^2\\
    &=2,25+4\\
    &=6,25
  \end{align*}
  Donc, $BD=\sqrt{6,25}=2,5~km$.
  \columnbreak

  \cnt
  \begin{itemize}
    \item $(BC)$ et $(CE)$ sont perpendiculaires
    \item $(EF)$ et $(CE)$ sont perpendiculaires
  \end{itemize}
  Donc, $(BC)$ et $(EG)$ sont parallèles.
\end{multicols}

\newpage



\begin{multicols}{2}
\cnt On a :
\begin{itemize}
  \item $(BC)$ et $(EF)$ sont parallèles
  \item Les points $B$, $D$ et $F$ sont alignés
  \item Les points $C$, $D$ et $E$ sont alignés
\end{itemize}
Ainsi, d'après le théorème de Thalès, on a :
\begin{align*}
  &\dfrac{DB}{DF}=\dfrac{DC}{DE}=\dfrac{BC}{FE}\\
  &\dfrac{2,5}{DF}=\dfrac{2}{6}=\dfrac{1,5}{FE}\\
  &\dfrac{2,5}{DF}=\dfrac{2}{6}\\
  &2,5\times 6=DF\times 2\\
  &15=DF\times 2\\
  &DF=15:2\\
  &DF=7,5~km
\end{align*}
Donc $DF$ fait 7,5 $km$.
\columnbreak

\cnt Ce n'est pas parce que la droite n'est pas tracée qu'elle n'existe pas ! On a :
\begin{itemize}
  \item $(AB)$ et $(FG)$ sont parallèles
  \item Les points $B$, $D$ et $F$ sont alignés
  \item Les points $A$, $D$ et $G$ sont alignés
\end{itemize}
Ainsi, d'après le théorème de Thalès, on a :
\begin{align*}
  &\dfrac{DB}{DF}=\dfrac{DA}{DG}=\dfrac{AB}{FG}\\
  &\dfrac{2,5}{7,5}=\dfrac{DA}{DG}=\dfrac{3}{FG}\\
  &\dfrac{2,5}{7,5}=\dfrac{3}{FG}\\
  &2,5\times FG=7,5\times 3\\
  &22,5=FG\times 2,5\\
  &FG=22,5:2,5\\
  &FG=9~km\\
\end{align*}
Donc $FG$ fait 9 $km$.
\end{multicols}

\vspace{1em}

\begin{multicols}{2}
  \cnt On a :
  \begin{align*}
    &AB+BD+DF+FG\\
    &=3+2,5+7,5+9\\
    &=22~km
  \end{align*}
  Donc, le parcours fait $22~km$ de long.
  \columnbreak

  \cnt On a :
  $\text{Vitesse}=\dfrac{\text{Distance}}{\text{Temps}}$, d'où
  \begin{align*}
    \text{Temps}&=\text{Distance}:\text{Vitesse}\\
    &= 22:10\\
    &=2,2~h
    &=2~h+0,2~h\\
    &=2~h+0,2\times 60 ~min\\
    &=2~h+12~min
  \end{align*}
  Donc, il faudra 2 heures et 12 minutes à Nicolas.
\end{multicols}

\vspace{1em}

\exo{/4 points}{Modéliser, communiquer}

\cnt et \cnt
\begin{figure}[H]
  \centering
  \begin{tikzpicture}[scale=1.5]
      \node (A) at (0,0) {}; %positions A
      \node (B) at (2,0) {}; %positions B
      \node (C) at (2,2) {}; %positions C
      \node (B') at (6,0) {}; %positions B'
      \node (C') at (6,6) {}; %positions C'
      \node [left=0.01cm of A] (A_label) {Oeil}; %label A
      \draw (A.base) -- (B.base) node [midway,below] {bras = 0,9 m};
      \draw (A.base) -- (C.base) ;
      \draw (B.base) -- (C.base) node [midway,right] {Pouce=0,08 m};
      \draw (B.base) -- (B'.base);
      \draw (C.base) -- (C'.base) ;
      \draw (B'.base) -- (C'.base) node [midway,right] {Champignon = 30 000 m} ;
      \draw [fill=black] (A) circle (0.21em) ;
      \draw [fill=black] (B) circle (0.21em) ;
      \draw [fill=black] (C) circle (0.21em) ;
      \draw [fill=black] (B') circle (0.21em) ;
      \draw [fill=black] (C') circle (0.21em) ;
  \end{tikzpicture}
\end{figure} 

\cnt 
\begin{itemize}
  \item L'oeil et les extremités du pouce et du champignon sont alignés.
  \item Le pouce et le champignon sont parallèles car verticales
\end{itemize}
Ainsi, d'après le théorème de Thalès (en appelant \textit{distance} la distance entre l'observateur et le champignon):
\begin{align*}
  &\dfrac{\text{bras}}{\text{distance}}=\dfrac{\text{pouce}}{\text{champignon}}\\
  &\dfrac{0,9}{\text{distance}}=\dfrac{0,08}{30 000}\\
  &0,9\times 30 000= \text{distance}\times 0,08\\
  &27 000 : 0,08 = \text{distance}\\
  &\text{distance} = 337 500~m\\
  &\text{distance} = 337,5~km
\end{align*}

Ainsi, si le champignon fait pile la taille de ton pouce, c'est qu'il est à 337 km. Ce qui est plus que 250, donc le repère donné dans la série est valable. Désolé Lucy.
\prop{parallèles et perpendiculaires}
{Soient trois droites $(a), (b)$ et $(c)$.
Si on a :\vspace{1em}\\
\begin{minipage}{0.65\textwidth}
    \begin{itemize}
        \item $(a)$ et $(b)$ sont perpendiculaires.
        \item $(a)$ et $(c)$ sont perpendiculaires.
    \end{itemize}
        Alors $(b)$ et $(c)$ sont parallèles.\vspace{1em} \\
        Dit autrement, si deux droites sont perpendiculaires à une même troisième, alors elles sont parallèles.
\end{minipage}
\hfill
\begin{minipage}{0.3\textwidth}
    \begin{figure}[H]
        \centering
        \begin{tikzpicture}[scale=0.75]
            \draw (0,0) -- (4,0) node [midway,below] {$(a)$} ;
            \draw (1,-1) -- (1,2) node [right] {$(b)$} ;
            \draw (3,-1) -- (3,2) node [right] {$(c)$} ;
            \draw (0.8,0) -- (0.8,0.2) ;
            \draw (1,0.2) -- (0.8,0.2) ;
            \draw (2.8,0) -- (2.8,0.2) ;
            \draw (3,0.2) -- (2.8,0.2) ;
        \end{tikzpicture}
    \end{figure}
\end{minipage}
%     \vspace{2em}\\
% \begin{minipage}{0.65\textwidth}
%     \begin{itemize}
%         \item $(a)$ et $(b)$ sont parallèles.
%         \item $(a)$ et $(c)$ sont parallèles.
%     \end{itemize}
%         Alors $(b)$ et $(c)$ sont parallèles.\vspace{1em}
%         \\
%         Dit autrement, si deux droites sont parallèles, alors toute parallèle à l'une est parallèle à l'autre.
% \end{minipage}
% \hfill
% \begin{minipage}{0.3\textwidth}
%     \begin{figure}[H]
%         \centering
%         \begin{tikzpicture}[scale=0.75]
%             \draw (2,-0.2) node [below] {$(a)$} -- (2,2) ;
%             \draw (1,-0.2) -- (1,2) node [left] {$(b)$} ;
%             \draw (3,-0.2) -- (3,2) node [right] {$(c)$} ;
%         \end{tikzpicture}
%     \end{figure}
% \end{minipage}
% \vspace{2em}\\
% \begin{minipage}{0.65\textwidth}
%     \begin{itemize}
%         \item $(a)$ et $(b)$ sont parallèle.
%         \item $(a)$ et $(c)$ sont perpendiculaires.
%     \end{itemize}
%         Alors $(b)$ et $(c)$ sont parallèles.\vspace{1em} \\
%         Dit autrement, si deux droites sont parallèles, alors toute perpendiculaires à l'une est perpendiculaires à l'autre.
% \end{minipage}
% \hfill
% \begin{minipage}{0.3\textwidth}
%     \begin{figure}[H]
%         \centering
%         \begin{tikzpicture}[scale=0.75]
%             \draw (0,0) -- (4,0) node [midway,below] {$(c)$} ;
%             \draw (1,-1) -- (1,2) node [right] {$(b)$} ;
%             \draw (3,-1) -- (3,2) node [right] {$(a)$} ;
%             \draw (0.8,0) -- (0.8,0.2) ;
%             \draw (1,0.2) -- (0.8,0.2) ;
%             \draw (2.8,0) -- (2.8,0.2) ;
%             \draw (3,0.2) -- (2.8,0.2) ;
%         \end{tikzpicture}
%     \end{figure}
% \end{minipage}
}

\stepcounter{exmplcntr}

\prop{Le théorème de Pythagore}
{Soit $ABC$ un triangle.\\
\begin{multicols}{3}
\textbf{Théorème}\\
    Si : 
\begin{itemize}
    \item $ABC$ rectangle en $A$
\end{itemize} 
Alors $AB^2+AC^2=BC^2$\\
\vfill\break
\textbf{Réciproque}\\
Si : 
\begin{itemize}
    \item $AB^2+AC^2=BC^2$
\end{itemize} 
Alors $ABC$ est rectangle en $A$\\
\vfill\break
\textbf{Contraposé}\\
Si : 
\begin{itemize}
    \item $AB^2+AC^2\neq BC^2$
    \item $BC>AB$ et $BC>AC$
\end{itemize} 
Alors $ABC$ n'est pas rectangle
\end{multicols}
}


\exmpl{\\
    \begin{minipage}{0.5\textwidth}
        \begin{figure}[H]
            \centering
            \begin{tikzpicture}[scale=1]
                \node (A) at (0,0) {}; %positions A
                \node (B) at (3,0) {}; %positions B
                \node (C) at (0,4) {}; %positions C
                \node [left=0.01cm of A] (A_label) {$A$}; %label A
                \node [right=0.01cm of B] (B_label) {$B$}; %label B
                \node [above=0.01cm of C] (C_label) {$C$}; %label C
                \draw (A.base) -- (B.base) node [midway,below] {3};
                \draw (A.base) -- (C.base) node [midway,right] {4};
                \draw (B.base) -- (C.base) node [midway,right] {5};
                \draw [fill=black] (A) circle (0.08) ;
                \draw [fill=black] (B) circle (0.08) ;
                \draw [fill=black] (C) circle (0.08) ;
            \end{tikzpicture}
        \end{figure}
        On a :
        \begin{itemize}
            \item $AB^2+AC^2=3^2+4^2=25$
            \item $BC^2=5^2=25$
        \end{itemize} 
        Donc, d'après la réciproque du théorème de Pythagore, le triangle $ABC$ est rectangle en $A$
    \end{minipage}
    \hfill
    \begin{minipage}{0.45\textwidth}
        \begin{figure}[H]
            \centering
            \begin{tikzpicture}[scale=1]
                \node (A) at (0,0) {}; %positions A
                \node (B) at (3,0) {}; %positions B
                \node (C) at (0,4) {}; %positions C
                \node [left=0.01cm of A] (A_label) {$A$}; %label A
                \node [right=0.01cm of B] (B_label) {$B$}; %label B
                \node [above=0.01cm of C] (C_label) {$C$}; %label C
                \draw (0.2,0)--(0.2,0.2) ;
                \draw (0.2,0.2) -- (0,0.2) ;
                \draw (A.base) -- (B.base) node [midway,below] {6};
                \draw (A.base) -- (C.base) ;
                \draw (B.base) -- (C.base) node [midway,right] {10};
                \draw [fill=black] (A) circle (0.08) ;
                \draw [fill=black] (B) circle (0.08) ;
                \draw [fill=black] (C) circle (0.08) ;
            \end{tikzpicture}
        \end{figure}
        On a :
        \begin{itemize}
            \item $ABC$ est rectangle en $A$
        \end{itemize} 
        D'après le théorème de Pythagore,
        \begin{align*}
            &AB^2+AC^2=BC^2\\
            &6^2 + AC^2=10^2\\
            &36 +AC^2=100\\
            &AC^2=64\\
            &AC=8
        \end{align*} 
        Donc, $AC=8$
    \end{minipage}
}
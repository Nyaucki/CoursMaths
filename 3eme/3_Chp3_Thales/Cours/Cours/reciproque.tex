\prop{Réciproque du théorème de Thalès}
{\begin{minipage}{0.40\textwidth}
    Soit deux triangles $ABC$ et $AB'C'$.\\
    Si on a  :
    \begin{itemize}
        \item $A$, $B$ et $B'$ sont alignés
        \item $A$, $C$ et $C'$ sont alignés
        \item L'égalité $\dfrac{AB}{AB'}=\dfrac{AC}{AC'}=\dfrac{BC}{B'C'}$ est vraie
    \end{itemize}
    \vspace{1em}
    Alors, $(BC)$ et $(B'C')$ sont parallèles.
\end{minipage}
\hfill
\begin{minipage}{0.55\textwidth}
    \begin{tikzpicture}[scale=1.5]
        \node (A) at (0,1) {}; %positions A
        \node (B) at (1,1) {}; %positions B
        \node (C) at (1,2) {}; %positions C
        \node (B') at (-2,1) {}; %positions B'
        \node (C') at (-2,-1) {}; %positions C'
        \node [above=0.01cm of A] (A_label) {$A$}; %label A
        \node [right=0.01cm of B] (B_label) {$B$}; %label B
        \node [above=0.01cm of C] (C_label) {$C$}; %label C
        \node [left=0.01cm of B'] (B'_label) {$B'$}; %label B'
        \node [left=0.01cm of C'] (C'_label) {$C'$}; %label C'
        \draw (A.base) -- (B.base);
        \draw (A.base) -- (C.base);
        \draw (B.base) -- (C.base);
        \draw (A.base) -- (B'.base);
        \draw (A.base) -- (C'.base);
        \draw (B'.base) -- (C'.base);
        \draw [fill=black] (A) circle (0.04) ;
        \draw [fill=black] (B) circle (0.04) ;
        \draw [fill=black] (C) circle (0.04) ;
        \draw [fill=black] (B') circle (0.04) ;
        \draw [fill=black] (C') circle (0.04) ;
    \end{tikzpicture}
\end{minipage}}

\prop{Contraposé du théorème de Thalès}
{\begin{minipage}{0.40\textwidth}
    Soit deux triangles $ABC$ et $AB'C'$.\\
    Si on a  :
    \begin{itemize}
        \item $A$, $B$ et $B'$ sont alignés
        \item $A$, $C$ et $C'$ sont alignés
        \item L'égalité $\dfrac{AB}{AB'}=\dfrac{AC}{AC'}=\dfrac{BC}{B'C'}$ est fausse
    \end{itemize}
    \vspace{1em}
    Alors, $(BC)$ et $(B'C')$ ne sont pas sont parallèles.
\end{minipage}
\hfill
\begin{minipage}{0.55\textwidth}
    \begin{tikzpicture}[scale=2]
        \node (A) at (0,1) {}; %positions A
        \node (B) at (1,1) {}; %positions B
        \node (C) at (2,2) {}; %positions C
        \node (B') at (-2,1) {}; %positions B'
        \node (C') at (-2,0) {}; %positions C'
        \node [above=0.01cm of A] (A_label) {$A$}; %label A
        \node [right=0.01cm of B] (B_label) {$B$}; %label B
        \node [above=0.01cm of C] (C_label) {$C$}; %label C
        \node [left=0.01cm of B'] (B'_label) {$B'$}; %label B'
        \node [left=0.01cm of C'] (C'_label) {$C'$}; %label C'
        \draw (A.base) -- (B.base);
        \draw (A.base) -- (C.base);
        \draw (B.base) -- (C.base);
        \draw (A.base) -- (B'.base);
        \draw (A.base) -- (C'.base);
        \draw (B'.base) -- (C'.base);
        \draw [fill=black] (A) circle (0.04) ;
        \draw [fill=black] (B) circle (0.04) ;
        \draw [fill=black] (C) circle (0.04) ;
        \draw [fill=black] (B') circle (0.04) ;
        \draw [fill=black] (C') circle (0.04) ;
    \end{tikzpicture}
\end{minipage}}

\rmq{Comme pour Pythagore : Si les conditions sont respectées, on parle de réciproque, sinon de contraposé.}

\exmpl
{Dans les deux triangles suivants, cherchons si $(BC)$ et $(B'C')$ sont parallèles.\\
\begin{minipage}{0.45\textwidth}
    \begin{figure}[H]
        \centering
        \begin{tikzpicture}[scale=2]
            \node (A) at (2,1) {}; %positions A
            \node (B) at (3,0.5) {}; %positions B
            \node (C) at (2.5,1.5) {}; %positions C
            \node (B') at (0,2) {}; %positions B'
            \node (C') at (1,0) {}; %positions C'
            \node [above=0.01cm of A] (A_label) {$A$}; %label A
            \node [right=0.01cm of B] (B_label) {$B$}; %label B
            \node [above=0.01cm of C] (C_label) {$C$}; %label C
            \node [left=0.01cm of B'] (B'_label) {$B'$}; %label B'
            \node [left=0.01cm of C'] (C'_label) {$C'$}; %label C'
            \draw (A.base) -- (B.base) node [midway,right] {1};
            \draw (A.base) -- (C.base) node [midway,right] {2};
            \draw (B.base) -- (C.base) node [midway,right] {1,5};
            \draw (A.base) -- (B'.base) node [midway,right] {3};
            \draw (A.base) -- (C'.base) node [midway,right] {6};
            \draw (B'.base) -- (C'.base) node [midway,right] {4,5};
            \draw [fill=black] (A) circle (0.04) ;
            \draw [fill=black] (B) circle (0.04) ;
            \draw [fill=black] (C) circle (0.04) ;
            \draw [fill=black] (B') circle (0.04) ;
            \draw [fill=black] (C') circle (0.04) ;
        \end{tikzpicture}
    \end{figure} 
    \textbf{Cherchons si $(B'C')\parallel (BC)$ :}\\\vspace{1em}    
    $\dfrac{AB}{AB'}=\dfrac{1}{3}$\\\vspace{1em}
    $\dfrac{AC}{AC'}=\dfrac{2}{6}=\dfrac{1}{3}$\\\vspace{1em}
    $\dfrac{BC}{B'C'}=\dfrac{1.5}{4.5}=\dfrac{1}{3}$\\\vspace{1em}
    On a donc : 
    \begin{itemize}
        \item $A$, $B$ et $B'$ sont alignés
        \item $A$, $C$ et $C'$ sont alignés
        \item $\dfrac{AB}{AB'}=\dfrac{AC}{AC'}=\dfrac{BC}{B'C'}$
    \end{itemize}
    Donc, d'après la réciproque du théorème de Thalès, $(B'C')$ et $ (BC)$ sont parallèles.
\end{minipage}
\hfill
\begin{minipage}{0.45\textwidth}
    \begin{figure}[H]
        \centering
        \begin{tikzpicture}[scale=1.5]
            \node (A) at (2,1) {}; %positions A
            \node (B) at (0.8,2.8) {}; %positions B
            \node (C) at (6.8,0.4) {}; %positions C
            \node (B') at (1.4,1.9) {}; %positions B'
            \node (C') at (4.4,0.7) {}; %positions C'
            \node [above=0.01cm of A] (A_label) {$A$}; %label A
            \node [right=0.01cm of B] (B_label) {$B$}; %label B
            \node [above=0.01cm of C] (C_label) {$C$}; %label C
            \node [left=0.01cm of B'] (B'_label) {$B'$}; %label B'
            \node [left=0.01cm of C'] (C'_label) {$C'$}; %label C'
            \draw (B'.base) -- (B.base) node [midway,left] {1,5};
            \draw (C'.base) -- (C.base) node [midway,below] {3};
            \draw (B.base) -- (C.base) node [midway,above] {10};
            \draw (A.base) -- (B'.base) node [midway,left] {1};
            \draw (A.base) -- (C'.base) node [midway,below] {2};
            \draw (B'.base) -- (C'.base) node [midway,above] {3};
            \draw [fill=black] (A) circle (0.04) ;
            \draw [fill=black] (B) circle (0.04) ;
            \draw [fill=black] (C) circle (0.04) ;
            \draw [fill=black] (B') circle (0.04) ;
            \draw [fill=black] (C') circle (0.04) ;
        \end{tikzpicture}
    \end{figure}
    \textbf{Cherchons si $(B'C')\parallel (BC)$ :}\\\vspace{1em}    
    $\dfrac{AB}{AB'}=\dfrac{1}{2,5}=\dfrac{2}{5}$\\\vspace{1em}
    $\dfrac{AC}{AC'}=\dfrac{2}{5}=\dfrac{1}{3}$\\\vspace{1em}
    $\dfrac{BC}{B'C'}=\dfrac{3}{10}\neq\dfrac{2}{5}$\\\vspace{1em}
    On a donc : 
    \begin{itemize}
        \item $A$, $B$ et $B'$ sont alignés
        \item $A$, $C$ et $C'$ sont alignés
        \item $\dfrac{AB}{AB'}\neq=\dfrac{BC}{B'C'}$
    \end{itemize}
    Donc, d'après la contraposé du théorème de Thalès, $(B'C')$ et $ (BC)$ ne sont pas parallèles.
\end{minipage}
}

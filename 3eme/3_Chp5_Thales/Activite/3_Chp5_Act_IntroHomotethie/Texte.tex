\section{Homothétie donnée}

\begin{minipage}{0.5\textwidth}
    On considère la figure ci-contre, avec les données suivantes :
    \begin{itemize}
        \item $(BC)//(DE)$
        \item $AB=1$
        \item $AD=2$
        \item $AC=2$
        \item $BC=1,5$
    \end{itemize}

    On s'intéresse à l'homothétie (noté $H_1$) de centre $A$ et de rapport 2.

    \begin{enumerate}
        \item Quelle est l'image de $B$ par l'homothétie $H_1$ ?
        \item Quelle est l'image de la droite $(BC)$ par $H_1$ ?
        \item En déduire l'image de $C$ par $H_1$.
        \item En déduire les longueurs $AE$ et $DE$.
    \end{enumerate}
\end{minipage}
\hfil
\begin{minipage}{0.45\textwidth}
    \begin{figure}[H]
        \centering
        \begin{tikzpicture}[scale=1.5]
            \node (A) at (0,0) {}; %positions A
            \node (B) at (0,-2) {}; %positions B
            \node (C) at (1.8,-1.5) {}; %positions C
            \node (B') at (0,-4) {}; %positions B'
            \node (C') at (3.6,-3) {}; %positions C'
            \node [above=0.01cm of A] (A_label) {$A$}; %label A
            \node [left=0.01cm of B] (B_label) {$B$}; %label B
            \node [above=0.01cm of C] (C_label) {$C$}; %label C
            \node [below=0.01cm of B'] (B'_label) {$D$}; %label B'
            \node [above=0.01cm of C'] (C'_label) {$E$}; %label C'
            \draw (A.base) -- (B.base) ;
            \draw (A.base) -- (C.base) ;
            \draw (B.base) -- (C.base) ;
            \draw (B.base) -- (B'.base) ;
            \draw (A.base) -- (C'.base);
            \draw (B'.base) -- (C'.base) ;
            \draw [fill=black] (A) circle (0.05em) ;
            \draw [fill=black] (B) circle (0.05em) ;
            \draw [fill=black] (C) circle (0.05em) ;
            \draw [fill=black] (B') circle (0.05em) ;
            \draw [fill=black] (C') circle (0.05em) ;
        \end{tikzpicture}
    \end{figure}
\end{minipage}

\section{Homothétie à trouver}

\begin{minipage}{0.5\textwidth}
    On considère la figure ci-contre, avec les données suivantes :
    \begin{itemize}
        \item $(BC)//(DE)$
        \item $AB=2$
        \item $AD=6$
        \item $AC=1$
        \item $DE=9$
    \end{itemize}

    On s'intéresse à l'homothétie (noté $H_2$) de centre $A$ et qui transforme $B$ en $D$.

    \begin{enumerate}
        \item Quel est le rapport de l'homothétie $H_2$ ?
        \item Quelle est l'image de la droite $(BC)$ par $H_2$ ?
        \item En déduire l'image de $C$ par $H_2$.
        \item En déduire les longueurs $AC$ et $BC$.
    \end{enumerate}
\end{minipage}
\hfil
\begin{minipage}{0.45\textwidth}
    \begin{figure}[H]
        \centering
        \begin{tikzpicture}[scale=1]
            \node (A) at (0,0) {}; %positions A
            \node (B) at (2,1.5) {}; %positions B
            \node (C) at (0,2) {}; %positions C
            \node (B') at (-4,-3) {}; %positions B'
            \node (C') at (0,-4) {}; %positions C'
            \node [above left=0.005cm of A] (A_label) {$A$}; %label A
            \node [right=0.01cm of B] (B_label) {$B$}; %label B
            \node [above=0.01cm of C] (C_label) {$C$}; %label C
            \node [left=0.01cm of B'] (B'_label) {$D$}; %label B'
            \node [below=0.01cm of C'] (C'_label) {$E$}; %label C'
            \draw (A.base) -- (B.base) ;
            \draw (A.base) -- (C.base) ;
            \draw (B.base) -- (C.base) ;
            \draw (A.base) -- (B'.base);
            \draw (A.base) -- (C'.base);
            \draw (B'.base) -- (C'.base) ;
            \draw [fill=black] (A) circle (0.05em) ;
            \draw [fill=black] (B) circle (0.05em) ;
            \draw [fill=black] (C) circle (0.05em) ;
            \draw [fill=black] (B') circle (0.05em) ;
            \draw [fill=black] (C') circle (0.05em) ;
        \end{tikzpicture}
    \end{figure} 
\end{minipage}

\section{Sans homothétie ?}

\begin{minipage}{0.5\textwidth}
    On considère la figure ci-contre, avec les données suivantes :
    \begin{itemize}
        \item $(BC)//(DE)$
        \item $AB=5$
        \item $AD=7$
        \item $AC=4$
        \item $DE=9$
    \end{itemize}

    \begin{enumerate}
        \item Existe-t-il une homothétie $H_3$ transformant $B$ en $D$ et $C$ en $E$ ?
        \item Quelle est le rapport de $H_3$
        \item En déduire les longueurs $AE$ et $BC$.
    \end{enumerate}
\end{minipage}
\hfil
\begin{minipage}{0.45\textwidth}
    \begin{figure}[H]
        \centering
        \begin{tikzpicture}[scale=2]
            \node (A) at (3.4,0.8) {}; %positions A
            \node (B) at (1,0.2) {}; %positions B
            \node (C) at (3.6,-0.4) {}; %positions C
            \node (B') at (0.2,0) {}; %positions B'
            \node (C') at (3.67,-0.8) {}; %positions C'
            \node [above=0.01cm of A] (A_label) {$A$}; %label A
            \node [above=0.01cm of B] (B_label) {$B$}; %label B
            \node [right=0.01cm of C] (C_label) {$C$}; %label C
            \node [left=0.01cm of B'] (B'_label) {$D$}; %label B'
            \node [below=0.01cm of C'] (C'_label) {$E$}; %label C'
            \draw (A.base) -- (B.base) ;
            \draw (A.base) -- (C.base) ;
            \draw (B.base) -- (C.base) ;
            \draw (A.base) -- (B'.base);
            \draw (A.base) -- (C'.base);
            \draw (B'.base) -- (C'.base) ;
            \draw [fill=black] (A) circle (0.05em) ;
            \draw [fill=black] (B) circle (0.05em) ;
            \draw [fill=black] (C) circle (0.05em) ;
            \draw [fill=black] (B') circle (0.05em) ;
            \draw [fill=black] (C') circle (0.05em) ;
        \end{tikzpicture}
    \end{figure} 
\end{minipage}
\documentclass{/home/nyaucki/Documents/Prof/CoursMaths/mycls/DevoirMaison}
\begin{document}



\renewcommand{\nom}{} 

\renewcommand{\prenom}{}

\hrulefill
\begin{figure}[H]
\centering
\begin{tabularx}{0.9\textwidth}{p{1.4cm}p{8cm}X}
\classe & \textbf{Devoir Maison \devoirNumero ~- Pour le \dateRendu} & Nom : \nom
\end{tabularx}
\end{figure}
\vspace{-1em}
\hrulefill

\begin{center}
	Si vous voulez aider \prenom , merci de ne pas juste lui donner la solution. 

	En prenant le temps de lui expliquer, vous l'aiderez beaucoup plus.
\end{center}

\medskip
\begin{minipage}{0.45\textwidth}
	On considère le parcours (en traits pleins) ci-contre, avec les informations ci-dessous. Toutes les longueurs sont données en mètre :
	\begin{itemize}
		\item $BC=\a1$
		\item $BD=\a2$
		\item $AC=\a3$
		\item $AD=\a4$
		\item $(AD)$ et $(CE)$ parallèles
		\item $(BE)$ et $(CD)$ perpendiculaires
	\end{itemize}
\end{minipage}
\hfil
\begin{minipage}{0.45\textwidth}
	\begin{figure}[H]
		\centering
		\begin{tikzpicture}
			\draw [dashed] (0,0) coordinate(B) node[above] {$B$} -- (0,-3) coordinate(A) node[left] {$A$} -- (0,-5) coordinate (C) node[left] {$C$} -- (4,-6) coordinate (E) node[right] {$E$} -- (2,-3) coordinate (D) node[right] {$D$} --cycle ; 
			\draw (B)--(A)--(D)--(C)--(E);
		\end{tikzpicture}
	\end{figure}
\end{minipage}

\begin{enumerate}
	\item Calculer $DC$
	\item Calculer $CE$
	\item Quelle est la longueur totale du parcours ?
	\item Si \prenom ~ cours le parcours à 12 km/h, combien de temps (en h, minutes et secondes) lui faut-il pour arriver au bout ?
\end{enumerate}

\end{document}
\begin{minipage}[t]{0.65\textwidth}
    \vspace*{-1.5em}
    \exo{4}{Raisonner}

    On considère le programme de calcul ci-contre :

    \begin{enumerate}
        \item Montrer qu'avec pour nombre de départs 3, on obtient 16.\vspace{2.2cm}
    \end{enumerate}
\end{minipage}
\hfil
\begin{minipage}[t]{0.32\textwidth}
    \begin{itemize}
        \item Choisir un nombre
        \item Le multiplier par 3
        \item Soustraire 2
        \item Multiplier le tout par 2
        \item Soustraire 2 fois le nombre de départs
        \item Ajouter 8
    \end{itemize}
\end{minipage}
\begin{enumerate}[start=2]
    \item Quel nombre obtient-on avec pour nombre de départs $-2$.\vspace{2.4cm}
    \item Quelle expression obtient-on avec pour nombre de départ $x$.\vspace{2.4cm}
    \item Comment peut-on simplifier le programme de calcul ?\vspace{2.4cm}
\end{enumerate}


\exo{4}{Calculer}

\begin{minipage}[t]{0.45\textwidth}
    \cnt $(BC)$ et $(FE)$ sont parallèles. Calculer $AE$. 
        \begin{figure}[H]
        \centering
        \begin{tikzpicture}[scale=0.6]
            \node (A) at (4,1) {}; %positions A
            \node (B) at (8,2) {}; %positions B
            \node (C) at (3,4) {}; %positions C
            \node (F) at (0,0) {}; %positions F
            \node (E) at (5,-2) {}; %positions E
            \node [above=0.01cm of A] (A_label) {$A$}; %label A
            \node [right=0.01cm of B] (B_label) {$B$}; %label B
            \node [above=0.01cm of C] (C_label) {$C$}; %label C
            \node [left=0.01cm of F] (F_label) {$F$}; %label F
            \node [below=0.01cm of E] (E_label) {$E$}; %label E
            \draw (A.base) -- (B.base);
            \draw (A.base) -- (C.base) node [midway,left] {2};
            \draw (B.base) -- (C.base) node [midway,above] {4};
            \draw (A.base) -- (F.base);
            \draw (A.base) -- (E.base);
            \draw (F.base) -- (E.base) node [midway,below] {6} ;
            \draw [fill=black] (A) circle (0.21em) ;
            \draw [fill=black] (B) circle (0.21em) ;
            \draw [fill=black] (C) circle (0.21em) ;
            \draw [fill=black] (F) circle (0.21em) ;
            \draw [fill=black] (E) circle (0.21em) ;
        \end{tikzpicture}
    \end{figure} 
\end{minipage}
\hfil
\begin{minipage}[t]{0.45\textwidth}
    
\end{minipage}

\begin{minipage}[t]{0.45\textwidth}
    \cnt $(BC)$ et $(FE)$ sont parallèles. Calculer $CE$.
    \begin{figure}[H]
        \centering
        \begin{tikzpicture}[scale=1]
            \node (A) at (0,0) {}; %positions A
            \node (B) at (0.8,-1.6) {}; %positions B
            \node (C) at (1.6,-0.8) {}; %positions C
            \node (F) at (2,-4) {}; %positions F
            \node (E) at (4,-2) {}; %positions E
            \node [above=0.01cm of A] (A_label) {$A$}; %label A
            \node [left=0.01cm of B] (B_label) {$B$}; %label B
            \node [above=0.01cm of C] (C_label) {$C$}; %label C
            \node [left=0.01cm of F] (F_label) {$F$}; %label F
            \node [right=0.01cm of E] (E_label) {$E$}; %label E
            \draw (A.base) -- (B.base) node [midway,left] {1};
            \draw (A.base) -- (C.base) node [midway,above] {3};
            \draw (B.base) -- (C.base) ;
            \draw (B.base) -- (F.base) node [midway,left] {2};
            \draw (A.base) -- (E.base) ;
            \draw (F.base) -- (E.base) ;
            \draw [fill=black] (A) circle (0.14em) ;
            \draw [fill=black] (B) circle (0.14em) ;
            \draw [fill=black] (C) circle (0.14em) ;
            \draw [fill=black] (F) circle (0.14em) ;
            \draw [fill=black] (E) circle (0.14em) ;
        \end{tikzpicture}
    \end{figure} 
\end{minipage}
\hfil
\begin{minipage}[t]{0.45\textwidth}
    
\end{minipage}

\vfil
\hrule
\exo{4}{Calculer}

\begin{minipage}[t]{0.45\textwidth}
    \cnt $(BC)$ et $(FE)$ sont elles parallèles ?
    \begin{figure}[H]
        \centering
        \begin{tikzpicture}[scale=1]
            \node (A) at (0,0) {}; %positions A
            \node (B) at (0.8,-1.6) {}; %positions B
            \node (C) at (1.6,-0.8) {}; %positions C
            \node (F) at (2,-4) {}; %positions F
            \node (E) at (4,-2) {}; %positions E
            \node [above=0.01cm of A] (A_label) {$A$}; %label A
            \node [left=0.01cm of B] (B_label) {$B$}; %label B
            \node [above=0.01cm of C] (C_label) {$C$}; %label C
            \node [left=0.01cm of F] (F_label) {$F$}; %label F
            \node [right=0.01cm of E] (E_label) {$E$}; %label E
            \draw (A.base) -- (B.base) node [midway,left] {1,5};
            \draw (A.base) -- (C.base) node [midway,above] {6};
            \draw (B.base) -- (C.base) node [midway,right] {3};
            \draw (B.base) -- (F.base) node [midway,left] {0,5};
            \draw (C.base) -- (E.base) node [midway,above] {2};
            \draw (F.base) -- (E.base) node [midway,right] {4};
            \draw [fill=black] (A) circle (0.14em) ;
            \draw [fill=black] (B) circle (0.14em) ;
            \draw [fill=black] (C) circle (0.14em) ;
            \draw [fill=black] (F) circle (0.14em) ;
            \draw [fill=black] (E) circle (0.14em) ;
        \end{tikzpicture}
    \end{figure} 
\end{minipage}
\hfil
\begin{minipage}[t]{0.45\textwidth}
    
\end{minipage}
\vfil\hrule
\begin{minipage}[t]{0.45\textwidth}
    \cnt $(BC)$ et $(FE)$ sont elles parallèles ?
        \begin{figure}[H]
        \centering
        \begin{tikzpicture}[scale=0.6]
            \node (A) at (4,1) {}; %positions A
            \node (B) at (8,2) {}; %positions B
            \node (C) at (3,4) {}; %positions C
            \node (F) at (0,0) {}; %positions F
            \node (E) at (5,-2) {}; %positions E
            \node [above=0.01cm of A] (A_label) {$A$}; %label A
            \node [right=0.01cm of B] (B_label) {$B$}; %label B
            \node [above=0.01cm of C] (C_label) {$C$}; %label C
            \node [left=0.01cm of F] (F_label) {$F$}; %label F
            \node [below=0.01cm of E] (E_label) {$E$}; %label E
            \draw (A.base) -- (B.base) node [midway,below] {2};
            \draw (A.base) -- (C.base) node [midway,left] {4};
            \draw (B.base) -- (C.base) node [midway,above] {3};
            \draw (A.base) -- (F.base) node [midway,above] {3};
            \draw (A.base) -- (E.base) node [midway,right] {6};
            \draw (F.base) -- (E.base) node [midway,below] {5} ;
            \draw [fill=black] (A) circle (0.21em) ;
            \draw [fill=black] (B) circle (0.21em) ;
            \draw [fill=black] (C) circle (0.21em) ;
            \draw [fill=black] (F) circle (0.21em) ;
            \draw [fill=black] (E) circle (0.21em) ;
        \end{tikzpicture}
    \end{figure} 
\end{minipage}\hfil
\begin{minipage}[t]{0.45\textwidth}
    
\end{minipage}

\vfil

\newpage

\exo{8}{Chercher}

\begin{minipage}[t]{0.45\textwidth}
    Nicolas doit faire un parcours en canoë. Le parcours est représenté sur la figure de droite en traits pleins.
    
    Les pointillés servent à tracer le parcours.

    On donne les informations suivantes :
    \begin{itemize}
        \item Les points $B$, $D$ et $F$ sont alignés
        \item Les points $C$, $D$ et $E$ sont alignés
        \item Les points $A$, $B$ et $C$ sont alignés
        \item Les points $E$, $F$ et $G$ sont alignés
        \item Les points $A$, $D$ et $G$ sont alignés
    \end{itemize}
    \vspace{1em}

    \cnt Montrer que $BD=2,5~km$.

    \cnt Montrer que les droites $(BC)$ et $(EF)$ sont parallèles.

    \cnt (/2 points) Montrer que $DF=7,5~km$.

    \cnt (/2 points) Montrer que $FG=9~km$.

    \cnt Calculer la distance parcourue par Nicolas.

    \cnt Sachant que Nicolas va à 10 km/h, combien de temps (en heures et minutes) lui faudra-t-il pour finir le parcours ? 
\end{minipage}
\hfill
\begin{minipage}[t]{0.55\textwidth}
        \begin{figure}[H]
        \centering
        Toutes les longueurs sont données en kilomètres.\\ \vspace{1em}
        \begin{tikzpicture}[scale=1]
            \node (A) at (0,0) {}; %positions A
            \node (B) at (2,0) {}; %positions B
            \node (C) at (3,0) {}; %positions C
            \node (D) at (3,-3) {}; %positions D
            \node (E) at (3,-8) {}; %positions E
            \node (F) at (4.67,-8) {}; %positions F
            \node (G) at (8,-8) {}; %positions G
            \node [above=0.01cm of A] (A_label) {$A$}; %label A
            \node [above=0.01cm of B] (B_label) {$B$}; %label B
            \node [above=0.01cm of C] (C_label) {$C$}; %label C
            \node [left=0.01cm of D] (D_label) {$D$}; %label D
            \node [below=0.01cm of E] (E_label) {$E$}; %label E
            \node [below=0.01cm of F] (F_label) {$F$}; %label F
            \node [below=0.01cm of G] (G_label) {$G$}; %label G
            \draw (A.base) -- (B.base) node [midway,above] {3};
            \draw [dotted] (B.base) -- (C.base) node [midway,above] {1,5};
            \draw (2.75,0)--(2.75,-0.25) ;
            \draw (3,-0.25)--(2.75,-0.25) ;
            \draw (3,-7.75) -- (3.25,-7.75) ;
            \draw (3.25,-8) -- (3.25,-7.75) ;
            \draw  [dotted] (C.base) -- (D.base) node [midway,right] {2};
            \draw (B.base) -- (D.base);
            \draw  [dotted] (D.base) -- (E.base) node [midway,left] {6};
            \draw (D.base) -- (F.base);
            \draw [dotted] (E.base) -- (F.base);
            \draw (F.base) -- (G.base) ;
            \draw [fill=black] (A) circle (0.21em) ;
            \draw  [dotted] (A.base) -- (G.base);
            \draw [fill=black] (B) circle (0.21em) ;
            \draw [fill=black] (C) circle (0.21em) ;
            \draw [fill=black] (D) circle (0.21em) ;
            \draw [fill=black] (E) circle (0.21em) ;
            \draw [fill=black] (G) circle (0.21em) ;
            \draw [fill=black] (F) circle (0.21em) ;
        \end{tikzpicture}
    \end{figure} 
\end{minipage}

\underline{Répondre aux questions ici : }

\newpage

L'exercice suivant est un exercice bonus. S'il est fait, il comptera comme une note supplémentaire (ne pouvant que remonter la moyenne).

\exo{4}{Modéliser}

\begin{minipage}[t]{0.6\textwidth}
    Dans la série Fallout, on peut entendre la phrase suivante :\\
    "En cas d'explosion nucléaire, si tu vois le champignon atomique plus petit que ton pouce à la verticale quand tu tends le bras, alors ça va."
    \vspace{1em}\\
    \cnt Faire un schéma représentant la situation avec uniquement des points et des segments. On représentera l'œil par un point ; et le bras, le pouce et le champignon atomique par des segments.

\end{minipage}
\hfill
\begin{minipage}[t]{0.35\textwidth}
    \begin{figure}[H]
        \centering
        \includegraphics[width=\textwidth]{fallout.png}
    \end{figure}
\end{minipage}

\vspace{4em}
On donne les longueurs suivantes (pour un adulte moyen) :

\begin{itemize}
\item Hauteur d'un pouce : $8~cm$
\item Longueur d'un bras : $90~cm$
\item Hauteur d'un champignon atomique : $30~km$
\end{itemize}
\vspace{1em}
\cnt Préciser sur le schéma précédent les longueurs connues (en mètres) ainsi que les segments parallèles.

Selon Lucy, la série dit n'importe quoi, parce qu'il faut être à plus de $250~km$ pour être en sécurité. 

\cnt (/2 points) Expliquez à Lucy pourquoi la série à aussi raison à l'aide de théorèmes du cours.

\vspace*{\fill}
\textit{Ceci est vrai, en théorie. En pratique, la zone de sécurité change en fonction du sens du vent. Donc si tu vois un champignon atomique, mieux vaut courir.}
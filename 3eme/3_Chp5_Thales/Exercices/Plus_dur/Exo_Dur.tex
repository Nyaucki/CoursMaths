
\exo{Contre exemple, chercher}{Dur4}

Chercher un exemple de figure de Thalès où les points sont bien alignés, mais les droites ne sont pas parallèles et l'égalité n'est pas respectée.

\exo{Contre exemple, chercher}{Dur5}

Chercher un exemple de figure de Thalès où les droites sont bien parallèles, mais les points ne sont pas alignés et l'égalité n'est pas respectée.

\exo{Problème ouvert}{Dur6}

L'exercice \ref{Concret1} explique comment Thalès a gagné le droit d'avoir un théorème à son nom en mesurant l'ombre d'une pyramide. 

Seulement, la situation est plus complexe. En effet, pour mesurer completement l'ombre, il aurait besoin de pouvoir commencer sa mesure pile sous le sommet de la pyramide. Or, la construction de la pyramide empêche d'y accéder.

Comment Thalès a-t-il pu faire ?
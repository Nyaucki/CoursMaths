\exo{Chercher, type brevet}{Brevet1}

\begin{minipage}[t]{0.60\textwidth}
    Le fonctionnement d'un vidéoprojecteur peut être représenté par la figure (du haut) ci-contre.

    La figure du bas représente le fonctionnement de la lentille et de la focale.

    Une image (à l'envers) est envoyé sur une lentille (le segment $[LT]$) qui va envoyer l'image toute entière en un point $F$ appelé le foyer. La distance entre la lentille et le foyer ($OF$) est appelé la distance focale.

    Après quoi, l'image continue en ligne droite jusqu'au mur (le segment $[MR]$) sur lequel elle est projeté.

    Le mur et la lentille sont considérés comme parfaitement verticaux, alors que le plafond est horizontal. 

    Maël a acheté un vidéoprojecteur dont la lentille a une longueur de 5~cm, une distance focale de 30~cm.

    Il veut que l'image remplisse entièrement le mur, mesurant 3~m. 

    \begin{enumerate}
        \item Pourquoi l'image arrivant sur la lentille doit-elle être à l'envers ?
        \item Maël a suivi le manuel et réglé l'appareil pour que $LO=4$. En déduire la longueur $LF$
        \item Calculer la longueur $FR$.
        \item A quelle distance Maël doit-il placer son vidéoprojecteur du mur ?
    \end{enumerate}


    
\end{minipage}
\hfill
%%%%%%%%%%%% Exercice 2+28%%%%
\begin{minipage}[t]{0.35\textwidth}
        \begin{figure}[H]
        \centering
        \begin{tikzpicture}[scale=1.5]
            \node (A) at (0,0) {}; %positions A
            \node (B) at (-1,0.5) {}; %positions B
            \node (C) at (-1,-0.166) {}; %positions C
            \node (B') at (3,-1.5) {}; %positions B'
            \node (C') at (3,0.5) {};
            \node [above=0.01cm of A] (A_label) {$F$}; %label A
            \node [left=0.01cm of B] (B_label) {$L$}; %label B
            \node [left=0.01cm of C] (C_label) {$T$}; %label C
            \node [right=0.01cm of B'] (B'_label) {$R$}; %label B'
            \node [right=0.01cm of C'] (C'_label) {$M$}; %label C'
            \draw (A.base) -- (B'.base);;
            \draw (A.base) -- (C.base);
            \draw (B.base) -- (C.base) ;
            \draw (B.base) -- (B'.base) ;
            \draw (A.base) -- (C'.base) ;
            \draw (B'.base) -- (C'.base) ;
            \draw[dashed] (C'.base)--(B.base) node[midway, above] {Plafond};
        \end{tikzpicture}
    \end{figure} 
    \vfil
    \begin{figure}[H]
        \centering
        \begin{tikzpicture}[scale=1.5]
            \node (A) at (1,0) {}; %positions A
            \node (B) at (-1,1) {}; %positions B
            \node (C) at (-1,-0.3) {}; %positions C
            \node (B') at (-1,0) {}; %positions B'
            \node [above=0.01cm of A] (A_label) {$F$}; %label A
            \node [above=0.01cm of B] (B_label) {$L$}; %label B
            \node [below=0.01cm of C] (C_label) {$T$}; %label C
            \node [left=0.01cm of B'] (B'_label) {$O$}; %label B'
            \draw[dashed] (A.base) -- (B'.base);
            \draw (A.base) -- (C.base);
            \draw (B.base) -- (C.base) ;
            \draw (B.base) -- (A.base) ;
            \draw [gray,right angle quadrant=1,right angle symbol={C}{B}{A}];
        \end{tikzpicture}
    \end{figure} 
\end{minipage}

\newpage


\exo{Calcul, type brevet}{Brevet2}

\begin{minipage}[t]{0.45\textwidth}
    Leïla suit une course de VTT dont le tracé est donné ci-contre en traits pleins. Les longueurs sont en kilomètres.
    \begin{enumerate}
        \item Montrer que $BG=5~km$.
        \item Calculer $GD$.%7.5
        \item Calculer $DF$.%6
        \item Quelle est la longueur totale du parcours ?%22.5
        \item Leïla avance à 20km/h. Combien de temps lui faut-il pour faire le parcours ?%1h7min30s
        \item Son père souhaite la rejoindre à l'arriver. Il prend un raccourci passe par $[AF]$. S'il avance à 10km/h, arrivera-t-il avant elle ?
    \end{enumerate}
\end{minipage}
\hfil
\begin{minipage}[t]{0.45\textwidth}
        \begin{figure}[H]
        \centering
        \begin{tikzpicture}[scale=1]
            \draw (0,0) coordinate (A) node[above] {$A$} --node [midway,above]{4}(2,0) coordinate (B) node [above] {$B$} -- (1,-1) coordinate (C) node[right] {$C$} --(-2,-4) coordinate (D) node [below] {$D$}--(0,-4) coordinate (E) node [below] {$E$} --(4,-4) coordinate (F) node [below] {$F$};
            \node (A_left) at (-0.1,0) {};
            \tkzInterLL(A,E)(B,D) \tkzGetPoint{G};
            \node[label=180:$G$] at (G) {};
            \draw[white] (A)--node[black] [midway,left]{3}(G);
            \draw[white] (E)--node[black] [midway,right]{4,5}(G);
            \draw[dashed] (A)--(C)--node [midway,above right]{6} (F);
            \draw [dashed] (A)--(E);
            \draw [gray,right angle quadrant=1,right angle symbol={D}{F}{A}];
            \draw [gray,right angle quadrant=1,right angle symbol={A_left}{B}{E}];
        \end{tikzpicture}
    \end{figure} 
\end{minipage}

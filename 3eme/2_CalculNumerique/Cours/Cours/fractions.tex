\subsection{Additions et soustractions}

\dfnt{Inverse}{
    On appelle \textbf{inverse} d'une fraction la fraction où les numérateur et dénominateur sont inversés.
}

\exmpl{$\dfrac{2}{3}$ est l'inverse de $\dfrac{3}{2}$}

\prop{Changement de dénominateur}{Si on multiplie ou divise le numérateur (en haut) et le dénominateur (en bas) d'une fraction par le même nombre, la valeur de celle-ci reste inchangée}

\exmpl{
\begin{itemize}
    \item $\dfrac{9}{12}=\dfrac{9:3}{12:3}=\dfrac{3}{4}$\medskip
    \item $\dfrac{2}{3}=\dfrac{2\times 5}{3\times 5}=\dfrac{10}{15}$
\end{itemize}}

\rmq {On rappelle que tout nombre entier peut être vu comme une fraction dont le dénominateur vaut 1. Par exemple $3=\dfrac{3}{1}$}

\prop{Addition ou soustraction}{Si deux fractions ont le même dénominateur, on peut additionner ou soustraire leur numérateur en gardant le dénominateur commun.}

\exmpl{
\begin{align*}
        \dfrac{2}{3}+\dfrac{4}{5}&=\dfrac{2\times 5}{3\times 5}+\dfrac{4\times 3}{5\times 3} & \dfrac{9}{12}-\dfrac{13}{12}&=\dfrac{9-13}{12}\\
        &=\dfrac{10}{15}+\dfrac{12}{15} & &=\dfrac{-4}{12}\\
        &=\dfrac{10+12}{15}\\
        &=\dfrac{22}{15}
    \end{align*}
}

\subsection{Multiplications et divisions}

\prop{Multiplications}{
    \begin{minipage}{0.75\textwidth}
        Dans un produit de deux fractions,
        \begin{itemize}
            \item Le numérateur du résultat est le produit des numérateurs
            \item Le dénominateur du résultat est le produit des dénominateurs
        \end{itemize}
    \end{minipage}
    \hfill
    \begin{minipage}{0.2\textwidth}
        $$\dfrac{a}{b}\times\dfrac{c}{d}=\dfrac{a\times c}{b\times d}$$
    \end{minipage}
}

\exmpl{
    \begin{align*}
        \dfrac{2}{3}\times\dfrac{4}{5}&=\dfrac{2\times 4}{3\times 5} & \dfrac{9}{12}\times\dfrac{-3}{2}&=\dfrac{9\times (-3)}{12\times 2}\\
        &=\dfrac{8}{15} & &=\dfrac{-27}{24}
    \end{align*}
}

\prop{division}{
    \begin{minipage}{0.75\textwidth}
        Diviser par une fractionrevient à multiplier par son inverse.
    \end{minipage}
    \hfill
    \begin{minipage}{0.2\textwidth}
        $$\dfrac{a}{b} :\dfrac{c}{d}=\dfrac{a}{b} \times \dfrac{d}{c}$$
    \end{minipage}
}

\exmpl{
    \begin{align*}
        \dfrac{2}{3}:\dfrac{4}{5}&=\dfrac{2}{3}\times\dfrac{5}{4}& \dfrac{9}{7}:\dfrac{-3}{2}&=\dfrac{9}{7}\times\dfrac{2}{-3}\\
        &=\dfrac{2\times 5}{3\times 4} & &=\dfrac{9\times 2}{7\times (-3)}\\
        &=\dfrac{10}{12} & &=\dfrac{18}{-21}\\
        &=\dfrac{5}{6} & &=\dfrac{-6}{7}
    \end{align*}
}
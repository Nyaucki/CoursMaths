\subsection{Rappels}

\dfnt{Puissance}{
    \begin{minipage}{0.65\textwidth}
        \begin{itemize}
            \item Un nombre à la puissance (ou exposant) 4 est un nombre multiplié 4 fois par lui même.
            \item Un nombre à la puissance n est un nombre multiplié n fois par lui même.
        \end{itemize}
    \end{minipage}
    \hfill
    \begin{minipage}{0.35\textwidth}
        \begin{itemize}
            \item $a^4=a\times a\times a\times a$
            \item $a^n=\underbrace{a\times a\times a\times a ... \times a}_{n\text{ fois}}$
        \end{itemize}
    \end{minipage}
}

\exmpl{
    \begin{align*}
        3^5&=3\times 3\times 3\times 3\times 3& \left(\dfrac{2}{5}\right)^2&=\dfrac{2}{5}\times \dfrac{2}{5}\\
        &=27\times 3\times 3\times 3\times 3 & &=\dfrac{2\times 2}{5\times 5}\\
        &=81\times 3\times 3\times 3 & &=\dfrac{4}{25}\\
        &=243\times 3\times 3&&\\
        &=729\times 3&&\\
        &=2187&&
    \end{align*}}

\rmq{ On a $\left(\dfrac{a}{b}\right)^n=\dfrac{a^n}{b^n}$}

\prop{Puissances remarquables}{
    Pour $a$ et $n$ deux nombres entiers non nuls.
    \begin{align*}
        a^0&=1 & 0^n&=0\\
        a^1&=a & 1^n&=1
    \end{align*}
}

\exmpl{
    \begin{itemize}
        \item $15^1=15$
        \item $3^0=1$
        \item $0^12=0$
        \item $1^3=1$
    \end{itemize}
}

\subsection{Exposant négatif}

\dfnt{Exposant négatif}
{\begin{align*}
    a^{-1}&=1:a=\dfrac{1}{a}\\
    a^{-n}&=1:a^n=\dfrac{1}{a^n}
\end{align*}}

\exmpl{
    \begin{align*}
        \left(\dfrac{2}{5}\right)^{-1} &= 1:\dfrac{2}{5} & \left(\dfrac{-3}{4}\right)^{-2} &= 1:\left(\dfrac{-3}{4}\right)^{2}\\
        &=1\times \dfrac{5}{2} & &=1\times \left(\dfrac{4}{-3}\right)^2\\
        &=\dfrac{5}{2} & &= \dfrac{4^2}{(-3)^2}\\
        & & &=\dfrac{16}{9}
    \end{align*}
}
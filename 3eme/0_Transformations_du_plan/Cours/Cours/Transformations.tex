\dfnt{Symétrie axiale}
{Deux figures sont symétriques par rapport à une droite si elles se supperposent en pliant par rapport à une droite (appelé \textbf{l'axe de symétrie}).\\
Dit autrement, l'axe de symétrie est la médiatrice du segment formé par un point et son image.}

\rmq{On rappelle que la médiatrice d'un segment le coupe en son milieu en formant un angle droit.}

\exmpl{
    \begin{figure} [H]
        \centering
        \begin{tikzpicture}
            \draw (0,0)--(4,4) node [right]{$(d)$};
            \draw [dotted] (0,3) node [above] {A} -- (3,0) node [right] {A'};
            \draw (1.1,1.5)--(1.3,1.3);
            \draw (1.3,1.7)--(1.1,1.5) ;
            \draw (0,3) -- (0,2) --(-1,2) -- (2,5)--(1.5,2.5)--(0,3);
            \draw [dashed] (3,0) -- (2,0) --(2,-1) -- (5,2)--(2.5,1.5)--(3,0);
        \end{tikzpicture}
    \end{figure}
}

\dfnt{Symétrie centrale}
{Deux figures sont symétriques par rapport à un point si elles se supperposent en faisant un demi tour autour de ce point (appelé \textbf{centre de symétrie}).
}

\rmq{Si $A'$ est l'image de $A$ et $O$ le centre de symétrie, alors $OA=OA'$ et $O,~A$ et $A'$ sont alignés.}

\exmpl{
    \begin{figure} [H]
        \centering
        \begin{tikzpicture}
            \draw (-1.5,1.5) -- (-1.5,0.5) --(-2.5,0.5) -- (0.5,3.5)--(0,1)--(-1.5,1.5);
            \draw [dashed,rotate=180] (-1.5,1.5) --(-1.5,0.5) --(-2.5,0.5) -- (0.5,3.5)--(0,1)--(-1.5,1.5);
            \node (0,0) [right]{O};
            \draw [dotted] (-1.5,1.5) -- (0,0) ;
            \draw [dotted,rotate=180] (-1.5,1.5) -- (0,0) ;
        \end{tikzpicture}
    \end{figure}
}

\dfnt{Translation}
{Un déplacement par translation est un déplacement où l'ont fait glisser la figure, sans la faire pivoter.
}


\exmpl{
    \begin{figure} [H]
        \centering
        \begin{tikzpicture}
            \draw (-1.5,1.5) -- (-1.5,0.5) --(-2.5,0.5) -- (0.5,3.5)--(0,1)--(-1.5,1.5);
            \draw [dashed,shift={(5,3)}] (-1.5,1.5) --(-1.5,0.5) --(-2.5,0.5) -- (0.5,3.5)--(0,1)--(-1.5,1.5);
        \end{tikzpicture}
    \end{figure}
}

\dfnt{Rotation}
{Un déplacement par rotation est un déplacement où l'ont fait tourner la figure autour d'un point \textbf{appelé centre de la rotation}.\\
Un point, le centre de la rotation et son image forment un angle appelé \textbf{angle de la rotation.}
}

\rmq{L'angle de la rotation est le même pour tous les points de la figure.}

\exmpl{
    \begin{figure} [H]
        \centering
        \begin{tikzpicture}
            \draw (-1.5,1.5) -- (-1.5,0.5) --(-2.5,0.5) -- (0.5,3.5)--(0,1)--(-1.5,1.5);
            \draw [dashed,rotate=143] (-1.5,1.5) --(-1.5,0.5) --(-2.5,0.5) -- (0.5,3.5)--(0,1)--(-1.5,1.5);
            \node (0,0) [right]{O};
            \draw [dotted] (-1.5,1.5) -- (0,0) ;
            \draw [dotted,rotate=143] (-1.5,1.5) -- (0,0) ;
        \end{tikzpicture}
    \end{figure}
}

\dfnt{Homotétie}
{Un homotétie de centre $O$ et de rapport $k$ transforme tous les points $M$ de la figure en $M'$ de sorte que :
\begin{itemize}
    \item $O,~M$ et $M'$ soient alignés.
    \item $OM'=k\times OM$
\end{itemize}
}

\rmq{Si $k$ est négatif, alors la figure est retournée, comme pour une symétrie centrale}

\rmq{Une symétrie centrale, une roation d'angle 180° et une homotétie de rapport -1 sont équivalentes.}

\exmpl{
    \begin{minipage}[t]{0.45\textwidth}
        \begin{figure} [H]
            \centering
            \begin{tikzpicture}
                \tikzset{
                    homothety at/.style args={#1 scaled by #2}{shift={($(#1)!#2!(0,0)$)},scale=#2},}
                \def\mypath{(-1.5,1.5) -- (-1.5,0.5) --(-2.5,0.5) -- (0.5,3.5)--(0,1)--(-1.5,1.5)};
                \draw\mypath ;
                \fill (0,0) coordinate (c) circle(2pt) node [above] {$O$};
                \begin{scope}[homothety at=c scaled by 2]
                    \draw[dashed] \mypath;
                \end{scope}
            \end{tikzpicture}
            \caption*{Homotétie de rapport $2$}
        \end{figure}
    \end{minipage}
    \hfill
    \begin{minipage}[t]{0.45\textwidth}
        \begin{figure} [H]
            \centering
            \begin{tikzpicture}
                \tikzset{
                    homothety at/.style args={#1 scaled by #2}{shift={($(#1)!#2!(0,0)$)},scale=#2},}
                \def\mypath{(-1.5,1.5) -- (-1.5,0.5) --(-2.5,0.5) -- (0.5,3.5)--(0,1)--(-1.5,1.5)};
                \draw\mypath ;
                \fill (0,0) coordinate (c) circle(2pt) node [above] {$O$};
                \begin{scope}[homothety at=c scaled by -0.6]
                    \draw[dashed] \mypath;
                \end{scope}
            \end{tikzpicture}
            \caption*{Homotétie de rapport -0,6}
        \end{figure}
    \end{minipage}
}
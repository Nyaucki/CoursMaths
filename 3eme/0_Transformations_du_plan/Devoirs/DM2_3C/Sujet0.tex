\begin{pycode}
from random import *
from sympy import *


x = Symbol('x')
n = Symbol('n')

#FONCTIONS AFFINES
def randex(a,b,sign=True):
	'''
	Renvoie un entier aléatoire n tel que |n| soit compris entre a et b
	'''
	r = randint(a,b)
	if sign :		
		r = (-1)**randint(1,2)*r
	return r

a1=randex(3,8)/10
b1=randint(2,6)
c1=randint(2,6)

a2=randint(3,8)/10
b2=randint(2,6)
c2=randint(2,6)

a3=1+randint(3,8)/10
b3=randint(1,15)/10


a4=1+randint(3,8)/10
b4=randint(1,15)/10
c4=randint(1,15)/10

\end{pycode}

%%%%%%%%%%%%%%%%%%%%%%%%%%%%%%%%%%%%%%%%%%%%%%%%%%%%%%%%%%%%%%%%%

\hrulefill
\begin{figure}[H]
\centering
\begin{tabularx}{0.9\textwidth}{p{2cm}p{8cm}X}
3eme C & \textbf{Devoir Maison 2 - Pour le 01/10/2024} & Nom : \nom
\end{tabularx}
\end{figure}
\vspace{-1em}
\hrulefill

\begin{center}
	Si vous voulez aider \prenom , merci de ne pas juste lui donner la solution. 

	En prenant le temps de lui expliquer, vous l'aiderez beaucoup plus.
\end{center}


\medskip

Trouver le centre de chacune des homothéties suivantes.
%a1=\py{a1} b1=\py{b1} c1=\py{c1} cc1=\py{cc1} e1=\py{e1}

\begin{minipage}[t]{0.45\textwidth}  
    \begin{figure}[H]
        \centering
		\begin{pysub} % Necessaire pour python dans tikz
			
			\begin{tikzpicture}[scale=1]
				\tikzset{
                homothety at/.style args={#1 scaled by #2}{shift={($(#1)!#2!(0,0)$)},scale=#2},}
				\def\mypath{(8,0)--(0,0)--(0,8)--(8,8)}
				\draw [white](0,0) grid (8,8);
				\draw [thick]\mypath ;
				\fill [white] (!{b1},!{c1}) coordinate (c) circle(2pt);
				\begin{scope}[homothety at=c scaled by !{a1}]
				    \draw[dashed] \mypath;
				\end{scope}
			\end{tikzpicture}
		\end{pysub}
    \end{figure}
\end{minipage}
\hfill
\begin{minipage}[t]{0.45\textwidth}  
    \begin{figure}[H]
        \centering
		\begin{pysub} % Necessaire pour python dans tikz
			
			\begin{tikzpicture}[scale=1]
				\tikzset{
                homothety at/.style args={#1 scaled by #2}{shift={($(#1)!#2!(0,0)$)},scale=#2},}
				\def\mypath{(8,0)--(4,1)--(4,0)--(0,0)--(0,3)--(1,4)--(0,5)--(1,8)}
				\draw [white](0,0) grid (8,8);
				\draw [thick]\mypath ;
				\fill [white] (!{b2},!{c2}) coordinate (c) circle(2pt);
				\begin{scope}[homothety at=c scaled by !{a2}]
				    \draw[dashed] \mypath;
				\end{scope}
			\end{tikzpicture}
		\end{pysub}
    \end{figure}
\end{minipage}

\begin{minipage}[t]{0.45\textwidth}  
    \begin{figure}[H]
        \centering
		\begin{pysub} % Necessaire pour python dans tikz
			
			\begin{tikzpicture}[scale=1]
				\tikzset{
                homothety at/.style args={#1 scaled by #2}{shift={($(#1)!#2!(0,0)$)},scale=#2},}
				\def\mypath{(2,0)--(2,3)--(3,4)--(3,0)}
				\draw [white](0,0) grid (8,8);
				\draw [thick]\mypath ;
				\fill [white] (!{b3},0) coordinate (c) circle(2pt);
				\begin{scope}[homothety at=c scaled by !{a3}]
				    \draw[dashed] \mypath;
				\end{scope}
			\end{tikzpicture}
		\end{pysub}
    \end{figure}
\end{minipage}
\hfill
\begin{minipage}[t]{0.45\textwidth}  
    \begin{figure}[H]
        \centering
		\begin{pysub} % Necessaire pour python dans tikz
			
			\begin{tikzpicture}[scale=1]
				\tikzset{
                homothety at/.style args={#1 scaled by #2}{shift={($(#1)!#2!(0,0)$)},scale=#2},}
				\def\mypath{(2,2)--(2,4)--(3,4)--(2,3)--(3,2)--(2,2)}
				\draw [white](0,0) grid (8,8);
				\draw [thick]\mypath ;
				\fill [white] (!{b4},!{c4}) coordinate (c) circle(2pt);
				\begin{scope}[homothety at=c scaled by !{a4}]
				    \draw[dashed] \mypath;
				\end{scope}
			\end{tikzpicture}
		\end{pysub}
    \end{figure}
\end{minipage}

\newpage

\hrulefill
\begin{figure}[H]
\centering
\begin{tabularx}{0.9\textwidth}{p{2cm}p{8cm}X}
3eme C & \textbf{Devoir Maison 2 - Correction} & Nom : \nom
\end{tabularx}
\end{figure}
\vspace{-1em}
\hrulefill

\begin{minipage}[t]{0.45\textwidth}
    
    \begin{figure}[H]
        \centering
		\begin{pysub} % Necessaire pour python dans tikz
			
			\begin{tikzpicture}[scale=1]
				\tikzset{
                homothety at/.style args={#1 scaled by #2}{shift={($(#1)!#2!(0,0)$)},scale=#2},}
				\def\mypath{(8,0)--(0,0)--(0,8)--(8,8)}
				\draw [white](0,0) grid (8,8);
				\draw [thick]\mypath ;
				\fill (!{b1},!{c1}) coordinate (c) circle(2pt) node [above] {$O$};
				\begin{scope}[homothety at=c scaled by !{a1}]
				    \draw[dashed] \mypath;
				\end{scope}
			\end{tikzpicture}
		\end{pysub}
    \end{figure}
\end{minipage}
\hfill
\begin{minipage}[t]{0.45\textwidth}  
    \begin{figure}[H]
        \centering
		\begin{pysub} % Necessaire pour python dans tikz
			
			\begin{tikzpicture}[scale=1]
				\tikzset{
                homothety at/.style args={#1 scaled by #2}{shift={($(#1)!#2!(0,0)$)},scale=#2},}
				\def\mypath{(8,0)--(4,1)--(4,0)--(0,0)--(0,3)--(1,4)--(0,5)--(1,8)}
				\draw [white](0,0) grid (8,8);
				\draw [thick]\mypath ;
				\fill (!{b2},!{c2}) coordinate (c) circle(2pt) node [above] {$O$};
				\begin{scope}[homothety at=c scaled by !{a2}]
				    \draw[dashed] \mypath;
				\end{scope}
			\end{tikzpicture}
		\end{pysub}
    \end{figure}
\end{minipage}

\begin{minipage}[t]{0.45\textwidth}  
    \begin{figure}[H]
        \centering
		\begin{pysub} % Necessaire pour python dans tikz
			
			\begin{tikzpicture}[scale=1]
				\tikzset{
                homothety at/.style args={#1 scaled by #2}{shift={($(#1)!#2!(0,0)$)},scale=#2},}
				\def\mypath{(2,0)--(2,3)--(3,4)--(3,0)}
				\draw [white](0,0) grid (8,8);
				\draw [thick]\mypath ;
				\fill (!{b3},0) coordinate (c) circle(2pt) node [above] {$O$};
				\begin{scope}[homothety at=c scaled by !{a3}]
				    \draw[dashed] \mypath;
				\end{scope}
			\end{tikzpicture}
		\end{pysub}
    \end{figure}
\end{minipage}
\hfill
\begin{minipage}[t]{0.45\textwidth}  
    \begin{figure}[H]
        \centering
		\begin{pysub} % Necessaire pour python dans tikz
			
			\begin{tikzpicture}[scale=1]
				\tikzset{
                homothety at/.style args={#1 scaled by #2}{shift={($(#1)!#2!(0,0)$)},scale=#2},}
				\def\mypath{(2,2)--(2,4)--(3,4)--(2,3)--(3,2)--(2,2)}
				\draw [white](0,0) grid (8,8);
				\draw [thick]\mypath ;
				\fill (!{b4},!{c4}) coordinate (c) circle(2pt) node [above] {$O$};
				\begin{scope}[homothety at=c scaled by !{a4}]
				    \draw[dashed] \mypath;
				\end{scope}
			\end{tikzpicture}
		\end{pysub}
    \end{figure}
\end{minipage}
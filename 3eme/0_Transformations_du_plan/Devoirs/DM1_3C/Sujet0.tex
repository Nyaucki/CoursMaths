\begin{pycode}
from random import *
from sympy import *


x = Symbol('x')
n = Symbol('n')

#FONCTIONS AFFINES
def randex(a,b,sign=True):
	'''
	Renvoie un entier aléatoire n tel que |n| soit compris entre a et b
	'''
	r = randint(a,b)
	if sign :		
		r = (-1)**randint(1,2)*r
	return r

a1=randint(2,7)
b1=randint(2,6)
if a1==b1:
	c1=(a1+randint(1,2))%6
	cc1=(c1+1)
else :
	c1=1
	cc1=a1

e1=randint(2,5)

a2=randint(3,5)
b2=randint(0,2)
c2=randint(1,3)
d2=randint(1,3)

a3=randint(2,7)
b3=randint(2,6)
if a3==b3:
	c3=(a3+randint(1,2))%6
	cc3=(c3+1)
else :
	c3=1
	cc3=a3

e3=randint(2,5)

a4=randint(4,6)
b4=randint(2,4)
d4=randint(0,1)

\end{pycode}

%%%%%%%%%%%%%%%%%%%%%%%%%%%%%%%%%%%%%%%%%%%%%%%%%%%%%%%%%%%%%%%%%

\hrulefill
\begin{figure}[H]
\centering
\begin{tabularx}{0.9\textwidth}{p{2cm}p{8cm}X}
3eme C & \textbf{Devoir Maison 1 - Pour le 20/09/2024} & Nom : \nom
\end{tabularx}
\end{figure}
\vspace{-1em}
\hrulefill

\begin{center}
	Si vous voulez aider \prenom , merci de ne pas juste lui donner la solution. 

	En prenant le temps de lui expliquer, vous l'aiderez beaucoup plus.
\end{center}


\medskip

%a1=\py{a1} b1=\py{b1} c1=\py{c1} cc1=\py{cc1} e1=\py{e1}




\begin{minipage}[t]{0.45\textwidth}
    Tracer le symétrique par rapport à la droite $(d)$
    
    \begin{figure}[H]
        \centering
		\begin{pysub} % Necessaire pour python dans tikz
			
			\begin{tikzpicture}[scale=0.85]
				\def\mypath{(!{a1},!{a1})--(!{b1},1)--(!{cc1},!{c1})--(7,!{e1})}
				\draw [thick]\mypath ;
				\draw (0,0)--(8,8) node [right]{$(d)$} ;
				% \draw [cm={0,1,1,0,(0,0)}] \mypath;%Matrice de transformation inverse X et Y
				\draw [dotted](0,0) grid (8,8);
			\end{tikzpicture}
		\end{pysub}
    \end{figure}
\end{minipage}
\hfill
\begin{minipage}[t]{0.45\textwidth}
    Tracer la translation qui envoie $A$ sur $B$.
    
    \begin{figure}[H]
        \centering
		\begin{pysub} % Necessaire pour python dans tikz

			\begin{tikzpicture}[scale=0.85]
				\def\mypath{(1,2) -- (5,5) --(3,5) -- (1,4)--(5,5)}
				\draw [thick]\mypath ;
				% \draw [shift={(\c2,\d2)}]\mypath ;
				\fill (!{a2},!{b2}) coordinate (c) circle(2pt) node [above] {$A$};
				\fill (!{a2}+!{c2},!{b2}+!{d2}) coordinate (b) circle(2pt) node [above] {$B$};
				\draw [dotted](0,0) grid (8,8);
			\end{tikzpicture}
		\end{pysub}
    \end{figure}
\end{minipage}

\begin{minipage}[t]{0.45\textwidth}
    Tracer le symétrique par rapport au point $O$
    
    \begin{figure}[H]
        \centering
		\begin{pysub} % Necessaire pour python dans tikz
			
			\begin{tikzpicture}[scale=0.85]
				\tikzset{
                homothety at/.style args={#1 scaled by #2}{shift={($(#1)!#2!(0,0)$)},scale=#2},}
				\def\mypath{(!{a3},!{a3})--(!{b3},1)--(!{cc3},!{c3})--(7,!{e3})}
				\draw [thick]\mypath ;
				\fill (4,4) coordinate (c) circle(2pt) node [above] {$O$};
				% \begin{scope}[homothety at=c scaled by -1]
				%     \draw \mypath;
				% \end{scope}
				\draw [dotted](0,0) grid (8,8);
			\end{tikzpicture}
		\end{pysub}
    \end{figure}
\end{minipage}
\hfill
\begin{minipage}[t]{0.45\textwidth}
    Tracer l'homotétie de rapport -2 et de centre $O$.
    
    \begin{figure}[H]
        \centering
		\begin{pysub} % Necessaire pour python dans tikz

			\begin{tikzpicture}[scale=0.85]
				\tikzset{
                homothety at/.style args={#1 scaled by #2}{shift={($(#1)!#2!(0,0)$)},scale=#2},}
				\def\mypath{(1,!{a4})--(2-!{d4},6+ !{d4})--(!{b4},7)--(1+ !{d4},7- !{d4})}
				\draw [thick]\mypath ;
				\fill (3,5) coordinate (c) circle(2pt) node [above] {$O$};
				% \begin{scope}[homothety at=c scaled by -2]
				%     \draw \mypath;
				% \end{scope}
				\draw [dotted](0,0) grid (8,8);
			\end{tikzpicture}
		\end{pysub}
    \end{figure}
\end{minipage}

\newpage

\hrulefill
\begin{figure}[H]
\centering
\begin{tabularx}{0.9\textwidth}{p{2cm}p{8cm}X}
3eme C & \textbf{Devoir Maison 1 - Correction} & Nom : \nom
\end{tabularx}
\end{figure}
\vspace{-1em}
\hrulefill



\begin{minipage}[t]{0.45\textwidth}
    Tracer le symétrique par rapport à la droite $(d)$
    
    \begin{figure}[H]
        \centering
		\begin{pysub} % Necessaire pour python dans tikz
			
			\begin{tikzpicture}[scale=0.85]
				\def\mypath{(!{a1},!{a1})--(!{b1},1)--(!{cc1},!{c1})--(7,!{e1})}
				\draw [thick]\mypath ;
				\draw (0,0)--(8,8) node [right]{$(d)$} ;
				\draw [cm={0,1,1,0,(0,0)}] \mypath;%Matrice de transformation inverse X et Y
				\draw [dotted](0,0) grid (8,8);
			\end{tikzpicture}
		\end{pysub}
    \end{figure}
\end{minipage}
\hfill
\begin{minipage}[t]{0.45\textwidth}
    Tracer la translation qui envoie $A$ sur $B$.
    
    \begin{figure}[H]
        \centering
		\begin{pysub} % Necessaire pour python dans tikz

			\begin{tikzpicture}[scale=0.85]
				\def\mypath{(1,2) -- (5,5) --(3,5) -- (1,4)--(5,5)}
				\draw [thick]\mypath ;
				\draw [shift={(!{c2},!{d2})}]\mypath ;
				\fill (!{a2},!{b2}) coordinate (c) circle(2pt) node [above] {$A$};
				\fill (!{a2}+!{c2},!{b2}+!{d2}) coordinate (b) circle(2pt) node [above] {$B$};
				\draw [dotted](0,0) grid (8,8);
			\end{tikzpicture}
		\end{pysub}
    \end{figure}
\end{minipage}

\begin{minipage}[t]{0.45\textwidth}
    Tracer le symétrique par rapport au point $O$
    
    \begin{figure}[H]
        \centering
		\begin{pysub} % Necessaire pour python dans tikz
			
			\begin{tikzpicture}[scale=0.85]
				\tikzset{
                homothety at/.style args={#1 scaled by #2}{shift={($(#1)!#2!(0,0)$)},scale=#2},}
				\def\mypath{(!{a3},!{a3})--(!{b3},1)--(!{cc3},!{c3})--(7,!{e3})}
				\draw [thick]\mypath ;
				\fill (4,4) coordinate (c) circle(2pt) node [above] {$O$};
				\begin{scope}[homothety at=c scaled by -1]
				    \draw \mypath;
				\end{scope}
				\draw [dotted](0,0) grid (8,8);
			\end{tikzpicture}
		\end{pysub}
    \end{figure}
\end{minipage}
\hfill
\begin{minipage}[t]{0.45\textwidth}
    Tracer l'homotétie de rapport -2 et de centre $O$.
    
    \begin{figure}[H]
        \centering
		\begin{pysub} % Necessaire pour python dans tikz

			\begin{tikzpicture}[scale=0.85]
				\tikzset{
                homothety at/.style args={#1 scaled by #2}{shift={($(#1)!#2!(0,0)$)},scale=#2},}
				\def\mypath{(1,!{a4})--(2-!{d4},6+ !{d4})--(!{b4},7)--(1+ !{d4},7- !{d4})}
				\draw [thick]\mypath ;
				\fill (3,5) coordinate (c) circle(2pt) node [above] {$O$};
				\begin{scope}[homothety at=c scaled by -2]
				    \draw \mypath;
				\end{scope}
				\draw [dotted](0,0) grid (8,8);
			\end{tikzpicture}
		\end{pysub}
    \end{figure}
\end{minipage}
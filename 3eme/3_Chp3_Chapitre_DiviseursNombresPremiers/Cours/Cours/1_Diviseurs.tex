\section{Diviseurs et divisibilités}

\dfnt{Diviseurs et multiples}
{Si la division euclidienne de $a$ par $b$ a pour reste 0, alors :
\begin{itemize}
    \item On dit que $b$ est un diviseur de $a$.
    \item On dit que $a$ est un multiple de $b$.
\end{itemize}
}

\prop{Diviseurs et multiples}
{Soient $a$ et $b$  deux entiers relatifs, avec $b$ non nul.
\begin{itemize}
    \item $b$ est un diviseur de $a$ s'il existe un nombre entier $n$ tel que $a=b\times n$.
    \item $a$ est un multiple de $b$ s'il existe un nombre entier $n$ tel que $a=b\times n$.
\end{itemize}
}

\rmq{\begin{itemize}
    \item 1 est un diviseur de tous les nombres
    \item N'importe quel nombre est un multiple et un diviseur de lui-même
    \item 0 est un multiple de tous les nombres
\end{itemize}}

\prop{Critères de divisibilité 2 ; 3 et 5}
{
    \begin{itemize}
        \item Un nombre est divisible par 2 si son chiffre des unités est 0 ; 2 ; 4 ; 6 ou 8.
        \item Un nombre est divisible par 3 si la somme de ses chiffres est un multiple de 3.
        \item Un nombre est divisible par 5 si son chiffre des unités est 0 ou 5.
    \end{itemize}
}

\rmq{Ces trois critères sont les plus importants. Les autres permettent avant tout de gagner du temps.}

\prop{Critères de divisibilité 4 et 9}
{
    \begin{itemize}
        \item Un nombre est divisible par 4 si le nombre composé par ses 2 derniers chiffres est un multiple de 4.
        \item Un nombre est divisible par 9 si la somme de ses chiffres est un multiple de 9.
    \end{itemize}
}

\prop{Critères de divisibilité 6 et 10}
{
    \begin{itemize}
        \item Un nombre est divisible par 6 s'il est divisible par 2 et 3
        \item Un nombre est divisible par 10 s'il est divisible par 2 et 5
    \end{itemize}
}
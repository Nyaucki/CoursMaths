\section{Equation de degré 1}

\dfnt{Equation de degré 1}
{Une équation de degré 1 (ou de premier degré) est une équation dont la forme après simplification ne comprends que des termes en $x$ et des nombres.
On appelle membre chaque côté de l'équation.}

\rmq{S'il y a du $x^2$ alors l'équation n'est pas du premier degré et ne peut pas être résolue de la même manière.}

\dfnt{Solution d'une équation}
{On appelle solution(s) d'une équation la (ou les) valeur(s) de $x$ pour laquelle l'égalité est vraie}

\rmq{Une équation de degré 1 aura toujours une unique solution.}

\prop{Equation équivalente}
{Si on ajoute ou soustrait le même terme d'une équation à chaque membre, on obtient une équation équivalente (elle aura la même solution)\\
De la même manière, multiplier ou soustraire chaque membre d'une équation par le même facteur donne une équation équivalente.
}

\textbf{Résoudre une équation du premier degré}

Résoudre une équation du premier degré revient à modifier son écriture jusqu'à aboutir à $x=\dots$
La valeur de $x$ obtenue est appelée solution.

Entre deux lignes de calcul, on ne peut pas mettre de signe égal puisqu'on risquerai de confondre avec celui de l'équation. On met donc le signe $\iff$ (en troisième, ne pas le mettre ne sera pas sanctionné).

\begin{align*}
    \text{Observons la méthode à travers un exemple :}& & &5(x-3)+4x=3x-7\\
    \text{On distribue pour supprimer les parenthèses :}& & \iff & 5x-5\times 3 +4x =3x-7\\
    \text{On simplifie en regroupant de chaque côté :}& & \iff & 9x-15=3x-7\\
    \text{On supprime le terme en $x$ à droite en ajoutant son opposé :} & & \iff & 9x-15-3x=3x-7-3x\\
    \text{On simplifie :}& & \iff &6x-15=-7\\
    \text{On supprime le terme sans $x$ à droite en ajoutant son opposé :}& & \iff &6x-15 +15=-7+15\\
    \text{On simplifie :}& &\iff & 6x=8\\
    \text{On se débarrasse du nombre multipliant le $x$ en divisant par celui-ci :} && \iff &6x\div 6=8\div 6\\
    \text{On peut laisser le résultat sous forme de fraction simplifié :} &&\iff &x=\dfrac{4}{3}
\end{align*}
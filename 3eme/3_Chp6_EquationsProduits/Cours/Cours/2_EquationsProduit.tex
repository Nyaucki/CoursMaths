\section{Équation produit}

\prop{Equation produit}
{
    Un produit de facteurs est nul si et seulement si l'un de ses facteurs est nul.
}
    
\exmpl{$(x-2)(x+2)=0$ Signifie que $x-2=0$ ou $x+2=0$. On a maintenant deux équations du premier degré à résoudre. Leurs deux solutions seront solutions de l'équation produit.}
    
\rmq{ La propriété reste vraie dans le cas d'un produit de trois facteurs ou plus.}

\rmq{La propriété ne marche que s'il y a un produit. Si on distribue, il n'y a plus de produit et on ne peut plus résoudre ! 

Parfois, il faudra même factoriser l'expression pour trouver un produit.}


\exo{Type Brevet}{Brevet Long1}

On considère le programme de calucl suivant :
\begin{itemize}
    \item Choisir un nombre
    \item Lui ajouter 1
    \item Calculer le carré de cette somme
    \item Enlever 16 au résultat obtenu
\end{itemize}

\begin{enumerate}
    \item Vérifier que lorsque le nombre de départ est 4, le résultat est 9.
    \item Quel résultat obtient-on en prenant -3 ?
    \item Le nombre de départ étant $x$, quelle expression obtient-on pour le résultat ? On appellera $P$ cette expression.
    \item Vérifier que $P=x^2+2x-15$
    \item Vérifer que $(x-3)(x+5)=P$
    \item Quels nombres peut-on choisir au départ pour que le résultat final soit 0 ?
\end{enumerate}

\exo{Type Brevet}{Brevet Long2}

On considère les programmes de calucl suivant :
\begin{multicols}{2}
    \textbf{Programme A :}
    \begin{itemize}
        \item Choisir un nombre
        \item Lui ajouter 1
        \item Calculer le carré du résultat obtenu
    \end{itemize}

    \textbf{Programme B :}
    \begin{itemize}
        \item Choisir un nombre
        \item Lui soustraire 7
        \item Calculer le carré du résultat obtenu
    \end{itemize}
\end{multicols}

\begin{enumerate}
    \item On choisit 5 comme nombre de départ. Montrer que le résultat du programme B est 4.
    \item On choisit -2 comme nombre de départ. Quel résultat obtient-on avec le programme A ?
    \item Quel nombre faut-il choisir pour que le résultat du programme 1 soit 0 ?
    \item Quel nombre faut-il choisir pour que le résultat du programme B soit 9 ?
    \item Quel nombre doit-on choisir pour obtenir le même résultat avec les deux programmes ?
\end{enumerate}

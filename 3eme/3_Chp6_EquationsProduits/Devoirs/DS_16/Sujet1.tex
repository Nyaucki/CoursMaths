Le devoir est sur 22 mais sera compté comme une note sur 20. Il y a donc 2 points bonus.

\exo{6}{Calculer} : Résoudre les équations suivantes : 

\begin{minipage}{0.45\textwidth}
    $$(3x+6)(4x-7)=0$$
    \vspace*{15cm}
\end{minipage}
\hfil
\vrule
\hfil
\begin{minipage}{0.45\textwidth}
    $$(3x-5)^2-(x+7)^2=0$$
    \vspace*{15cm}
\end{minipage}

\newpage

\exo{6}{Calculer} : À l'aide du programme de calcul suivant :


\begin{multicols}{2}
    \textbf{Programme A :}
    \begin{itemize}
        \item Choisir un nombre
        \item multiplier par 3
        \item Lui ajouter 2
        \item Calculer le carré du résultat obtenu
    \end{itemize}

    \textbf{Programme B :}
    \begin{itemize}
        \item Choisir un nombre
        \item Multiplier par -2
        \item Lui soustraire 7
        \item Calculer le carré du résultat obtenu
    \end{itemize}
\end{multicols}

\begin{enumerate}
    \item (\hspace{5ex}/0,75) Avec 1 comme nombre de départ, montrer que le résultat du programme B est 81.
    \item (\hspace{5ex}/0,75) Avec -2 comme nombre de départ quel résultat obtient-on avec le programme A ?
    \item (\hspace{5ex}/1,5)Quel nombre faut-il choisir pour que le résultat du programme 1 soit 0 ?
    \item (\hspace{5ex}/1,5)Quel nombre faut-il choisir pour que le résultat du programme B soit 9 ?
    \item (\hspace{5ex}/1,5)Quel nombre doit-on choisir pour obtenir le même résultat avec les deux programmes ?
\end{enumerate}

\newpage

\exo{6}{Modéliser} : Problème

\begin{minipage}[t]{0.45\textwidth}
    Leïla suit une course de VTT dont le tracé est donné ci-contre en traits pleins. Les longueurs sont en kilomètres.
    \begin{enumerate}
        \item Montrer que $BG=5~km$.
        \item Montrer que $(AB)$ est parallèle à $(DF)$.
        \item Montrer que $GD=7.5~km$.
        \item Calculer $DF$.%6
        \item Quelle est la longueur totale du parcours ?%22.5
        \item Leïla avance à 20km/h. Combien de temps lui faut-il pour faire le parcours ?%1h7min30s
    \end{enumerate}
\end{minipage}
\hfil
\begin{minipage}[t]{0.45\textwidth}
        \begin{figure}[H]
        \centering
        \begin{tikzpicture}[scale=1]
            \draw (0,0) coordinate (A) node[above] {$A$} --node [midway,above]{4}(2,0) coordinate (B) node [above] {$B$} -- (1,-1) coordinate (C) node[right] {$C$} --(-2,-4) coordinate (D) node [below] {$D$}--(0,-4) coordinate (E) node [below] {$E$} --(4,-4) coordinate (F) node [below] {$F$};
            \node (A_left) at (-0.1,0) {};
            \tkzInterLL(A,E)(B,D) \tkzGetPoint{G};
            \node[label=180:$G$] at (G) {};
            \draw[white] (A)--node[black] [midway,left]{3}(G);
            \draw[white] (E)--node[black] [midway,right]{4,5}(G);
            \draw[dashed] (A)--node [midway, right]{3} (C)--node [midway,above right]{6} (F);
            \draw [dashed] (A)--(E);
            \draw [gray,right angle quadrant=1,right angle symbol={D}{F}{A}];
            \draw [gray,right angle quadrant=1,right angle symbol={A_left}{B}{E}];
        \end{tikzpicture}
    \end{figure} 
\end{minipage}

\newpage

\exo{4}{Modéliser} : Problème

Squeezie cherche à faire une vidéo de très haute qualité en mettant du Mentos dans des bouteilles de Coca.

Pour cela, il dispose de 1980 boites de Mentos et de 1100 bouteilles de Coca.

Il cherche à les partager en groupes égaux et en utilisant tout le matériel à sa disposition.

\begin{enumerate}
    \item Peut-il faire 33 groupes ?
    \item Donner les décompositions en produits de facteurs premiers de 1980 et 1100.
    \item Quel est le nombre maximum de groupes qu'il peut faire ?
    \item Combien de bouteilles de Coca et de boites de Mentos y aurait-il alors par groupe ?
\end{enumerate}


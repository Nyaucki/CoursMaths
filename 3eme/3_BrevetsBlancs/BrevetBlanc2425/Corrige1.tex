\exo{20}{}

\begin{enumerate}
    \vspace*{1em}\item Les côtés sont multipliés par 3. L'aire est donc multipliée par $3^2=9$. \underbar{Réponse C}
    \vspace*{1em}\item $253,6\times 10^{-5}=2,536\times 10^{2}\times 10^{-5}=2,536\times 10^{2-5}2,536\times 10^{-3}$. \underbar{Réponse C}
    \vspace*{1em}\item 39 et 65 ne sont pas premiers. La bonne réponse est donc $3\times 5\times 13$. \underbar{Réponse B}
    \vspace*{1em}\item La figure n'a pas changé de sens (pas une symétrie) et n'a pas changé de dimension (pas une homothétie). C'est donc une translation. \underbar{Réponse A}
    \vspace*{1em}\item $(-5)^3=(-5)\times (-5)\times (-5)=-125$. \underbar{Réponse A}
\end{enumerate}

\exo{20}{}

\begin{enumerate}
        \item\noindent
        \begin{align*}
            &(2+2)\times 4 \times (2\times 5-3)\\
            =&4\times 4 \times7\\
            =&16\times 7\\
            =&112
        \end{align*}
        
        \item\noindent
        \begin{align*}
            &(-3+2)\times 4 \times (-3\times 5-3)\\
            =&(-1)\times 4 \times(-18)\\
            =&(-4)\times (-18)\\
            =&72
        \end{align*}

        \vspace*{1em}\item Dans la question on demande quelles expressions sont les bonnes. Le pluriel indique qu'au moins deux réponses sont attendues. 
        
        Faire le programme de calcul avec $x$ nous donne :
        $(x+2)\times 4 \times (x\times 5-3)=(x+2)\times 4 \times (5\times x-3)$ soit \underbar{la réponse D}. 

        En distribuant le $(x+2)\times 4$, on obtient $(4x+8)$. On retrouve donc $(4x+8)(5x+3)$ soit \underbar{la réponse C}.

        \vspace*{1em}\item Pour répondre à cette question, on va repartir d'une des expressions trouvée à la question précédente. Je choisis de prendre la C, et je cherche pour quelle valeur de $x$ elle est nulle.
        $$(4x+8)(5x+3)=0$$ Il s'agit d'un produit nul. Au moins un de ses facteurs est donc nul. On a donc :
        \begin{multicols}{2}
            \begin{align*}
                &4x+8=0\\
                \iff & 4x+8-8=0-8\\
                \iff & 4x=-8\\
                \iff & 4x\div 4=-8\div4\\
                \iff & x=-2
            \end{align*}

            \begin{align*}
                &5x-3=0\\
                \iff & 5x-3+3=0+3\\
                \iff & 5x=3\\
                \iff & 5x\div 5=3\div5\\
                \iff & x=\dfrac{3}{5}
            \end{align*}
        \end{multicols}
        $$S=\left\{-2;\dfrac{3}{5}\right\}$$
       Donc les deux nombres permettant d'obtenir 0 sont $-2$ et $\dfrac{3}{5}$.

       \vspace*{1em}\item 
       \begin{align*}
            &(4x+2)(5x-3)\\
            =&4x\times 5x +4x\times (-3)+2\times 5x +2\times (-3)\\
            =&20x^2-12x+10x-6\\
            =&20x^2 -2x-6
       \end{align*}
\end{enumerate}

\exo{10}{}

\begin{enumerate}
    \vspace*{1em}\item 
    \begin{enumerate}[label=\alph*.]
        \vspace*{1em}\item 
    Voici la figure : 
    
    \begin{tikzpicture}
        \draw[gray,dotted] (-3,-3) grid (3,3);
        \draw (0,0)-|(2,2)-|(-2,-2);
    \end{tikzpicture}

    \vspace*{1em}\item Le stylo est orienté vers la droite au début du tracé. Il a tourné 4 fois de 90°. $4\times 90=360$, il a donc fait un tour complet et est de nouveau \underbar{orienté vers la droite}.
    \end{enumerate}
    
    \vspace*{1em}\item 
    \begin{enumerate}[label=\alph*.]
        \vspace*{1em}\item
        La figure 1 n'est pas possible, car la longueur du segment augmente après chaque tournant et non tous les deux tournants.
        \\ La figure 2 n'est pas possible car les angles ne font pas 90°. (On peut aussi le déduire en lisant la question suivante)
        \\ La figure 3 est la suite logique du dessin fait à la question 1.
        \\ La bonne réponse est donc \underbar{la figure 3}.

        \vspace*{1em}\item Pour la figure 2, il faut tourner 6 fois avant de faire un tour complet. On a $360\div 6=60$, il faudrait donc tourner de 60°.
    \end{enumerate}
\end{enumerate}

\exo{15}{}

\begin{enumerate}
    \vspace*{1em}\item La courbe commence à monter au jour 2. L'injection a eu lieu au jour 0. \underbar{Il a donc fallu attendre 2 jours.}
    \vspace*{1em}\item La courbe monte pour la deuxième injection au jour 30. D'après la question précédente, il faut 2 jours pour que les anticorps soient visible. \underbar{l'injection a donc eu lieue à 30-2=28 jours}.
    \vspace*{1em}\item La courbe retourne sur l'axe des abscisse au jour 12. \underbar{Il a donc fallu attendre 12 jours.}
    \vspace*{1em}\item La courbe reste en dessous de 100. On peut donc estimer le taux d'anticorps à \underbar{environ 90}.
    \vspace*{1em}\item Le taux est supérieur à partir du jour 34 et jusqu'au jour 36, soit pendant une durée de \underbar{deux jours}.
\end{enumerate}

\exo{20}{}
\begin{enumerate}
    \vspace*{1em}\item Les points D, K et L sont alignés. On a donc \begin{align*}
        &DK+KL=DL\\
        \iff & DK+ 120=600\\
        \iff & DK + 120-120=600-120\\
        \iff & DK=480
    \end{align*}

    \vspace*{1em}\item On va chercher à utiliser la réciproque du théorème de Pythagore dans le triangle DKJ.
    DJ, le plus long côté, serait l'hypoténuse. On a d'une part :
    $DJ^2=520^2=270400$
    
    Et d'autre part : 
    $DK^2+KJ^2=480^2+200^2=270400$

    Ainsi, $DK^2+KJ^2=DJ^2$. D'après la réciproque du théorème de Pythagore,\underbar{ le triangle DJK est donc rectangle en K}.

    \vspace*{1em}\item (KJ) et (LA) sont toutes deux perpendiculaire à (DL). Elles sont donc parallèles.
    
    \vspace*{1em}\item Les droites (KJ) et (LA) sont parallèles.
    
    Les droites (LK) et (JA) se coupent en D. 

    Ainsi, d'après le théorème de Thalès :

    \begin{align*}
        &\dfrac{DK}{DL}=\dfrac{DJ}{DA}=\dfrac{KJ}{LA}\\
        \Rightarrow & \dfrac{480}{600}=\dfrac{520}{DA}=\dfrac{200}{LA}\\
        \Rightarrow & \dfrac{480}{600}=\dfrac{520}{DA}\\
        \Rightarrow & DA=600\times 520 \div 480\\
        \Rightarrow & DA=650
    \end{align*}
    Donc, \underbar{[DA] mesure 650 mètres}

    \vspace*{1em}\item Comme à la question 1, D, J et A sont alignés dans cet ordre, donc DJ+JA=DA. On en déduis que JA=130m.
    
    \underbar{Le trajet DKJA a donc pour longueur 480+200+130=810m.}
\end{enumerate}

\exo{15}{}
\begin{enumerate}
    \vspace*{1em}\item La moyenne est la somme de toutes les valeurs, divisée par le nombre de valeurs. Ici :
    $$\overline{x}=\dfrac{453+649+786+854+860+1003+957+838}{8}=800$$

    \vspace*{1em}\item Au cours de la période, on a vendu 453+649+786+854+860+1003+957+838=6400 pots de glace.
    
    On a donc $67\%$ de 6400 soit $0,67\times 6400=4288$ pots de glace à une boule et 6400-4288=2112 pots à deux boules. 

    \underbar{La somme rapportée est donc $4288\times 2,8 +2112\times 3,5 =19398,4 $ \texteuro}

    \vspace*{1em}\item \begin{enumerate}[label=\alph*.]
        \vspace*{1em}\item Le diamètre de la boule de glace est de 4,2cm. Son rayon est donc de $4,2\div 2=2,1$cm.
        
        En utilisant la formule donnée, on obtient : $V=\dfrac{4}{3}\times\pi\times (2,1)^3=38,7cm^3$

        \vspace*{1em}\item On commence par convertir en L : $39cm^3=0,039$L.
        
        Il suffit ensuite de diviser : $10\div 0,039=256$. \underbar{On peut donc faire 256 boules avec un bac.}
    \end{enumerate}
\end{enumerate}



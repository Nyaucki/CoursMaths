\begin{minipage}[t]{0.55\textwidth}
    \exo{}{}

    La situation peut être représenté par le schéma ci-contre : 
    (Comme seule la hauteur de la pyramide nous intéresse, on peut juste représenter le segment reliant le somment au sol)

    Ainsi on a :
    \begin{itemize}
        \item Thalès et la pyramide sont parallèles car verticales
        \item La droites reliant les sommets de Thalès et de la pyramide et celle reliant leurs pieds sont séquentes
    \end{itemize}

    On peut donc utiliser le théorème de Thalès, et on a :\\
    $\dfrac{\text{Thalès}}{\text{Pyramide}}=\dfrac{\text{Ombre de Thalès}}{\text{Ombre de la Pyramide}}$

    Donc, en connaissant sa taille et les deux ombres, Thalès peut trouver celle de la pyramide.
    
\end{minipage}
\hfill
%%%%%%%%%%%% Exercice 2+28%%%%
\begin{minipage}[t]{0.35\textwidth}
        \begin{figure}[H]
        \centering
        \begin{tikzpicture}[scale=1.5]
            \node (A) at (0,0) {}; %positions A
            \node (B) at (1,0) {}; %positions B
            \node (C) at (1,1) {}; %positions C
            \node (B') at (3,0) {}; %positions B'
            \node (C') at (3,3) {};
            \draw (A.base) -- (B'.base) node [midway,below] {Ombres sur le sol};
            \draw (A.base) -- (C.base);
            \draw (B.base) -- (C.base) node [midway,right] {Thalès};
            \draw (B.base) -- (B'.base) ;
            \draw (A.base) -- (C'.base) ;
            \draw (B'.base) -- (C'.base) node [midway,right] {Pyramide};
        \end{tikzpicture}
    \end{figure} 
\end{minipage}

\newpage

\begin{minipage}[t]{0.55\textwidth}
    \exo{}{}

    \cnt 
    \begin{align*}
        \text{distance}&=\text{vitesse}\times\text{temps}\\
        &=18\times3:60\\
        &=0,9~km
    \end{align*}
    Il y a 0,9 km entre la station et le premier refuge.

    \cnt 

\end{minipage}
\hfill
%%%%%%%%%%%% Exercice 2+28%%%%
\begin{minipage}[t]{0.35\textwidth}
        \begin{figure}[H]
        \centering
        \begin{tikzpicture}[scale=1.5]
            \node (A) at (0,0) {}; %positions A
            \node (B) at (1,0) {}; %positions B
            \node (C) at (1,1) {}; %positions C
            \node (B') at (3,0) {}; %positions B'
            \node (C') at (3,3) {};
            \draw (A.base) -- (B'.base) node [midway,below] {Ombres sur le sol};
            \draw (A.base) -- (C.base);
            \draw (B.base) -- (C.base) node [midway,right] {Thalès};
            \draw (B.base) -- (B'.base) ;
            \draw (A.base) -- (C'.base) ;
            \draw (B'.base) -- (C'.base) node [midway,right] {Pyramide};
        \end{tikzpicture}
    \end{figure} 
\end{minipage}
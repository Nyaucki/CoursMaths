
%%%%%%%%%%%% Exercie 1 +36 %%%%%%%%
\exo{communiquer}{Concret1} 

Le théorème de Thalès n'a pas été découvert par Thalès. Ni démontré par lui.

Par contre, Thalès est célèbre pour avoir utiliser ce théorème pour trouver la hauteur des pyramides d'Egyptes. Il aurait dit :

"Je n'ai qu'à mesurer mon ombre et celle de la pyramide. Comme je connais ma taille, je pourrais trouver celle de la pyramide."

Expliquer en quoi est-ce une utilisation du théorème de Thalès ?

%%%%%%%%%%%% Exercie 2 +36 %%%%%%%
\exo{modéliser}{Concret2} 

Une station de ski est située à 1650m d'alititude. 

Depuis cette station, un télésiège avançant à 18$km/h$ permet de rejoindre deux refuges : 

Le télésiège met 3 minutes pour rejoindre le premier, qui est à 1930m d'alititude. Il continue ensuite jusqu'au deuxième refuge, à 2470m d'alititude.

On supposera que la station et les deux refuges sont alignés dans l'axe du télésiège.

\cnt Quelle distance parcours le télésiège entre la station et le premier refuge ?

\cnt Quelle distance parcours le télésiège entre la station et le deuxième refuge ?

\cnt Combien de temps faut-il pour rejoindre le deuxième refuge depuis la station ?



\exo{modéliser, communiquer}{Concret3}

Prenez un stylo à la verticale dans la main et tender le bras, en vous mettant face à un camarade. 

Reculez jusqu'à ce que le stylo semble faire la même taille que votre camarade.

\cnt Faire un schéma avec (avec juste des points et segments) représentant la situation.

\cnt Expliquer comment, en connaissant la longueur de votre bras, la taille du stylo et en mesurant la distance vous séparant de votre camarade, vous pourriez connaître sa taille.
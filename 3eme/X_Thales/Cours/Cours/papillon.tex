\prop{Le théorème de Thalès}
{\begin{minipage}{0.40\textwidth}
    Soit deux triangles $ABC$ et $AB'C'$.\\
    Si on a  :
    \begin{itemize}
        \item $A$, $B$ et $B'$ sont alignés
        \item $A$, $C$ et $C'$ sont alignés
        \item ($BC$) et ($B'C'$) sont parallèles
    \end{itemize}
    \vspace{1em}
    Alors, $\dfrac{AB}{AB'}=\dfrac{AC}{AC'}=\dfrac{BC}{B'C'}$
\end{minipage}
\hfill
\begin{minipage}{0.55\textwidth}
    \begin{tikzpicture}[scale=2]
        \node (A) at (0,1) {}; %positions A
        \node (B) at (1,1) {}; %positions B
        \node (C) at (1,2) {}; %positions C
        \node (B') at (-2,1) {}; %positions B'
        \node (C') at (-2,-1) {}; %positions C'
        \node [above=0.01cm of A] (A_label) {$A$}; %label A
        \node [right=0.01cm of B] (B_label) {$B$}; %label B
        \node [above=0.01cm of C] (C_label) {$C$}; %label C
        \node [left=0.01cm of B'] (B'_label) {$B'$}; %label B'
        \node [left=0.01cm of C'] (C'_label) {$C'$}; %label C'
        \draw (A.base) -- (B.base);
        \draw (A.base) -- (C.base);
        \draw (B.base) -- (C.base);
        \draw (A.base) -- (B'.base);
        \draw (A.base) -- (C'.base);
        \draw (B'.base) -- (C'.base);
        \draw [fill=black] (A) circle (0.04) ;
        \draw [fill=black] (B) circle (0.04) ;
        \draw [fill=black] (C) circle (0.04) ;
        \draw [fill=black] (B') circle (0.04) ;
        \draw [fill=black] (C') circle (0.04) ;
    \end{tikzpicture}
\end{minipage}}

\rmq{On a aussi $\dfrac{AB}{AB'}=\dfrac{AC}{AC'}=\dfrac{BC}{B'C'}$ : tant que $B'$ et $C'$ sont au même niveau dans les fractions, l'égalité est juste.}

\rmq{Les hypothèses "$A$, $B$ et $B'$ sont alignés" et "$A$, $C$ et $C'$ sont alignés" sont équivalentes à : "Les droites $(BB')$ et $(CC')$ sont sécantes en $A$". On peut donc écrire l'un ou l'autre.}

\exmpl
{Dans le triangle suivant, $(BC)$ est parallèle à $(B'C')$.\\
    \begin{figure}[H]
        \centering
        \begin{tikzpicture}[scale=1]
            \node (A) at (4,1) {}; %positions A
            \node (B) at (8,2) {}; %positions B
            \node (C) at (3,4) {}; %positions C
            \node (B') at (0,0) {}; %positions B'
            \node (C') at (5,-2) {}; %positions C'
            \node [above=0.01cm of A] (A_label) {$A$}; %label A
            \node [right=0.01cm of B] (B_label) {$B$}; %label B
            \node [above=0.01cm of C] (C_label) {$C$}; %label C
            \node [left=0.01cm of B'] (B'_label) {$B'$}; %label B'
            \node [left=0.01cm of C'] (C'_label) {$C'$}; %label C'
            \draw (A.base) -- (B.base);
            \draw (A.base) -- (C.base) node [midway,right] {2};
            \draw (B.base) -- (C.base) ;
            \draw (A.base) -- (B'.base);
            \draw (A.base) -- (C'.base) node [midway,right] {3};
            \draw (B'.base) -- (C'.base) node [midway,above] {4} ;
            \draw [fill=black] (A) circle (0.08) ;
            \draw [fill=black] (B) circle (0.08) ;
            \draw [fill=black] (C) circle (0.08) ;
            \draw [fill=black] (B') circle (0.08) ;
            \draw [fill=black] (C') circle (0.08) ;
        \end{tikzpicture}
    \end{figure} 
    \textbf{Cherchons à trouver la longueur BC :}
    \begin{itemize}
        \item $A$, $B$ et $B'$ sont alignés
        \item $A$, $C$ et $C'$ sont alignés
        \item ($BC$) et ($B'C'$) sont parallèles
    \end{itemize}
    D'après le théorème de Thalès, on a l'égalité : 
    \begin{align*}
        \dfrac{AB}{AB'}&=\dfrac{AC}{AC'}=\dfrac{BC}{B'C'}&&\\
        \dfrac{AB}{AB'}&=\dfrac{2}{3}=\dfrac{BC}{4}&&\\
        \dfrac{2}{3}&=\dfrac{BC}{4}&&\text{On retire la fraction inutile}\\
        \dfrac{2}{3}\times 4&=BC&&\text{On isole BC}\\
        \dfrac{8}{3}&=BC&&
    \end{align*}
    Donc, $BC=\dfrac{8}{3}$}

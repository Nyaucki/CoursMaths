\section{Histogramme}

Un histogramme ou diagramme à bâton est un autre moyen de représenter une série :

\begin{minipage}[c]{0.3\textwidth}
    \begin{center}
        \begin{tabular}{c*{5}{|c}}
            Valeurs & 1 & 3 & 4 &5 & 6 \\ \hline
            Effectifs & 3 & 2 & 3 & 2 & 1
        \end{tabular}
    \end{center}
\end{minipage}
\hfil
\begin{minipage}[c]{0.2\textwidth}
    \begin{center}
        Est équivalent à 
    \end{center}
\end{minipage}
\hfil
\begin{minipage}[c]{0.45\textwidth}
    \begin{center}
        \begin{tikzpicture}
            \draw [<->] (0,4) node[left]{Effectifs} |- (5.5,0) node[right]{Valeurs};
            \foreach \z [count=\zi] in {1,3,4,5,6}
                {
                    \node at (\zi - 0.5,-0.5) {\z};
                }
            \foreach \x in {0,...,3}
                {
                    \draw (-0.1,\x) node[left]{\x} --(0.1,\x);
                    \draw[lightgray,thin] (0,\x) --(5,\x);
                }
            \foreach \y [count=\yi] in {3,2,3,2,1}
            {
                \draw[fill=white] (\yi -1,0) rectangle (\yi,\y);
            }
        \end{tikzpicture}
    \end{center}
\end{minipage}
\section{Les basiques de statistiques}

\dfnt{Vocabulaire}
{\textbf{Une série statistiques} est un ensemble de données, chiffrées ou non.
Chaque élément de la série a une \textbf{valeur} et on appelle \textbf{effectif} d'une valeur le nombre de fois où elle est répétée dans la série.\\
L'\textbf{effectif total} d'une série est le nombre d'éléments dans la série.}

\exmpl{\begin{itemize}
    \item (1 ; 3 ; 3 ; 4 ; 4) est une série chiffrée. Les valeurs de cette série sont 1 ; 3 et 4. L'effectif de la valeur 3 est 2. L'effectif total est de 5.
    \item (bleu, rose, rose, rose, rouge, jaune) est une série non chifrée. Les valeurs de cette série sont bleu, rose, rouge et jaune. L'effectif de la valeur rose est 3. L'effectif total est de 6.
\end{itemize}
}

\dfnt{Etendue d'une série}
{Pour une série chiffrée, l'étendue correspond à l'écart entre la valeur minimale et la valeur maximale. $$\text{Etendue}=\text{Valeur}_{\text{max}}-\text{Valeur}_{\text{min}}$$}

\exmpl{Pour la série (1 ; 3 ; 3 ; 4 ; 4 ; 4), l'étendue est $4-1=3$}

\dfnt{Fréquence d'une valeur}
{Chaque valeur d'une série possède une fréquence correspondant au taux d'apparition de cette valeur au sein de la série. $$\text{fréquence}=\dfrac{\text{effectif}_{\text{valeur}}}{\text{effectif}_{\text{total}}}$$}

\exmpl{Pour la série (1 ; 3 ; 3 ; 4 ; 4 ; 4) la fréquence de la valeur 3 est $\dfrac{2}{6}=\dfrac{1}{3}$}

\rmq{La somme des fréquences de toutes les valeurs d'une série est toujours égale à 1.}

Pour les séries contenant un grand nombre de valeurs, on préfèrera les noter sous forme de tableau indiquant les valeurs et leurs effectifs. Ainsi,  la série (1 ; 3 ; 3 ; 4 ; 4 ; 4 ; 5 ; 5 ; 5 ; 5 ; 5 ; 6 ; 6) correspond au tableau :
\begin{figure}[H]
    \center
    \begin{tabular}{c*{5}{|c}}
        Valeurs & 1 & 3 & 4 &5 & 6 \\ \hline
        Effectifs & 1 & 2 & 3 & 5 & 2
    \end{tabular}
\end{figure}

Souvent, on en profitera pour ajouter une colonne total et une ligne fréquence, ce qui donnerai le tableau suivant :

\begin{figure}[H]
    \center
    \begin{tabular}{c*{6}{|c}}
        Valeurs & 1 & 3 & 4 &5 & 6 & Total \\ \hline
        Effectifs & 1 & 2 & 3 & 5 & 2 & 13\\ \hline
        Fréquences &$\dfrac{1}{13}$ & $\dfrac{2}{13}$  & $\dfrac{3}{13}$  & $\dfrac{5}{13}$  & $\dfrac{2}{13}$  & $\dfrac{13}{13}$ 
    \end{tabular}
\end{figure}
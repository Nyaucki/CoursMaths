\documentclass[12pt,a4paper,french,fleqn]{beamer}

\usepackage{tikz}

\begin{document}

\begin{frame}
        \begin{itemize}
            \item Une voiture roule à 50 $km/h$ pendant 12 minutes. Quelle distance a-t-elle parcouru ?
            \item Un cycliste mets 20 minutes à faire les 6 kilomètres qui le séparent de chez lui. À quelle vitesse a-t-il roulé ?
            \item Un chameau avance sans se presser à 10 $km/h$. Combien de temps lui faudra-t-il pour traverser un désert de 25 $km$ ?
        \end{itemize}
\end{frame}

\begin{frame}
    \begin{center}
        Un lutin veut décorer des sapins de Noël. Pour cela, il dispose de  84 guirlandes et 154 boules de Noël. Il veut être sûr que chaque sapin est décoré de la même manière, et d'utiliser toutes ses décorations.
    \end{center}
    \begin{enumerate}
        \item Peut-il décorer 21 sapins ?
        \item Combien de sapins peut-il décorer au maximum.
        \item Même question avec 4680 boules et 1800 guirlandes.
        \item Même question avec 8281 boules et 2730 guirlandes.
    \end{enumerate}
\end{frame}


\end{document}

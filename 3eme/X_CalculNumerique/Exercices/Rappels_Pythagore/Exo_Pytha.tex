%%%%%%%%%%%% Exercice 1+12 %%%%%%%%%%%
\vspace{-1em}
\begin{minipage}[t]{0.45\textwidth}
    \exo{cours}{Pythagore1} 

    Calculer $AB$ \vspace{-1em}
    \begin{figure}[H]
        \centering
        \begin{tikzpicture}[scale=2]
            \node (A) at (0,0) {}; %positions A
            \node (B) at (2,0) {}; %positions B
            \node (C) at (2,1.5) {}; %positions C
            \node [left=0.005cm of A] (A_label) {$A$}; %label A
            \node [right=0.01cm of B] (B_label) {$B$}; %label B
            \node [above=0.01cm of C] (C_label) {$C$}; %label C
            \draw (1.8,0) -- (1.8,0.2) ;
            \draw (2,0.2) -- (1.8,0.2) ;
            \draw (A.base) -- (B.base);%12
            \draw (A.base) -- (C.base) node [midway,above] {13};
            \draw (B.base) -- (C.base) node [midway,right] {5};
            \draw [fill=black] (A) circle (0.05em) ;
            \draw [fill=black] (B) circle (0.05em) ;
            \draw [fill=black] (C) circle (0.05em) ;
        \end{tikzpicture}
    \end{figure} 
\end{minipage}
\hfill
%%%%%%%%%%%% Exercice 2+12 %%%%%%%%%%%
\begin{minipage}[t]{0.45\textwidth}
    \exo{cours}{Pythagore2} 

    Calculer $AC$ \vspace{-1em}
    \begin{figure}[H]
        \centering
        \begin{tikzpicture}[scale=2]
            \node (A) at (-5,1) {}; %positions A
            \node (B) at (-3.5,0) {}; %positions B
            \node (C) at (-6,-0.5) {}; %positions C
            \node [above=0.005cm of A] (A_label) {$A$}; %label A
            \node [right=0.01cm of B] (B_label) {$B$}; %label B
            \node [left=0.01cm of C] (C_label) {$C$}; %label C
            \draw (A.base) -- (B.base)node [midway,above right] {6};
            \draw (A.base) -- (C.base) ;
            \draw (B.base) -- (C.base) node [midway,below] {10};
            \draw (-5.11,0.83) -- (-4.95,0.72) ;
            \draw (-4.83,0.89) -- (-4.95,0.72) ;
            \draw [fill=black] (A) circle (0.05em) ;
            \draw [fill=black] (B) circle (0.05em) ;
            \draw [fill=black] (C) circle (0.05em) ;
        \end{tikzpicture}
    \end{figure}  
\end{minipage}


%%%%%%%%%%%% Exercice 3+12 %%%%%%%%%%%

\begin{minipage}[t]{0.45\textwidth}
    \exo{cours}{Pythagore3} 

    Le triangle $ABC$ est-il rectangle ?
    \begin{figure}[H]
        \centering
        \begin{tikzpicture}[scale=2]
            \node (A) at (0,0) {}; %positions A
            \node (B) at (2,0) {}; %positions B
            \node (C) at (2,1.5) {}; %positions C
            \node [left=0.005cm of A] (A_label) {$A$}; %label A
            \node [right=0.01cm of B] (B_label) {$B$}; %label B
            \node [above=0.01cm of C] (C_label) {$C$}; %label C
            \draw (A.base) -- (B.base) node [midway,above] {5};%12
            \draw (A.base) -- (C.base) node [midway,above] {7};
            \draw (B.base) -- (C.base) node [midway,right] {5};
            \draw [fill=black] (A) circle (0.05em) ;
            \draw [fill=black] (B) circle (0.05em) ;
            \draw [fill=black] (C) circle (0.05em) ;
        \end{tikzpicture}
    \end{figure} 
\end{minipage}
\hfill
%%%%%%%%%%%% Exercice 4+12 %%%%%%%%%%%
\begin{minipage}[t]{0.45\textwidth}
    \exo{cours}{Pythagore4} 

    Le triangle $ABC$ est-il rectangle ? 
    \begin{figure}[H]
        \centering
        \begin{tikzpicture}[scale=2]
            \node (A) at (-5,1) {}; %positions A
            \node (B) at (-3.5,0) {}; %positions B
            \node (C) at (-6,-0.5) {}; %positions C
            \node [above=0.005cm of A] (A_label) {$A$}; %label A
            \node [right=0.01cm of B] (B_label) {$B$}; %label B
            \node [left=0.01cm of C] (C_label) {$C$}; %label C
            \draw (A.base) -- (B.base)node [midway,right] {15};
            \draw (A.base) -- (C.base) node [midway,left] {8};
            \draw (B.base) -- (C.base) node [midway,below] {17};
            \draw [fill=black] (A) circle (0.05em) ;
            \draw [fill=black] (B) circle (0.05em) ;
            \draw [fill=black] (C) circle (0.05em) ;
        \end{tikzpicture}
    \end{figure}  
\end{minipage}
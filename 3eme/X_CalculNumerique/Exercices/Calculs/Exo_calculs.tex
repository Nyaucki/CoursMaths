%%%%%%%%%%%% Exercice 1+28%%%%
\vspace{-1em}
\begin{minipage}[t]{0.45\textwidth}
    \exo{calcul}{Calcul1} 

    $(EG)$ et $(DJ)$ sont parallèles. Calculer $HJ$
        \begin{figure}[H]
        \centering
        \begin{tikzpicture}[scale=2]
            \node (A) at (0.5,0.5) {}; %positions A
            \node (B) at (1.75,-0.250) {}; %positions B
            \node (C) at (2,0) {}; %positions C
            \node (B') at (3,-1) {}; %positions B'
            \node (C') at (3.5,-0.5) {}; %positions C'
            \node [above=0.01cm of A] (A_label) {$H$}; %label A
            \node [below=0.01cm of B] (B_label) {$E$}; %label B
            \node [above=0.01cm of C] (C_label) {$G$}; %label C
            \node [below=0.01cm of B'] (B'_label) {$D$}; %label B'
            \node [above=0.01cm of C'] (C'_label) {$J$}; %label C'
            \draw (A.base) -- (B.base) node [midway,below] {1};
            \draw (A.base) -- (C.base) node [midway,above] {3};
            \draw (B.base) -- (C.base) ;
            \draw (B.base) -- (B'.base) node [midway,below] {5};
            \draw (A.base) -- (C'.base);
            \draw (B'.base) -- (C'.base) ;
            \draw [fill=black] (A) circle (0.05em) ;
            \draw [fill=black] (B) circle (0.05em) ;
            \draw [fill=black] (C) circle (0.05em) ;
            \draw [fill=black] (B') circle (0.05em) ;
            \draw [fill=black] (C') circle (0.05em) ;
        \end{tikzpicture}
    \end{figure} 
\end{minipage}
\hfill
%%%%%%%%%%%% Exercice 2+28%%%%
\begin{minipage}[t]{0.45\textwidth}
    \exo{calcul}{Calcul2} 

    $(MW)$ et $(ZB)$ sont-elles parallèles ? 
        \begin{figure}[H]
        \centering
        \begin{tikzpicture}[scale=0.75]
            \node (A) at (4,1) {}; %positions A
            \node (B) at (8,2) {}; %positions B
            \node (C) at (3,4) {}; %positions C
            \node (B') at (0,0) {}; %positions B'
            \node (C') at (5,-2) {}; %positions C'
            \node [above right=0.01cm of A] (A_label) {$Y$}; %label A
            \node [right=0.01cm of B] (B_label) {$M$}; %label B
            \node [above=0.01cm of C] (C_label) {$W$}; %label C
            \node [left=0.01cm of B'] (B'_label) {$Z$}; %label B'
            \node [right=0.01cm of C'] (C'_label) {$B$}; %label C'
            \draw (A.base) -- (B.base) node [midway,below] {1};
            \draw (A.base) -- (C.base) node [midway,left] {2};
            \draw (B.base) -- (C.base) node [midway,above] {4};
            \draw (A.base) -- (B'.base) node [midway,above] {1,5};
            \draw (A.base) -- (C'.base) node [midway,right] {2,5} ;
            \draw (B'.base) -- (C'.base) node [midway,below] {5};
            \draw [fill=black] (A) circle (0.13em) ;
            \draw [fill=black] (B) circle (0.13em) ;
            \draw [fill=black] (C) circle (0.13em) ;
            \draw [fill=black] (B') circle (0.13em) ;
            \draw [fill=black] (C') circle (0.13em) ;
        \end{tikzpicture}
    \end{figure} 
\end{minipage}

%%%%%%%%%%%% Exercice 3+28%%%%
\vspace{-2em}
\begin{minipage}[t]{0.45\textwidth}
    \exo{calcul}{Calcul3} 

    $(BC)$ et $(B'C')$ sont parallèles. Calculer $AB'$
        \begin{figure}[H]
        \centering
        \begin{tikzpicture}[scale=2]
            \node (A) at (2,0.2) {}; %positions A
            \node (B) at (2.72,0.36) {}; %positions B
            \node (C) at (2,0.6) {}; %positions C
            \node (B') at (0.2,-0.2) {}; %positions B'
            \node (C') at (2,-0.8) {}; %positions C'
            \node [above left=0.005cm of A] (A_label) {$A$}; %label A
            \node [right=0.01cm of B] (B_label) {$B$}; %label B
            \node [above=0.01cm of C] (C_label) {$C$}; %label C
            \node [left=0.01cm of B'] (B'_label) {$B'$}; %label B'
            \node [below=0.01cm of C'] (C'_label) {$C'$}; %label C'
            \draw (A.base) -- (B.base)node [midway,below] {$\dfrac{3}{4}$} ;
            \draw (A.base) -- (C.base) ;
            \draw (B.base) -- (C.base) node [midway,above] {$\dfrac{1}{3}$};
            \draw (A.base) -- (B'.base);
            \draw (A.base) -- (C'.base);
            \draw (B'.base) -- (C'.base) node [midway,below] {$\dfrac{5}{2}$} ;
            \draw [fill=black] (A) circle (0.05em) ;
            \draw [fill=black] (B) circle (0.05em) ;
            \draw [fill=black] (C) circle (0.05em) ;
            \draw [fill=black] (B') circle (0.05em) ;
            \draw [fill=black] (C') circle (0.05em) ;
        \end{tikzpicture}
    \end{figure} 
\end{minipage}
\hfill
%%%%%%%%%%%% Exercice 4+28%%%%
\begin{minipage}[t]{0.45\textwidth}
    \exo{calcul}{Calcul4} 

    $(BC)$ et $(B'C')$ sont-elles parallèles ? 
        \begin{figure}[H]
        \centering
        \begin{tikzpicture}[scale=3]
            \node (A) at (1.2,0) {}; %positions A
            \node (B) at (0.2,0.6) {}; %positions B
            \node (C) at (0.4,-0.4) {}; %positions C
            \node (B') at (1.7,-0.3) {}; %positions B'
            \node (C') at (1.6,0.2) {}; %positions C'
            \node [above=0.005cm of A] (A_label) {$A$}; %label A
            \node [left=0.01cm of B] (B_label) {$B$}; %label B
            \node [left=0.01cm of C] (C_label) {$C$}; %label C
            \node [right=0.01cm of B'] (B'_label) {$B'$}; %label B'
            \node [right=0.01cm of C'] (C'_label) {$C'$}; %label C'
            \draw (A.base) -- (B.base)node [midway,above] {$\dfrac{2}{3}$} ;
            \draw (A.base) -- (C.base) node [midway,below] {$\dfrac{1}{4}$};
            \draw (B.base) -- (C.base) node [midway,left] {$\dfrac{1}{2}$};
            \draw (A.base) -- (B'.base) node [midway,below] {$\dfrac{2}{5}$};
            \draw (A.base) -- (C'.base) node [midway,above] {$\dfrac{3}{20}$};
            \draw (B'.base) -- (C'.base) node [midway,right] {$\dfrac{3}{10}$} ;
            \draw [fill=black] (A) circle (0.028em) ;
            \draw [fill=black] (B) circle (0.028em) ;
            \draw [fill=black] (C) circle (0.028em) ;
            \draw [fill=black] (B') circle (0.028em) ;
            \draw [fill=black] (C') circle (0.028em) ;
        \end{tikzpicture}
    \end{figure}  
\end{minipage}

%%%%%%%%%%%% Exercice 5+28%%%%
\vspace{-2em}
\begin{minipage}[t]{0.45\textwidth}
    \exo{calcul}{Calcul5} 

    $(BC) \mathbin{\!/\mkern-5mu/\!} (ED)$ et $(EB) \mathbin{\!/\mkern-5mu/\!} (DC)
    $\\
    Calculer $BC$ puis $ED$.
        \begin{figure}[H]
        \centering
        \begin{tikzpicture}[scale=1.5]
            \node (A) at (0,0) {}; %positions A
            \node (B) at (1.5,0) {}; %positions B
            \node (C) at (3,0) {}; %positions C
            \node (D) at (3,2) {}; %positions D
            \node (E) at (1.5,2) {}; %positions C'
            \node (F) at (3,4) {}; %positions C'
            \node [below=0.005cm of A] (A_label) {$A$}; %label A
            \node [below=0.01cm of B] (B_label) {$B$}; %label B
            \node [below=0.01cm of C] (C_label) {$C$}; %label C
            \node [right=0.01cm of D] (D_label) {$D$}; %label D
            \node [above left=0.01cm of E] (E_label) {$E$}; %label E
            \node [right=0.01cm of F] (F_label) {$F$}; %label F
            \draw (A.base) -- (B.base) node [midway,below] {3} ;
            \draw (A.base) -- (E.base) node [midway,above left] {2} ;
            \draw (B.base) -- (C.base) ;
            \draw (B.base) -- (E.base);
            \draw (E.base) -- (D.base);
            \draw (C.base) -- (D.base) node [midway,right] {3} ;
            \draw (E.base) -- (F.base) node [midway,above left] {4} ;
            \draw (F.base) -- (D.base) ;
        \end{tikzpicture}
    \end{figure} 
\end{minipage}
\hfill
%%%%%%%%%%%% Exercice 6+28%%%%
\begin{minipage}[t]{0.45\textwidth}
    \exo{calcul}{Calcul6} 

    $(BA),~(ED)$ et $(GF)$ sont parallèles.\\
    Calculer $ED$ puis $CG$.
        \begin{figure}[H]
        \centering
        \begin{tikzpicture}[scale=1.5]
            \node (A) at (1,4) {}; %positions A
            \node (B) at (3,4) {}; %positions B
            \node (C) at (1.89,2.52) {}; %positions C
            \node (D) at (2.2,2) {}; %positions D
            \node (E) at (1.5,2) {}; %positions C'
            \node (F) at (0,0) {}; %positions C'
            \node (G) at (3.4,0) {}; %positions C'
            \node [left=0.005cm of A] (A_label) {$A$}; %label A
            \node [right=0.01cm of B] (B_label) {$B$}; %label B
            \node [left=0.01cm of C] (C_label) {$C$}; %label C
            \node [right=0.01cm of D] (D_label) {$D$}; %label D
            \node [left=0.01cm of E] (E_label) {$E$}; %label E
            \node [below=0.01cm of F] (F_label) {$F$}; %label F
            \node [below=0.01cm of G] (G_label) {$G$}; %label G
            \draw (A.base) -- (B.base) node [midway,below] {2} ;
            \draw (A.base) -- (C.base) node [midway,left] {3} ;
            \draw (D.base) -- (C.base) node [midway,right] {1} ;
            \draw (C.base) -- (E.base);
            \draw (C.base) -- (B.base);
            \draw (E.base) -- (D.base);
            \draw (F.base) -- (E.base)  ;
            \draw (D.base) -- (G.base) ;
            \draw (F.base) -- (G.base) node [midway,below] {5} ;
        \end{tikzpicture}
    \end{figure} 
\end{minipage}

%%%%%%%%%%%% Exercice 7+28%%%%
\vspace{-1em}
\begin{minipage}[t]{0.45\textwidth}
    \exo{calcul}{Calcul7} 

    Montrer que $(BG)$ et $(DI)$ sont parallèles.\\
    Montrer que $(CH)$ et $(EF)$ sont parallèles.
        \begin{figure}[H]
        \centering
        \begin{tikzpicture}[scale=1.4]
            \node (A) at (0,0) {}; %positions A
            \node (B) at (0.75,0) {}; %positions B
            \node (C) at (1.5,0) {}; %positions C
            \node (D) at (3,0) {}; %positions D
            \node (E) at (4,0) {}; %positions C'
            \node (F) at (5,5) {}; %positions C'
            \node (I) at (2.46,2.46) {}; %positions C'
            \node (H) at (1.88,1.88) {}; %positions C'
            \node (G) at (0.62,0.62) {}; %positions C'
            \node [below left=0.005cm of A] (A_label) {$A$}; %label A
            \node [below=0.01cm of B] (B_label) {$B$}; %label B
            \node [below right=0.01cm of C] (C_label) {$C$}; %label C
            \node [below=0.01cm of D] (D_label) {$D$}; %label D
            \node [below =0.01cm of E] (E_label) {$E$}; %label E
            \node [left=0.01cm of F] (F_label) {$F$}; %label F
            \node [above left=0.01cm of G] (G_label) {$G$}; %label G
            \node [above left=0.01cm of H] (H_label) {$H$}; %label H
            \node [above=0.01cm of I] (I_label) {$I$}; %label I
            \draw (A.base) -- (B.base) node [midway,below] {0,75} ;
            \draw (B.base) -- (C.base) node [midway,below] {0,75} ;
            \draw (C.base) -- (D.base) node [midway,below] {1,5} ;
            \draw (D.base) -- (E.base) node [midway,below] {1} ;
            \draw (A.base) -- (G.base) node [midway,above left] {1} ;
            \draw (G.base) -- (H.base) node [midway,above left] {2} ;
            \draw (H.base) -- (I.base) node [midway,above left] {1} ;
            \draw (I.base) -- (F.base) node [midway,above left] {4} ;
            \draw (E.base) -- (F.base) node [midway,right] {12} ;
            \draw (D.base) -- (I.base) node [midway,right] {6} ;
            \draw (C.base) -- (H.base) node [midway,right] {4,5} ;
            \draw (B.base) -- (G.base) node [midway,right] {1,5} ;
        \end{tikzpicture}
    \end{figure} 
\end{minipage}
\hfill
%%%%%%%%%%%% Exercice 8+28%%%%
\begin{minipage}[t]{0.45\textwidth}
    \exo{calcul}{Calcul8} 

    Montrer que $(AB),~(DE)$ et $(GH)$ sont parallèles.
        \begin{figure}[H]
        \centering
        \begin{tikzpicture}[scale=1.5]
            \node (A) at (0,4) {}; %positions A
            \node (B) at (1,5) {}; %positions B
            \node (C) at (1,4) {}; %positions C
            \node (D) at (1,2) {}; %positions D
            \node (E) at (3,4) {}; %positions C'
            \node (F) at (2,2) {}; %positions C'
            \node (G) at (1,0) {}; %positions C'
            \node (H) at (3,2) {}; %positions C'
            \node [left=0.005cm of A] (A_label) {$A$}; %label A
            \node [right=0.01cm of B] (B_label) {$B$}; %label B
            \node [below left=0.01cm of C] (C_label) {$C$}; %label C
            \node [left=0.01cm of D] (D_label) {$D$}; %label D
            \node [right=0.01cm of E] (E_label) {$E$}; %label E
            \node [below=0.01cm of F] (F_label) {$F$}; %label F
            \node [below=0.01cm of G] (G_label) {$G$}; %label G
            \node [right=0.01cm of H] (H_label) {$H$}; %label H
            \draw (A.base) -- (B.base) node [midway,above] {2} ;
            \draw (A.base) -- (C.base) node [midway,below] {4} ;
            \draw (B.base) -- (C.base) node [midway,right] {3} ;
            \draw (C.base) -- (E.base) node [midway,above] {2} ;
            \draw (C.base) -- (D.base) node [midway,left] {1,5} ;
            \draw (E.base) -- (D.base) node [midway,above] {1} ;
            \draw (F.base) -- (E.base)  node [midway,right] {2} ;
            \draw (D.base) -- (F.base) node [midway,below] {1,5} ;
            \draw (F.base) -- (G.base) node [midway,left] {6} ;
            \draw (H.base) -- (F.base) node [midway,above] {4,5} ;
            \draw (G.base) -- (H.base) node [midway,right] {3} ;
        \end{tikzpicture}
    \end{figure} 
\end{minipage}
\vspace{-1em}
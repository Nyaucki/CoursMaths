\begin{multicols}{2}
    %%%%%%%%%%%% Exercie 1 %%%%%%%%
    \exo{}{} On a :
    \begin{itemize}
      \item $A$, $B$ et $C$ sont alignés
      \item $A$, $B'$ et $C'$ sont alignés
      \item $(BC)$ et $(B'C')$ sont parallèles
    \end{itemize}
    Ainsi, d'après le théorème de Thalès on a :
    \begin{align*}
      &\dfrac{AB}{AB'}=\dfrac{AC}{AC'}=\dfrac{BC}{B'C'}&&\\
      &\dfrac{2}{4}=\dfrac{3}{AC'}=\dfrac{BC}{B'C'}&&\text{Remplacer les valeurs}\\
      &\dfrac{2}{4}=\dfrac{3}{AC'}&&\text{Garder les fractions utiles}\\
      &4\times 3=AC'\times 2 &&\text{Egalité produits en croix}\\
      &12=AC'\times 2 &&\\
      &AC'=12:2&&\\
      &AC'=6&&
    \end{align*}
    %%%%%%%%%%%% Exercie 2 %%%%%%%
    \exo{}{} On a :
    \vspace{-1em} %Pour aligner. Pas compris
    \begin{itemize}
      \item $A$, $B$ et $C$ sont alignés
      \item $A$, $B'$ et $C'$ sont alignés
      \item $(BC)$ et $(B'C')$ sont parallèles
    \end{itemize}
    Ainsi, d'après le théorème de Thalès on a :
    \begin{align*}
      &\dfrac{AB}{AB'}=\dfrac{AC}{AC'}=\dfrac{BC}{B'C'}&&\\
      &\dfrac{5}{AB'}=\dfrac{AC}{AC'}=\dfrac{4}{6}&&\text{Remplacer les valeurs}\\
      &\dfrac{5}{AB'}=\dfrac{4}{6} &&\text{Garder les fractions utiles}\\
      &5\times 6=AB'\times 4&&\text{Egalité produits en croix}\\
      &AB'\times 4=30 &&\\
      &AB'=7,5 &&\\
      &BB'=2,5&&\text{Car }AB'=AB+BB'
    \end{align*}
  \end{multicols}
\hrule \vspace{-0.5em}%séparation
  \begin{multicols}{2}
    %%%%%%%%%%%% Exercie 3 %%%%%%%%
    \exo{}{} On a :
    \begin{itemize}
      \item $A$, $B$ et $C$ sont alignés
      \item $A$, $B'$ et $C'$ sont alignés
      \item $(BC)$ et $(B'C')$ sont parallèles
    \end{itemize}
    Ainsi, d'après le théorème de Thalès on a :
    \begin{align*}
      &\dfrac{AB}{AB'}=\dfrac{AC}{AC'}=\dfrac{BC}{B'C'}&&\\
      &\dfrac{2}{AB'}=\dfrac{AC}{AC'}=\dfrac{3}{9}&&\text{Remplacer les valeurs}\\
      &\dfrac{2}{AB'}=\dfrac{3}{9}&&\text{Garder les fractions utiles}\\
      &AB'\times 3=2\times 9 &&\text{Egalité produits en croix}\\
      &AB'\times 3=18&&\\
      &AB'=18:3&&\\
      &AB'=6&&
    \end{align*}
    %%%%%%%%%%%% Exercie 4 %%%%%%%
    \exo{}{} On a :
    \vspace{-1em} %Pour aligner. Pas compris
    \begin{itemize}
      \item $A$, $B$ et $C$ sont alignés
      \item $A$, $B'$ et $C'$ sont alignés
      \item $(BC)$ et $(B'C')$ sont parallèles
    \end{itemize}
    Ainsi, d'après le théorème de Thalès on a :
    \begin{align*}
      &\dfrac{AB}{AB'}=\dfrac{AC}{AC'}=\dfrac{BC}{B'C'}&&\\
      &\dfrac{AB}{AB'}=\dfrac{2}{2,5}=\dfrac{4}{BC'}&&\text{Remplacer les valeurs}\\
      &\dfrac{2}{2,5}=\dfrac{4}{BC'} &&\text{Garder les fractions utiles}\\
      &BC'\times 2=2,5\times 4&&\text{Egalité produits en croix}\\
      &BC'\times 2=10 &&\\
      &BC'=10:2 &&\\
      &BC'=5
    \end{align*}
  \end{multicols}
  \hrule \vspace{-0.5em}
  \begin{multicols}{2}
    %%%%%%%%%%%% Exercie 5 %%%%%%%%
    \exo{}{} On a :
    \begin{itemize}
      \item $A$, $B$ et $C$ sont alignés
      \item $A$, $B'$ et $C'$ sont alignés
      \item $(BC)$ et $(B'C')$ sont parallèles
    \end{itemize}
    Ainsi, d'après le théorème de Thalès on a :
    \begin{align*}
      &\dfrac{AB}{AB'}=\dfrac{AC}{AC'}=\dfrac{BC}{B'C'}&&\\
      &\dfrac{8}{AB'}=\dfrac{AC}{AC'}=\dfrac{6}{3}&&\text{Remplacer les valeurs}\\
      &\dfrac{8}{AB'}=\dfrac{6}{3}&&\text{Garder les fractions utiles}\\
      &AB'\times 6=8\times 3 &&\text{Egalité produits en croix}\\
      &AB'\times 6=24 &&\\
      &AB'=24:6&&\\
      &AC'=4&&
    \end{align*}
    %%%%%%%%%%%% Exercie 6 %%%%%%%
    \exo{}{} On a :
    \vspace{-1em} %Pour aligner. Pas compris
    \begin{itemize}
      \item $A$, $B$ et $C$ sont alignés
      \item $A$, $B'$ et $C'$ sont alignés
      \item $(BC)$ et $(B'C')$ sont parallèles
    \end{itemize}
    Ainsi, d'après le théorème de Thalès on a :
    \begin{align*}
      &\dfrac{AB}{AB'}=\dfrac{AC}{AC'}=\dfrac{BC}{B'C'}&&\\
      &\dfrac{1}{3}=\dfrac{AC}{AC'}=\dfrac{BC}{4}&&\text{Remplacer les valeurs}\\
      &\dfrac{1}{3}=\dfrac{BC}{4}&&\text{Garder les fractions utiles}\\
      &BC\times 3=1\times 4&&\text{Egalité produits en croix}\\
      &BC\times 3=4 &&\\
      &BC=4:3 &&\\
      &BC\approx 1,33&&
    \end{align*}
  \end{multicols}
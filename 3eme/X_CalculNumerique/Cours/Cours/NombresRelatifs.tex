\subsection{Nombres relatifs}

\prop{Additions et soustraction}
{\begin{minipage}{0.45\textwidth}
    \textbf{Même signe}
    \begin{itemize}
    \item On prend le signe des deux nombres
    \item On additionne les deux nombres
    \end{itemize}
\end{minipage}
\hfill
\begin{minipage}{0.45\textwidth}
    \textbf{Signes opposés}
    \begin{itemize}
    \item On prend le signe du plus grand
    \item On soustrait les deux nombres
    \end{itemize}    
\end{minipage}  }

\exmpl{
    \begin{itemize}
        \item $4-7=-3$ On a prit le signe du plus grand (\textbf{-}7) et calculé 7-4
        \item $-8-15=-23$ On a prit le signe des nombres (\textbf{-}) et calculé 8+15
    \end{itemize}  
}

\prop{Multiplications et divisions}
{\textbf{Même signe}
\begin{itemize}
    \item Le produit ou la division de deux nombres positifs est positif
    \item Le produit ou la division de deux nombres négatifs est positif
\end{itemize}
\vspace{1em}
\textbf{Signes opposés}
\begin{itemize}
    \item Le produit ou la division de deux nombres de signes opposés est négatif
\end{itemize}
}

\exmpl{
    \begin{itemize}
        \item $3\times 4=12$ 
        \item $(-3)\times (-4)=12$
        \item $(-3)\times 4=-12$
        \item $3\times (-4)=-12$
    \end{itemize}  
}

\subsection{Priorités}

\prop{Priorités des opérations}
{On fait les opérations dans l'ordre suivant :
\begin{enumerate}
    \item Les parenthèses
    \item Les exposasnts (les puissances)
    \item Les multiplications et les divisions
    \item Les additions et les soustractions
\end{enumerate}}

\rmq {On pourra retenir PEMDAS (Parenthèses Exposants Multiplications Divisions Additions Soustractions)}

\exmpl{
    \begin{align*}
        4-3\times(-2+3\times 4)² & = 4-(-2+12)² & &\text{On commence dans la parenthèses, par la multiplication}\\
        & = 4-3\times(10)² & &\text{On calcule la parenthèses}\\
        & = 4-3\times 100 & &\text{On calcule la puissance}\\
        & = 4-300 & &\text{On calcule la multiplication}\\
        & = -296 & &\text{On termine avec la soustraction}
    \end{align*}
}
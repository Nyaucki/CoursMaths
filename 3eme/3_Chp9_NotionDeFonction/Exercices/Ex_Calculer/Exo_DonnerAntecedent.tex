\consigne{CalcAntecedent1}{CalcAntecedent12} (Calculer) Pour ces exercices, on considère les fonctions $f:x\mapsto 2x+4$, $g:x\mapsto 4x^2-2$ et $h:x\mapsto \dfrac{1}{x}$

\begin{multicols}{3}
\exo{}{CalcAntecedent1} 


$g(\fillin[1cm])=38$

\exo{}{CalcAntecedent2} 


$f(\fillin[1cm])=-14$

\exo{}{CalcAntecedent3} 


Les antécédents de 3 par la fonction $g$ sont \fillin[1cm]

\exo{}{CalcAntecedent4} 


Les antécédents de 6 par la fonction $h$ sont \fillin[1cm]

\exo{}{CalcAntecedent5} 


$h(\fillin[1cm])=7$

\exo{}{CalcAntecedent6} 


$f(\fillin[1cm])=-1$

\exo{}{CalcAntecedent7} 


Les antécédents de 9 par la fonction $h$ sont \fillin[1cm]

\exo{}{CalcAntecedent8} 


Les antécédents de -13 par la fonction $f$ sont \fillin[1cm]\columnbreak

\exo{}{CalcAntecedent9} 


$f(\fillin[1cm])=12$

\exo{}{CalcAntecedent10} 


$f(\fillin[1cm])=11$

\exo{}{CalcAntecedent11} 


Les antécédents de 10 par la fonction $h$ sont \fillin[1cm]

\exo{}{CalcAntecedent12} 


Les antécédents de 1 par la fonction $h$ sont \fillin[1cm]\columnbreak

\end{multicols}
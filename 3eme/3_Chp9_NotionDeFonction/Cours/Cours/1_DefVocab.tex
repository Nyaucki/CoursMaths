\dfnt{fonction}{
    Une \textbf{fonction} est un objet mathématique qui associe à un nombre un autre nombre. On y associe le vocabulaire suivant :
    \begin{itemize}
        \item Son \textbf{Ensemble de définition} est l'ensemble des nombres avec lesquels on peut utiliser la fonction.
        \item Sa \textbf{forme algébrique} est la manière mathématique de la décrire (avec $f(x)=$ ou $f:x\mapsto$)
    \end{itemize}
}

\exmpl{$f(x)=4x+2$ (ou $f:x\mapsto 4x+2$) est la fonction qui a un nombre $x$ associe 4 fois ce nombre puis y ajoute 2.
Ainsi, $f(5)=4\times 5 +2=22$}


\dfnt{Image et antécédent}
{
    Soit une fonction $f$ et deux nombres $a$ et $b$ tels que $f(a)=b$. On dit alors que :
    \begin{itemize}
        \item $a$ est \textbf{un antécédent} de $b$ par la fonction $f$.
        \item $b$ est \textbf{l'image} de $a$ par la fonction $f$.
    \end{itemize}
}

\rmq{Un nombre ne peut avoir qu'une image, mais il peut avoir plusieurs antécédents (ou aucun).}

\exmpl{$f(5)=22$ signifie que 22 est l'image de 5 par $f$ et que 5 est un antécédent de 22 par $f$.}

\rmq{Pour calculer un antécédent, il faut résoudre une équation. Pour une image, il faut calculer.}
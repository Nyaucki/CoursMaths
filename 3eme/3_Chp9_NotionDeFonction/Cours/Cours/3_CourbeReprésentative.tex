\dfnt{Courbe représentative d'une fonction}
{
    \textbf{La courbe représentative d'une fonction} est la courbe montrant toutes les images d'une fonction pour un intervalle donné.
    \begin{multicols}{2}
        \begin{itemize}
            \item Les antécédents sur l'axe des abscisses.
            \item Les images sont sur l'axe des ordonnés.
        \end{itemize}
    \end{multicols}
}

\exmpl{
    La courbe représentative de la fonction $f:x\mapsto \dfrac{2}{5}x^3+\dfrac{4}{5}x^2-x-3$ sur l'intervalle [-2;2] est :
    \begin{figure}[H]
        \center
        \begin{tikzpicture}[xscale=4]
            \draw [thick,<->] (2.2,0) node [below] {$x$}-|(0,2.4) node[left] {$f(x)$};
            \draw [thick] (-2.2,0)-|(0,-4.4);
            \draw[domain=-2:2]  plot (\x,{0.4*\x*\x*\x+0.8*\x*\x-\x-3}) node [left] {$f$};
            \foreach \y in {2,...,1}
            {
                \draw (\y,-0.2) node[below] {\y}--(\y,0.2);
                \draw (-\y,-0.2) node[below] {-\y}--(-\y,0.2);
            }
            \foreach \y in {2,...,1}
            {
                \draw (-0.05,\y) node[left] {\y}--(0.05,\y);
            }
            \foreach \y in {4,...,1}
            {
                \draw (-0.05,-\y) node[left] {-\y}--(0.05,-\y);
            }
            \node[below left] at (0,0) {0};
        \end{tikzpicture}
    \end{figure}
}


\begin{minipage}{0.48\textwidth}
    \subsection*{Trouver graphiquement une image}

    On considère toujours la courbe représentative de la fonction $f:x\mapsto \dfrac{2}{5}x^3+\dfrac{4}{5}x^2-x-3$ et on cherche l'image de 1.

    \begin{itemize}
        \item \textcolor{teal}{On trace la droite verticale partant de l'abscisse souhaitée (ici 1) jusqu'à la courbe.}
        \item \textcolor{red}{On cherche l'intersection avec la courbe.}
        \item \textcolor{blue}{On trace le segment depuis ce point jusqu'à l'axe des ordonnés.}
        \item \textcolor{magenta}{On lit l'image sur l'axe des ordonnés}
    \end{itemize}
    \begin{figure}[H]
        \center
        \begin{tikzpicture}[xscale=2]
            \draw [thick,<->] (2.2,0) node [below] {$x$}-|(0,2.4) node[left] {$f(x)$};
            \draw [thick] (-2.2,0)-|(0,-4.4);
            \draw[domain=-2:2,name path=curb]  plot (\x,{0.4*\x*\x*\x+0.8*\x*\x-\x-3}) node [left] {$f(x)$};
            \foreach \y in {2,...,1}
            {
                \draw (\y,-0.2) node[below] {\y}--(\y,0.2);
                \draw (-\y,-0.2) node[below] {-\y}--(-\y,0.2);
            }
            \foreach \y in {2,...,1}
            {
                \draw (-0.05,\y) node[left] {\y}--(0.05,\y);
            }
            \foreach \y in {4,...,1}
            {
                \draw (-0.05,-\y) node[left] {-\y}--(0.05,-\y);
            }
            \node[below left] at (0,0) {0};
            \node (abs-) at (-2,0) {};
            \node (abs+) at (2,0) {};
            \node (ord-) at (0,-4) {};
            \node (ord+) at (0,4) {};
            \node [draw,red,cross] (X) at (1,0.4*1*1*1+0.8*1*1-1-3) {};
            \draw[dashed,teal] ($(abs+)!(X)!(abs-)$) -- (X);
            \draw[dashed,blue] ($(ord-)!(X)!(ord+)$) -- (X);
            \node [magenta,above right] at ($(ord-)!(X)!(ord+)$) {-2,8};
        \end{tikzpicture}
    \end{figure}
\end{minipage}
\hfil
\vrule
\hfil
\begin{minipage}{0.48\textwidth}
    \subsection*{Trouver graphiquement un antécédent}

    On considère toujours la courbe représentative de la fonction $f:x\mapsto \dfrac{2}{5}x^3+\dfrac{4}{5}x^2-x-3$ et on cherche les antécédents de -2.

    \begin{itemize}
        \item \textcolor{teal}{On trace la droite horizontale partant de l'ordonnée souhaitée (ici -2).}
        \item \textcolor{red}{On cherche les intersections avec la courbe.}
        \item \textcolor{blue}{On trace le segment depuis ces points jusqu'à l'axe des abscisses.}
        \item \textcolor{magenta}{On lit l(es) image(s) sur l'axe des ordonnés}
    \end{itemize}
    \begin{figure}[H]
        \center
        \begin{tikzpicture}[xscale=2]
            \draw [thick,<->] (2.2,0) node [below] {$x$}-|(0,2.4) node[left] {$f(x)$};
            \draw [thick] (-2.2,0)-|(0,-4.4);
            \draw[domain=-2:2,name path=curb]  plot (\x,{0.4*\x*\x*\x+0.8*\x*\x-\x-3}) node [left] {$f(x)$};
            \foreach \y in {2,...,1}
            {
                \draw (\y,-0.2) node[below] {\y}--(\y,0.2);
                \draw (-\y,-0.2) node[below] {-\y}--(-\y,0.2);
            }
            \foreach \y in {2,...,1}
            {
                \draw (-0.05,\y) node[left] {\y}--(0.05,\y);
            }
            \foreach \y in {4,...,1}
            {
                \draw (-0.05,-\y) node[left] {-\y}--(0.05,-\y);
            }
            \node[below left] at (0,0) {0};
            \node (abs-) at (-2,0) {};
            \node (abs+) at (2,0) {};
            \node (ord-) at (0,-4) {};
            \node (ord+) at (0,4) {};
            \draw[dashed,teal,name path=hotline] (-2,-2) -- (2,-2);
            \fill[red,name intersections={of=hotline and curb,total=\t}];
            \node[draw,red,cross] (X1) at (intersection-1) {};
            \node[draw,red,cross] (X2) at (intersection-2) {}; 
            \draw[dashed,blue] ($(abs-)!(X1)!(abs+)$) -- (X1);
            \draw[dashed,blue] ($(abs-)!(X2)!(abs+)$) -- (X2);
            \node [magenta,above] at ($(abs-)!(X1)!(abs+)$) {-0,73};
            \node [magenta,above] at ($(abs-)!(X2)!(abs+)$) {1,32};
        \end{tikzpicture}
    \end{figure}
\end{minipage}

\rmq{Résoudre graphiquement une équation revient à résoudre en utilisant une courbe comme ci-dessus. La solution est généralement une approximation.}
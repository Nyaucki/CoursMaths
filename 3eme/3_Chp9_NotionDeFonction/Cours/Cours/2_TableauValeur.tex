\dfnt{tableau de valeurs}
{
    Pour se faire une idée du comportement d'une fonction, on peut faire un tableau de valeurs, donnant directement les images de plusieurs nombres. Il est de la forme :\hspace{2em}
    \begin{center}
        \begin{tabular}{c*{3}{|c}}
            $x$ & antécédent 1 & antécédent 2 & \dots \\ \hline
            $f(x)$ &image 1 & image 2 & \dots
        \end{tabular}
    \end{center}
}

\exmpl{
    Voici un tableau de valeurs de la fonction $f:x\mapsto 4x-3$ : \hfil
    \begin{tabular}{c*{5}{|c}}
        $x$ & 0 & 1 & 2 & 3 & 4 \\ \hline
        $f(x)$ &-3&1& 5 & 9 & 13
    \end{tabular}
}

\exmpl{
    Les fonctions $f:x\mapsto 0$, $g:x\mapsto x^2 -4$ et $h:x\mapsto (x-2)(x+2)(4x-7)$ correspondent toutes les trois au tableau :\hspace{2em}
    \begin{tabular}{c*{2}{|c}}
        $x$ & -2 & 2 \\ \hline
        $f(x)$ &0&0
    \end{tabular}
}

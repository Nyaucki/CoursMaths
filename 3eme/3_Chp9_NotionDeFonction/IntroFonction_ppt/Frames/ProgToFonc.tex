\begin{frame}
    \begin{minipage}[t]{0.5\textwidth}
        \underbar{On considère le programme $A$ :}
        \begin{itemize}
            \item Prendre un nombre 
            \item Multiplier par 4
            \item Enlever 3
        \end{itemize}
    \end{minipage}
    \hfil
    \vrule
    \hfil
    \begin{minipage}[t]{0.45\textwidth}
        \underbar{Qu'obtient-on avec $x$ :}
        
    \end{minipage}
\end{frame}

\begin{frame}
    \begin{minipage}[t]{0.5\textwidth}
        \underbar{On considère le programme $A$ :}
        \begin{itemize}
            \item Prendre un nombre 
            \item Multiplier par 4
            \item Enlever 3
        \end{itemize}
    \end{minipage}
    \hfil
    \vrule
    \hfil
    \begin{minipage}[t]{0.45\textwidth}
        \underbar{Qu'obtient-on avec  $x$ :}
        $$4x-3$$
        \vfil
        \underbar{On simplifie le programme ainsi :}
        $$A:x\mapsto 4x-3$$ 

        On appelle cela une fonction.
    \end{minipage}
\end{frame}

\begin{frame}
    \begin{minipage}[t]{0.5\textwidth}
        \underbar{On considère le programme $B$ :}
        \begin{itemize}
            \item Prendre un nombre 
            \item Ajouter 3
            \item Multiplier par 2
            \item Mettre au carré
        \end{itemize}
    \end{minipage}
    \hfil
    \vrule
    \hfil
    \begin{minipage}[t]{0.45\textwidth}
        \underbar{Écrire la fonction associée :}
    \end{minipage}
\end{frame}

\begin{frame}
    \begin{minipage}[t]{0.5\textwidth}
        \underbar{On considère le programme $B$ :}
        \begin{itemize}
            \item Prendre un nombre 
            \item Ajouter 3
            \item Multiplier par 2
            \item Mettre au carré
        \end{itemize}
    \end{minipage}
    \hfil
    \vrule
    \hfil
    \begin{minipage}[t]{0.45\textwidth}
        \underbar{Écrire la fonction associée :}
        $$B:x\mapsto \left((x+3)\times 2\right)^2$$ Qu'on peut simplifier en 
        $$B:x\mapsto (2x+6)^2$$
    \end{minipage}
\end{frame}

\begin{frame}
    Écrire les fonctions associées aux programmes suivants :
    \vspace*{1em}

    \begin{minipage}[t]{0.45\textwidth}
        \underbar{Programme $C$ :}
        \begin{itemize}
            \item Prendre un nombre 
            \item Mettre au carré
            \item Multiplier par -4
            \item Ajouter 3 fois le nombre de départs
            \item Soustraire 7
        \end{itemize}
    \end{minipage}
    \hfil
    \vrule
    \hfil
    \begin{minipage}[t]{0.45\textwidth}
        \underbar{Programme $D$ :}
        \begin{itemize}
            \item Prendre un nombre 
            \item Soustraire 7
            \item Multiplier par -4
            \item Mettre au carré
            \item Ajouter 3 fois le nombre de départs
        \end{itemize}
    \end{minipage}
\end{frame}

\begin{frame}
    Écrire les fonctions associées aux programmes suivants :
    \vspace*{1em}

    \begin{minipage}[t]{0.45\textwidth}
        \underbar{Programme $C$ :}
        \begin{itemize}
            \item Prendre un nombre 
            \item Mettre au carré
            \item Multiplier par -4
            \item Ajouter 3 fois le nombre de départs
            \item Soustraire 7
        \end{itemize}
        \begin{center}
            $$C:x\mapsto x^2-4+3x-7$$ ou $$C:x\mapsto x^2+3x-11$$
        \end{center}
    \end{minipage}
    \hfil
    \vrule
    \hfil
    \begin{minipage}[t]{0.45\textwidth}
        \underbar{Programme $D$ :}
        \begin{itemize}
            \item Prendre un nombre 
            \item Soustraire 7
            \item Multiplier par -4
            \item Mettre au carré
            \item Ajouter 3 fois le nombre de départs
        \end{itemize}
        \begin{center}
            $$D:x\mapsto ((x-7)\times (-4))+3x$$ ou
            $$D:x\mapsto (-4x+28)^2+3x$$
        \end{center}
    \end{minipage}
\end{frame}
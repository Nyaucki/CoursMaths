\begin{frame}
    \underbar{On considère la fonction $K:x\mapsto (x-3)(2x+4)$}\vspace*{1em}

    Trouver le nombre $x$ tel que $K(x)=0$. 
\end{frame}

\begin{frame}
    \underbar{On considère la fonction $K:x\mapsto (x-3)(2x+4)$}\vspace*{1em}

    Trouver le nombre $x$ tel que $K(x)=0$ \vspace*{1em}

    On dit que 3 et -2 sont des antécédents de 0 par la fonction $K$.
\end{frame}

\begin{frame}
    \underbar{On considère la fonction $L:x\mapsto 2x+5$}\vspace*{1em}

    \begin{enumerate}
        \item Trouver le(s) antécédent(s) de 11 par $L$.
        \item Trouver le(s) antécédent(s) de 2 par $L$.
    \end{enumerate}
\end{frame}
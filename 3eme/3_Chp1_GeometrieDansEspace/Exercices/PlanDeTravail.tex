\section*{\titre - Plan de Travail}

\pdt[5]{Faire une transformation avec un quadrillage}
{\begin{multicols}{4}
\begin{itemize}
    \itemindent=-25pt
        \item \exref{TracerSymAxQuadri1}
        \item \exref{TracerSymAxQuadri2}
        \item \exref{TracerSymAxQuadri3}
        \item \exref{TracerSymAxQuadri4}
        \item \exref{TracerSymCentQuadri1}
        \item \exref{TracerSymCentQuadri2}
        \item \exref{TracerSymCentQuadri3}
        \item \exref{TracerSymCentQuadri4}
        \item \exref{TracerTransQuadri1}
        \item \exref{TracerTransQuadri2}
        \item \exref{TracerTransQuadri3}
        \item \exref{TracerTransQuadri4}
        \item \exref{TracerRotaQuadri1}
        \item \exref{TracerRotaQuadri2}
        \item \exref{TracerRotaQuadri3}
        \item \exref{TracerRotaQuadri4}
        \item \exref{TracerHomotQuadri1}
        \item \exref{TracerHomotQuadri2}
        \item \exref{TracerHomotQuadri3}
        \item \exref{TracerHomotQuadri4}
    \end{itemize}
\end{multicols}}

\begin{plandetravailDS}
    Programme de l'interro (2 questions seront les mêmes que des exercices ):
    \begin{itemize}
        \item Faire une transformation avec Quadrillage (4 points)
        \item Faire une transformation sans Quadrillage (2 points)
        \item Retrouver les éléments caractéristiques d'une transformation (4 points)
    \end{itemize}
\end{plandetravailDS}

\pdt[4]{Faire une transformation sans un quadrillage}
{\begin{multicols}{4}
\begin{itemize}
    \itemindent=-25pt
        \item \exref{TracerSymAxBlanc1}
        \item \exref{TracerSymAxBlanc2}
        \item \exref{TracerSymAxBlanc3}
        \item \exref{TracerSymAxBlanc4}
        \item \exref{TracerSymCentBlanc1}
        \item \exref{TracerSymCentBlanc2}
        \item \exref{TracerSymCentBlanc3}
        \item \exref{TracerSymCentBlanc4}
        \item \exref{TracerTransBlanc1}
        \item \exref{TracerTransBlanc2}
        \item \exref{TracerTransBlanc3}
        \item \exref{TracerTransBlanc4}
        \item \exref{TracerRotaBlanc1}
        \item \exref{TracerRotaBlanc2}
        \item \exref{TracerRotaBlanc3}
        \item \exref{TracerRotaBlanc4}
        \item \exref{TracerHomotBlanc1}
        \item \exref{TracerHomotBlanc2}
        \item \exref{TracerHomotBlanc3}
        \item \exref{TracerHomotBlanc4}
    \end{itemize}
\end{multicols}}

\pdt[6]{Retrouver les éléments caractéristiques}{
    \begin{multicols}{4}
        \begin{itemize}
            \itemindent=-25pt
                \item \exref{CentreSymAx1}
                \item \exref{CentreSymAx2}
                \item \exref{CentreSymAx3}
                \item \exref{CentreSymAx4}
                \item \exref{CentreSymCent1}
                \item \exref{CentreSymCent2}
                \item \exref{CentreSymCent3}
                \item \exref{CentreSymCent4}
                \item \exref{CentreRota1}
                \item \exref{CentreRota2}
                \item \exref{CentreRota3}
                \item \exref{CentreRota4}
                \item \exref{CentreHomot1}
                \item \exref{CentreHomot2}
                \item \exref{CentreHomot3}
                \item \exref{CentreHomot4}
            \end{itemize}
    \end{multicols}
}


\vspace{-1em}

\begin{minipage}[t]{0.5\textwidth}
    \vspace{-0.25em}
    \pdt[2]{Exercices Plus difficiles}{
        \begin{multicols}{2}
            \begin{itemize}
                \itemindent=-25pt
                \item \exref{Exceptions1}
                \item \exref{Exceptions2}
                \item \exref{Exceptions3}
                \item \exref{Dur1}%Avec exceptions
            \end{itemize}
        \end{multicols}
    }
\end{minipage}  
\hfill
\begin{minipage}[t]{0.47\textwidth}
    \vspace{-0.25em}
    \textbf{Que mettre dans les cases ?}
    \begin{itemize}
        \item \textbf{TB} \textit{(Très bien)} Si tout est juste
        \item \textbf{B} \textit{(Bien)} J'ai le bon résultat, mais pas la bonne rédaction
        \item \textbf{AB} \textit{(Assez bien)} J'ai une faute, mais je peux comprendre avec la correction
        \item  \textbf{AA} \textit{(Avec de l'Aide)} Si j'ai eu besoin d'aide pour réussir l'exercice 
        \item \textbf{A} \textit{(Au secours!)} J'ai besoin que quelqu'un m'explique.
    \end{itemize}
\end{minipage}
\exo{4}{Calculer} : Développer les expressions suivantes :

\begin{multicols}{4}
    \noindent$$3(2x+7)$$
    \vspace*{10em}\columnbreak


    \noindent$$-7(-x+2)$$
    \vspace*{10em}\columnbreak

    \noindent$$x(2x-1)$$
    \vspace*{10em}\columnbreak

    \noindent$$-4x(3x-2)$$
    \vspace*{10em}\columnbreak
\end{multicols}

\exo{4}{Calculer} : Développer les expressions suivantes :

\begin{multicols}{2}
    \noindent$$(x+3)(2x+7)$$
    \vspace*{10em}\columnbreak

    \noindent$$(2x-7)(-x+2)$$
    \vspace*{10em}\columnbreak
\end{multicols}

\exo{4}{Raisonner} : Factoriser les expressions suivantes :*

\begin{multicols}{4}
    \noindent$$3x+6$$
    \vspace*{5em}\columnbreak

    \noindent$$10x+15$$
    \vspace*{5em}\columnbreak

    \noindent$$12x-8$$
    \vspace*{5em}\columnbreak

    \noindent$$-4x(3x-2)+2(3x-2)$$
    \vspace*{5em}\columnbreak
\end{multicols}

\newpage

\exo{4}{Modéliser} : Programme de calcul

\begin{minipage}[t]{0.5\textwidth}
    On considère le programme de calcul ci-contre :

    \begin{enumerate}
        \item Vérifier qu'avec pour nombre de départs 3, on obtient 26.\vspace{2.5cm}
        \item Quel nombre obtient-on avec pour nombre de départs $-2$.\vspace{2.5cm}
        \item Quelle expression obtient-on avec pour nombre de départ $x$.\vspace{2.5cm}
        \item Peut-on simplifier le programme de calcul ?\vspace{2.5cm}
    \end{enumerate}
\end{minipage}
\hfil
\begin{minipage}[t]{0.45\textwidth}
    \begin{itemize}
        \item Choisir un nombre
        \item Le multiplier par 2
        \item Ajouter 3
        \item Multiplier le tout par 4
        \item Soustraire 6 fois le nombre de départs
        \item Ajouter 8
    \end{itemize}
\end{minipage}




\exo{4}{Représenter} : Compléter avec = ou $\neq$. Justifier.



\begin{multicols}{2}
    \noindent$$(2x+1)(2x-7)+2\filling[1cm]4x^2-12x-5$$
    \vspace*{10em}\columnbreak

    \noindent$$(2x-6)(x+3)\filling[1cm](x-3)(2x+6)$$
    \vspace*{10em}\columnbreak
\end{multicols}
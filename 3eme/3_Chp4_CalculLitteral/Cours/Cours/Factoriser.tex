\section{Factoriser}

\subsection{Facteur commun}

\prop{Factoriser}{
Pour tous réels $k,a \text{ et } b$, on a :
\begin{align*}
    k\times a + k \times b=k\times(a+b)\\
    k\times a - k \times b= k\times(a-b)
\end{align*}
On dit alors que $k$ est un \textbf{facteur commun}.
}

\exmpl
{\begin{itemize}
    \item $8x+16=8(x+2)$
    \item $x^2-3x=x(x-3)$
    \item $12-9x=3(4-3x)$
\end{itemize}
}

\rmq{ Factoriser, c'est transformer une somme en un produit. }

\rmq{ Factorisation et développement sont deux opérations contraires. Faire l'une puis l'autre \textbf{doit} ramener au point de départ.

\begin{figure}[H]
    \centering
    \begin{tikzpicture}[node distance = 6cm, thick]% 
        \node (1) {$2(4x-7)$};
        \node (2) [right=of 1] {$8x-14$};
        \draw[->] (1) to [bend right=15] node [midway, below] {développer} (2);
        \draw[->] (2) to [bend right=15] node [midway, above] {factoriser} (1);
    \end{tikzpicture}
\end{figure}}


\subsection{Factoriser à l'aide des identités remarquables}


\prop{Identité remarquable 1}{
    Pour tous réels $a \text{ et } b$, on a :
    \begin{align*}
        (a+b)^2=a^2+2ab+b^2\\%au tableau, mettre des couleur pour expliquer
        (a-b)^2=a^2-2ab+b^2
    \end{align*}
}

\prop{Identité remarquable 2}{
    Pour tous réels $a \text{ et } b$, on a :
    \begin{equation*}
        (a+b)(a-b)=a^2-b^2%au tableau, mettre des couleur pour expliquer
    \end{equation*}
}

La factorisation à l'aide des identités remarquable se fait selon l'algorithme suivant :

\begin{figure}[H]
    \centering
    \begin{tikzpicture}[node distance = 3cm, thick, text centered]% 
        \node (1) { Combien de terme ?};
        \node (2) [right=2cm of 1] {};
        \node (3) [left=2cm of 1] {};
        \node (4) [below=of 2,text width=4cm] {\textbf{Identité 2} \\ Signes différents ?};
        \node (5) [below=of 3,text width=4cm] {\textbf{Identité 1} \\ Signes différents ?};
        \node (6) [below=3.5cm of 4] {};
        \node (7) [left=1.3cm of 6] {Impossible};
        \node (8) [right=1.3cm of 6,text width=2.5cm] {$a^2-b^2=$ \\ $(a+b)(a-b)$};
        \node (9) [below=3.5cm of 5] {};
        \node (10) [left=1.3cm of 9,text width=2.5cm] {$a^2+2ab+b^2$\\$=(a+b)^2$};
        \node (11) [right=1.3cm of 9,text width=2.5cm] {$a^2-2ab+b^2$\\$=(a-b)^2$};
        \draw[->] (1) to [] node [midway, above, sloped] {2} (4);
        \draw[->] (1) to [] node [midway, below, sloped] {ex : $x^2+6x+9$} (5);
        \draw[->] (1) to [] node [midway, above, sloped] {3} (5);
        \draw[->] (1) to [] node [midway, below, sloped] {ex : $x^2-16$} (4);
        \draw[->] (4) to [] node [midway, below, sloped] {ex : $x^2+16$} (7);
        \draw[->] (4) to [] node [midway, above, sloped] {non} (7);
        \draw[->] (4) to [] node [midway, below, sloped] {ex : $x^2-16$} (8);
        \draw[->] (4) to [] node [midway, above, sloped] {oui} (8);
        \draw[->] (5) to [] node [midway, below, sloped] {ex : $x^2+6x+9$} (10);
        \draw[->] (5) to [] node [midway, above, sloped] {non} (10);
        \draw[->] (5) to [] node [midway, below, sloped] {ex : $x^2-6x+9$} (11);
        \draw[->] (5) to [] node [midway, above, sloped] {oui} (11);
    \end{tikzpicture}
\end{figure}

\exmpl{ 
\begin{itemize}
    \item $4x^2 - 12x+9$.
    \item Il y a trois termes. Nous avons donc à faire à l'identité 1.
    \item Un terme est négatif alors que les autres sont positifs. Il y a donc des signes différents.
    \item Dans ce cas, on utilisera la formule $a^2-2ab+b^2=(a-b)^2$.
    \item $4x^2$ est le carré de $2x$. 9 est le carré de $3$. Et on a bien $2\times 2x\times 3=12x$.
    \item La factorisation est donc $(2x-3)^2$.
\end{itemize}
}

\exmpl{
\begin{itemize}
    \item $-x^2-2x-1$
    \item Il y a trois termes. Nous avons donc à faire à l'identité 1.
    \item Tous les termes sont négatifs. Il n'y a donc des signes différents.
    \item Dans ce cas, on utilisera la formule $a^2+2ab+b^2=(a+b)^2$.
    \item Dans cette formule, tous les signes sont positifs. Pour s'en rapprocher, on peut factoriser l'expression par $(-1)$, qui est un facteur commun.
    \item On a alors : $-(x^2+2x+1)$
    \item $x^2$ est le carré de $x$. 1 est le carré de $1$. Et on a bien $2\times x\times 1=2x$.
    \item La factorisation est donc $-(x+1)^2$.
\end{itemize}
}

\exmpl{ 
\begin{itemize}
    \item $9x^2-2$
    \item Il y a deux termes. Nous avons donc à faire à l'identité 2.
    \item Un terme est négatif alors que l'autre est positif. Il y a donc des signes différents.
    \item Dans ce cas, on utilisera la formule $a^2-b^2=(a-b)(a+b)$.
    \item $9^2$ est le carré de $3x$. 2 est le carré de $\sqrt{2}$.
    \item La factorisation est donc $(3x-\sqrt{2})^2$.
\end{itemize}    
}

\rmq{ On n'oubliera pas qu'il est aussi possible de factoriser à l'aide d'un facteur commun, comme dit à la propriété \ref{factoriser}.}
    



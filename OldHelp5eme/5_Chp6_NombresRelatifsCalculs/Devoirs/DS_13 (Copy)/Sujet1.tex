\exo{4}{Calculer} : Calculer en détaillant les étapes

\begin{multicols}{2}
    $$(+12)+(-11)+(-6)+(+1)$$
    \vspace*{16em}

    $$(-10)-(-5)+(-8)-(+5)$$
    \vspace*{16em}\columnbreak

    $$(-1)-(-1)+(-6)+(-1)-(+8)-(-5)$$
    \vspace*{16em}

    $$(+6)-(+3)-(-8)-(+4)-(-3)-(+6)$$
    \vspace*{16em}
\end{multicols}

\newpage



\exo{2}{Raisonner} : Calculer en détaillant les étapes :

\begin{multicols}{2}
    $$-5+7+5-3-1-4$$
    \vspace*{16em}\columnbreak

    $$+12-7+5+6-1+6+4-2+3$$
    \vspace*{16em}

\end{multicols}

\exo{4}{Modéliser} : Résoudre le problème suivant :

Crabe et Goyle jouent à un jeu de société. Crabe a 12 points et Goyle en a 27.
Crabe lui échange un jeton valant -5 points contre un jeton en valant +3. 
Qui a le plus de point désormais ?


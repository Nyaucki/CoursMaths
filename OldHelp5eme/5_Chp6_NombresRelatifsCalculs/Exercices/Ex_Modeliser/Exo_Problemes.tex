\consigne{Pb1}{Pb4} Fred et George s'affrontent à un jeu de société. Chacun possède des jetons pouvant aller de -10 à +10 points. Les joueurs peuvent échanger des jetons, mais ceux échangés doivent être choisis au hasard.

\vspace*{-1em}

\begin{minipage}[t]{0.45\textwidth}
    \exo{Modeliser}{Pb1}
    
    Fred a 22 points, et George en a 37. Au tour de Fred, celui ci échange un de ses jetons valant -5 points contre un jeton de George valant +3 points. Qui a le plus de points maintenant ? 
\end{minipage}
\hfil
\vrule
\hfil
\begin{minipage}[t]{0.45\textwidth}
    \exo{Modeliser}{Pb2}
    
    Fred a 42 points, et George en a 37. Au tour de Fred, celui ci échange un de ses jetons valant 7 points contre un jeton de George valant +3 points. Qui a le plus de points maintenant ? 
\end{minipage}

\begin{minipage}[t]{0.45\textwidth}
    \exo{Modeliser}{Pb3}
    
    Fred a 27 points, et George en a 34. Au tour de Fred, celui ci échange un de ses jetons valant -7 points contre un jeton de George valant -3 points. Qui a le plus de points maintenant ? 
\end{minipage}
\hfil
\vrule
\hfil
\begin{minipage}[t]{0.45\textwidth}
    \exo{Modeliser}{Pb4}
    
    Fred a -5 points, et George en a 4. Au tour de Fred, celui ci échange un de ses jetons valant -2 points contre un jeton de George valant +3 points. Qui a le plus de points maintenant ? 
\end{minipage}
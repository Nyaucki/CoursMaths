\section*{Les carrés magiques}

\begin{minipage}[t]{0.60\textwidth}
    Pour remplir un carré magique, il faut compléter chaque case de sorte à ce que la somme de chaque ligne, de chaque colone et de chaque diagonale soit égale au même nombre (15 dans l'exemple ci-contre.)
\end{minipage}
\hfill
\begin{minipage}[t]{0.30\textwidth}
    \vspace{-4em}
    \begin{figure}[H]
        \centering
        \includegraphics[width=\textwidth]{carre33.png}
    \end{figure}
\end{minipage}

\vspace{-2em}
\textbf{Remplir les carrés magiques ci-dessous :}

\begin{minipage}[t]{0.23\textwidth}
    \begin{tabularx}{\textwidth}{|Y|Y|Y|}
        \hline
        8 &   &   \\\hline
          & 5 &   \\\hline
        4 &   & 2 \\\hline
    \end{tabularx}
\end{minipage}
\hfill
\begin{minipage}[t]{0.23\textwidth}
    \begin{tabularx}{\textwidth}{|Y|Y|Y|}
        \hline
        18 &   &  24 \\\hline
          & 15 &   \\\hline
         &   & 12 \\\hline
    \end{tabularx}
\end{minipage}
\hfill
\begin{minipage}[t]{0.23\textwidth}
    \begin{tabularx}{\textwidth}{|Y|Y|Y|}
        \hline
        2 & 7 &   \\\hline
          & 3 &   \\\hline
          &   & 4 \\\hline
    \end{tabularx}
\end{minipage}
\hfill
\begin{minipage}[t]{0.23\textwidth}
    \begin{tabularx}{\textwidth}{|Y|Y|Y|}
        \hline
          & 1 & 4  \\\hline
          & 7 &   \\\hline
        10&   &   \\\hline
    \end{tabularx}
\end{minipage}
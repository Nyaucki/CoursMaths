\section*{Exercice 1}

\begin{minipage}{0.45\textwidth}
    \pts{3}
    \begin{align*}
        &-6+18\div 3\times (7\times2-11)\\
        =&-6+18\div 3\times (14-11)\\
        =&-6+18\div 3\times 3\\
        =&-6+6\times 3\\
        =&-6+18\\
        =&12
    \end{align*}
    
\end{minipage}
\hfil
\vrule
\hfil
\begin{minipage}{0.45\textwidth}
    \pts{3}
    \begin{align*}
        &-(-4)+(3\times2-9)-(7+(-3))\\
        =&-(-4)+(6-9)-(7+(-3))\\
        =&-(-4)+(-3)-(7+(-3))\\
        =&-(-4)+(-3)-(+4)\\
        =&+4-3-4\\
        =&-3\\
    \end{align*}
\end{minipage}

\section*{Exercice 2}

\subsection*{Triangle GHI}  \pts{2}
On additionne les trois angles : $\widehat{G}+\widehat{H}+\widehat{I}=57+78+45=180$.

Ce triangle est donc constructible.


\subsection*{Triangle WXY} \pts{2}
Le plus grand côté est $WY=9$cm

En additionnant les deux autres, on trouve : $WX+XY=3+4,5=7,5$ cm

Donc, $WX+XY<WY$, le triangle ne peut pas être construit.

\section*{Exercice 3}

\subsection*{Triangle VER} \pts{2}

Les angles $\widehat{R}$ et $\widehat{E}$ sont égaux, donc le triangle est isocèle en $V$.

De plus, l'angle $\widehat{V}$ mesure $180-\widehat{R}-\widehat{E}=180-45-45=90$. Le triangle est donc rectangle en $V$.

Donc, le triangle $VER$ est isocèle et rectangle en $V$.


\subsection*{Triangle PUR} \pts{2}

L'angle $\widehat{R}$ mesure $180-\widehat{P}-\widehat{U}=180-32-58=90$.

Ainsi, le triangle PUR est rectangle en R.

\subsection*{Triangle TYP} \pts{2}

Le triangle $TYP$ a deux angles mesurant 60 degrés. Le troisième mesure donc $180-60-60=60$°, c'est donc un triangle équilatéral.




\section*{Exercice 4}


Selon l'énoncé, Célimène doit être rajoutée aux invités pendant le partage. Elle a donc autant de fondant que tous les autres plus ce qui reste après le partage. 

Néanmoins, certains élèves n'ont pas compris ceci à la lecture de l'énoncé et ont conclu que Célimène ne recevait que les restes. \textcolor{gray}{Pour les questions 1 et 4, la réponse en partant de ce postulat sera écrit en gris.} (on retirera un point aux élèves ayant fait cette confusion.) 

\begin{enumerate}
    \item \begin{enumerate}[label=(\alph*)]
        \item \pts{3} On n'oublie pas de compter Célimène dans le partage. Il y aura donc 4+1=5 mangeurs de fondants.
        
        $27\div5=5$ et il reste 2. 
        
        Célimène aura donc 5+2=7 fondants.

        \textcolor{gray}{Si Célimène ne participe pas au partage, on a : $27\div4=6$ et il reste 1. Célimène n'aura donc qu'un seul fondant.}

        \item \pts{2} $27\div7=3$ et il reste 6. 
        
        Célimène aura donc 3+6=9 fondants.

        \textcolor{gray}{Si Célimène ne participe pas au partage, on a : $27\div6=4$ et il reste 3. Célimène aura donc trois fondants.}
    \end{enumerate}
    \item \pts{2} $27=3\times9$ donc 27 n'est pas un nombre premier.
    \item \pts{2} 23 ; 29 et 31 sont des nombres premiers proches de 27.
    \item \pts{3} Si on se réfère à la question 1, mieux vaut que Célimène invite 6 personnes que 4. 
    
    Considérons maintenant tous les cas possibles : 

    \begin{tabular}{c*{10}{|c}}
        nombre d'invités & 1 & 2 & 3 & 4 & 5 & 6 & 7 & 8 & 9 & 10\\\hline
        fondants par personne &13 & 9 & 6 & 5 & 4 & 3 & 3 & 3 & 2 & 2\\\hline
        fondants restant & 1 & 0 & 3 & 2 & 3 &6 & 3 & 0 & 7 & 5\\\hline
        fondants pour Célimène & 14 & 9 & 9 & 7 & 7 & 9 & 6 & 3 & 9 & 7
    \end{tabular}

    Au maximum, Célimène peut avoir 14 fondants si elle n'invite qu'une personne. \\Comme ce n'est pas vraiment une super fête avec un seul invité, supposons qu'elle souhaite plus d'invité. 
    \\ Elle peut avoir 9 gâteaux en invitant 2 , 3 , 6 ou 9 amis. 

    \textcolor{gray}{Si Célimène ne reçoit que les restes (pas cool pour sa fête...), on obtient le tableau suivant :\\[1ex]
    \begin{tabular}{c*{10}{|c}}
        nombre d'invités & 1 & 2 & 3 & 4 & 5 & 6 & 7 & 8 & 9 & 10\\\hline
        fondants pour Célimène & 0 & 1 & 0 & 3 & 2 & 3 & 6 & 3 & 0 & 7
    \end{tabular}\\[1ex]
    Il vaut donc  mieux que Célimène invite 10 copines.}

\end{enumerate}

\section*{Exercice 5}

\begin{enumerate}
    \begin{minipage}{0.45\textwidth}
        \item \pts{3} $2400\times\dfrac{2}{5}=960$. Elle a économisé 960€ le premier mois.
        \item \pts{4} \begin{align*}
            &\dfrac{2}{5}+\dfrac{3}{20}\\
            =&\dfrac{2\times4}{5\times4}+\dfrac{3}{20}\\
            =&\dfrac{8}{20}+\dfrac{3}{20}\\
            =&\dfrac{11}{20}
        \end{align*}
        Elle a donc économisé $\dfrac{11}{20}$ de la somme avant le 6ème mois. 
    
        Il lui restait donc à économiser $1-\dfrac{11}{20}=\dfrac{9}{20}$ le 6ème mois.
    \end{minipage}
    \hfil\vrule\hfil
    \begin{minipage}{0.45\textwidth}
        3. \pts{3}
        
        \begin{align*}
            &\dfrac{7}{9}+\dfrac{5}{6}-\dfrac{1}{3}\\
            =&\dfrac{7\times2}{9\times2}+\dfrac{5\times3}{6\times3}-\dfrac{1\times6}{3\times6}\\
            =&\dfrac{14}{18}+\dfrac{15}{18}-\dfrac{6}{18}\\
            =&\dfrac{23}{18}
        \end{align*}
    \end{minipage}
\end{enumerate}
\section*{Exercice 6}

\renewcommand{\ax}{-1}
\renewcommand{\ay}{5}
\renewcommand{\bx}{3}
\renewcommand{\by}{7}
\renewcommand{\cx}{2}
\renewcommand{\cy}{1}
\renewcommand{\dx}{-7}
\renewcommand{\dy}{2}

\begin{enumerate}
    \item voir figure  \pts{3} 
    \item voir figure \pts{3} 
\begin{figure}[H]
    \center
    \begin{tikzpicture}
        \draw[dotted] (-8,-5) grid (8,7);
        \draw [ultra thick,->] (0,-5)-- (0,7);
        \draw [ultra thick,->] (-8,0)-- (8,0);
        \node at (0,0) [below  left]{0};
        \draw (1,0.2)--(1,-0.2) node [below] {1};
        \draw (0.2,1)--(-0.2,1) node [left] {1};>
        \fill (\ax , \ay) coordinate (A) circle(1.5pt)node [above  left]{$A$};
        \fill[red] (\bx , \by) coordinate (B) circle(1.5pt)node [below  left]{$B$};
        \fill[red] (\cx , \cy) coordinate (C) circle(1.5pt)node [below  left]{$C$};
        \fill[red] (\dx , \dy) coordinate (D) circle(1.5pt)node [below  left]{$D$};
        \fill[red] ($(D)!1!-90:(A)$)  coordinate (E) circle(1.5pt)node  [below  left]{$E$};
        \fill[red] ($(A)!2!(C)$) coordinate (A')  circle(1.5pt)node [below  right]{$A'$};
        \fill[red] ($(B)!2!(C)$) coordinate (B')  circle(1.5pt)node [above  left]{$B'$};
        \draw [red,dashed] (A)--(E)--(D)--(A)--(B)--(A')--(B')--(A);
    \end{tikzpicture}
\end{figure}

    \item  \pts{3} ABA'B' est un quadrilatère avec un centre de symétrie. C'est donc un parallélogramme.
    
    On aurait aussi pu raisonner en disant que par construction de la symétrie centrale, $CA=CA'$ et $CB=CB'$. Les diagonales ont donc le même milieu et ABA'B' est donc un parallélogramme.

    Enfin, on aurait aussi pu utiliser les propriétés de conservation des distances (ou de parallélisme de l'image) de la symétrie centrale pour démontrer que les côtés opposés sont égaux (ou parallèles), et que ABA'B' est donc un parallélogramme.

    \item  \pts{3} 
    La symétrie centrale conserve les angles, donc $\widehat{B'AB}=\widehat{B'A'B}=100$.

    De plus, le triangle $DAE$ est rectangle et isocèle en $D$ donc $\widehat{DAE}=45$.

    Enfin, $\widehat{DAB}$ étant un angle plat, il mesure 180°.

    On a donc : \begin{align*}
    \widehat{B'AE}&=\widehat{BAB'}-\widehat{BAB'}-\widehat{DAE}\\
    &=180-100-45\\
    &=35
    \end{align*}
\end{enumerate}
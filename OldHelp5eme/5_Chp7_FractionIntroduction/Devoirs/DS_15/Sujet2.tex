\exo{4}{Calculer} Calculer les multiplications suivantes.

\begin{multicols}{4}
    $$424\times \dfrac{357}{424}$$\vspace*{1.7cm}
    
    \columnbreak
    $$14\times \dfrac{3}{7}$$\vspace*{1.7cm}
    
    \columnbreak
    $$32\times \dfrac{5}{4}$$\vspace*{1.7cm}
    
    \columnbreak
    $$48\times \dfrac{7}{6}$$\vspace*{1.7cm}
\end{multicols}

\exo{4}{Représenter} Compléter pour rendre les fractions égales.

\begin{multicols}{4}
    $$ \dfrac{16}{\text{\filling[1cm]}}=\dfrac{2}{9}$$

    $$ \dfrac{5}{3}=\dfrac{\text{\filling[1cm]}}{18}$$

    $$ \dfrac{\text{\filling[1cm]}}{30}=\dfrac{5}{10}$$

    $$ \dfrac{21}{35}=\dfrac{3}{\text{\filling[1cm]}}$$
\end{multicols}

\exo{4}{Représenter} Simplifier les fractions suivantes.

\begin{multicols}{4}
    $$\dfrac{60}{90}$$\vspace*{1.7cm}
    
    \columnbreak
    $$\dfrac{32}{40}$$\vspace*{1.7cm}
    
    \columnbreak
    $$\dfrac{45}{75}$$\vspace*{1.7cm}
    
    \columnbreak
    $$\dfrac{84}{78}$$\vspace*{1.7cm}
\end{multicols}

\exo{4}{Chercher} Compléter avec > ou <.

\begin{multicols}{4}
    $$\dfrac{7}{13}\text{\filling}\dfrac{6}{13}$$\vspace*{1.7cm}
    
    \columnbreak
    $$\dfrac{21}{130}\text{\filling}\dfrac{21}{5}$$\vspace*{1.7cm}
    
    \columnbreak
    $$\dfrac{3}{4}\text{\filling}\dfrac{5}{8}$$\vspace*{1.7cm}
    
    \columnbreak
    $$\dfrac{3}{5}\text{\filling}\dfrac{2}{3}$$\vspace*{1.7cm}
\end{multicols}

\newpage

\exo{4}{Calculer} : Pour chacune des figures suivantes, déterminer l'angle $\alpha$.

\begin{minipage}[t]{0.45\textwidth}
    \begin{figure}[H]
        \center
        \begin{tikzpicture}
            \draw (0,0) coordinate (A1) -- (4,0) coordinate (A2) node[right] {$(a)$};
            \draw (0,1.5) coordinate (B1) -- (4,1.5) coordinate (B2) node[right] {$(b)$};
            \draw (0.5,-1) coordinate (C1) -- (4,2.5) coordinate (C2);
            \node (I1) at (1.5,0) {};
            \node (I2) at (3,1.5) {};
            \draw pic["$\alpha$",draw=blue,fill=blue!20,angle eccentricity=1.3, angle radius=0.8cm]{angle=C1--I2--B2};
            \draw pic["29",draw=orange,fill=orange!20,angle eccentricity=1.3, angle radius=0.8cm]{angle=A2--I1--I2};
        \end{tikzpicture}

        $\alpha$=\filling[1.7cm]
    \end{figure}
\end{minipage}
\hfill
\begin{minipage}[t]{0.45\textwidth}
    \begin{figure}[H]
        \center
        \begin{tikzpicture}
            \draw (0,0.5) coordinate (A1) -- (0,3.75) coordinate (A2) node[right] {$(a)$};
            \draw (1.5,0.5) coordinate (B1) node[right] {$(b)$} -- (1.5,3.95) coordinate (B2) ;
            \draw (-1,0.5) coordinate (C1) -- (2.5,4) coordinate (C2);
            \node (I1) at (0,1.5) {};
            \node (I2) at (1.5,3) {};
            \draw pic["$\alpha$",draw=blue,fill=blue!20,angle eccentricity=1.3, angle radius=0.8cm]{angle=C2--I2--B2};
            \draw pic["103",draw=orange,fill=orange!20,angle eccentricity=1.4, angle radius=0.8cm]{angle=A1--I1--I2};
        \end{tikzpicture}

        $\alpha$=\filling[1.7cm]
    \end{figure}
\end{minipage}

\begin{minipage}[t]{0.45\textwidth}
    \begin{figure}[H]
        \center
        \begin{tikzpicture}[scale=1.25]
            \draw (0,0) coordinate (A1) -- (4,0) coordinate (A2) node[right] {$(a)$};
            \draw (0,1.5) coordinate (B1) -- (4,1.5) coordinate (B2) node[right] {$(b)$};
            \draw (0.5,-1) coordinate (C1) -- (4,2.5) coordinate (C2);
            \draw (1.5,-1) coordinate (D1) --(1.5,2.5) coordinate (D2) node [left] {$(d)$};
            \node (I1) at (1.5,0) {};
            \node (I2) at (3,1.5) {};
            \draw [gray,right angle quadrant=2,right angle symbol={A1}{A2}{D2}];
            \draw [gray,right angle quadrant=2,right angle symbol={B1}{B2}{D2}];
            \draw pic["$\alpha$",draw=blue,fill=blue!20,angle eccentricity=1.3, angle radius=0.8cm]{angle=C1--I2--B2};
            \draw pic["32",draw=orange,fill=orange!20,angle eccentricity=1.3, angle radius=0.8cm]{angle=C1--I1--D1};
        \end{tikzpicture}

        $\alpha$=\filling[1.7cm]
    \end{figure}
\end{minipage}
\hfill
\begin{minipage}[t]{0.45\textwidth}
    \begin{figure}[H]
        \center
        \begin{tikzpicture}[scale=1.25]
            \draw (0,0.5) coordinate (A1) -- (0,3.75) coordinate (A2) node[right] {$(a)$};
            \draw (1.5,0.5) coordinate (B1) node[right] {$(b)$} -- (1.5,3.75) coordinate (B2) ;
            \draw (-1,0.5) coordinate (C1) -- (2.5,4) coordinate (C2);
            \draw (-1,1.5) coordinate (D1) --(2.5,1.5) coordinate (D2) node [below] {$(d)$};
            \node (I1) at (0,1.5) {};
            \node (I2) at (1.5,3) {};
            \draw [gray,right angle quadrant=2,right angle symbol={A1}{A2}{D2}];
            \draw [gray,right angle quadrant=2,right angle symbol={B1}{B2}{D2}];
            \draw pic["$\alpha$",draw=blue,fill=blue!20,angle eccentricity=1.3, angle radius=0.8cm]{angle=D2--I1--I2};
            \draw pic["107",draw=orange,fill=orange!20,angle eccentricity=1.4, angle radius=0.8cm]{angle=B1--I2--C2};
        \end{tikzpicture}

        $\alpha$=\filling[1.7cm]
    \end{figure}
\end{minipage}

\exo{}{Modéliser} 

Des élèves ont acheté un paquet de bonbons.

\begin{itemize}
    \item Joe en mange la moitié. 
    
    \item Puis Jack mange $\dfrac{2}{3}$ du reste.
    
    \item William mange $\dfrac{4}{5}$ de ce qu'ont laissé les deux précédents.
    
    \item À la fin, il ne reste plus que 3 bonbons à Averell.
\end{itemize}

Combien de bonbons y avait-il dans le paquet ?


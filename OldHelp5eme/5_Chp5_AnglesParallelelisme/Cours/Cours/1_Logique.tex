\section{Logique mathématique}

\dfnt{Ce que je sais et ce que je devine}{
En mathématiques, je ne considère que deux types d'informations :
\\\textbf{Ce que je sais}, qui est généralement donné par l'énoncé.
\\\textbf{Ce que je peux deviner}, qui peut être déduit des autres informations à l'aide de propriétés.
}

\rmq{Il est conseillé en début d'exercice de faire un schéma pour y noter toutes les informations que l'on sait ou peut deviner.}

\dfnt{Démonstration}
{Lorsqu'on utilise une propriété pour ajouter une information à la liste de ce qui peut être deviné, on appelle cela \textbf{une démonstration}}

\rmq {Une information que l'on ne sait pas, \textbf{même si elle semble vraie sur le dessin}, ne peut pas être utilisée ou considérée vraie tant quelle n'a pas été démontré.}



\begin{tcolorbox}
    \textbf{Pour rédiger une démonstration :}

    On commence par identifier la propriété qui sera utilisé.

    On écrit alors les informations qui seront utilisés (Celles avec des $\bullet$ dans le cours)
    
    Puis on écrit la conclusion de la propriété (ce qui est après le "Alors" dans le cours.)
\end{tcolorbox}

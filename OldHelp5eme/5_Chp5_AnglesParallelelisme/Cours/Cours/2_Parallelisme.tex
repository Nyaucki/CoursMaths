\section{Propriétés de parallélisme}

\prop{Perpendiculaires à même troisième}
{Soient trois droites $(a), (b)$ et $(c)$.
Si on a :\vspace{1em}\\
\begin{minipage}{0.65\textwidth}
    \begin{itemize}
        \item $(a)$ et $(b)$ sont perpendiculaires.
        \item $(a)$ et $(c)$ sont perpendiculaires.
    \end{itemize}
        Alors $(b)$ et $(c)$ sont parallèles.\vspace{1em} \\
        Dit autrement, si deux droites sont perpendiculaires à une même troisième, alors elles sont parallèles.
\end{minipage}
\hfill
\begin{minipage}{0.3\textwidth}
    \begin{figure}[H]
        \center
        \begin{tikzpicture}[scale=0.75]
            \draw (0,0) -- (4,0) node [midway,below] {$(a)$} ;
            \draw (1,-1) -- (1,2) node [right] {$(b)$} ;
            \draw (3,-1) -- (3,2) node [right] {$(c)$} ;
            \draw (0.8,0) -- (0.8,0.2) ;
            \draw (1,0.2) -- (0.8,0.2) ;
            \draw (2.8,0) -- (2.8,0.2) ;
            \draw (3,0.2) -- (2.8,0.2) ;
        \end{tikzpicture}
    \end{figure}
\end{minipage}
}

\prop{Troisième parallèle}{
Soient trois droites $(a), (b)$ et $(c)$.
Si on a :\vspace{1em}\\
\begin{minipage}{0.65\textwidth}
    \begin{itemize}
        \item $(a)$ et $(b)$ sont parallèles.
        \item $(a)$ et $(c)$ sont parallèles.
    \end{itemize}
        Alors $(b)$ et $(c)$ sont parallèles.\vspace{1em}
        \\
        Dit autrement, si deux droites sont parallèles, alors toute parallèle à l'une est parallèle à l'autre.
\end{minipage}
\hfill
\begin{minipage}{0.3\textwidth}
    \begin{figure}[H]
        \center
        \begin{tikzpicture}[scale=0.75]
            \draw (2,-0.2) node [below] {$(a)$} -- (2,2) ;
            \draw (1,-0.2) -- (1,2) node [left] {$(b)$} ;
            \draw (3,-0.2) -- (3,2) node [right] {$(c)$} ;
        \end{tikzpicture}
    \end{figure}
\end{minipage}
}

\prop{Perpendiculaire à une parallèle}{
Soient trois droites $(a), (b)$ et $(c)$.
Si on a :\vspace{1em}\\
\begin{minipage}{0.65\textwidth}
    \begin{itemize}
        \item $(a)$ et $(b)$ sont parallèles.
        \item $(a)$ et $(c)$ sont perpendiculaires.
    \end{itemize}
        Alors $(b)$ et $(c)$ sont parallèles.\vspace{1em} \\
        Dit autrement, si deux droites sont parallèles, alors toute perpendiculaires à l'une est perpendiculaires à l'autre.
\end{minipage}
\hfill
\begin{minipage}{0.3\textwidth}
    \begin{figure}[H]
        \center
        \begin{tikzpicture}[scale=0.75]
            \draw (0,0) -- (4,0) node [midway,below] {$(c)$} ;
            \draw (1,-1) -- (1,2) node [right] {$(b)$} ;
            \draw (3,-1) -- (3,2) node [right] {$(a)$} ;
            \draw (0.8,0) -- (0.8,0.2) ;
            \draw (1,0.2) -- (0.8,0.2) ;
            \draw (2.8,0) -- (2.8,0.2) ;
            \draw (3,0.2) -- (2.8,0.2) ;
        \end{tikzpicture}
    \end{figure}
\end{minipage}}

% \dfnt{Parallélogramme}
% {Un parallélogramme est un quadrilatère avec les propriétés suivantes :
% \begin{itemize}
%     \item Les côtés opposés sont égaux
%     \item Les côtés opposés sont parallèles
%     \item Les diagonales se coupent en leurs milieux
% \end{itemize}}

% \prop{Être un parallélogramme}
% {Soit $ABCD$ un quadrilatère. Si on a :
% \begin{itemize}
%     \item $AB=CD$
%     \item $AD=BC$
% \end{itemize}
% Dit autrement, les côtés opposés sont égaux deux à deux.
% % {\center \textbf{OU}}
% \begin{itemize}
%     \item $AB||CD$
%     \item $AD||BC$
% \end{itemize}
% Dit autrement, les côtés opposés sont parallèles.
% % {\center \textbf{OU}}
% \begin{itemize}
%     \item $AB||CD$
%     \item $AB=CD$
% \end{itemize}
% Dit autrement, deux côtés opposés sont égaux et parallèles.
% % {\center \textbf{OU}}
% \begin{itemize}
%     \item $[AC]$ et $[BD]$ ont le même milieu
% \end{itemize}
% Dit autrement, les diagonales se coupent en leur milieu.\\
% Alors $ABCD$ est un parallélogramme.
% }

% \dfnt{Losange}
% {}

% \dfnt{Rectangle}
% {}

% \dfnt{Carré}
% {}
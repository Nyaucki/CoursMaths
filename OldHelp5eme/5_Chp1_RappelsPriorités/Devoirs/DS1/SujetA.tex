\rfoot{A}

\exo{4}{Calcul}

Effectuer les calculs suivants en détaillant les étapes.

\begin{multicols}{2}
    \cnt\\ 
    $12-1+3\times 2$

    \vspace*{14em}
    \columnbreak
    \cnt\\
    $(7\times (2+3)-5):6$

    \vspace*{14em}
\end{multicols}

\begin{multicols}{2}
    \exo{2}{Cours}
    
    Retrouve le nombre de départs en sachant que : 
    
    Si j'ajoute 3, je multiplie par 2, j'enlève 2 et je multiplie le tout par 9, j'obtiens 90.

    \vspace*{14em}
    \columnbreak
    \exo{2}{Chercher}
    
    Placer les parenthèses au bon endroit pour rendre l'égalité vraie :
    
    $2+3\times 5-1=24$

    \vspace*{14em}
\end{multicols}

\begin{multicols}{2}
    \exo{1}{Modéliser}\\    
    Écrire le Calcul permettant de trouver le nombre de bonbons de Zoé : \vspace*{1em} \\     
    Elle en achète 50, elle en donne 13 à son frère qui lui rends 3 lots de 2.
    \vspace*{6em}
    
    \columnbreak
    \exo{1}{Communiquer}
    
    Trouve l'erreur :
    \begin{align*}
        &10-(1+3\times 2)\\
        =&10-(1+6)\\
        =&9+6\\
        =&15
    \end{align*}
    \vspace*{6em}
\end{multicols}



\begin{minipage}[t]{0.45\textwidth}
  \exo{Représenter}{PlacerDroite1}
  
  Placer chacun des nombres sur la droite graduée ci-dessous :
  0,4 ; 0,2 ; 0,5 ; 0,7
  \vspace{2em}
  
  
  \tikzmath{\ya =0; \scl=10; \rcl= 0; \dclg =8; \pas=1/\scl ; \yb =\ya +\dclg * \pas; \dprt = \ya - \rcl * \pas; \y2 =\dprt +\pas; \fin =\dprt +8*\pas ; \grad = 0.1/\scl ; } %modifier yA, scl (scale), rcl (reculer) et dclg (decalage )uniquelent
  
  \begin{figure}[H]
      \centering
      \begin{tikzpicture}[scale=\scl]
          \draw (\dprt,0) -- (\fin,0) node[midway, sloped]{};
          \foreach \x in {\dprt,\y2,...,\fin}
          {
            \draw (\x,\grad) -- (\x,-\grad) ;
          }
          \node (A) at (\ya,-\grad) [below] {\pgfmathprintnumber[use comma]{\ya}} ;
          \node (B) at (\yb,-\grad) [below] {\pgfmathprintnumber[use comma,precision=3]{\yb}} ;
      \end{tikzpicture} 
  \end{figure}
\end{minipage}
\hfill
\begin{minipage}[t]{0.45\textwidth}
  \exo{Représenter}{PlacerDroite2}
  
  Placer chacun des nombres sur la droite graduée ci-dessous :
  0,8 ; 1,4 ; 1,1 ; 0,7 
  \vspace{2em}
  
  
  \tikzmath{\ya =1; \scl=10; \rcl= 3; \dclg =2; \pas=1/\scl ; \yb =\ya +\dclg * \pas; \dprt = \ya - \rcl * \pas; \y2 =\dprt +\pas; \fin =\dprt +8*\pas ; \grad = 0.1/\scl ; } %modifier yA, scl (scale), rcl (reculer) et dclg (decalage )uniquelent
  
  \begin{figure}[H]
      \centering
      \begin{tikzpicture}[scale=\scl]
          \draw (\dprt,0) -- (\fin,0) node[midway, sloped]{};
          \foreach \x in {\dprt,\y2,...,\fin}
          {
            \draw (\x,\grad) -- (\x,-\grad) ;
          }
          \node (A) at (\ya,-\grad) [below] {\pgfmathprintnumber[use comma]{\ya}} ;
          \node (B) at (\yb,-\grad) [below] {\pgfmathprintnumber[use comma,precision=4]{\yb}} ;
      \end{tikzpicture} 
  \end{figure}
\end{minipage}


\begin{minipage}[t]{0.45\textwidth}
  \exo{Représenter}{PlacerDroite3}
  
  Placer chacun des nombres sur la droite graduée ci-dessous :
  1 ; 0,96 ; 1,02 ; 1,04
  \vspace{2em}
  
  
  \tikzmath{\ya =0.98; \scl=100; \rcl= 2; \dclg =5; \pas=1/\scl ; \yb =\ya +\dclg * \pas; \dprt = \ya - \rcl * \pas; \y2 =\dprt +\pas; \fin =\dprt +8*\pas ; \grad = 0.1/\scl ; } %modifier yA, scl (scale), rcl (reculer) et dclg (decalage )uniquelent
  
  \begin{figure}[H]
      \centering
      \begin{tikzpicture}[scale=\scl]
          \draw (\dprt,0) -- (\fin,0) node[midway, sloped]{};
          \foreach \x in {\dprt,\y2,...,\fin}
          {
            \draw (\x,\grad) -- (\x,-\grad) ;
          }
          \node (A) at (\ya,-\grad) [below] {\pgfmathprintnumber[use comma]{\ya}} ;
          \node (B) at (\yb,-\grad) [below] {\pgfmathprintnumber[use comma,precision=3]{\yb}} ;
      \end{tikzpicture} 
  \end{figure}
\end{minipage}
\hfill
\begin{minipage}[t]{0.45\textwidth}
  \exo{Représenter}{PlacerDroite4}
  
  Placer chacun des nombres sur la droite graduée ci-dessous :
  1,48 ; 1,5 ; 1,43 ; 1.45
  \vspace{2em}
  
  
  \tikzmath{\ya =1.44; \scl=100; \rcl= 1; \dclg =2; \pas=1/\scl ; \yb =\ya +\dclg * \pas; \dprt = \ya - \rcl * \pas; \y2 =\dprt +\pas; \fin =\dprt +8*\pas ; \grad = 0.1/\scl ; } %modifier yA, scl (scale), rcl (reculer) et dclg (decalage )uniquelent
  
  \begin{figure}[H]
      \centering
      \begin{tikzpicture}[scale=\scl]
          \draw (\dprt,0) -- (\fin,0) node[midway, sloped]{};
          \foreach \x in {\dprt,\y2,...,\fin}
          {
            \draw (\x,\grad) -- (\x,-\grad) ;
          }
          \node (A) at (\ya,-\grad) [below] {\pgfmathprintnumber[use comma]{\ya}} ;
          \node (B) at (\yb,-\grad) [below] {\pgfmathprintnumber[use comma,precision=4]{\yb}} ;
      \end{tikzpicture} 
  \end{figure}
\end{minipage}

\begin{minipage}[t]{0.45\textwidth}
  \exo{Représenter}{PlacerDroite5}
  
  Placer chacun des nombres sur la droite graduée ci-dessous :
  4,8 ; 4,2 ; 3,8 ; 5,2
  \vspace{2em}
  
  
  \tikzmath{\ya =4; \scl=5; \rcl= 2; \dclg =2; \pas=1/\scl ; \yb =\ya +\dclg * \pas; \dprt = \ya - \rcl * \pas; \y2 =\dprt +\pas; \fin =\dprt +8*\pas ; \grad = 0.1/\scl ; } %modifier yA, scl (scale), rcl (reculer) et dclg (decalage )uniquelent
  
  \begin{figure}[H]
      \centering
      \begin{tikzpicture}[scale=\scl]
          \draw (\dprt,0) -- (\fin,0) node[midway, sloped]{};
          \foreach \x in {\dprt,\y2,...,\fin}
          {
            \draw (\x,\grad) -- (\x,-\grad) ;
          }
          \node (A) at (\ya,-\grad) [below] {\pgfmathprintnumber[use comma]{\ya}} ;
          \node (B) at (\yb,-\grad) [below] {\pgfmathprintnumber[use comma,precision=3]{\yb}} ;
      \end{tikzpicture} 
  \end{figure}
\end{minipage}
\hfill
\begin{minipage}[t]{0.45\textwidth}
  \exo{Représenter}{PlacerDroite6}
  
  Placer chacun des nombres sur la droite graduée ci-dessous :
  8,1 ; 7,7 ; 8,9 ; 7,3 
  \vspace{2em}
  
  
  \tikzmath{\ya =8.5; \scl=5; \rcl= 6; \dclg =1; \pas=1/\scl ; \yb =\ya +\dclg * \pas; \dprt = \ya - \rcl * \pas; \y2 =\dprt +\pas; \fin =\dprt +8*\pas ; \grad = 0.1/\scl ; } %modifier yA, scl (scale), rcl (reculer) et dclg (decalage )uniquelent
  
  \begin{figure}[H]
      \centering
      \begin{tikzpicture}[scale=\scl]
          \draw (\dprt,0) -- (\fin,0) node[midway, sloped]{};
          \foreach \x in {\dprt,\y2,...,\fin}
          {
            \draw (\x,\grad) -- (\x,-\grad) ;
          }
          \node (A) at (\ya,-\grad) [below] {\pgfmathprintnumber[use comma]{\ya}} ;
          \node (B) at (\yb,-\grad) [below] {\pgfmathprintnumber[use comma,precision=4]{\yb}} ;
      \end{tikzpicture} 
  \end{figure}
\end{minipage}


\begin{minipage}[t]{0.45\textwidth}
  \exo{Représenter}{PlacerDroite7}
  
  Placer chacun des nombres sur la droite graduée ci-dessous :
  1.58 ; 1.54 ; 1,46 ; 1,6
  \vspace{2em}
  
  
  \tikzmath{\ya =1.5; \scl=50; \rcl= 2; \dclg =3; \pas=1/\scl ; \yb =\ya +\dclg * \pas; \dprt = \ya - \rcl * \pas; \y2 =\dprt +\pas; \fin =\dprt +8*\pas ; \grad = 0.1/\scl ; } %modifier yA, scl (scale), rcl (reculer) et dclg (decalage )uniquelent
  
  \begin{figure}[H]
      \centering
      \begin{tikzpicture}[scale=\scl]
          \draw (\dprt,0) -- (\fin,0) node[midway, sloped]{};
          \foreach \x in {\dprt,\y2,...,\fin}
          {
            \draw (\x,\grad) -- (\x,-\grad) ;
          }
          \node (A) at (\ya,-\grad) [below] {\pgfmathprintnumber[use comma]{\ya}} ;
          \node (B) at (\yb,-\grad) [below] {\pgfmathprintnumber[use comma,precision=3]{\yb}} ;
      \end{tikzpicture} 
  \end{figure}
\end{minipage}
\hfill
\begin{minipage}[t]{0.45\textwidth}
  \exo{Représenter}{PlacerDroite8}
  
  Placer chacun des nombres sur la droite graduée ci-dessous :
  10 ; 9,84 ; 10,12 ; 9.92
  \vspace{2em}
  
  
  \tikzmath{\ya =9.88; \scl=25; \rcl= 1; \dclg =2; \pas=1/\scl ; \yb =\ya +\dclg * \pas; \dprt = \ya - \rcl * \pas; \y2 =\dprt +\pas; \fin =\dprt +8*\pas ; \grad = 0.1/\scl ; } %modifier yA, scl (scale), rcl (reculer) et dclg (decalage )uniquelent
  
  \begin{figure}[H]
      \centering
      \begin{tikzpicture}[scale=\scl]
          \draw (\dprt,0) -- (\fin,0) node[midway, sloped]{};
          \foreach \x in {\dprt,\y2,...,\fin}
          {
            \draw (\x,\grad) -- (\x,-\grad) ;
          }
          \node (A) at (\ya,-\grad) [below] {\pgfmathprintnumber[use comma]{\ya}} ;
          \node (B) at (\yb,-\grad) [below] {\pgfmathprintnumber[use comma,precision=4]{\yb}} ;
      \end{tikzpicture} 
  \end{figure}
\end{minipage}

\vspace{-1em}


\begin{minipage}[t]{0.45\textwidth}
  \exo{Représenter}{LireDroite1}
    
  Quels nombre correspondent à A, B, C et D?
  
  
  \tikzmath{\ya =0; \scl=10; \rcl= 0; \dclg =8; \pas=1/\scl ; \yb =\ya +\dclg * \pas; \dprt = \ya - \rcl * \pas; \y2 =\dprt +\pas; \fin =\dprt +8*\pas ; \grad = 0.1/\scl ; } %modifier yA, scl (scale), rcl (reculer) et dclg (decalage )uniquelent
  
  \begin{figure}[H]
      \centering
      \begin{tikzpicture}[scale=\scl]
          \draw (\dprt,0) -- (\fin,0) node[midway, sloped]{};
          \foreach \x in {\dprt,\y2,...,\fin}
          {
            \draw (\x,\grad) -- (\x,-\grad) ;
          }
          \foreach \z [count=\zi] in {0.2,0.7,0.4,0.1}
          {
            \node at (\z,\grad) [above] {\makeAlph{\zi}};
          }
          \node (A) at (\ya,-\grad) [below] {\pgfmathprintnumber[use comma]{\ya}} ;
          \node (B) at (\yb,-\grad) [below] {\pgfmathprintnumber[use comma,precision=3]{\yb}} ;
      \end{tikzpicture} 
  \end{figure}
\end{minipage}
\hfill
\begin{minipage}[t]{0.45\textwidth}
  \exo{Représenter}{LireDroite2}
    
  Quels nombre correspondent à A, B, C et D?
  
  
  \tikzmath{\ya =2; \scl=10; \rcl= 3; \dclg =1; \pas=1/\scl ; \yb =\ya +\dclg * \pas; \dprt = \ya - \rcl * \pas; \y2 =\dprt +\pas; \fin =\dprt +8*\pas ; \grad = 0.1/\scl ; } %modifier yA, scl (scale), rcl (reculer) et dclg (decalage )uniquelent
  
  \begin{figure}[H]
      \centering
      \begin{tikzpicture}[scale=\scl]
          \draw (\dprt,0) -- (\fin,0) node[midway, sloped]{};
          \foreach \x in {\dprt,\y2,...,\fin}
          {
            \draw (\x,\grad) -- (\x,-\grad) ;
          }
          \foreach \z [count=\zi] in {2.3,1.8,2.5,1.7}
          {
            \node at (\z,\grad) [above] {\makeAlph{\zi}};
          }
          \node (A) at (\ya,-\grad) [below] {\pgfmathprintnumber[use comma]{\ya}} ;
          \node (B) at (\yb,-\grad) [below] {\pgfmathprintnumber[use comma,precision=3]{\yb}} ;
      \end{tikzpicture} 
  \end{figure}
\end{minipage}
\vspace{-1em}

\begin{minipage}[t]{0.45\textwidth}
  \exo{Représenter}{LireDroite3}
    
  Quels nombre correspondent à A, B, C et D?
  
  
  \tikzmath{\ya =8.95; \scl=50; \rcl= 1; \dclg =2; \pas=1/\scl ; \yb =\ya +\dclg * \pas; \dprt = \ya - \rcl * \pas; \y2 =\dprt +\pas; \fin =\dprt +8*\pas ; \grad = 0.1/\scl ; } %modifier yA, scl (scale), rcl (reculer) et dclg (decalage )uniquelent
  
  \begin{figure}[H]
      \centering
      \begin{tikzpicture}[scale=\scl]
          \draw (\dprt,0) -- (\fin,0) node[midway, sloped]{};
          \foreach \x in {\dprt,\y2,...,\fin}
          {
            \draw (\x,\grad) -- (\x,-\grad) ;
          }
          \foreach \z [count=\zi] in {9.03,8.93,9.05,9.01}
          {
            \node at (\z,\grad) [above] {\makeAlph{\zi}};
          }
          \node (A) at (\ya,-\grad) [below] {\pgfmathprintnumber[use comma]{\ya}} ;
          \node (B) at (\yb,-\grad) [below] {\pgfmathprintnumber[use comma,precision=3]{\yb}} ;
      \end{tikzpicture} 
  \end{figure}
\end{minipage}
\hfill
\begin{minipage}[t]{0.45\textwidth}
  \exo{Représenter}{LireDroite4}
    
  Quels nombre correspondent à A, B, C et D?
  
  
  \tikzmath{\ya =2.13; \scl=100; \rcl= 2; \dclg =1; \pas=1/\scl ; \yb =\ya +\dclg * \pas; \dprt = \ya - \rcl * \pas; \y2 =\dprt +\pas; \fin =\dprt +8*\pas ; \grad = 0.1/\scl ; } %modifier yA, scl (scale), rcl (reculer) et dclg (decalage )uniquelent
  
  \begin{figure}[H]
      \centering
      \begin{tikzpicture}[scale=\scl]
          \draw (\dprt,0) -- (\fin,0) node[midway, sloped]{};
          \foreach \x in {\dprt,\y2,...,\fin}
          {
            \draw (\x,\grad) -- (\x,-\grad) ;
          }
          \foreach \z [count=\zi] in {2.11,2.17,2.19,2.14}
          {
            \node at (\z,\grad) [above] {\makeAlph{\zi}};
          }
          \node (A) at (\ya,-\grad) [below] {\pgfmathprintnumber[use comma]{\ya}} ;
          \node (B) at (\yb,-\grad) [below] {\pgfmathprintnumber[use comma,precision=3]{\yb}} ;
      \end{tikzpicture} 
  \end{figure}
\end{minipage}
\vspace{-1em}


\begin{minipage}[t]{0.45\textwidth}
  \exo{Représenter}{LireDroite5}
    
  Quels nombre correspondent à A, B, C et D?
  
  
  \tikzmath{\ya =7.51; \scl=5; \rcl= 4; \dclg =2; \pas=1/\scl ; \yb =\ya +\dclg * \pas; \dprt = \ya - \rcl * \pas; \y2 =\dprt +\pas; \fin =\dprt +8*\pas ; \grad = 0.1/\scl ; } %modifier yA, scl (scale), rcl (reculer) et dclg (decalage )uniquelent
  
  \begin{figure}[H]
      \centering
      \begin{tikzpicture}[scale=\scl]
          \draw (\dprt,0) -- (\fin,0) node[midway, sloped]{};
          \foreach \x in {\dprt,\y2,...,\fin}
          {
            \draw (\x,\grad) -- (\x,-\grad) ;
          }
          \foreach \z [count=\zi] in {8.11,7.31,6.91,6.71}
          {
            \node at (\z,\grad) [above] {\makeAlph{\zi}};
          }
          \node (A) at (\ya,-\grad) [below] {\pgfmathprintnumber[use comma]{\ya}} ;
          \node (B) at (\yb,-\grad) [below] {\pgfmathprintnumber[use comma,precision=3]{\yb}} ;
      \end{tikzpicture} 
  \end{figure}
\end{minipage}
\hfill
\begin{minipage}[t]{0.45\textwidth}
  \exo{Représenter}{LireDroite6}
    
  Quels nombre correspondent à A, B, C et D?
  
  
  \tikzmath{\ya =2.13; \scl=200; \rcl= 2; \dclg =2; \pas=1/\scl ; \yb =\ya +\dclg * \pas; \dprt = \ya - \rcl * \pas; \y2 =\dprt +\pas; \fin =\dprt +8*\pas ; \grad = 0.1/\scl ; } %modifier yA, scl (scale), rcl (reculer) et dclg (decalage )uniquelent
  
  \begin{figure}[H]
      \centering
      \begin{tikzpicture}[scale=\scl]
          \draw (\dprt,0) -- (\fin,0) node[midway, sloped]{};
          \foreach \x in {\dprt,\y2,...,\fin}
          {
            \draw (\x,\grad) -- (\x,-\grad) ;
          }
          \foreach \z [count=\zi] in {2.12,2.15,2.16,2.135}
          {
            \node at (\z,\grad) [above] {\makeAlph{\zi}};
          }
          \node (A) at (\ya,-\grad) [below] {\pgfmathprintnumber[use comma]{\ya}} ;
          \node (B) at (\yb,-\grad) [below] {\pgfmathprintnumber[use comma,precision=3]{\yb}} ;
      \end{tikzpicture} 
  \end{figure}
\end{minipage}
\vspace{-1em}
\begin{minipage}[t]{0.45\textwidth}
  \exo{Représenter}{LireDroite7}
    
  Quels nombre correspondent à A, B, C et D?
  
  
  \tikzmath{\ya =5.51; \scl=5; \rcl= 1; \dclg =6; \pas=1/\scl ; \yb =\ya +\dclg * \pas; \dprt = \ya - \rcl * \pas; \y2 =\dprt +\pas; \fin =\dprt +8*\pas ; \grad = 0.1/\scl ; } %modifier yA, scl (scale), rcl (reculer) et dclg (decalage )uniquelent
  
  \begin{figure}[H]
      \centering
      \begin{tikzpicture}[scale=\scl]
          \draw (\dprt,0) -- (\fin,0) node[midway, sloped]{};
          \foreach \x in {\dprt,\y2,...,\fin}
          {
            \draw (\x,\grad) -- (\x,-\grad) ;
          }
          \foreach \z [count=\zi] in {5.71,6.11,6.91,6.51}
          {
            \node at (\z,\grad) [above] {\makeAlph{\zi}};
          }
          \node (A) at (\ya,-\grad) [below] {\pgfmathprintnumber[use comma]{\ya}} ;
          \node (B) at (\yb,-\grad) [below] {\pgfmathprintnumber[use comma,precision=3]{\yb}} ;
      \end{tikzpicture} 
  \end{figure}
\end{minipage}
\hfill
\begin{minipage}[t]{0.45\textwidth}
  \exo{Représenter}{LireDroite8}
    
  Quels nombre correspondent à A, B, C et D?
  
  
  \tikzmath{\ya =5.14; \scl=100; \rcl= 1; \dclg =4; \pas=1/\scl ; \yb =\ya +\dclg * \pas; \dprt = \ya - \rcl * \pas; \y2 =\dprt +\pas; \fin =\dprt +8*\pas ; \grad = 0.1/\scl ; } %modifier yA, scl (scale), rcl (reculer) et dclg (decalage )uniquelent
  
  \begin{figure}[H]
      \centering
      \begin{tikzpicture}[scale=\scl]
          \draw (\dprt,0) -- (\fin,0) node[midway, sloped]{};
          \foreach \x in {\dprt,\y2,...,\fin}
          {
            \draw (\x,\grad) -- (\x,-\grad) ;
          }
          \foreach \z [count=\zi] in {5.2,5.13,5.16,5.15}
          {
            \node at (\z,\grad) [above] {\makeAlph{\zi}};
          }
          \node (A) at (\ya,-\grad) [below] {\pgfmathprintnumber[use comma]{\ya}} ;
          \node (B) at (\yb,-\grad) [below] {\pgfmathprintnumber[use comma,precision=3]{\yb}} ;
      \end{tikzpicture} 
  \end{figure}
\end{minipage}
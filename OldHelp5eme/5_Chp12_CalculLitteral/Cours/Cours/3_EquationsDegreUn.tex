\section{Premières équations}

\dfnt{Equation}
{Une équation est une égalité entre deux expressions (appelées \textbf{membres}) de part et d'autre du signe égale dont au moins une est littérale.}

\exmpl{
    \begin{multicols}{2}
        \begin{itemize}
        \item $3x+2=7$ est une équation
        \item $14+5=19$ n'est pas une équation (pas de lettre)
        \item $-7=2x^2-5$ est une équation
        \item $3x-7x^2$ n'est pas une équation (une seule expression)
    \end{itemize}
    \end{multicols}
}

\rmq{Au sein d'une équation, les lettres ont le rôle d'inconnues.}

\dfnt{Solution d'une équation}
{On appelle solution(s) d'une équation la (ou les) valeur(s) de $x$ pour laquelle l'égalité est vraie}


\prop{Tester une solution}
{Pour tester si un nombre est solution, on remplace toutes les lettres par ce nombre  et on observe si les membres sont égaux. 
\begin{itemize}
    \item Si les membres sont égaux, alors le nombre est solution.
    \item Si les membres ne sont pas égaux, alors le nombre n'est pas solution.
\end{itemize}}


\prop{Equation équivalente}
{Si on ajoute ou soustrait le même terme d'une équation à chaque membre, on obtient une équation équivalente (elle aura la même solution)\\
De la même manière, multiplier ou diviser chaque membre d'une équation par le même facteur donne une équation équivalente.
}

\textbf{Résoudre une équation}

Résoudre une équation du premier degré revient à modifier son écriture (en passant par des équations équivalentes) jusqu'à aboutir à $x=\dots$
La valeur de $x$ obtenue est alors la solution.

Entre deux lignes de calcul, on ne peut pas mettre de signe égal puisqu'on risquerai de confondre avec celui de l'équation. On met donc le signe $\iff$ (au collège, ne pas le mettre ne sera pas sanctionné).

\begin{align*}
    \text{Observons la méthode à travers un exemple :}& & &5(x-3)+4x=3x-7\\
    \text{On distribue pour supprimer les parenthèses :}& & \iff & 5x-5\times 3 +4x =3x-7\\
    \text{On simplifie en regroupant de chaque côté :}& & \iff & 9x-15=3x-7\\
    \text{On supprime le terme en $x$ à droite en ajoutant son opposé :} & & \iff & 9x-15-3x=3x-7-3x\\
    \text{On simplifie :}& & \iff &6x-15=-7\\
    \text{On supprime le terme sans $x$ à droite en ajoutant son opposé :}& & \iff &6x-15 +15=-7+15\\
    \text{On simplifie :}& &\iff & 6x=8\\
    \text{On se débarrasse du nombre multipliant le $x$ en divisant par celui-ci :} && \iff &6x\div 6=8\div 6\\
    \text{On peut laisser le résultat sous forme de fraction simplifié :} &&\iff &x=\dfrac{4}{3}
\end{align*}
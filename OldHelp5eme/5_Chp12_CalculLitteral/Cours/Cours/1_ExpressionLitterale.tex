\section{Expression littérale}

\dfnt{Expression littérale}{Une expression littérale est un calcul contenant au moins une lettre représentant un nombre inconnu.}

\rmq{On peut utiliser une expression littérale pour écrire une formule générale à l'aide de lettres qui seront ensuite remplacés par les nombres voulus lorsqu'on utilisera la formule. On dit alors que la lettre a un rôle de variable.}

\prop{Règles d'écriture}
{Pour gagner du temps à l'écriture, on adoptera les conventions suivantes :
\begin{multicols}{2}
    \begin{itemize}
        \item $3\times a=3a$
        \item $a\times b =ab$
        \item $4\times(a-2)=4(a-2)$
        \item $1\times a =1a=a$
        \item $a\times a =a^2$
        \item $a\times a \times a=a^3$
    \end{itemize}
\end{multicols} }

\rmq{Pour simplifier, on pourra retenir qu'il n'est pas nécessaire d'écrire le symbole multiplication quand il y a une lettre.}


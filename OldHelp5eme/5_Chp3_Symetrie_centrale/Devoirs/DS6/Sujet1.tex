\exo{4}{Chercher} : 
S'agit-il d'une symétrie centrale ou d'une symétrie axiale ?
\begin{minipage}[t]{0.45\textwidth} 
    \begin{figure}[H]
        \centering 
        Il s'agit d'une symétrie \fillin[3cm]
        \begin{tikzpicture}[scale=0.8]
            \tikzset{
            homothety at/.style args={#1 scaled by #2}{shift={($(#1)!#2!(0,0)$)},scale=#2},}
            \draw [white](-2,-2) grid (3,3.5);
            \def\mypath{(-2,2) -- (-1,0) --(-2,0) -- (1,3)--(-2,2)}
            \draw [thick]\mypath ;
            \fill[white] (-2,1) coordinate (c) circle(2pt) node [above]{$O$};     
            \begin{scope}[homothety at=c scaled by -1]
                \draw[dashed] \mypath;
            \end{scope}        
        \end{tikzpicture}
    \end{figure}
\end{minipage}
\hfill
\begin{minipage}[t]{0.45\textwidth}
    \begin{figure}[H]
        \centering 
        Il s'agit d'une symétrie \fillin[3cm]
        \begin{tikzpicture}[scale=0.8]
            \tikzset{
            homothety at/.style args={#1 scaled by #2}{shift={($(#1)!#2!(0,0)$)},scale=#2},}
            \def\mypath{(-2,-2) -- (-1,2)--(0,-2)--(-2,-2)}
            \fill [white] (-3,2.5) coordinate (d) circle(2pt); 
            \fill[white] (-3,0) coordinate (c) circle(2pt) node [above]{$O$}; 
            \begin{scope}[homothety at=c scaled by -1]
                \draw[dashed] \mypath;
            \end{scope} 
            \draw [thick]\mypath ;    
        \end{tikzpicture}
    \end{figure}
\end{minipage}

\begin{minipage}[t]{0.45\textwidth} 
    \begin{figure}[H]
        \centering 
        Il s'agit d'une symétrie \fillin[3cm]
        \begin{tikzpicture}[scale=0.8]
            \tikzset{
            homothety at/.style args={#1 scaled by #2}{shift={($(#1)!#2!(0,0)$)},scale=#2},}
            \draw [white](-2,-2) grid (3,3.5);
            \def\mypath{(-3,2) -- (-1,0) -- (1,3)--(-3,2)}
            \draw [thick]\mypath ;   
            \draw [cm={-1,0,0,1,(0,0)},dashed] \mypath;%Matrice de transformation inverse X et Y        
        \end{tikzpicture}
    \end{figure}
\end{minipage}
\hfill
\begin{minipage}[t]{0.45\textwidth}
    \begin{figure}[H]
        \centering 
        Il s'agit d'une symétrie \fillin[3cm]
        \begin{tikzpicture}[scale=0.8]
            \def\mypath{(-2,2) -- (-1,0) --(-2,0) -- (1,3)--(-2,2)}
            \draw [thick]\mypath ;
            \draw [white] (-2,-2)--(3,3) node [right]{$(d)$} ;
            \draw [cm={0,1,1,0,(0,0)},dashed] \mypath;%Matrice de transformation inverse X et Y
            \draw [opacity=0](-3,0) grid (4,2.5);
        \end{tikzpicture}
    \end{figure}
\end{minipage}

\exo{4}{Représenter} : 
Effectuer les symétries centrales de centre $O$.

\begin{minipage}[t]{0.45\textwidth} 
    \begin{figure}[H]
        \centering
        \begin{tikzpicture}[scale=1]
            \tikzset{
            homothety at/.style args={#1 scaled by #2}{shift={($(#1)!#2!(0,0)$)},scale=#2},}
            \draw [white](-4,-4) grid (3,3);
            \def\mypath{(0,0) -- (-2,3) --(-2,0) -- (-1,-2)--(0,0)}
            \draw [thick]\mypath ;
            \fill (1,0) coordinate (c) circle(2pt) node [above]{$O$};     
            \begin{scope}[homothety at=c scaled by -1]
                \draw[white] \mypath;
            \end{scope}        
        \end{tikzpicture}
    \end{figure}
\end{minipage}
\hfill
\begin{minipage}[t]{0.45\textwidth}
    \begin{figure}[H]
        \centering
        \begin{tikzpicture}[scale=1]
            \tikzset{
            homothety at/.style args={#1 scaled by #2}{shift={($(#1)!#2!(0,0)$)},scale=#2},}
            \draw [white](-3,-3) grid (4,4);
            \def\mypath{(-2,-2) -- (-1,1) -- (-1,0)--(1,-2)--(-2,-2)}
            \fill (-3,0) coordinate (c) circle(2pt) node [above]{$O$}; 
            \begin{scope}[homothety at=c scaled by -1]
                \draw[white] \mypath;
            \end{scope} 
            \draw [thick]\mypath ;    
        \end{tikzpicture}
    \end{figure}
\end{minipage}

% \begin{minipage}[t]{0.45\textwidth} 
%     \begin{figure}[H]
%         \centering
%         \begin{tikzpicture}[scale=1]
%             \tikzset{
%             homothety at/.style args={#1 scaled by #2}{shift={($(#1)!#2!(0,0)$)},scale=#2},}
%             \draw [white](-4,-4) grid (3,3);
%             \def\mypath{(-2,2) -- (-1,0) --(-2,0) -- (1,3)--(-2,2)}
%             \draw [thick]\mypath ;
%             \fill (-2,2) coordinate (c) circle(2pt) node [above]{$O$};     
%             \begin{scope}[homothety at=c scaled by -1]
%                 \draw[opacity=0] \mypath;
%             \end{scope}        
%         \end{tikzpicture}
%     \end{figure}
% \end{minipage}
% \hfill
% \begin{minipage}[t]{0.45\textwidth}
%     \begin{figure}[H]
%         \centering
%         \begin{tikzpicture}[scale=1]
%             \tikzset{
%             homothety at/.style args={#1 scaled by #2}{shift={($(#1)!#2!(0,0)$)},scale=#2},}
%             \draw [white](-3,-3) grid (4,4);
%             \def\mypath{(-2,-2) -- (-1,2)--(0,-2)--(-2,-2)}
%             \fill (-3,0) coordinate (c) circle(2pt) node [above]{$O$}; 
%             \begin{scope}[homothety at=c scaled by -1]
%                 \draw[white] \mypath;
%             \end{scope} 
%             \draw [thick]\mypath ;    
%         \end{tikzpicture}
%     \end{figure}
% \end{minipage}

\begin{minipage}[t]{0.45\textwidth} 
    \begin{figure}[H]
        \centering
        \begin{tikzpicture}[scale=1]
            \tikzset{
            homothety at/.style args={#1 scaled by #2}{shift={($(#1)!#2!(0,0)$)},scale=#2},}
            \def\mypath{(-2,2) -- (-1,0) --(-2,0) -- (1,3)--(-2,2)}
            \draw [thick]\mypath ;
            \fill (0,0) coordinate (c) circle(2pt) node [above]{$O$};     
            % \begin{scope}[homothety at=c scaled by -1]
            %     \draw[dashed] \mypath;
            % \end{scope}        
            \draw [dotted](-3,-3) grid (4,4);
        \end{tikzpicture}
    \end{figure}
\end{minipage}
\hfill
\begin{minipage}[t]{0.45\textwidth}
    \begin{figure}[H]
        \centering
        \begin{tikzpicture}[scale=1]
            \tikzset{
            homothety at/.style args={#1 scaled by #2}{shift={($(#1)!#2!(0,0)$)},scale=#2},}
            \def\mypath{(-2,-2) -- (-1,1) --(-2,0) -- (-1,0)--(0,-2)--(-2,-2)}
            \draw [thick]\mypath ;
            \fill (0,0) coordinate (c) circle(2pt) node [above]{$O$};     
            % \begin{scope}[homothety at=c scaled by -2]
            %     \draw[dashed] \mypath;
            % \end{scope}        
            \draw [dotted](-3,-3) grid (4,4);
        \end{tikzpicture}
    \end{figure}
\end{minipage}

\exo{4}{Raisonner}

Trouver le centre de chacune des symétries centrales suivantes.

\begin{minipage}[t]{0.45\textwidth}  
    \begin{figure}[H]
        \centering
		\begin{tikzpicture}[scale=0.8]
            \tikzset{
            homothety at/.style args={#1 scaled by #2}{shift={($(#1)!#2!(0,0)$)},scale=#2},}
            \def\mypath{(8,0)--(0,0)--(0,6)--(8,6)}
            \draw [white](0,0) grid (8,6);
            \draw [thick]\mypath ;
            \fill [white] (4.7,4.2) coordinate (c) circle(2pt);
            \begin{scope}[homothety at=c scaled by -1]
                \draw[dashed] \mypath;
            \end{scope}
        \end{tikzpicture}
    \end{figure}
\end{minipage}
\hfill
\begin{minipage}[t]{0.45\textwidth}  
    \begin{figure}[H]
        \centering
        \begin{tikzpicture}[scale=0.8]
            \tikzset{
            homothety at/.style args={#1 scaled by #2}{shift={($(#1)!#2!(0,0)$)},scale=#2},}
            \def\mypath{(7,0)--(4,1)--(4,0)--(0,0)--(0,3)--(4,1)}
            \draw [white](0,0) grid (8,8);
            \draw [thick]\mypath ;
            \fill [white] (4,2) coordinate (c) circle(2pt);
            \begin{scope}[homothety at=c scaled by -1]
                \draw[dashed] \mypath;
            \end{scope}
        \end{tikzpicture}
    \end{figure}
\end{minipage}

\begin{minipage}[t]{0.45\textwidth}  
    \begin{figure}[H]
        \centering
        \begin{tikzpicture}[scale=0.8]
            \tikzset{
            homothety at/.style args={#1 scaled by #2}{shift={($(#1)!#2!(0,0)$)},scale=#2},}
            \def\mypath{(2,0)--(2,3)--(3,4)--(2,0)}
            \draw [white](0,0) grid (8,5);
            \draw [thick]\mypath ;
            \fill [white] (0,0) coordinate (c) circle(2pt);
            \begin{scope}[homothety at=c scaled by -1]
                \draw[dashed] \mypath;
            \end{scope}
        \end{tikzpicture}
    \end{figure}
\end{minipage}
\hfill
\begin{minipage}[t]{0.45\textwidth}  
    \begin{figure}[H]
        \centering
        \begin{tikzpicture}[scale=0.8]
            \tikzset{
            homothety at/.style args={#1 scaled by #2}{shift={($(#1)!#2!(0,0)$)},scale=#2},}
            \def\mypath{(2,2)--(2,4)--(3,4)--(4,4)--(3,2)--(2,2)}
            \draw [white](0,0) grid (8,5);
            \draw [thick]\mypath ;
            \fill [white] (5,1) coordinate (c) circle(2pt);
            \begin{scope}[homothety at=c scaled by -1]
                \draw[dashed] \mypath;
            \end{scope}
        \end{tikzpicture}
    \end{figure}
\end{minipage}
\section*{Chapitre 10 : Retrouvailles avec les parallélogrammes}

\begin{enumerate}[leftmargin=0cm]
    \item Trouver quelles caractéristiques correspondent à quel quadrilatère :

\begin{multicols}{2}
\begin{enumerate}[label=\alph*.]
    \item Possède deux côtés parallèles
    \item Ses côtés opposés sont parallèles
    \item Possède 4 angles droits
    \item Possède 2 côtés égaux
    \item Possède 4 côtés égaux
    \item Ses côtés opposés sont égaux
    \item Possède deux côtés égaux et parallèles
    \item Ses diagonales se coupent en leurs milieux
    \item Ses diagonales ont la même longueur
    \item Ses diagonales sont perpendiculaires
\end{enumerate}
\end{multicols}


\newcommand{\hot}{\vspace*{12cm}}

\begin{tabularx}{\textwidth}{Y*{3}{|Y}}
    Parallélogramme & Rectangle & Losange & Carré \\\hline
     \hot &\hot&\hot&\hot\\
\end{tabularx}

\item Parmi ces caractéristiques, entoure celles qui suffisent à elles seules pour démontrer qu'il s'agit d'un parallélogramme/rectangle/losange/...

\item Complète les figures suivantes pour faire de $ABCD$ le quadrilatère voulu.

\begin{multicols}{3}
    $ABCD$ est un rectangle

    \begin{tikzpicture}
        \draw(0,0) node [below left]{A}--(4,0) node [below right]{B};
        \draw(0,3) node [above left]{D}--(4,3) node [above right]{C};
    \end{tikzpicture}

    $ABCD$ est un rectangle

    \begin{tikzpicture}
        \draw (0,0) node [below left]{A}--(4,0) node [below right]{B}--(4,3) node [above right] {C};
    \end{tikzpicture}

    $ABCD$ est un rectangle

    \begin{tikzpicture}
        \draw[dashed] (0,0) node [below left]{A}--(4,3) node [above right] {C};
    \end{tikzpicture}

    $ABCD$ est un parallélogramme

    \begin{tikzpicture}
        \draw (3,0) node [below right]{B}--(2,3) node [above right] {C};
        \draw (0,0) node [below left]{A}--(-1,3) node [above left]{D};
    \end{tikzpicture}

    $ABCD$ est un parallélogramme

    \begin{tikzpicture}
        \draw[white] (0,0) node [below left]{A}--(4,0) node [below right]{B}--(4,3) node [above right] {C};
        \draw (0,0) node [below left]{A}--(3,0) node [below right]{B}--(4,3) node [above right] {C};
    \end{tikzpicture}

    $ABCD$ est un parallélogramme

    \begin{tikzpicture}
        \draw [dashed] (0,0) node [below left]{A}--(4.5,3) node [above right] {C};
        \draw [dashed] (2.25,1.5) --(2.5,0) node [below right] {B};
    \end{tikzpicture}

    $ABCD$ est un carré

    \begin{tikzpicture}
        \draw[white] (0,0) node [below left]{A}--(4,0) node [below right]{B}--(4,3) node [above right] {C};
        \draw (0,0) node [below left]{A}--(3,0) node [below right]{B};
    \end{tikzpicture}

    $ABCD$ est un losange

    \begin{tikzpicture}
        \draw[white] (0,0) node [below left]{A}--(4,0) node [below right]{B}--(4,1.5) node [above right] {C};
        \draw (0,0) node [ left]{A}--(2,-1.5) node [below ]{B}--(4,0) node [ right] {C};
    \end{tikzpicture}

    $ABCD$ est un carré

    \begin{tikzpicture}
        \draw[white] (0,0) node [below left]{A}--(4,3) node [above right] {C};
        \draw[dashed] (1,0) node [below left]{A}--(3,1) node [above right] {C};
    \end{tikzpicture}
\end{multicols}
\end{enumerate}

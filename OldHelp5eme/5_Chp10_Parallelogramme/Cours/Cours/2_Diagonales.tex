\section{Diagonales}

\dfnt{Diagonales d'un quadrilatère}
{Dans un quadrilatère ABCD, les diagonales sont les segments [AC] et [BD].}

\prop{Déterminisme par diagonales}
{On considère un quadrilatère.
\begin{itemize}[leftmargin=0.5em]
    \item Si ses diagonales se coupent en leurs milieux, c'est un parallélogramme.
    \begin{multicols}{2}
        \item Si en plus elles font la même longueur,\\ c'est un rectangle.
        \item Si en plus elles sont perpendiculaires,\\ c'est un losange.
    \end{multicols}
\end{itemize}}
\rmq{Les réciproques sont aussi vraies : tous les rectangles ont leurs diagonales de même longueur.}
\exo{8}{Calculer} : Effectuer les calculs suivants.

\begin{multicols}{2}
    $$\dfrac{5}{7}+\dfrac{17}{14}$$\vspace*{-0.5em}\dotlines[0]{3}
    
    $$\dfrac{5}{2}+\dfrac{8}{7}$$\vspace*{-0.5em}\dotlines[0]{3}

    \columnbreak
    $$4-\dfrac{5}{8}$$\vspace*{-0.5em}\dotlines[0]{3}

    $$\dfrac{5}{8}-\dfrac{3}{7}-\dfrac{9}{56}$$\vspace*{-0.5em}\dotlines[0]{3}

\end{multicols}

\exo{6}{Représenter} : Donner les fractions associées aux positions suivantes.


    \fracdgmult[0]{4}{0}{3}{10}

    \fracdgmult[2]{3}{3}{13}{19}

    \fracdgmult[4]{5}{1}{3}{8}

\exo{6}{Modéliser} :    


Hisoka achète un paquet de 400 chewing gum. En rentrant chez lui :
\begin{itemize}
    \item Il en mange $\dfrac{2}{5}$.
    \item Il en donne $\dfrac{1}{8}$ à Kurapika.
    \item Son sac était troué et il en a perdu $\dfrac{1}{5}$
\end{itemize}
Combien de chewing gum lui reste-t-il ?
\dotlines[0]{6}


\exo{}{} :  Calculer  

$$1+\dfrac{1}{2}-\dfrac{1}{3}+\dfrac{1}{4}-\dfrac{1}{6}+\dfrac{1}{8}$$
\dotlines[0]{6}


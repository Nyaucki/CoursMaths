
\begin{minipage}[t]{0.45\textwidth}
    \exo{Modéliser}{Pb1}

Joe cuisine des coockies. 
\begin{itemize}
    \item Il en mange $\dfrac{1}{8}$
    \item Son frère en prend $\dfrac{1}{4}$
    \item Il en vend $\dfrac{1}{2}$
\end{itemize}
\begin{enumerate}
    \item Quelle fraction des coockies lui reste-t-il à la fin ?
    \item S'il avait cuisiné 40 coockies, combien en resterait-il ?
    \item S'il avait cuisiné 64 coockies, combien en resterait-il ?
\end{enumerate}
\end{minipage}
\hfil
\vrule
\hfil
\begin{minipage}[t]{0.45\textwidth}
    \exo{Modéliser}{Pb2}

Jack cuisine des crèpes. 
\begin{itemize}
    \item Il en mange $\dfrac{2}{3}$
    \item Son frère en prend $\dfrac{3}{9}$
    \item Il en vend $\dfrac{4}{27}$
\end{itemize}
\begin{enumerate}
    \item Quelle fraction des crèpes lui reste-t-il à la fin ?
    \item S'il avait cuisiné 270 crèpes, combien en resterait-il ?
    \item S'il avait cuisiné 81 crèpes, combien en resterait-il ?
\end{enumerate}
\end{minipage}

\begin{minipage}[t]{0.45\textwidth}
    \exo{Modéliser}{Pb3}

William cuisine des muffins. 
\begin{itemize}
    \item Il en mange $\dfrac{3}{7}$
    \item Son frère en prend $\dfrac{5}{42}$
    \item Il en vend $\dfrac{2}{6}$
\end{itemize}
\begin{enumerate}
    \item Quelle fractionrait des muffins lui reste-t-il à la fin ?
    \item S'il avait cuisiné 84 muffins, combien en resterait-il ?
    \item S'il avait cuisiné 210 muffins, combien en resterait-il ?
\end{enumerate}
\end{minipage}
\hfil
\vrule
\hfil
\begin{minipage}[t]{0.45\textwidth}
    \exo{Modéliser}{Pb4}

Averelle cuisine des tartines. 
\begin{itemize}
    \item Il en mange $\dfrac{5}{9}$
    \item Son frère en prend $\dfrac{3}{36}$
    \item Il en vend $\dfrac{3}{8}$
\end{itemize}
\begin{enumerate}
    \item Quelle fraction des tartines lui reste-t-il à la fin ?
    \item S'il avait cuisiné 144 tartines, combien en resterait-il ?
    \item S'il avait cuisiné 360 tartines, combien en resterait-il ?
\end{enumerate}
\end{minipage}


\begin{minipage}[t]{0.45\textwidth}
    \exo{Modéliser}{Pb5}

Fred a un troupeau de chèvres. 
\begin{itemize}
    \item $\dfrac{3}{8}$ vont dans les alpages.
    \item $\dfrac{1}{4}$ restent à la bergerie.
\end{itemize}
Est-il possible que $ \dfrac{2}{4}$ du troupeau soient à l'abrevoir ?
\end{minipage}
\hfil
\vrule
\hfil
\begin{minipage}[t]{0.45\textwidth}
    \exo{Modéliser}{Pb6}

Daphné a un troupeau de moutons. 
\begin{itemize}
    \item $\dfrac{3}{7}$ vont dans les alpages.
    \item $\dfrac{3}{14}$ restent à la bergerie.
\end{itemize}
Est-il possible que $ \dfrac{2}{7}$ du troupeau soient à l'abrevoir ?
\end{minipage}

\begin{minipage}[t]{0.45\textwidth}
    \exo{Modéliser}{Pb7}

Verra a un troupeau de moutons. 
\begin{itemize}
    \item $\dfrac{2}{5}$ vont dans les alpages.
    \item $\dfrac{4}{15}$ restent à la bergerie.
\end{itemize}
Est-il possible que $ \dfrac{2}{3}$ du troupeau soient à l'abrevoir ?
\end{minipage}
\hfil
\vrule
\hfil
\begin{minipage}[t]{0.45\textwidth}
    \exo{Modéliser}{Pb8}

Samy a un troupeau d'élèves. 
\begin{itemize}
    \item $\dfrac{2}{5}$ vont dans les alpages.
    \item $\dfrac{4}{15}$ restent à la bergerie.
\end{itemize}
Est-il possible que $ \dfrac{2}{3}$ du troupeau soient à l'abrevoir ?
\end{minipage}
\section*{Chapitre 2 : Symétries centrales - Plan de Travail}

\pdt[]{Rappel : Symétrie axiale}
{\begin{multicols}{4}
\begin{itemize}
    \itemindent=-25pt
        \item \exref{TracerSymAxQuadri1}
        \item \exref{TracerSymAxQuadri2}
        \item \exref{TracerSymAxQuadri3}
        \item \exref{TracerSymAxQuadri4}
        \item \exref{TracerSymAxBlanc1}
        \item \exref{TracerSymAxBlanc2}
        \item \exref{TracerSymAxBlanc3}
        \item \exref{TracerSymAxBlanc4}
        \item \exref{CentreSymAx1}
        \item \exref{CentreSymAx2}
        \item \exref{CentreSymAx3}
        \item \exref{CentreSymAx4}
    \end{itemize}
\end{multicols}}

\begin{plandetravailDS}
    Programme de l'interro :
    \begin{itemize}
        \item Revoir chapitre 1 : faire un calcul avec priorité (2 points) 
        \item Faire une symétrie sans quadrillage (2 points)
        \item Faire une symétrie avec quadrillage (4 points)
        \item Retrouver le centre d'une symétrie (2 points)
    \end{itemize}
\end{plandetravailDS}

\begin{minipage}[t]{0.5\textwidth}
    \pdt[6]{Symétrie centrale avec quadrillage}
    {\begin{multicols}{2}
    \begin{itemize}
        \itemindent=-25pt
            \item \exref{TracerSymCentQuadri1}
            \item \exref{TracerSymCentQuadri2}
            \item \exref{TracerSymCentQuadri3}
            \item \exref{TracerSymCentQuadri4}
            \item \exref{TracerSymCentQuadri5}
            \item \exref{TracerSymCentQuadri6}
            \item \exref{TracerSymCentQuadri7}
            \item \exref{TracerSymCentQuadri8}
        \end{itemize}
    \end{multicols}}
\end{minipage}
\hfill
\begin{minipage}[t]{0.5\textwidth}
    \pdt[4]{Symétrie centrale sans quadrillage}
    {\begin{multicols}{2}
    \begin{itemize}
        \itemindent=-25pt
            \item \exref{TracerSymCentBlanc1}
            \item \exref{TracerSymCentBlanc2}
            \item \exref{TracerSymCentBlanc3}
            \item \exref{TracerSymCentBlanc4}
            \item \exref{TracerSymCentBlanc5}
            \item \exref{TracerSymCentBlanc6}
            \item \exref{TracerSymCentBlanc7}
            \item \exref{TracerSymCentBlanc8}
        \end{itemize}
    \end{multicols}}
\end{minipage}

\pdt[4]{Retrouver le centre d'une symétrie}{
    \begin{multicols}{4}
        \begin{itemize}
            \itemindent=-25pt
                \item \exref{CentreSymCent1}
                \item \exref{CentreSymCent2}
                \item \exref{CentreSymCent3}
                \item \exref{CentreSymCent4}
                \item \exref{CentreSymCent5}
                \item \exref{CentreSymCent6}
                \item \exref{CentreSymCent7}
                \item \exref{CentreSymCent8}
            \end{itemize}
    \end{multicols}
}


\vspace{-1em}

\begin{minipage}[t]{0.5\textwidth}
    \vspace{-0.25em}
    \pdt[1]{Exercices plus difficiles}{
        \begin{multicols}{2}
            \begin{itemize}
                \itemindent=-25pt
                \item \exref{Dur1}
                \item \exref{Dur2}
            \end{itemize}
        \end{multicols}
    }
\end{minipage}  
\hfill
\begin{minipage}[t]{0.47\textwidth}
    \vspace{-0.25em}
    \textbf{Que mettre dans les cases ?}
    \begin{itemize}
        \item \textbf{TB} \textit{(Très bien)} Si tout est juste
        \item \textbf{B} \textit{(Bien)} J'ai le bon résultat, mais pas la bonne rédaction
        \item \textbf{AB} \textit{(Assez bien)} J'ai une faute, mais je peux comprendre avec la correction
        \item  \textbf{AA} \textit{(Avec de l'Aide)} Si j'ai eu besoin d'aide pour réussir l'exercice 
        \item \textbf{A} \textit{(Au secours!)} J'ai besoin que quelqu'un m'explique.
    \end{itemize}
\end{minipage}
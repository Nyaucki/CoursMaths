\section{Multiple et diviseurs}

\dfnt{diviseur et multiple}
{On considère deux nombres entiers $a$ et $b$
    On dit que $a$ est un diviseur de $b$ si le reste de la division euclidienne de $b$ par $a$ est 0.\\
    Dans ce cas, on peut aussi dire que $b$ est un multiple de $a$.}

\exmpl{ 4 est un diviseur de 20 et 20 est un multiple de 4.}


\prop{Autre caractérisation des diviseurs}
{$a$ est un diviseur de $b$ \ssi il existe un nombre $n$ tel que $a\times n = b$.}

\rmq {Pour n'importe quel nombre $n$, $n\times 1=n$. Donc :
\begin{itemize}
\item 1 est un diviseur de tous les nombres.
\item Tout nombre est son propre diviseur.
\end{itemize}}

\exmpl{ 4 est un diviseur de 20 parce que $4\times 5 =20$}


\prop{Critères de divisibilité}{Un nombre est divisible par : 
\begin{itemize}
    \item 2 s'il se termine par 0 ; 2 ; 4 ; 6 ou 8.
    \begin{itemize}
        \item \textcolor{gray}{122228 est divisible par 2 car il se termine par 8}
    \end{itemize}
    \item 3 si la somme de ses chiffres est divisible par 3.
    \begin{itemize}
        \item \textcolor{gray}{132465 est divisible par 3 car 1+3+2+4+6+5=21 qui est divisible par 3}
    \end{itemize}
    \item 5 s'il se termine par 0 ou 5.
    \begin{itemize}
        \item \textcolor{gray}{12225 est divisible par 5 car il se termine par 5}
    \end{itemize}
\end{itemize}}
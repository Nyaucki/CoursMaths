%%% Pour les commandes relatives aux fractions. Include dans les documents le nécessitant. %%%%


%%% Inscrire une fraction dans une droite graduée %%%%
\newcommand{\fracdg}[2]
    {\tikzmath{\num=#1 ; \den = #2 ;\invden= 1/\den ;}   
    \begin{figure}[H]
        \centering
        \begin{tikzpicture}
            \draw (0,0) node[below=0.2] {0} -- (5,0) node[below=0.2] {1} ;
            \foreach \x in {0,\invden,...,1}
            {
            \draw (5*\x,0.2) -- (5*\x,-0.2) ;
            }
            \draw[-Stealth] (5*\num/\den,0.6) -- (5*\num/\den,0.25);
        \end{tikzpicture} 
    \end{figure}}


%%% Inscrire une fraction dans une pizza %%%%
\newcommand{\fracpizza}[3][orange]
{
    \tikzmath{\num=#2 ; \den = #3 ;\invden= 1/\den ;} 
    \begin{figure}[H]
        \centering
        \begin{tikzpicture}
            \draw[white,fill=#1!30] (0,0)--(0:1) arc (0:360*\num/\den : 1)--(0,0);
            \draw (0,0) circle (1) ;
            \foreach \x in {0,\invden,...,1}
            {
            \draw (0,0) -- (360*\x : 1) ;
            }
        \end{tikzpicture} 
    \end{figure}
}

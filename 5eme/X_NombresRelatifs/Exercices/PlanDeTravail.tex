\section*{Chapitre 2 : Thalès - Plan de Travail}

\pdt[]{Rappels}{
    Exercices de rappels à faire pour se remettre à niveau (si besoin).
    \begin{multicols}{4}
        \textbf{Pythagore}
        \begin{itemize}
            \itemindent=-25pt
            \item \exref{Pythagore1}
            \item \exref{Pythagore2}
            \item \exref{Pythagore3}
            \item \exref{Pythagore4}
        \end{itemize}
        \textbf{Produits en croix}
        \begin{itemize}
            \itemindent=-25pt
            \item \exref{ProduitX1}
            \item \exref{ProduitX2}
            \item \exref{ProduitX3}
            \item \exref{ProduitX4}
        \end{itemize}
        \textbf{Conversion heure}
        \begin{itemize}
            \itemindent=-25pt
            \item \exref{Heures1}
            \item \exref{Heures2}
            \item \exref{Heures3}
            \item \exref{Heures4}
        \end{itemize}
        \textbf{Vitesse et distance}
        \begin{itemize}
            \itemindent=-25pt
            \item \exref{Vitesse1}
            \item \exref{Vitesse2}
            \item \exref{Vitesse3}
            \item \exref{Vitesse4}
        \end{itemize}
    \end{multicols}
}

\begin{plandetravailDS}
    \begin{itemize}
        \item Calcul numérique sur 4 points
        \item Application de cours sur 4 points
        \item Problème type brevet (\textit{Calcul}) sur 8 points
        \item Cas concret (\textit{Modéliser, communiquer}) sur 4 points
    \end{itemize}
\end{plandetravailDS}



\begin{minipage}[t]{0.5\textwidth}
    \pdt[2]{Application du théorème}{
        \begin{multicols}{2}
            \begin{itemize}
                \itemindent=-25pt
                \item \exref{Direct1}
                \item \exref{Direct2}
                \item \exref{Direct3}
                \item \exref{Direct4}
                \item \exref{Direct5}
                \item \exref{Direct6}
            \end{itemize}
        \end{multicols}
    }
\end{minipage}
\hfill
\begin{minipage}[t]{0.5\textwidth}
    \pdt[2]{Application de la réciproque}{
        \begin{multicols}{2}
            \begin{itemize}
                \itemindent=-25pt
                \item \exref{Reciproque1}
                \item \exref{Reciproque2}
                \item \exref{Reciproque3}
                \item \exref{Reciproque4}
                \item \exref{Reciproque5}
                \item \exref{Reciproque6}
            \end{itemize}
        \end{multicols}
    }
\end{minipage}

\pdt[3]{Exercices de calcul}{
    \begin{multicols}{4}
        \begin{itemize}
            \itemindent=-25pt
            \item \exref{Calcul1}
            \item \exref{Calcul2}
            \item \exref{Calcul3}
            \item \exref{Calcul4}
            \item \exref{Calcul5}
            \item \exref{Calcul6}
            \item \exref{Calcul7}
            \item \exref{Calcul8}
        \end{itemize}
    \end{multicols}
}

\begin{minipage}[t]{0.5\textwidth}
    \pdt[2]{Petits problèmes}{
        \begin{multicols}{2}
            \begin{itemize}
                \itemindent=-25pt
                \item \exref{Concret1}
                \item \exref{Concret2}
                \item \exref{Concret3}
            \end{itemize}
        \end{multicols}
    }
\end{minipage}
\hfill
\begin{minipage}[t]{0.5\textwidth}
    \pdt[2]{Problèmes type brevet}{
        \begin{multicols}{2}
            \begin{itemize}
                \itemindent=-25pt
                \item \exref{Brevet1}
                \item \exref{Brevet2}
                \item \exref{Brevet3}
                \item \exref{Brevet4}
            \end{itemize}
        \end{multicols}
    }
\end{minipage}  

\begin{minipage}[t]{0.5\textwidth}
    \pdt[]{Exercices plus difficiles}{
        \begin{multicols}{2}
            \begin{itemize}
                \itemindent=-25pt
                \item \exref{Dur1}
                \item \exref{Dur2}
                \item \exref{Dur3}
                \item \exref{Dur4}
                \item \exref{Dur5}
                \item \exref{Dur6}
            \end{itemize}
        \end{multicols}
    }
\end{minipage}  
\hfill
\begin{minipage}[t]{0.47\textwidth}
    \vspace{-6em}
    \textbf{Que mettre dans les cases ?}
    \begin{itemize}
        \item \textbf{TB} \textit{(Très bien)} Si tout est juste
        \item \textbf{B} \textit{(Bien)} J'ai le bon résultat, mais pas la bonne rédaction
        \item \textbf{AB} \textit{(Assez bien)} J'ai une faute, mais je peux comprendre avec la correction
        \item  \textbf{AA} \textit{(Avec de l'Aide)} Si j'ai eu besoin d'aide pour réussir l'exercice 
        \item \textbf{A} \textit{(Au secours!)} J'ai besoin que quelqu'un m'explique.
    \end{itemize}
\end{minipage}
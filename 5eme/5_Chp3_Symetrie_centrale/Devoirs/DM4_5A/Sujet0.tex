\begin{pycode}
from random import *
from sympy import *


x = Symbol('x')
n = Symbol('n')

#FONCTIONS AFFINES
def randex(a,b,sign=True):
    '''
    Renvoie un entier aléatoire n tel que |n| soit compris entre a et b
    '''
    r = randint(a,b)
    if sign :		
        r = (-1)**randint(1,2)*r
    return r

a1=randint(2,7)
b1=randint(2,6)
if a1==b1:
    c1=(a1+randint(1,2))%6
    cc1=(c1+1)
else :
    c1=1
    cc1=a1

e1=randint(2,5)

a2=randint(0,3)
b2=(a2+randint(1,2))%4
c2=(b2+randint(1,2))%4
d2=(c2+randint(1,2))%4
e2=(d2+randint(1,2))%4
f2=(e2+randint(1,2))%4
g2=(f2+randint(1,2))%4

a3=randint(2,7)
b3=randint(2,6)
if a3==b3:
    c3=(a3+randint(1,2))%6
    cc3=(c3+1)
else :
    c3=1
    cc3=a3

e3=randint(2,5)

a4=randint(0,3)
b4=(a4+randint(1,2))%4
c4=(b4+randint(1,2))%4
d4=(c4+randint(1,2))%4
e4=(d4+randint(1,2))%4
f4=(e4+randint(1,2))%4
g4=(f4+randint(1,2))%4

\end{pycode}

%%%%%%%%%%%%%%%%%%%%%%%%%%%%%%%%%%%%%%%%%%%%%%%%%%%%%%%%%%%%%%%%%

\hrulefill
\begin{figure}[H]
\centering
\begin{tabularx}{0.9\textwidth}{p{2cm}p{8cm}X}
5A & \textbf{Devoir Maison 4 - Pour le 04/10/2024} & Nom : \nom
\end{tabularx}
\end{figure}
\vspace{-1em}
\hrulefill

\begin{center}
    Si vous voulez aider \prenom , merci de ne pas juste lui donner la solution. 

    En prenant le temps de lui expliquer, vous l'aiderez beaucoup plus.
\end{center}


\medskip

%a1=\py{a1} b1=\py{b1} c1=\py{c1} cc1=\py{cc1} e1=\py{e1}




\begin{minipage}[t]{0.45\textwidth}
    Tracer le symétrique par rapport à la droite $(d)$
    
    \begin{figure}[H]
        \centering
        \begin{pysub} % Necessaire pour python dans tikz
            
            \begin{tikzpicture}[scale=0.85]
                \def\mypath{(!{a1},!{a1})--(!{b1},1)--(!{cc1},!{c1})--(7,!{e1})}
                \draw [thick]\mypath ;
                \draw (0,0)--(8,8) node [right]{$(d)$} ;
                % \draw [cm={0,1,1,0,(0,0)}] \mypath;%Matrice de transformation inverse X et Y
                \draw [dotted](0,0) grid (8,8);
            \end{tikzpicture}
        \end{pysub}
    \end{figure}
\end{minipage}
\hfill
\begin{minipage}[t]{0.45\textwidth}
    Tracer le symétrique par rapport à la droite $(d)$
    
    \begin{figure}[H]
        \centering
        \begin{pysub} % Necessaire pour python dans tikz
            
            \begin{tikzpicture}[scale=0.85]
                \def\mypath{(1,!{a2})--(2,!{b2})--(3,!{c2})--(4,!{d2})--(5,!{e2})--(6,!{f2})--(7,!{g2})}
                \draw [thick]\mypath ;
                \draw (0,0)--(8,0) node [above]{$(d)$} ;
                % \draw [cm={0,1,0,-1,(0,0)}] \mypath;%Matrice de transformation 
                \draw [dotted](0,-4) grid (8,4);
            \end{tikzpicture}
        \end{pysub}
    \end{figure}
\end{minipage}

\begin{minipage}[t]{0.45\textwidth}
    Tracer le symétrique par rapport à la droite $(d)$
    
    \begin{figure}[H]
        \centering
        \begin{pysub} % Necessaire pour python dans tikz
            
            \begin{tikzpicture}[scale=0.85]
                \def\mypath{(!{a3},!{a3})--(!{b3},1)--(!{cc3},!{c3})--(7,!{e3})}
                \draw [white](0,0) grid (8,8);
                \draw [thick]\mypath ;
                \draw (0,0)--(8,8) node [right]{$(d)$} ;
                % \draw [cm={0,1,1,0,(0,0)}] \mypath;%Matrice de transformation inverse X et Y
            \end{tikzpicture}
        \end{pysub}
    \end{figure}
\end{minipage}
\hfill
\begin{minipage}[t]{0.45\textwidth}
    Tracer le symétrique par rapport à la droite $(d)$
    
    \begin{figure}[H]
        \centering
        \begin{pysub} % Necessaire pour python dans tikz
            
            \begin{tikzpicture}[scale=0.85]
                \def\mypath{(1,!{a4})--(2,!{b4})--(3,!{c4})--(4,!{d4})--(5,!{e4})--(6,!{f4})--(7,!{g4})}
                \draw [white](0,-4) grid (8,4);
                \draw [thick]\mypath ;
                \draw (0,0)--(8,0) node [above]{$(d)$} ;
                % \draw [cm={0,1,0,-1,(0,0)}] \mypath;%Matrice de transformation 
            \end{tikzpicture}
        \end{pysub}
    \end{figure}
\end{minipage}

\newpage

\hrulefill
\begin{figure}[H]
\centering
\begin{tabularx}{0.9\textwidth}{p{2cm}p{8cm}X}
5A & \textbf{Devoir Maison 4 - Correction} & Nom : \nom
\end{tabularx}
\end{figure}
\vspace{-1em}
\hrulefill



\begin{minipage}[t]{0.45\textwidth}
    Tracer le symétrique par rapport à la droite $(d)$
    
    \begin{figure}[H]
        \centering
        \begin{pysub} % Necessaire pour python dans tikz
            
            \begin{tikzpicture}[scale=0.85]
                \def\mypath{(!{a1},!{a1})--(!{b1},1)--(!{cc1},!{c1})--(7,!{e1})}
                \draw [thick]\mypath ;
                \draw (0,0)--(8,8) node [right]{$(d)$} ;
                \draw [cm={0,1,1,0,(0,0)}] \mypath;%Matrice de transformation inverse X et Y
                \draw [dotted](0,0) grid (8,8);
            \end{tikzpicture}
        \end{pysub}
    \end{figure}
\end{minipage}
\hfill
\begin{minipage}[t]{0.45\textwidth}
    Tracer le symétrique par rapport à la droite $(d)$
    
    \begin{figure}[H]
        \centering
        \begin{pysub} % Necessaire pour python dans tikz
            
            \begin{tikzpicture}[scale=0.85]
                \def\mypath{(1,!{a2})--(2,!{b2})--(3,!{c2})--(4,!{d2})--(5,!{e2})--(6,!{f2})--(7,!{g2})}
                \draw [thick]\mypath ;
                \draw (0,0)--(8,0) node [above]{$(d)$} ;
                \draw [cm={1,0,0,-1,(0,0)}] \mypath;%Matrice de transformation 
                \draw [dotted](0,-4) grid (8,4);
            \end{tikzpicture}
        \end{pysub}
    \end{figure}
\end{minipage}

\begin{minipage}[t]{0.45\textwidth}
    Tracer le symétrique par rapport à la droite $(d)$
    
    \begin{figure}[H]
        \centering
        \begin{pysub} % Necessaire pour python dans tikz
            
            \begin{tikzpicture}[scale=0.85]
                \def\mypath{(!{a3},!{a3})--(!{b3},1)--(!{cc3},!{c3})--(7,!{e3})}
                \draw [white](0,0) grid (8,8);
                \draw [thick]\mypath ;
                \draw (0,0)--(8,8) node [right]{$(d)$} ;
                \draw [cm={0,1,1,0,(0,0)}] \mypath;%Matrice de transformation inverse X et Y
            \end{tikzpicture}
        \end{pysub}
    \end{figure}
\end{minipage}
\hfill
\begin{minipage}[t]{0.45\textwidth}
    Tracer le symétrique par rapport à la droite $(d)$
    
    \begin{figure}[H]
        \centering
        \begin{pysub} % Necessaire pour python dans tikz
            
            \begin{tikzpicture}[scale=0.85]
                \def\mypath{(1,!{a4})--(2,!{b4})--(3,!{c4})--(4,!{d4})--(5,!{e4})--(6,!{f4})--(7,!{g4})}
                \draw [white](0,-4) grid (8,4);
                \draw [thick]\mypath ;
                \draw (0,0)--(8,0) node [above]{$(d)$} ;
                \draw [cm={1,0,0,-1,(0,0)}] \mypath;%Matrice de transformation 
            \end{tikzpicture}
        \end{pysub}
    \end{figure}
\end{minipage}
\consigne{Deduction1}{Deduction6}Dire ce que l'on sait et ce que l'on peut deviner.

\begin{minipage}[t]{0.45\textwidth}
    \exo{Raisonner}{Deduction1}
    \begin{figure}[H]
        \center
        \begin{tikzpicture}[scale=1.25]
            \draw (0,0) coordinate(A1) node [left,above] {$(a)$}-- (6,0)coordinate (A2) ;
            \draw (3,-1) coordinate (B) node[left] {$(b)$} -- (3,3);
            \draw (5,-1) coordinate (C) node[left] {$(c)$} -- (5,3);
            \draw [gray,right angle quadrant=4,right angle symbol={A1}{A2}{B}];\draw [gray,right angle quadrant=4,right angle symbol={A1}{A2}{C}];
        \end{tikzpicture}
    \end{figure}
\end{minipage}
\hfill
\begin{minipage}[t]{0.45\textwidth}
    \exo{Raisonner}{Deduction2}
    \begin{figure}[H]
    \center
    \begin{tikzpicture}[scale=1.25]
        \draw (0,0) coordinate(A1) node [left,above] {$(a)$}-- (6,0)coordinate (A2) ;
        \draw (3,-1) coordinate (B) node[left] {$(b)$} -- (3,2);
        \draw (5,-1) coordinate (C) node[left] {$(c)$} -- (5,2);
        \draw [gray,right angle quadrant=4,right angle symbol={A1}{A2}{B}];
    \end{tikzpicture}

    $(b)$ et $(c)$ sont parallèles.
    \end{figure}
\end{minipage}

\begin{minipage}[t]{0.45\textwidth}
    \exo{Raisonner}{Deduction3}
    \begin{figure}[H]
        \center
        \begin{tikzpicture}[scale=1.25]
            \draw (0,0) coordinate(A1) node [left,above] {$(a)$}-- (6,0) coordinate (A2) ;
            \draw (0,1) coordinate(B1) node [left,above] {$(b)$}-- (6,1) coordinate (B2) ;
            \draw (0,2) coordinate(C1) node [left,above] {$(c)$}-- (6,2) coordinate (C2) ;
            \draw (0,3) coordinate(D1) node [left,above] {$(d)$}-- (6,3) coordinate (D2) ;
            \draw (2,-0.5) coordinate (E) node[left] {$(e)$} -- (2,1.5);
            \draw (4,1.5) coordinate (F) node[left] {$(f)$} -- (4,3.5);
            \draw [gray,right angle quadrant=4,right angle symbol={A1}{A2}{E}];
            \draw [gray,right angle quadrant=4,right angle symbol={C1}{C2}{F}];
            \draw [gray,right angle quadrant=4,right angle symbol={D1}{D2}{F}];
        \end{tikzpicture}

        $(a)//(b)$ et $(b)//(c)$.
    \end{figure}
\end{minipage}
\hfill
\begin{minipage}[t]{0.45\textwidth}
    \exo{Raisonner}{Deduction4}
    \begin{figure}[H]
    \center
    \begin{tikzpicture}[scale=1.25]
        \draw (1,0) coordinate(A) node [left] {$A$}-- (5,0)coordinate (B) node [right] {$B$} -- (5,3) coordinate (C) node [right] {$C$}--(1,3) coordinate (D) node [left] {$D$}--cycle;
        \draw (3,-0.5) coordinate (E) node [right] {$(e)$}--(3,2);
        \draw (-1,-0.5) coordinate (F) node [right] {$(f)$}--(-1,3.5);
        \draw [gray,right angle quadrant=4,right angle symbol={A}{B}{E}];
    \end{tikzpicture}

    $ABCD$ est un rectangle et $(f)//(AD)$.
    \end{figure}
\end{minipage}

\begin{minipage}[t]{0.45\textwidth}
    \exo{Raisonner}{Deduction5}
    \begin{figure}[H]
        \center
        \begin{tikzpicture}[scale=1.25]
            \draw (0,0) node [left,above] {$(a)$}-- (6,0)  ;
            \draw (1,-1) -- (1,2) node [right] {$(b)$} ;
            \draw (3,-1) -- (3,2) node [left] {$(c)$} ;
            \draw (0.5,-1)--(3.5,2) node [right] {$(d)$} ;
            \draw (2.5,-1) coordinate (E1) --(5.5,2) node [right] {$(e)$} ;
            \draw (6,-0.5)node [right] {$(f)$}--(4,1.5)  ;
            \node (A1) at (0,0) {};
            \node (A2) at (6,0) {};
            \node (B) at (1,-1) {};
            \node (C) at (3,-1) {};
            % \node (E1) at (2.5,-1) {};
            \node (E2) at (5.5,2) {};
            \node (F) at (4,1.5) {};
            \draw [gray,right angle quadrant=4,right angle symbol={A1}{A2}{B}];
            \draw [gray,right angle quadrant=4,right angle symbol={A1}{A2}{C}];
            \draw [gray,right angle quadrant=4,right angle symbol={E1}{E2}{F}];
        \end{tikzpicture}
        
        $(e)$ et $(d)$ sont parallèles.
    \end{figure}
\end{minipage}
\hfill
\begin{minipage}[t]{0.45\textwidth}
    \exo{Raisonner}{Deduction6}
    \begin{figure}[H]
        \center
        \begin{tikzpicture}[scale=1.25]
            \draw (0,0) coordinate(A1) node [left,above] {$(a)$}-- (6,-1)coordinate (A2)  ;
            \draw (0,-1) coordinate (B1) -- (5,2) coordinate (B2) node [right] {$(b)$} ;
            \draw (3,-1) coordinate (C) -- (3.5,2) node [left] {$(c)$} ;
            \draw (0.5,-1) coordinate (D) -- (1,2) node [left] {$(d)$} ;
            \draw (0,1.5) coordinate(E1) node [left,above] {$(e)$}-- (6,0.5)coordinate(E2) ;
            \draw (4,2) coordinate (F) -- (5.5,-0.5) node [right] {$(f)$} ;
            \draw (2,2) coordinate (G) node [right] {$(g)$}  -- (3.5,-0.5) ;
            \draw [gray,right angle quadrant=4,right angle symbol={A1}{A2}{C}];
            \draw [gray,right angle quadrant=4,right angle symbol={A1}{A2}{D}];
            \draw [gray,right angle quadrant=4,right angle symbol={E1}{E2}{D}];
            \draw [gray,right angle quadrant=4,right angle symbol={B1}{B2}{F}];
            \draw [gray,right angle quadrant=4,right angle symbol={B1}{B2}{G}];
        \end{tikzpicture}
    \end{figure}
\end{minipage}


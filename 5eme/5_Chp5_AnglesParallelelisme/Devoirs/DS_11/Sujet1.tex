\exo{4}{Représenter} : Compléter les propriétés suivantes.

\begin{minipage}{0.23\textwidth}
    \begin{figure}[H]
        \center
        \begin{tikzpicture}[scale=0.75]
            \draw (0,0) coordinate (A) -- (4,0) coordinate (B) ;
            \draw (1,1) coordinate (C) -- (3,-1) coordinate (D)  ;
            \node (O) at (2,0) {};
            \draw pic["$a$",draw=blue,fill=blue!20,angle eccentricity=1.3, angle radius=0.8cm]{angle=C--O--A};
            \draw pic["$b$",draw=blue,fill=blue!20,angle eccentricity=1.3, angle radius=0.8cm]{angle=D--O--B};
        \end{tikzpicture}
    \end{figure}
    Les angles $a$ et $b$ sont égaux car ce sont des angles \filling[2.95cm]
\end{minipage}
\hfil
\vline
\hfil
\begin{minipage}{0.23\textwidth}
    \begin{figure}[H]
        \center
        \begin{tikzpicture}
            \draw (1,0) coordinate (A1) -- (3.5,0) coordinate (A2) node[right] {$(a)$};
            \draw (1,1.5) coordinate (B1) -- (3.5,1.5) coordinate (B2) node[right] {$(b)$};
            \draw (1,-0.5) coordinate (C1) -- (3.5,2) coordinate (C2);
            \node (I1) at (1.5,0) {};
            \node (I2) at (3,1.5) {};
            \draw pic["$a$",draw=blue,fill=blue!20,angle eccentricity=1.3, angle radius=0.8cm]{angle=B1--I2--I1};
            \draw pic["$b$",draw=orange,fill=orange!20,angle eccentricity=1.3, angle radius=0.8cm]{angle=A2--I1--I2};
        \end{tikzpicture}
    \end{figure}
    \vspace*{-1em}
    Les angles $a$ et $b$ sont égaux car ce sont des angles \filling[2.95cm]
\end{minipage}
\hfil
\vline
\hfil
\begin{minipage}{0.23\textwidth}
    \begin{figure}[H]
        \center
        \begin{tikzpicture}
            \draw (-2,0) coordinate (A) -- (2,0) coordinate (B)--(0,3) coordinate (C)--cycle;
            \draw pic["$b$",draw=blue,fill=blue!20,angle eccentricity=1.3, angle radius=0.8cm]{angle=C--B--A};
            \draw pic["$a$",draw=orange,fill=orange!20,angle eccentricity=1.3, angle radius=0.9cm]{angle=B--A--C};
            \draw pic["$c$",draw=green,fill=green!20,angle eccentricity=1.3, angle radius=0.9cm]{angle=A--C--B};
        \end{tikzpicture}
    \end{figure}
    $a+b+c$=\filling[2cm]
\end{minipage}
\hfil
\vline
\hfil
\begin{minipage}{0.23\textwidth}
    \begin{figure}[H]
        \center
        \begin{tikzpicture}[scale=0.75]
            \draw (0,0) coordinate (A) -- (4,0) coordinate (B) ;
            \node (O) at (2,0) {};
            \draw pic["$a$",draw=blue,fill=blue!20,angle eccentricity=1.3, angle radius=0.8cm]{angle=B--O--A};
        \end{tikzpicture}
    \end{figure}
    $a$=\filling[2cm]
\end{minipage}

\exo{4}{Calculer} : Pour chacune des figures suivantes, déterminer l'angle $\alpha$.

\begin{minipage}[t]{0.45\textwidth}
    \begin{figure}[H]
        \center
        \begin{tikzpicture}
            \draw (0,0) coordinate (A1) -- (4,0) coordinate (A2) node[right] {$(a)$};
            \draw (0,1.5) coordinate (B1) -- (4,1.5) coordinate (B2) node[right] {$(b)$};
            \draw (0.5,-1) coordinate (C1) -- (4,2.5) coordinate (C2);
            \node (I1) at (1.5,0) {};
            \node (I2) at (3,1.5) {};
            \draw pic["$\alpha$",draw=blue,fill=blue!20,angle eccentricity=1.3, angle radius=0.8cm]{angle=C1--I2--B2};
            \draw pic["28",draw=orange,fill=orange!20,angle eccentricity=1.3, angle radius=0.8cm]{angle=A2--I1--I2};
        \end{tikzpicture}

        $\alpha$=\filling[2cm]
    \end{figure}
\end{minipage}
\hfill
\begin{minipage}[t]{0.45\textwidth}
    \begin{figure}[H]
        \center
        \begin{tikzpicture}
            \draw (0,0.5) coordinate (A1) -- (0,3.75) coordinate (A2) node[right] {$(a)$};
            \draw (1.5,0.5) coordinate (B1) node[right] {$(b)$} -- (1.5,3.95) coordinate (B2) ;
            \draw (-1,0.5) coordinate (C1) -- (2.5,4) coordinate (C2);
            \node (I1) at (0,1.5) {};
            \node (I2) at (1.5,3) {};
            \draw pic["$\alpha$",draw=blue,fill=blue!20,angle eccentricity=1.3, angle radius=0.8cm]{angle=C2--I2--B2};
            \draw pic["113",draw=orange,fill=orange!20,angle eccentricity=1.4, angle radius=0.8cm]{angle=A1--I1--I2};
        \end{tikzpicture}

        $\alpha$=\filling[2cm]
    \end{figure}
\end{minipage}

\begin{minipage}[t]{0.45\textwidth}
    \begin{figure}[H]
        \center
        \begin{tikzpicture}[scale=1.25]
            \draw (0,0) coordinate (A1) -- (4,0) coordinate (A2) node[right] {$(a)$};
            \draw (0,1.5) coordinate (B1) -- (4,1.5) coordinate (B2) node[right] {$(b)$};
            \draw (0.5,-1) coordinate (C1) -- (4,2.5) coordinate (C2);
            \draw (1.5,-1) coordinate (D1) --(1.5,2.5) coordinate (D2) node [left] {$(d)$};
            \node (I1) at (1.5,0) {};
            \node (I2) at (3,1.5) {};
            \draw [gray,right angle quadrant=2,right angle symbol={A1}{A2}{D2}];
            \draw [gray,right angle quadrant=2,right angle symbol={B1}{B2}{D2}];
            \draw pic["$\alpha$",draw=blue,fill=blue!20,angle eccentricity=1.3, angle radius=0.8cm]{angle=C1--I2--B2};
            \draw pic["42",draw=orange,fill=orange!20,angle eccentricity=1.3, angle radius=0.8cm]{angle=C1--I1--D1};
        \end{tikzpicture}

        $\alpha$=\filling[2cm]
    \end{figure}
\end{minipage}
\hfill
\begin{minipage}[t]{0.45\textwidth}
    \begin{figure}[H]
        \center
        \begin{tikzpicture}[scale=1.25]
            \draw (0,0.5) coordinate (A1) -- (0,3.75) coordinate (A2) node[right] {$(a)$};
            \draw (1.5,0.5) coordinate (B1) node[right] {$(b)$} -- (1.5,3.75) coordinate (B2) ;
            \draw (-1,0.5) coordinate (C1) -- (2.5,4) coordinate (C2);
            \draw (-1,1.5) coordinate (D1) --(2.5,1.5) coordinate (D2) node [below] {$(d)$};
            \node (I1) at (0,1.5) {};
            \node (I2) at (1.5,3) {};
            \draw [gray,right angle quadrant=2,right angle symbol={A1}{A2}{D2}];
            \draw [gray,right angle quadrant=2,right angle symbol={B1}{B2}{D2}];
            \draw pic["$\alpha$",draw=blue,fill=blue!20,angle eccentricity=1.3, angle radius=0.8cm]{angle=D2--I1--I2};
            \draw pic["103",draw=orange,fill=orange!20,angle eccentricity=1.4, angle radius=0.8cm]{angle=B1--I2--C2};
        \end{tikzpicture}

        $\alpha$=\filling[2cm]
    \end{figure}
\end{minipage}

\newpage
\exo{4}{Raisonner} : Pour chacune des figures, dire si $(a)$ et $(b)$ sont parallèles.
\vspace*{-1em}

\begin{minipage}[t]{0.45\textwidth}
    \begin{figure}[H]
        \center
        \begin{tikzpicture}
            \draw (0,0.5) coordinate (A1) -- (0,3.75) coordinate (A2) node[right] {$(a)$};
            \draw (1.5,0.5) coordinate (B1) node[right] {$(b)$} -- (1.5,3.95) coordinate (B2) ;
            \draw (-1,0.5) coordinate (C1) -- (2.5,4) coordinate (C2);
            \node (I1) at (0,1.5) {};
            \node (I2) at (1.5,3) {};
            \draw pic["$56$",draw=blue,fill=blue!20,angle eccentricity=1.3, angle radius=0.8cm]{angle=C2--I2--B2};
            \draw pic["46",draw=orange,fill=orange!20,angle eccentricity=1.3, angle radius=0.8cm]{angle=I2--I1--A2};
        \end{tikzpicture}

        \filling[3cm]
    \end{figure}
\end{minipage}
\hfill
\begin{minipage}[t]{0.45\textwidth}
    \begin{figure}[H]
        \center
        \begin{tikzpicture}
            \draw (0,0) coordinate (A1) -- (2,2) coordinate (A2) node[right] {$(a)$};
            \draw (2.5,0) coordinate (B1) -- (4.5,2) coordinate (B2) node[right] {$(b)$};
            \draw (0,1) coordinate (C1) -- (4.5,1) coordinate (C2);
            \node (I1) at (1,1) {};
            \node (I2) at (3.5,1) {};
            \draw pic["$165$",draw=blue,fill=blue!20,angle eccentricity=1.5, angle radius=0.8cm]{angle=B1--I2--C2};
            \draw pic["15",draw=orange,fill=orange!20,angle eccentricity=1.3, angle radius=0.8cm]{angle=C1--I1--A1};
        \end{tikzpicture}

        \filling[3cm]
    \end{figure}
\end{minipage}

\begin{minipage}[t]{0.45\textwidth}
    \begin{figure}[H]
        \center
        \begin{tikzpicture}[scale=1.25]
            \draw (0,0) coordinate (A1) -- (4,0) coordinate (A2) node[right] {$(a)$};
            \draw (0,1.5) coordinate (B1) -- (4,1.5) coordinate (B2) node[right] {$(b)$};
            \draw (0.5,-1) coordinate (C1) -- (4,2.5) coordinate (C2);
            \draw (2.5,-1) coordinate (D1) --(2.5,2.5) coordinate (D2) node [left] {$(d)$};
            \node (I1) at (1.5,0) {};
            \node (I2) at (3,1.5) {};
            \node (I3) at (2.5,1) {};
            \draw [gray,right angle quadrant=4,right angle symbol={A1}{A2}{D2}];
            \draw pic["$111$",draw=blue,fill=blue!20,angle eccentricity=1.3, angle radius=0.8cm]{angle=C1--I2--B2};
            \draw pic["21",draw=orange,fill=orange!20,angle eccentricity=1.3, angle radius=0.8cm]{angle=C1--I3--D1};
        \end{tikzpicture}

        \filling[3cm]

    \end{figure}
\end{minipage}
\hfill
\begin{minipage}[t]{0.45\textwidth}
    \begin{figure}[H]
        \center
        \begin{tikzpicture}[scale=1.25]
            \draw (0,0.5) coordinate (A1) -- (0,3.75) coordinate (A2) node[right] {$(a)$};
            \draw (1.5,0.5) coordinate (B1) node[right] {$(b)$} -- (1.5,3.75) coordinate (B2) ;
            \draw (-1,0.5) coordinate (C1) -- (2.5,4) coordinate (C2);
            \draw (-1,2.5) coordinate (D1) --(2.5,2.5) coordinate (D2) node [below] {$(d)$};
            \node (I1) at (0,1.5) {};
            \node (I2) at (1.5,3) {};
            \node (I3) at (1,2.5) {};
            \draw [gray,right angle quadrant=4,right angle symbol={A1}{A2}{D2}];
            \draw pic["$37$",draw=blue,fill=blue!20,angle eccentricity=1.3, angle radius=0.8cm]{angle=D1--I3--I1};
            \draw pic["53",draw=orange,fill=orange!20,angle eccentricity=1.4, angle radius=0.8cm]{angle=C2--I2--B2};
        \end{tikzpicture}

        \filling[3cm]

    \end{figure}
\end{minipage}

\exo{4}{Calculer} Calculer en détaillant les étapes.

\begin{multicols}{2}
$13+(8\times 5 +4)\div (3+1)-1+10$
\vspace*{13em}
\columnbreak

$23-3\times 4 + 2 + 10\div 5 \times 2 $
\vspace*{13em}
\end{multicols}

\vspace*{-1em}

\exo{4}{Chercher} Déterminer  $\alpha$. $ABCD$ est un rectangle, $(DF)//(EB)$ et $(AE)//(FC)$.


\begin{figure}[H]
    \center
    \begin{tikzpicture}[scale=1.25]
        \draw (0,0) coordinate (A) node [left] {$A$} -- (12,0) coordinate (B) node [right] {$B$} -- (12,4) coordinate (C) node [right] {$C$}-- (0,4) coordinate (D) node [left] {$D$}--cycle;
        \draw (A) -- (5,4) coordinate (E) node [above] {$E$}--(B);
        \draw (D) -- (7,0) coordinate (F) node [below] {$F$}--(C);
        \node (I) at (9.1,1.67) {};
        \draw pic["$60$",draw=blue,fill=blue!20,angle eccentricity=1.3, angle radius=0.8cm]{angle=E--A--D};
        \draw pic["40",draw=orange,fill=orange!20,angle eccentricity=1.4, angle radius=0.8cm]{angle=F--D--E};
        \draw pic["$\alpha$",draw=green,fill=green!20,angle eccentricity=1.4, angle radius=0.8cm]{angle=B--I--C};
    \end{tikzpicture}

    \filling[3cm]

\end{figure}
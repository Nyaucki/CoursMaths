\consigne{AireFacile1}{AireFacile6} Calculer les aires des figures suivantes (en traits pleins). Tous les quadrilatères sont des parallélogrammes.

\begin{minipage}[t]{0.45\textwidth}
    \exo{Raisonner}{AireDur1}

    \begin{figure}[H]
        \centering
        \begin{tikzpicture}
            \node at (-0.1,0) (A) {};
            \draw (0,0)--(3,0) coordinate (B) node[midway,above] {4} ;
            \draw (0,0)--(0,-2) coordinate (C) node[midway,left] {3} ;
            \draw (0,-2)--(3,-2) coordinate (D);
            \draw (3,-2)--(3,0) ;
            \draw (3,0)--(4,1)--(4,-1)--(3,-2);
            \draw [dotted] (3,-0.5)-- node [midway, above] {1} (4,-0.5) coordinate (E);
            \draw [gray,right angle quadrant=1,right angle symbol={A}{B}{C}];
            \draw [gray,right angle quadrant=2,right angle symbol={D}{B}{E}];
        \end{tikzpicture}
    \end{figure}
\end{minipage}
\hfill
\begin{minipage}[t]{0.45\textwidth}
    \exo{Raisonner}{AireDur2}

    \begin{figure}[H]
        \centering
        \begin{tikzpicture}
            \draw (0,0)--node [midway,left] {2} (0,3)--(2,4) coordinate (A)--(2,1) coordinate (B)--cycle;
            \draw (2,4)--(5,3)--(5,0)--(2,1);
            \draw [dotted] (0,2) coordinate (C)--node [midway, above]{3}(2,2)--node [midway, above]{4} (5,2);
            \draw [gray,right angle quadrant=1,right angle symbol={A}{B}{C}];
        \end{tikzpicture}
    \end{figure}
\end{minipage}

\begin{minipage}[t]{0.45\textwidth}
    \exo{Raisonner}{AireDur3}

    \begin{figure}[H]
        \centering
        \begin{tikzpicture}
            \draw [dotted] (0,0)--(3,0)--(3,2)--(0,2)--cycle ;
            \draw (0,0) coordinate (A)-- node [midway, below] {8} (5,0) coordinate (B)-- node[sloped, rotate=-40] {$\parallel$}(5,2)-- node[sloped, rotate=-40] {$\parallel$}(3,2) coordinate (C)--(3,4) node [midway, right] {4}-- node [midway, above] {6}(0,4)-- cycle ;
            \draw [gray,right angle quadrant=1,right angle symbol={A}{B}{C}];
        \end{tikzpicture}
    \end{figure}
\end{minipage}
\hfill
\begin{minipage}[t]{0.45\textwidth}
    \exo{Raisonner}{AireDur4}

    \begin{figure}[H]
        \centering
        \begin{tikzpicture}
            \draw [dotted] (0,0) coordinate (A) -- node [midway, above] {1} (1,0) coordinate (M);
            \draw [dotted] (M) -- (8,0) coordinate (B);
            \draw [dotted] (2,0) coordinate (C) --  node [midway,right]{2} (2,2) coordinate (D);
            \draw [dotted] (A) -- (1,2) coordinate (E) -- (2,0) coordinate (O)-- (1,-2) coordinate (F)--(A);
            \draw [dotted] (E)--(5,2) coordinate (G)--(O);
            \draw [dotted] (G)--(B)--(5,-2) coordinate (I)--(O);
            \draw [dotted] (3,-0.6666)coordinate (J)--(2,-2.6666)coordinate (K)--(F);
            \draw [dotted] (E)--(F);
            \draw [dotted] (G)--(I);
            \draw [dotted] (1.5,-1) coordinate (L)--node [midway, above] {1} ($(J)!(L)!(K)$);
            \draw (A)--(E)--node [midway, above] {4} (G)--(B)--(I)-- (J)--(K)--(F)-- node [midway, below left] {3} cycle;
            \draw [gray,right angle quadrant=1,right angle symbol={J}{K}{L}];
            \draw [gray,right angle quadrant=1,right angle symbol={A}{B}{E}];
            \draw [gray,right angle quadrant=1,right angle symbol={A}{B}{G}];
            \draw [gray,right angle quadrant=1,right angle symbol={E}{G}{O}];
        \end{tikzpicture}
    \end{figure}
\end{minipage}

\textbf{Pour les exercices \ref{PlacerDroite1} à \ref{PlacerDroite8} :} Placer chacun des nombres sur la droite graduée.

\begin{minipage}[t]{0.45\textwidth}
    \exo{Représenter}{PlacerDroite1}
    
    $$-2 ~;~ -4 ~;~ 2 ~;~ 3$$    
    
    \tikzmath{\ya =0; \scl=1; \rcl= 4; \dclg =1; \pas=1/\scl ; \yb =\ya +\dclg * \pas; \dprt = \ya - \rcl * \pas; \y2 =\dprt +\pas; \fin =\dprt +8*\pas ; \grad = 0.1/\scl ; } %modifier yA, scl (scale), rcl (reculer) et dclg (decalage )uniquelent
    
    \begin{figure}[H]
        \centering
        \begin{tikzpicture}[scale=\scl]
            \draw (\dprt,0) -- (\fin,0) node[midway, sloped]{};
            \foreach \x in {\dprt,\y2,...,\fin}
            {
              \draw (\x,\grad) -- (\x,-\grad) ;
            }
            \node (A) at (\ya,-\grad) [below] {\pgfmathprintnumber[use comma]{\ya}} ;
            \node (B) at (\yb,-\grad) [below] {\pgfmathprintnumber[use comma,precision=3]{\yb}} ;
        \end{tikzpicture} 
    \end{figure}
  \end{minipage}
  \hfill
  \begin{minipage}[t]{0.45\textwidth}
    \exo{Représenter}{PlacerDroite2}
    
    $$ 3 ~;~ -4 ~;~ -2 ~;~ 3$$ 
    
    \tikzmath{\ya =-1; \scl=1; \rcl= 4; \dclg =1; \pas=1/\scl ; \yb =\ya +\dclg * \pas; \dprt = \ya - \rcl * \pas; \y2 =\dprt +\pas; \fin =\dprt +8*\pas ; \grad = 0.1/\scl ; } %modifier yA, scl (scale), rcl (reculer) et dclg (decalage )uniquelent
    
    \begin{figure}[H]
        \centering
        \begin{tikzpicture}[scale=\scl]
            \draw (\dprt,0) -- (\fin,0) node[midway, sloped]{};
            \foreach \x in {\dprt,\y2,...,\fin}
            {
              \draw (\x,\grad) -- (\x,-\grad) ;
            }
            \node (A) at (\ya,-\grad) [below] {\pgfmathprintnumber[use comma]{\ya}} ;
            \node (B) at (\yb,-\grad) [below] {\pgfmathprintnumber[use comma,precision=4]{\yb}} ;
        \end{tikzpicture} 
    \end{figure}
  \end{minipage}

  \begin{minipage}[t]{0.45\textwidth}
    \exo{Représenter}{PlacerDroite3}
    
    $$-2 ~;~ 1 ~;~ -5 ~;~ -7$$    
    
    \tikzmath{\ya =-4; \scl=1; \rcl= 3; \dclg =1; \pas=1/\scl ; \yb =\ya +\dclg * \pas; \dprt = \ya - \rcl * \pas; \y2 =\dprt +\pas; \fin =\dprt +8*\pas ; \grad = 0.1/\scl ; } %modifier yA, scl (scale), rcl (reculer) et dclg (decalage )uniquelent
    
    \begin{figure}[H]
        \centering
        \begin{tikzpicture}[scale=\scl]
            \draw (\dprt,0) -- (\fin,0) node[midway, sloped]{};
            \foreach \x in {\dprt,\y2,...,\fin}
            {
              \draw (\x,\grad) -- (\x,-\grad) ;
            }
            \node (A) at (\ya,-\grad) [below] {\pgfmathprintnumber[use comma]{\ya}} ;
            \node (B) at (\yb,-\grad) [below] {\pgfmathprintnumber[use comma,precision=3]{\yb}} ;
        \end{tikzpicture} 
    \end{figure}
  \end{minipage}
  \hfill
  \begin{minipage}[t]{0.45\textwidth}
    \exo{Représenter}{PlacerDroite4}
    
    $$ -3 ~;~ -4 ~;~ -2 ~;~ -5$$ 
    
    \tikzmath{\ya =1; \scl=1; \rcl= 6; \dclg =1; \pas=1/\scl ; \yb =\ya +\dclg * \pas; \dprt = \ya - \rcl * \pas; \y2 =\dprt +\pas; \fin =\dprt +8*\pas ; \grad = 0.1/\scl ; } %modifier yA, scl (scale), rcl (reculer) et dclg (decalage )uniquelent
    
    \begin{figure}[H]
        \centering
        \begin{tikzpicture}[scale=\scl]
            \draw (\dprt,0) -- (\fin,0) node[midway, sloped]{};
            \foreach \x in {\dprt,\y2,...,\fin}
            {
              \draw (\x,\grad) -- (\x,-\grad) ;
            }
            \node (A) at (\ya,-\grad) [below] {\pgfmathprintnumber[use comma]{\ya}} ;
            \node (B) at (\yb,-\grad) [below] {\pgfmathprintnumber[use comma,precision=4]{\yb}} ;
        \end{tikzpicture} 
    \end{figure}
  \end{minipage}

  \begin{minipage}[t]{0.45\textwidth}
    \exo{Représenter}{PlacerDroite5}
    
    $$-0,2 ~;~ -0,4 ~;~0,3 $$    
    
    \tikzmath{\ya =0; \scl=10; \rcl= 4; \dclg =1; \pas=1/\scl ; \yb =\ya +\dclg * \pas; \dprt = \ya - \rcl * \pas; \y2 =\dprt +\pas; \fin =\dprt +8*\pas ; \grad = 0.1/\scl ; } %modifier yA, scl (scale), rcl (reculer) et dclg (decalage )uniquelent
    
    \begin{figure}[H]
        \centering
        \begin{tikzpicture}[scale=\scl]
            \draw (\dprt,0) -- (\fin,0) node[midway, sloped]{};
            \foreach \x in {\dprt,\y2,...,\fin}
            {
              \draw (\x,\grad) -- (\x,-\grad) ;
            }
            \node (A) at (\ya,-\grad) [below] {\pgfmathprintnumber[use comma]{\ya}} ;
            \node (B) at (\yb,-\grad) [below] {\pgfmathprintnumber[use comma,precision=4]{\yb}} ;
        \end{tikzpicture} 
    \end{figure}
  \end{minipage}
  \hfill
  \begin{minipage}[t]{0.45\textwidth}
    \exo{Représenter}{PlacerDroite6}
    
    $$ -0,7 ~;~ -1,2 ~;~ -1,4$$ 
    
    \tikzmath{\ya =-1; \scl=10; \rcl= 4; \dclg =1; \pas=1/\scl ; \yb =\ya +\dclg * \pas; \dprt = \ya - \rcl * \pas; \y2 =\dprt +\pas; \fin =\dprt +8*\pas ; \grad = 0.1/\scl ; } %modifier yA, scl (scale), rcl (reculer) et dclg (decalage )uniquelent
    
    \begin{figure}[H]
        \centering
        \begin{tikzpicture}[scale=\scl]
            \draw (\dprt,0) -- (\fin,0) node[midway, sloped]{};
            \foreach \x in {\dprt,\y2,...,\fin}
            {
              \draw (\x,\grad) -- (\x,-\grad) ;
            }
            \node (A) at (\ya,-\grad) [below] {\pgfmathprintnumber[use comma]{\ya}} ;
            \node (B) at (\yb,-\grad) [below] {\pgfmathprintnumber[use comma,precision=4]{\yb}} ;
        \end{tikzpicture} 
    \end{figure}
  \end{minipage}

  \begin{minipage}[t]{0.45\textwidth}
    \exo{Représenter}{PlacerDroite7}
    
    $$-3,97 ~;~ -4,02 ~;~ -3,95 $$    
    
    \tikzmath{\ya =-4; \scl=100; \rcl= 3; \dclg =1; \pas=1/\scl ; \yb =\ya +\dclg * \pas; \dprt = \ya - \rcl * \pas; \y2 =\dprt +\pas; \fin =\dprt +8*\pas ; \grad = 0.1/\scl ; } %modifier yA, scl (scale), rcl (reculer) et dclg (decalage )uniquelent
    
    \begin{figure}[H]
        \centering
        \begin{tikzpicture}[scale=\scl]
            \draw (\dprt,0) -- (\fin,0) node[midway, sloped]{};
            \foreach \x in {\dprt,\y2,...,\fin}
            {
              \draw (\x,\grad) -- (\x,-\grad) ;
            }
            \node (A) at (\ya,-\grad) [below] {\pgfmathprintnumber[use comma]{\ya}} ;
            \node (B) at (\yb,-\grad) [below] {\pgfmathprintnumber[use comma,precision=3]{\yb}} ;
        \end{tikzpicture} 
    \end{figure}
  \end{minipage}
  \hfill
  \begin{minipage}[t]{0.45\textwidth}
    \exo{Représenter}{PlacerDroite8}
    
    $$ -1,15 ~;~ -1,13 ~;~ -1,17 $$ 
    
    \tikzmath{\ya =-1.11; \scl=100; \rcl= 6; \dclg =1; \pas=1/\scl ; \yb =\ya +\dclg * \pas; \dprt = \ya - \rcl * \pas; \y2 =\dprt +\pas; \fin =\dprt +8*\pas ; \grad = 0.1/\scl ; } %modifier yA, scl (scale), rcl (reculer) et dclg (decalage )uniquelent
    
    \begin{figure}[H]
        \centering
        \begin{tikzpicture}[scale=\scl]
            \draw (\dprt,0) -- (\fin,0) node[midway, sloped]{};
            \foreach \x in {\dprt,\y2,...,\fin}
            {
              \draw (\x,\grad) -- (\x,-\grad) ;
            }
            \node (A) at (\ya,-\grad) [below] {\pgfmathprintnumber[use comma]{\ya}} ;
            \node (B) at (\yb,-\grad) [below] {\pgfmathprintnumber[use comma,precision=4]{\yb}} ;
        \end{tikzpicture} 
    \end{figure}
  \end{minipage}

%% Exercices avec repère %%

\textbf{Exercice \hspace*{1cm} / 12 points : }À l'aide du repère ci-dessous :

\newcommand{\ax}{-1}
\newcommand{\ay}{5}
\newcommand{\bx}{3}
\newcommand{\by}{7}
\newcommand{\cx}{2}
\newcommand{\cy}{1}
\newcommand{\dx}{-7}
\newcommand{\dy}{2}


\begin{figure}[H]
    \center
    \begin{tikzpicture}
        \draw[dotted] (-8,-5) grid (8,7);
        \draw [ultra thick,->] (0,-5)-- (0,7);
        \draw [ultra thick,->] (-8,0)-- (8,0);
        \node at (0,0) [below  left]{0};
        \draw (1,0.2)--(1,-0.2) node [below] {1};
        \draw (0.2,1)--(-0.2,1) node [left] {1};>
        \fill (\ax , \ay) coordinate (A) circle(1.5pt)node [above  left]{$A$};
        % \fill[red] (\bx , \by) coordinate (B) circle(1.5pt)node [below  left]{$B$};
        % \fill[red] (\cx , \cy) coordinate (C) circle(1.5pt)node [below  left]{$C$};
        % \fill[red] (\dx , \dy) coordinate (D) circle(1.5pt)node [below  left]{$D$};
        % \fill[red] ($(D)!1!-90:(A)$)  coordinate (E) circle(1.5pt)node  [below  left]{$E$};
        % \fill[red] ($(A)!2!(C)$) coordinate (A')  circle(1.5pt)node [below  right]{$A'$};
        % \fill[red] ($(B)!2!(C)$) coordinate (B')  circle(1.5pt)node [above  left]{$B'$};
        % \draw [red,dashed] (A)--(E)--(D)--(A)--(B)--(A')--(B')--(A);
    \end{tikzpicture}
\end{figure}

\begin{enumerate}
    \item Placer les points suivants (\hspace*{1cm}/3 points) :
        \begin{multicols}{3}
            \begin{itemize}
                \item $B~(\bx;\by)$
                \item $C~(\cx;\cy)$
                \item $D~(\dx;\dy)$
            \end{itemize}
        \end{multicols}
    \item Placer les points suivants (\hspace*{1cm}/3 points) : 
    \begin{itemize}
        \item $E$ tel que le triangle $ADE$ soit rectangle et isocèle en $D$.
        \item $[A'B']$ l'image de $[AB]$ par la symétrie centrale de centre $C$.
    \end{itemize}
    \item Quelle est la nature du quadrilatère $ABA'B'$. Justifier. (\hspace*{1cm}/3 points) :
    \foreach \m in {0,...,1}
    {
        \\*[0.3cm]
    
        \dotfill
    }
    \item L'angle $\widehat{B'A'B}$ mesure 100°, quelle est la mesure de l'angle $\widehat{B'AE}$ ? Justifier. (\hspace*{1cm}/3 points) :
    \foreach \m in {0,...,1}
    {
        \\*[0.3cm]
    
        \dotfill
    }
\end{enumerate}
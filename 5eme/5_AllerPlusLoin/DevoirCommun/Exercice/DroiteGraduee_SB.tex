

%%%%% DEVOIR 6 EME %%%%

\newpage

\newcommand{\fracdgmult}[5][1]{
% #1 : Combien de grd avant le départ
% #2 : Dénominateur
% #3 : Graduation connue
% #4 : Numérateur 1
% #5 : Numérateur 2
\tikzmath{\den=#2; \ya =#3;  \rcl= #1 ; \pas=1/\den ; \yb =\ya +1; \dprt = \ya - \rcl * \pas; \y2 =\dprt +\pas; \fin =\dprt +17*\pas ; \grad = 0.1/\den ; } %modifier yA, scl (scale), rcl (reculer) et dclg (decalage )uniquelent
\begin{figure}[H]
    \centering
    \begin{tikzpicture}[scale=\den]
        \draw (\dprt,0) -- (\fin,0) node[midway, sloped]{};
        \foreach \x in {\dprt,\y2,...,\fin}
        {
          \pgfmathparse{int(Mod(\x * \den +\pas ,\den))}
          \ifnum\pgfmathresult>0
            \draw (\x,\grad) -- (\x,-\grad) ;
          \else 
            \draw[ultra thick] (\x,2*\grad) -- (\x,-2*\grad) ;
          \fi
        }
        \foreach \z [count=\zi] in {#4 * \pas , #5 * \pas}
        {
          % \node at (\z,\grad) [above] {\makeAlph{\zi}};
          \draw[-Stealth] (\z,5*\grad) node [above] {\filling[1cm]}--(\z,1.5*\grad);
        }
        \node (A) at (\ya,-2*\grad) [below] {\pgfmathprintnumber[use comma]{\ya}} ;
        \node (B) at (\yb,-2*\grad) [below] {\pgfmathprintnumber[use comma]{\yb}} ;
    \end{tikzpicture} 
\end{figure}}

\newcommand{\fracdplace}[3][1]{
% #1 : Combien de grd avant le départ
% #2 : Dénominateur
% #3 : Graduation connue
\tikzmath{\den=#2; \ya =#3;  \rcl= #1 ; \pas=1/\den ; \yb =\ya +1; \dprt = \ya - \rcl * \pas; \y2 =\dprt +\pas; \fin =\dprt +17*\pas ; \grad = 0.1/\den ; } %modifier yA, scl (scale), rcl (reculer) et dclg (decalage )uniquelent
\begin{figure}[H]
    \centering
    \begin{tikzpicture}[scale=\den]
        \draw (\dprt,0) -- (\fin,0) node[midway, sloped]{};
        \foreach \x in {\dprt,\y2,...,\fin}
        {
          \pgfmathparse{int(Mod(\x * \den +\pas ,\den))}
          \ifnum\pgfmathresult>0
            \draw (\x,\grad) -- (\x,-\grad) ;
          \else 
            \draw[ultra thick] (\x,2*\grad) -- (\x,-2*\grad) ;
          \fi
        }
        \node (A) at (\ya,-2*\grad) [below] {\pgfmathprintnumber[use comma]{\ya}} ;
        \node (B) at (\yb,-2*\grad) [below] {\pgfmathprintnumber[use comma]{\yb}} ;
    \end{tikzpicture} 
\end{figure}}


\textbf{Exercice : } Sur les droites graduées

\begin{enumerate}
    \item Donner les fractions associées aux positions suivantes.

    \fracdgmult[4]{5}{1}{2}{7}

    \fracdgmult[0]{4}{0}{2}{9}

    \item Donner les nombres associés aux positions suivantes.

    \fracdgmult[0]{10}{0}{3}{11}

    \item Placer les lettres associées aux nombres ou fractions suivants sur une des droites graduées ci-dessous.
    \begin{multicols}{6}
        \begin{itemize}[label={}, leftmargin=*]
            \item $$A=\dfrac{2}{4}$$
            \item $$B=\dfrac{7}{4}$$
            \item $$C=\dfrac{38}{10}$$
            \item $$D=\dfrac{14}{4}$$
            \item $$E=4,1$$
            \item $$F=3,96$$
        \end{itemize}
    \end{multicols}
    \vspace{1em}

    \fracdplace[0]{4}{0}

    \fracdplace[3]{10}{4}

    \tikzmath{\den=100; \ya =3.9;  \rcl= 1 ; \pas=1/\den ; \yb =\ya +0.1; \dprt = \ya - \rcl * \pas; \y2 =\dprt +\pas; \fin =\dprt +17*\pas ; \grad = 0.1/\den ; } 
    \begin{figure}[H]
        \centering
        \begin{tikzpicture}[scale=\den]
            \draw (\dprt,0) -- (\fin,0) node[midway, sloped]{};
            \foreach \x in {\dprt,\y2,...,\fin}
            {
              \pgfmathparse{int(Mod(\x * \den +\pas ,\den))}
              \ifnum\pgfmathresult>0
                \draw (\x,\grad) -- (\x,-\grad) ;
              \else 
                \draw[ultra thick] (\x,2*\grad) -- (\x,-2*\grad) ;
              \fi
            }
            \draw[ultra thick] (3.9,2*\grad) -- (3.9,-2*\grad) ;
            \node (A) at (\ya,-2*\grad) [below] {\pgfmathprintnumber[use comma]{\ya}} ;
            \node (B) at (\yb,-2*\grad) [below] {\pgfmathprintnumber[use comma]{\yb}} ;
        \end{tikzpicture} 
    \end{figure}


    
\end{enumerate}
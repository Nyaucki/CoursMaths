
%% Exercices avec repère %%

\textbf{Exercice : }À l'aide du repère ci-dessous :

\newcommand{\ax}{-1}
\newcommand{\ay}{2}
\newcommand{\bx}{1}
\newcommand{\by}{-1}
\newcommand{\cx}{-5}
\newcommand{\cy}{-2}
\newcommand{\dx}{7}
\newcommand{\dy}{2}


\begin{figure}[H]
    \center
    \begin{tikzpicture}
        \draw[dotted] (-8,-5) grid (8,7);
        \draw [ultra thick,->] (0,-5)-- (0,7);
        \draw [ultra thick,->] (-8,0)-- (8,0);
        \node at (0,0) [below  left]{0};
        \draw (1,0.2)--(1,-0.2) node [below] {1};
        \draw (0.2,1)--(-0.2,1) node [left] {1};
        % \fill[red] (\ax , \ay) coordinate (A) circle(1.5pt)node [below  left]{$A$};
        % \fill[red] (\bx , \by) coordinate (B) circle(1.5pt)node [below  left]{$B$};
        % \fill[red] (\cx , \cy) circle(1.5pt)node [below  left]{$C$};
        % \fill[red] (\dx , \dy) coordinate (D) circle(1.5pt)node [below  left]{$D$};
    \end{tikzpicture}
\end{figure}

\begin{enumerate}
    \item Placer les points suivants :
        \begin{multicols}{4}
            \begin{itemize}
                \item $A~(\ax;\ay)$
                \item $B~(\bx;\by)$
                \item $C~(\cx;\cy)$
                \item $D~(\dx;\dy)$
            \end{itemize}
        \end{multicols}
    \item Placer les points suivants : 
    \begin{itemize}
        \item $B'$ l'image de $B$ par la symétrie d'axe $(AC)$
        \item $C'$ l'image de $C$ par la symétrie d'axe $(AD)$
        \item $[A'D']$ l'image de $[AD]$ par la symétrie de centre $B$.
    \end{itemize}
    \item Donner les coordonnées des points placés précédemment :
    \begin{multicols}{4}
        \begin{itemize}
            \item $A'~\filling$
            \item $B'~\filling$
            \item $C'~\filling$
            \item $D'~\filling$
        \end{itemize}
    \end{multicols}
    \item Que peut-on dire des droites $(AD)$ et $(A'D')$. Justifier.
    \foreach \m in {0,...,2}
    {
        \\*[0.3cm]
    
        \dotfill
    }
\end{enumerate}
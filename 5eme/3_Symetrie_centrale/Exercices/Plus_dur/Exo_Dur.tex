\newpage

\exo{Modéliser}{Dur1}
On considère un triangle $ABC$ quelconque. On appelle :
\begin{itemize}
    \item $D$ l'image de $A$ par la symétrie de centre $A$
    \item $E$ l'image de $B$ par la symétrie de centre $A$
    \item $F$ l'image de $C$ par la symétrie de centre $A$
    \item $G$ l'image de $A$ par la symétrie de centre $B$
    \item $H$ l'image de $B$ par la symétrie de centre $B$
    \item $I$ l'image de $C$ par la symétrie de centre $B$
    \item $J$ l'image de $D$ par la symétrie de centre $E$
    \item $K$ l'image de $E$ par la symétrie de centre $E$
    \item $L$ l'image de $F$ par la symétrie de centre $E$
\end{itemize}

Quels sont les points occupant la même position ?

\exo{Chercher}{Dur2}
    
Si $(d)$ est une droite et $(d')$ son image par une symétrie axiale, à quelle condition est-ce que $(d)$ et $(d')$ sont parallèles ?

\exo{4}{Représenter} : Tracer les triangles ABC suivants.

\begin{minipage}{0.45\textwidth}
    \begin{itemize}
        \item $AB=5$cm
        \item $AC=3$cm
        \item $BC=4$cm
    \end{itemize}
    \vspace*{10cm}
\end{minipage}
\hfil
\vrule
\hfil
\begin{minipage}{0.45\textwidth}
    \begin{itemize}
        \item $AB=4$cm
        \item $\widehat{ABC}=30$°
        \item $\widehat{BAC}=100$°
    \end{itemize}
    \vspace*{10cm}
\end{minipage}


\exo{4}{Raisonner} : Pour chacun des triangles $ABC$ suivants, dire s'il est Rectangle, Isocèle, Équilatéral, Impossible ou quelconque.

\begin{minipage}{0.22\textwidth}
    \begin{itemize}
        \item $AB=5$cm
        \item $AC=3$cm
        \item $BC=1$cm
    \end{itemize}
    \begin{center}
        \filling
    \end{center}
\end{minipage}
\hfil
\vrule
\hfil
\begin{minipage}{0.22\textwidth}
    \begin{itemize}
        \item $AB=2$cm
        \item $AC=3$cm
        \item $BC=2$cm
    \end{itemize}
    \begin{center}
        \filling
    \end{center}
\end{minipage}
\hfil
\vrule
\hfil
\begin{minipage}{0.22\textwidth}
    \begin{itemize}
        \item $AB=7$cm
        \item $AC=3$cm
        \item $BC=6$cm
    \end{itemize}
    \begin{center}
        \filling
    \end{center}
\end{minipage}
\hfil
\vrule
\hfil
\begin{minipage}{0.22\textwidth}
    \begin{itemize}
        \item $AB=2$cm
        \item $AC=2$cm
        \item $BC=2$cm
    \end{itemize}
    \begin{center}
        \filling
    \end{center}
\end{minipage}
\vfil

\begin{minipage}{0.22\textwidth}
    \begin{itemize}
        \item $\widehat{ABC}=40$°
        \item $\widehat{BAC}=40$°
    \end{itemize}
    \begin{center}
        \filling
    \end{center}
\end{minipage}
\hfil
\vrule
\hfil
\begin{minipage}{0.22\textwidth}
    \begin{itemize}
        \item $\widehat{ABC}=63$°
        \item $\widehat{BAC}=144$°
    \end{itemize}
    \begin{center}
        \filling
    \end{center}
\end{minipage}
\hfil
\vrule
\hfil
\begin{minipage}{0.22\textwidth}
    \begin{itemize}
        \item $\widehat{ABC}=67$°
        \item $\widehat{BAC}=23$°
    \end{itemize}
    \begin{center}
        \filling
    \end{center}
\end{minipage}
\hfil
\vrule
\hfil
\begin{minipage}{0.22\textwidth}
    \begin{itemize}
        \item $\widehat{ABC}=60$°
        \item $\widehat{BAC}=60$°
    \end{itemize}
    \begin{center}
        \filling
    \end{center}
\end{minipage}

\newpage

\exo{4}{Calculer} : Calculer l'aire des deux triangles suivants.

\begin{minipage}{0.45\textwidth}
    \begin{figure}[H]
        \center
        \begin{tikzpicture}
            \draw (0,0) coordinate (A) --node [midway,below] {6} (5,0) coordinate (B) --node [midway,above right] {5} (2,3) coordinate (C) --node [midway,above left] {4} cycle;
            \draw[dashed] ($(A)!(C)!(B)$)--node [midway,right]{3} (C); %Perpendiculaire
            \draw [gray,right angle quadrant=1,right angle symbol={A}{B}{C}];
        \end{tikzpicture}

        Aire=\filling[3cm]
    \end{figure}
\end{minipage}
\hfil
\vrule
\hfil
\begin{minipage}{0.45\textwidth}
    \begin{figure}[H]
        \center
        \begin{tikzpicture}
            \draw (-0.5,-0.5) coordinate (A) --node [midway,below] {6} (5,-1) coordinate (B) --node [midway,below left] {3} (2,1.2) coordinate (C) --node [midway,below right] {4} cycle;
            \draw[dashed] ($(A)!(B)!(C)$)--node [midway,right]{2} (B);%Perpendiculaire
            \draw [dashed,shorten >=-2cm] (A)--(C); %Extension
            \draw [gray,right angle quadrant=1,right angle symbol={C}{A}{B}];
        \end{tikzpicture}

        Aire=\filling[3cm]
    \end{figure}
\end{minipage}

\exo{4}{Raisonner} : Pour chacune des figures suivantes, déterminer l'angle $\alpha$.

\begin{minipage}[t]{0.45\textwidth}
    \begin{figure}[H]
        \center
        \begin{tikzpicture}
            \draw (0,0) coordinate (A) -- (3,0) coordinate (B) -- (1,2) coordinate (C) --cycle;
            \draw (B)--(5,0)coordinate (D)--(C);
            \draw (D)--(5,-2) coordinate (F)--(B);
            \node (E) at (5.2,0){};
            \draw [gray,right angle quadrant=1,right angle symbol={E}{A}{F}];
            \draw pic["55°",draw=green,fill=green!20,angle eccentricity=1.4, angle radius=0.7cm]{angle=B--A--C};
            \draw pic["10°",draw=green,fill=green!20,angle eccentricity=1.3, angle radius=0.8cm]{angle=C--D--B};
            \draw pic["60°",draw=green,fill=green!20,angle eccentricity=1.3, angle radius=0.7cm]{angle=A--C--B};
            \draw pic["$\alpha$",draw=blue,fill=blue!20,angle eccentricity=1.3, angle radius=0.8cm]{angle=D--F--B};
        \end{tikzpicture}
        
        $\alpha$=\filling[3cm]
    \end{figure}
\end{minipage}
\hfill
\vrule
\hfil
\begin{minipage}[t]{0.45\textwidth}
    \begin{figure}[H]
        \center
        \begin{tikzpicture}
            \draw (0.5,0) coordinate (A) -- node[midway,above] {2} (3,0) coordinate (B) -- (7,0) coordinate (C) --(4.5,-2) coordinate (D) --node[midway,below] {2} (1.5,-2) coordinate (E)--cycle;
            \draw (E)--node[midway,below right] {2}(B)--node[midway,below left] {2}(D);
            \draw pic["55°",draw=green,fill=green!20,angle eccentricity=1.4, angle radius=0.7cm]{angle=B--E--A};
            \draw pic["50°",draw=green,fill=green!20,angle eccentricity=1.3, angle radius=0.8cm]{angle=C--D--B};
            \draw pic["$\alpha$",draw=blue,fill=blue!20,angle eccentricity=1.3, angle radius=0.8cm]{angle=B--C--D};
        \end{tikzpicture}

        $\alpha$=\filling[1.7cm]
    \end{figure}
\end{minipage}

\exo{4}{Raisonner} 

\begin{minipage}[t]{0.45\textwidth}
    \begin{figure}[H]
        \center
        \begin{tikzpicture}
            \draw (0,0) coordinate (A) node[below left]{A}--node [midway,below] {5} (5,0) coordinate (B) node [below right]{B} --(0,3) coordinate (C) node[above]{C} --node [midway,above left] {4} cycle;
            \draw[dashed] ($(C)!(A)!(B)$) node[above right]{H}--node [midway,right]{2} (A); 
            %Perpendiculaire
            \draw [gray,right angle quadrant=1,right angle symbol={C}{B}{A}];
            \node (D)at(-0.5,0){} ;
            \draw [gray,right angle quadrant=1,right angle symbol={D}{B}{C}];
        \end{tikzpicture}
    \end{figure}
\end{minipage}
\hfill
\begin{minipage}[t]{0.45\textwidth}
    Dans la figure ci-contre :
    \begin{enumerate}
        \item Calculer l'aire du triangle $ABC$.
        \vspace*{1.2cm}
        \item En écrivant le calcul de l'aire du triangle $ABC$ à l'aide de la longueur $AH$, déterminer la longueur $BC$.
        \vspace*{2.3cm}
    \end{enumerate}
\end{minipage}

\exo{}{} Si réussi, l'exercice comptera comme une note en plus. Déterminer l'angle $\alpha$.

\begin{minipage}[m]{0.45\textwidth}
    \begin{figure}[H]
        \center
        \begin{tikzpicture}[yscale=0.75]
            \draw (0,0) coordinate (A) node [below left] {A}--(3.5,0) coordinate (J) node [below]{J} --(7,0) coordinate (B) node [below right] {B}--(4,2.5) coordinate (G) node [right] {G} --(1,5) coordinate (C) node [above]{C}--(0.5,2.5) coordinate (E) node [left]{E}--cycle;
            \draw (0.2,1) coordinate (I)node [left] {I} --(3.7,0.9) coordinate (K) node[right] {K}--(4.45,4.81) coordinate (H) node[above right]{H}--cycle;
            \draw (J)--(K)--(G)--(E)--cycle ;
            \draw pic["$\alpha$",draw=blue,fill=blue!20,angle eccentricity=1.3, angle radius=0.8cm]{angle=I--H--G};
            \draw pic["40°",draw=blue,fill=blue!20,angle eccentricity=1.3, angle radius=0.8cm]{angle=A--C--B};
            \draw pic["30°",draw=blue,fill=blue!20,angle eccentricity=1.3, angle radius=0.8cm]{angle=E--J--A};
            \draw pic["40°",draw=blue,fill=blue!20,angle eccentricity=1.3, angle radius=0.8cm]{angle=H--I--C};
        \end{tikzpicture}
    \end{figure}
\end{minipage}
\hfil
\begin{minipage}[m]{0.45\textwidth}
    \begin{itemize}
        \item (AC)//(JG)
        \item (AB)//(EG)
        \item (EJ)//(BC)
        \item (IK)//(AB)
    \end{itemize}
\end{minipage}
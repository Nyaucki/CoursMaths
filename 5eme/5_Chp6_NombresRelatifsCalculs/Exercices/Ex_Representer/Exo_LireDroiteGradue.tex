\textbf{Pour les exercices \ref{LireDroite1} à \ref{LireDroite8} :} dire à quels nombres correspondent à A, B et C ?

\begin{minipage}[t]{0.45\textwidth}
    \exo{Représenter}{LireDroite1} 
    
    \tikzmath{\ya =0; \scl=1; \rcl= 6; \dclg =1; \pas=1/\scl ; \yb =\ya +\dclg * \pas; \dprt = \ya - \rcl * \pas; \y2 =\dprt +\pas; \fin =\dprt +8*\pas ; \grad = 0.1/\scl ; } %modifier yA, scl (scale), rcl (reculer) et dclg (decalage )uniquelent
    
    \begin{figure}[H]
        \centering
        \begin{tikzpicture}[scale=\scl]
            \draw (\dprt,0) -- (\fin,0) node[midway, sloped]{};
            \foreach \x in {\dprt,\y2,...,\fin}
            {
              \draw (\x,\grad) -- (\x,-\grad) ;
            }
            \foreach \z [count=\zi] in {-5,-3,2}
            {
              \node at (\z,\grad) [above] {\makeAlph{\zi}};
            }
            \node (A) at (\ya,-\grad) [below] {\pgfmathprintnumber[use comma]{\ya}} ;
            \node (B) at (\yb,-\grad) [below] {\pgfmathprintnumber[use comma,precision=3]{\yb}} ;
        \end{tikzpicture} 
    \end{figure}
\end{minipage}
\hfill
\begin{minipage}[t]{0.45\textwidth}
    \exo{Représenter}{LireDroite2}  
    
    \tikzmath{\ya =-8; \scl=1; \rcl= 3; \dclg =1; \pas=1/\scl ; \yb =\ya +\dclg * \pas; \dprt = \ya - \rcl * \pas; \y2 =\dprt +\pas; \fin =\dprt +8*\pas ; \grad = 0.1/\scl ; } %modifier yA, scl (scale), rcl (reculer) et dclg (decalage )uniquelent
    
    \begin{figure}[H]
        \centering
        \begin{tikzpicture}[scale=\scl]
            \draw (\dprt,0) -- (\fin,0) node[midway, sloped]{};
            \foreach \x in {\dprt,\y2,...,\fin}
            {
              \draw (\x,\grad) -- (\x,-\grad) ;
            }
            \foreach \z [count=\zi] in {-5,-3,-9}
            {
              \node at (\z,\grad) [above] {\makeAlph{\zi}};
            }
            \node (A) at (\ya,-\grad) [below] {\pgfmathprintnumber[use comma]{\ya}} ;
            \node (B) at (\yb,-\grad) [below] {\pgfmathprintnumber[use comma,precision=3]{\yb}} ;
        \end{tikzpicture} 
    \end{figure}
\end{minipage}


\begin{minipage}[t]{0.45\textwidth}
    \exo{Représenter}{LireDroite3} 
    
    \tikzmath{\ya =0; \scl=10; \rcl= 6; \dclg =1; \pas=1/\scl ; \yb =\ya +\dclg * \pas; \dprt = \ya - \rcl * \pas; \y2 =\dprt +\pas; \fin =\dprt +8*\pas ; \grad = 0.1/\scl ; } %modifier yA, scl (scale), rcl (reculer) et dclg (decalage )uniquelent
    
    \begin{figure}[H]
        \centering
        \begin{tikzpicture}[scale=\scl]
            \draw (\dprt,0) -- (\fin,0) node[midway, sloped]{};
            \foreach \x in {\dprt,\y2,...,\fin}
            {
              \draw (\x,\grad) -- (\x,-\grad) ;
            }
            \foreach \z [count=\zi] in {-0.1,-0.6,-0.4}
            {
              \node at (\z,\grad) [above] {\makeAlph{\zi}};
            }
            \node (A) at (\ya,-\grad) [below] {\pgfmathprintnumber[use comma]{\ya}} ;
            \node (B) at (\yb,-\grad) [below] {\pgfmathprintnumber[use comma,precision=4]{\yb}} ;
        \end{tikzpicture} 
    \end{figure}
\end{minipage}
\hfill
\begin{minipage}[t]{0.45\textwidth}
    \exo{Représenter}{LireDroite4}  
    
    \tikzmath{\ya =-7.3; \scl=10; \rcl= 1; \dclg =1; \pas=1/\scl ; \yb =\ya +\dclg * \pas; \dprt = \ya - \rcl * \pas; \y2 =\dprt +\pas; \fin =\dprt +8*\pas ; \grad = 0.1/\scl ; } %modifier yA, scl (scale), rcl (reculer) et dclg (decalage )uniquelent
    
    \begin{figure}[H]
        \centering
        \begin{tikzpicture}[scale=\scl]
            \draw (\dprt,0) -- (\fin,0) node[midway, sloped]{};
            \foreach \x in {\dprt,\y2,...,\fin}
            {
              \draw (\x,\grad) -- (\x,-\grad) ;
            }
            \foreach \z [count=\zi] in {-7,-6.8,-6.6}
            {
              \node at (\z,\grad) [above] {\makeAlph{\zi}};
            }
            \node (A) at (\ya,-\grad) [below] {\pgfmathprintnumber[use comma]{\ya}} ;
            \node (B) at (\yb,-\grad) [below] {\pgfmathprintnumber[use comma,precision=3]{\yb}} ;
        \end{tikzpicture} 
    \end{figure}
\end{minipage}


\begin{minipage}[t]{0.45\textwidth}
    \exo{Représenter}{LireDroite5} 
    
    \tikzmath{\ya =0; \scl=100; \rcl= 7; \dclg =1; \pas=1/\scl ; \yb =\ya +\dclg * \pas; \dprt = \ya - \rcl * \pas; \y2 =\dprt +\pas; \fin =\dprt +8*\pas ; \grad = 0.1/\scl ; } %modifier yA, scl (scale), rcl (reculer) et dclg (decalage )uniquelent
    
    \begin{figure}[H]
        \centering
        \begin{tikzpicture}[scale=\scl]
            \draw (\dprt,0) -- (\fin,0) node[midway, sloped]{};
            \foreach \x in {\dprt,\y2,...,\fin}
            {
              \draw (\x,\grad) -- (\x,-\grad) ;
            }
            \foreach \z [count=\zi] in {-0.01,-0.05,-0.07}
            {
              \node at (\z,\grad) [above] {\makeAlph{\zi}};
            }
            \node (A) at (\ya,-\grad) [below] {\pgfmathprintnumber[use comma]{\ya}} ;
            \node (B) at (\yb,-\grad) [below] {\pgfmathprintnumber[use comma,precision=4]{\yb}} ;
        \end{tikzpicture} 
    \end{figure}
\end{minipage}
\hfill
\begin{minipage}[t]{0.45\textwidth}
    \exo{Représenter}{LireDroite6}  
    
    \tikzmath{\ya =-1.31; \scl=100; \rcl= 1; \dclg =1; \pas=1/\scl ; \yb =\ya +\dclg * \pas; \dprt = \ya - \rcl * \pas; \y2 =\dprt +\pas; \fin =\dprt +8*\pas ; \grad = 0.1/\scl ; } %modifier yA, scl (scale), rcl (reculer) et dclg (decalage )uniquelent
    
    \begin{figure}[H]
        \centering
        \begin{tikzpicture}[scale=\scl]
            \draw (\dprt,0) -- (\fin,0) node[midway, sloped]{};
            \foreach \x in {\dprt,\y2,...,\fin}
            {
              \draw (\x,\grad) -- (\x,-\grad) ;
            }
            \foreach \z [count=\zi] in {-1.26,-1.29,-1.24}
            {
              \node at (\z,\grad) [above] {\makeAlph{\zi}};
            }
            \node (A) at (\ya,-\grad) [below] {\pgfmathprintnumber[use comma]{\ya}} ;
            \node (B) at (\yb,-\grad) [below] {\pgfmathprintnumber[use comma,precision=3]{\yb}} ;
        \end{tikzpicture} 
    \end{figure}
\end{minipage}

\begin{minipage}[t]{0.45\textwidth}
    \exo{Représenter}{LireDroite7} 
    
    \tikzmath{\ya =-0.03; \scl=100; \rcl= 4; \dclg =1; \pas=1/\scl ; \yb =\ya +\dclg * \pas; \dprt = \ya - \rcl * \pas; \y2 =\dprt +\pas; \fin =\dprt +8*\pas ; \grad = 0.1/\scl ; } %modifier yA, scl (scale), rcl (reculer) et dclg (decalage )uniquelent
    
    \begin{figure}[H]
        \centering
        \begin{tikzpicture}[scale=\scl]
            \draw (\dprt,0) -- (\fin,0) node[midway, sloped]{};
            \foreach \x in {\dprt,\y2,...,\fin}
            {
              \draw (\x,\grad) -- (\x,-\grad) ;
            }
            \foreach \z [count=\zi] in {0.01,-0.05,-0.06}
            {
              \node at (\z,\grad) [above] {\makeAlph{\zi}};
            }
            \node (A) at (\ya,-\grad) [below] {\pgfmathprintnumber[use comma,precision=4]{\ya}} ;
            \node (B) at (\yb,-\grad) [below] {\pgfmathprintnumber[use comma,precision=4]{\yb}} ;
        \end{tikzpicture} 
    \end{figure}
\end{minipage}
\hfill
\begin{minipage}[t]{0.45\textwidth}
    \exo{Représenter}{LireDroite8}  
    
    \tikzmath{\ya =-4.5; \scl=100; \rcl= 6; \dclg =1; \pas=1/\scl ; \yb =\ya +\dclg * \pas; \dprt = \ya - \rcl * \pas; \y2 =\dprt +\pas; \fin =\dprt +8*\pas ; \grad = 0.1/\scl ; } %modifier yA, scl (scale), rcl (reculer) et dclg (decalage )uniquelent
    
    \begin{figure}[H]
        \centering
        \begin{tikzpicture}[scale=\scl]
            \draw (\dprt,0) -- (\fin,0) node[midway, sloped]{};
            \foreach \x in {\dprt,\y2,...,\fin}
            {
              \draw (\x,\grad) -- (\x,-\grad) ;
            }
            \foreach \z [count=\zi] in {-4.48,-4.53,-4.51}
            {
              \node at (\z,\grad) [above] {\makeAlph{\zi}};
            }
            \node (A) at (\ya,-\grad) [below] {\pgfmathprintnumber[use comma]{\ya}} ;
            \node (B) at (\yb,-\grad) [below] {\pgfmathprintnumber[use comma,precision=3]{\yb}} ;
        \end{tikzpicture} 
    \end{figure}
\end{minipage}
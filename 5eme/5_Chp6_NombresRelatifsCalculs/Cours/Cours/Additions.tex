\prop{Additions et soustraction}
{\begin{minipage}{0.45\textwidth}
    \textbf{Même signe}
    \begin{itemize}
    \item On prend le signe des deux nombres
    \item On additionne les deux nombres
    \end{itemize}
\end{minipage}
\hfill
\begin{minipage}{0.45\textwidth}
    \textbf{Signes opposés}
    \begin{itemize}
    \item On prend le signe du plus grand
    \item On soustrait les deux nombres
    \end{itemize}    
\end{minipage}  }

\exmpl{
    \begin{itemize}
        \item $4-7=-3$ On a prit le signe du plus grand (\textbf{-}7) et calculé 7-4
        \item $-8-15=-23$ On a prit le signe des nombres (\textbf{-}) et calculé 8+15
    \end{itemize}  
}

\prop{Avec paranthèses}
{
    S'il n'y a \textbf{qu'un seul} nombre dans une paranthèse et un signe avant celle-ci :
    \begin{itemize}
        \item Un + devant la paranthèse signifie qu'on ne change pas le signe du nombre.
        \item Un - devant la paranthèse signifie qu'on prend l'opposé.
    \end{itemize}
}

\rmq{S'il y a plusieurs nombres (et donc au moins une oprération) entre paranthèses, la règle des priorités s'applique, et c'est au moment d'enlever les paranthèses que se pose la question du signe.}

\exmpl{
    \begin{itemize}
        \item $+(-4)=-4$
        \item $+(+16)=+16$
        \item $-(+12)=-12$
        \item $-(-13)=13$
        \item $-(4-7)=-(-3)=3$
    \end{itemize}
}

\rmq{Cela signifie que soustraire un nombre est la même chose qu'additionner l'opposé.}
\section*{Découverte du calcul littéral}


Un jeune dresseur affronte un Fantominus avec son Pikachu. 
Avant le combat, l'ennemi avait 10 Points de Vie (PV). Après 4 attaques Tonnerre, le Fantomins n'a plus que 2 PV et s'enfuit.

Plus tard dans son aventure, le même dresseur rencontre un Flagadoss avec 50 PV. Heureusement, Pikachu est super efficace et après seulement 3 attaques, le Pokémon ennemi s'enfuit avec juste 2 PV.

Cette fois ci, notre dresseur fait face à un Dracaufeu ! Celui-ci à 150 PV. Le combat est rude, mais après 7 attaques Tonnere, l'ennemi est affaibli et abandonne. Il n'a que 3 PV.

\cnt Pour chaque rencontre, calculer combien de dégâts à fait chaque attaque Tonnerre ?

\cnt Établir une formule pour calculer facilement les dégâts de l'attaque Tonnerre.

\cnt Utiliser cette formule pour calculer rapidement les dégâts effectués dans les cas suivants :

\vspace*{-1em}

\renewcommand{\arraystretch}{2}

\begin{tabularx}{\textwidth}{YYYY}
Pokémon & Absol & Ronflex & Mewtoo \\
\hline
PV avant combat & 230 & 500 & $431$  \\
\hline
Nombre d'attaques & 5 & 8 & 11 \\
\hline
PV après combat & 5 & 20 & 2
\end{tabularx}

\cnt Peut-on écrire une formule similaire donnant le nombre de PV avant le combat en connaissant le nombre d'attaques, le nombre de dégâts par attaque et les PV après combat ?

\cnt Après seulement 3 attaques tonnerre contre un Carapuce, celui-ci est KO et ses PV sont à -4. Sachant que chaque attaque a fait 35 dégâts, combien de PV avait-il au début du combat ?

\vspace*{-2em}
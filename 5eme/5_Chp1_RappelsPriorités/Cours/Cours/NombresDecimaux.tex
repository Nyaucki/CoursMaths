\dfnt{Fraction décimale}{\textbf{Une fraction décimale} est une fraction pouvant s'écrire avec un numérateur entier et un dénominateur sous forme de puissance de 10}

\rmq{Parfois, il faut simplifier la fraction pour la mettre sous la forme décimale.}

\exmpl{\begin{itemize}
    \item $\dfrac{3}{10}$ ; $\dfrac{127}{10000}$ ; $\dfrac{3}{5}$ et 5 sont des fractions décimales.
    \item $\dfrac{4}{3}$ n'en est pas une.
\end{itemize}}

\prop{Nombre et fraction décimale}{Tout nombre décimale avec \textbf{un nombre fini} de chiffre après la virgule peur s'écrire sous forme de fraction décimale}

\exmpl{$31,2345=\dfrac{312345}{10000}$}

\rmq{Si le nombre de chiffre après la virgule est infini, alors on ne peut pas écrire sous forme de fraction décimale}

\prop{Décomposition}{Lorsqu'on décompose un nombre décimale, on peut écrire le chiffre des dixièmes comme une fraction sur 10, celui des centièmes comme une fraction sur 100, celui des millième comme une fraction sur 1000...}

\exmpl{$62,432=61+\dfrac{4}{10}+\dfrac{3}{100}+\dfrac{2}{1000}$}
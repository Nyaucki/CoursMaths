\consigne{SousMul1}{SousMul10} (Raisonner) Effectuer les calculs suivants

\begin{multicols}{5}
\exo{}{SousMul1} 
$$\dfrac{12}{3}-\dfrac{19}{24}+\dfrac{16}{24}$$

\exo{}{SousMul2} 
$$\dfrac{13}{7}-\dfrac{16}{70}-\dfrac{2}{70}$$

\exo{}{SousMul3} 
$$\dfrac{4}{3}-\dfrac{2}{30}-\dfrac{20}{30}$$

\exo{}{SousMul4} 
$$\dfrac{3}{4}-\dfrac{19}{20}-\dfrac{9}{20}$$

\exo{}{SousMul5} 
$$\dfrac{1}{3}-\dfrac{12}{6}+\dfrac{15}{24}$$

\exo{}{SousMul6} 
$$\dfrac{10}{9}-\dfrac{19}{36}-\dfrac{11}{72}$$

\exo{}{SousMul7} 
$$\dfrac{7}{5}-\dfrac{18}{20}-\dfrac{14}{40}$$

\exo{}{SousMul8} 
$$\dfrac{15}{8}-\dfrac{15}{40}-\dfrac{4}{88}$$

\exo{}{SousMul9} 
$$\dfrac{5}{9}-\dfrac{16}{27}+\dfrac{15}{81}$$

\exo{}{SousMul10} 
$$\dfrac{16}{7}-\dfrac{6}{14}+\dfrac{8}{63}$$

\end{multicols}
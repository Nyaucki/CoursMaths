\exo{8}{Calculer} : Effectuer les calculs suivants.

\begin{multicols}{2}
    $$\dfrac{8}{7}+\dfrac{17}{14}$$\vspace*{-0.5em}\dotlines[0]{3}
    
    $$\dfrac{6}{2}+\dfrac{8}{7}$$\vspace*{-0.5em}\dotlines[0]{3}

    \columnbreak
    $$3-\dfrac{5}{8}$$\vspace*{-0.5em}\dotlines[0]{3}

    $$\dfrac{7}{8}-\dfrac{2}{7}-\dfrac{9}{56}$$\vspace*{-0.5em}\dotlines[0]{3}

\end{multicols}


\exo{6}{Représenter} : Donner les fractions associées aux positions suivantes.

\newcommand{\fracdgmult}[5][1]{
% #1 : Combien de grd avant le départ
% #2 : Dénominateur
% #3 : Graduation connue
% #4 : Numérateur 1
% #5 : Numérateur 2
\tikzmath{\den=#2; \ya =#3;  \rcl= #1 ; \pas=1/\den ; \yb =\ya +1; \dprt = \ya - \rcl * \pas; \y2 =\dprt +\pas; \fin =\dprt +17*\pas ; \grad = 0.1/\den ; } %modifier yA, scl (scale), rcl (reculer) et dclg (decalage )uniquelent
\begin{figure}[H]
    \centering
    \begin{tikzpicture}[scale=\den]
        \draw (\dprt,0) -- (\fin,0) node[midway, sloped]{};
        \foreach \x in {\dprt,\y2,...,\fin}
        {
          \pgfmathparse{int(Mod(\x * \den +\pas ,\den))}
          \ifnum\pgfmathresult>0
            \draw (\x,\grad) -- (\x,-\grad) ;
          \else 
            \draw[ultra thick] (\x,2*\grad) -- (\x,-2*\grad) ;
          \fi
        }
        \foreach \z [count=\zi] in {#4 * \pas , #5 * \pas}
        {
          % \node at (\z,\grad) [above] {\makeAlph{\zi}};
          \draw[-Stealth] (\z,5*\grad) node [above] {\filling[1cm]}--(\z,1.5*\grad);
        }
        \node (A) at (\ya,-2*\grad) [below] {\pgfmathprintnumber[use comma]{\ya}} ;
        \node (B) at (\yb,-2*\grad) [below] {\pgfmathprintnumber[use comma]{\yb}} ;
    \end{tikzpicture} 
\end{figure}}

\newcommand{\fracdplace}[3][1]{
% #1 : Combien de grd avant le départ
% #2 : Dénominateur
% #3 : Graduation connue
\tikzmath{\den=#2; \ya =#3;  \rcl= #1 ; \pas=1/\den ; \yb =\ya +1; \dprt = \ya - \rcl * \pas; \y2 =\dprt +\pas; \fin =\dprt +17*\pas ; \grad = 0.1/\den ; } %modifier yA, scl (scale), rcl (reculer) et dclg (decalage )uniquelent
\begin{figure}[H]
    \centering
    \begin{tikzpicture}[scale=\den]
        \draw (\dprt,0) -- (\fin,0) node[midway, sloped]{};
        \foreach \x in {\dprt,\y2,...,\fin}
        {
          \pgfmathparse{int(Mod(\x * \den +\pas ,\den))}
          \ifnum\pgfmathresult>0
            \draw (\x,\grad) -- (\x,-\grad) ;
          \else 
            \draw[ultra thick] (\x,2*\grad) -- (\x,-2*\grad) ;
          \fi
        }
        \node (A) at (\ya,-2*\grad) [below] {\pgfmathprintnumber[use comma]{\ya}} ;
        \node (B) at (\yb,-2*\grad) [below] {\pgfmathprintnumber[use comma]{\yb}} ;
    \end{tikzpicture} 
\end{figure}}


    \fracdgmult[0]{4}{0}{2}{9}

    \fracdgmult[2]{3}{3}{11}{17}

    \fracdgmult[4]{5}{1}{2}{7}

\exo{6}{Modéliser} :    


Hisoka achète un paquet de 400 chewing gum. En rentrant chez lui :
\begin{itemize}
    \item Il en mange $\dfrac{2}{5}$.
    \item Il en donne $\dfrac{1}{4}$ à Kurapika.
    \item Son sac était troué et il en a perdu $\dfrac{1}{5}$
\end{itemize}
Combien de chewing gum lui reste-t-il ?
\dotlines[0]{6}

\exo{}{} :  Calculer  

$$1+\dfrac{1}{2}-\dfrac{1}{3}+\dfrac{1}{4}-\dfrac{1}{6}+\dfrac{1}{8}$$
\dotlines[0]{6}

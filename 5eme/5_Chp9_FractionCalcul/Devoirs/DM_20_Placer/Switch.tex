%%%%%%%%%%%%%%%%%%%%%%%%%%%% A compléter pour l'entête %%%%%%%%%%%%%%%%%%%%%%%%%

\newcommand{\classe}{3emeC} % A compléter pour la classe cooncernée

\newcommand{\dateRendu}{15/05/2025} % A compléter pour la date de rendue

\newcommand{\devoirNumero}{20} % A compléter pour la date de rendue


%%%%%%%%%%%%%%%%%%%%%%%%%%%% Autres réglages %%%%%%%%%%%%%%%%%%%%%%%%%

\setlength{\columnseprule}{0.4pt} %Avoir les lignes entre les multicols

\newcommand{\fracdgmult}[5][1]{
% #1 : Combien de grd avant le départ
% #2 : Dénominateur
% #3 : Graduation connue
% #4 : Numérateur 1
% #5 : Numérateur 2
\tikzmath{\den=#2; \ya =#3;  \rcl= #1 ; \pas=1/\den ; \yb =\ya +1; \dprt = \ya - \rcl * \pas; \y2 =\dprt +\pas; \fin =\dprt +17*\pas ; \grad = 0.1/\den ; } %modifier yA, scl (scale), rcl (reculer) et dclg (decalage )uniquelent
\begin{figure}[H]
    \centering
    \begin{tikzpicture}[scale=\den]
        \draw (\dprt,0) -- (\fin,0) node[midway, sloped]{};
        \foreach \x in {\dprt,\y2,...,\fin}
        {
          \pgfmathparse{int(Mod(\x * \den +\pas ,\den))}
          \ifnum\pgfmathresult>0
            \draw (\x,\grad) -- (\x,-\grad) ;
          \else 
            \draw[ultra thick] (\x,2*\grad) -- (\x,-2*\grad) ;
          \fi
        }
        \foreach \z [count=\zi] in {#4 * \pas , #5 * \pas}
        {
          % \node at (\z,\grad) [above] {\makeAlph{\zi}};
          \draw[-Stealth] (\z,5*\grad) node [above] {\filling[1cm]}--(\z,1.5*\grad);
        }
        \node (A) at (\ya,-2*\grad) [below] {\pgfmathprintnumber[use comma]{\ya}} ;
        \node (B) at (\yb,-2*\grad) [below] {\pgfmathprintnumber[use comma]{\yb}} ;
    \end{tikzpicture} 
\end{figure}}
\section{Rappels}

\dfnt{fractions}
{Pour $a$ un nombre et $b\neq 0$, $\dfrac{a}{b}$ est le nombre qui multiplié par $b$ donne $a$.}

\exmpl{
    \begin{multicols}{3}
    \noindent$$5\times\dfrac{3}{5}=3$$
    $$12\times\dfrac{-7}{12}=-7$$
    $$127\times\dfrac{3456}{127}=3456$$
    \end{multicols}}

\prop{fraction égale}
{Une fraction reste la même si on multiplie ou divise le numérateur ET le dénominateur par le même nombre.}\label{propdenomin}

\exmpl{
    \begin{multicols}{3}
        \noindent$$\dfrac{4}{3}=\dfrac{4\times 5}{3\times 5}=\dfrac{20}{15}$$
%
        $$\dfrac{24}{6}=\dfrac{24\div 6}{6\div 6}=\dfrac{4}{1}=24$$
%
        $$\dfrac{4}{3}=\dfrac{4\times 132}{3\times 132}=\dfrac{528}{396}$$
    \end{multicols}    
}
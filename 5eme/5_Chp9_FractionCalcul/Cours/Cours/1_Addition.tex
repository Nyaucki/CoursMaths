\section{Additions}

\prop{Addition avec même dénominateur}
{Soient $a,b$ et $c$ trois nombres réels, $c\neq 0$. On a : $$\dfrac{a}{c}+\dfrac{b}{c}=\dfrac{a+b}{c}$$
Dit autrement, le dénominateur reste inchangé et le numérateur est la somme des numérateurs.}

\exmpl{
    \begin{multicols}{3}
        $$\dfrac{13}{12}+\dfrac{21}{12}=\dfrac{13+21}{12}=\dfrac{34}{12}$$
        %
        $$\dfrac{3}{4}+\dfrac{7}{4}=\dfrac{3+7}{4}=\dfrac{10}{4}$$
        %
        $$\dfrac{128}{412}+\dfrac{721}{412}=\dfrac{128+721}{412}=\dfrac{849}{412}$$
    \end{multicols}}

\rmq{Pour additionner deux fractions qui ne sont pas au même dénominateur, on utilisera la propriété \ref{propdenomin}}.

\newpage

\exmpl{\\
    \begin{minipage}[t]{0.45\textwidth}
      \begin{align*}
        &\dfrac{1}{4}+\dfrac{3}{36} & &\text{On remarque que $36=4\times 9$}\\
        =&\dfrac{1\times 9}{4\times 9}+\dfrac{3}{36} & &\text{On met au même dénominateur}\\
        =&\dfrac{9}{36}+\dfrac{3}{36} & & \text{On simplifie}\\
        =&\dfrac{9+3}{36} & &\text{On peut maintenant additionner}\\
        =&\dfrac{12}{36} &&
      \end{align*} 
    \end{minipage} 
      \hfil
      \vrule
      \hfil
    \begin{minipage}[t]{0.45\textwidth}
      \begin{align*}
        &\dfrac{1}{\textcolor{blue}{4}}+\dfrac{2}{\textcolor{red}{3}} & &\text{On met sur $\textcolor{red}{3}\times \textcolor{blue}{4}$}\\
        =&\dfrac{1\times \textcolor{red}{3}}{4\times \textcolor{red}{3}}+\dfrac{2\times \textcolor{blue}{4}}{3\times \textcolor{blue}{4}} & &\text{On multiplie par l'autre  dénominateur}\\
        =&\dfrac{3}{12}+\dfrac{8}{12} & & \text{On simplifie}\\
        =&\dfrac{3+8}{12} & &\text{On peut maintenant additionner}\\
        =&\dfrac{11}{12} &&
      \end{align*}  
    \end{minipage}
}
\section{Soustractions}

\prop{Soustraction avec même dénominateur}
{Soient $a,b$ et $c$ trois nombres réels, $c\neq 0$. On a : $$\dfrac{a}{c}-\dfrac{b}{c}=\dfrac{a-b}{c}$$
Dit autrement, le dénominateur reste inchangé et le numérateur est la somme des numérateurs.}

\exmpl{
    \begin{multicols}{3}
        $$\dfrac{13}{12}-\dfrac{2}{12}=\dfrac{13-2}{12}=\dfrac{11}{12}$$
        %
        $$\dfrac{3}{4}-\dfrac{7}{4}=\dfrac{3-7}{4}=\dfrac{-4}{4}=-1$$
        %
        $$\dfrac{128}{412}+\dfrac{321}{412}=\dfrac{128-321}{412}=\dfrac{-193}{412}$$
    \end{multicols}}

\rmq{Pour soustraire deux fractions qui n'ont pas le même dénominateur, comme pour les additions, on utilisera la propriété \ref{propdenomin}}.

\exmpl{\\
    \begin{minipage}[t]{0.45\textwidth}
      \begin{align*}
        &\dfrac{1}{4}-\dfrac{3}{36} & &\text{On remarque que $36=4\times 9$}\\
        =&\dfrac{1\times 9}{4\times 9}-\dfrac{3}{36} & &\text{On met au même dénominateur}\\
        =&\dfrac{9}{36}-\dfrac{3}{36} & & \text{On simplifie}\\
        =&\dfrac{9-3}{36} & &\text{On peut maintenant additionner}\\
        =&\dfrac{6}{36} &&
      \end{align*} 
    \end{minipage} 
      \hfil
      \vrule
      \hfil
    \begin{minipage}[t]{0.45\textwidth}
      \begin{align*}
        &\dfrac{1}{\textcolor{blue}{4}}-\dfrac{2}{\textcolor{red}{3}} & &\text{On met sur $\textcolor{red}{3}\times \textcolor{blue}{4}$}\\
        =&\dfrac{1\times \textcolor{red}{3}}{4\times \textcolor{red}{3}}-\dfrac{2\times \textcolor{blue}{4}}{3\times \textcolor{blue}{4}} & &\text{On multiplie par l'autre  dénominateur}\\
        =&\dfrac{3}{12}-\dfrac{8}{12} & & \text{On simplifie}\\
        =&\dfrac{3-8}{12} & &\text{On peut maintenant additionner}\\
        =&\dfrac{-5}{12} &&
      \end{align*}  
    \end{minipage}
}
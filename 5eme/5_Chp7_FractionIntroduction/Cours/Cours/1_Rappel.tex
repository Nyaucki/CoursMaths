\section{Rappel}

\dfnt{Vocabulaire}
{   \begin{minipage}{0.7\textwidth}
        Le nombre du haut d'une fraction est appelé le \textbf{numérateur}. 
        \\Celui du bas est appelé \textbf{dénominateur}.
    \end{minipage}
    \hfil
    \begin{minipage}{0.25\textwidth}
        $$\dfrac{\text{Numérateur}}{\text{Dénominateur}}$$
    \end{minipage}
}

\rmq{
\textbf{Nu}mérateur commence comme \textbf{Nu}age, \textbf{Dé}nominateur commence comme \textbf{De}ssous.}

\dfnt{fractions}
{La fraction $\dfrac{a}{b}$ correspond à $a$ part d'un partage en $b$ parties. 
}
\exmpl{\center
    \begin{multicols}{3}
        \fracpizza{3}{4} correspond à $\dfrac{3}{4}$
        \fracpizza{7}{9} correspond à $\dfrac{7}{9}$
        \fracpizza{3}{13} correspond à $\dfrac{3}{13}$
    \end{multicols}
}

\rmq{Avec cette définition, $\dfrac{a}{b}$ correspond à la division de $a$ par $b$.}

\prop{Fraction entière}
{Tout nombre entier peut être vu comme une fraction dont le dénominateur est 1.}

% \exmpl{$$4=\dfrac{4}{1}$$}
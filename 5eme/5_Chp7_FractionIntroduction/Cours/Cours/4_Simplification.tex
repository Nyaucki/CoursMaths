\section{Simplifications et comparaisons}

\prop{Simplification}
{Une fraction reste la même si on multiplie ou divise le numérateur ET le dénominateur par le même nombre.}

\exmpl{$$\dfrac{3}{4}=\dfrac{3\times 5}{4\times 5}=\dfrac{15}{20}$$}

\rmq{On utilise généralement cette propriété pour simplifier des fractions}

\exmpl{$$\dfrac{15}{3}=\dfrac{15\div 3}{3\div 3}=\dfrac{5}{1}=5$$}

\vspace*{-1em}

\prop{Comparaison de fractions}
{Pour comparer deux fractions, il faut qu'elles aient le même dénominateur. Il suffit ensuite de comparer leurs numérateurs}

\exmpl{
    \begin{multicols}{2}
        \noindent$$\dfrac{6}{5}>\dfrac{4}{5} \text{ ~~~car~~~ } 4<5$$
        $$\dfrac{4}{5}=\dfrac{8}{10} \text{ ~~~donc~~~ } \dfrac{3}{5}>\dfrac{7}{10}$$
    \end{multicols}
}

\prop{Egalité des produits en croix}
{$$\dfrac{a}{b}=\dfrac{c}{d} \text{~~~si et seulement si~~~} a\times d = b\times c $$}

\exmpl{
    \begin{multicols}{2}
        $4\times 6 = 2\times 12$ donc $\dfrac{4}{12}=\dfrac{2}{6}$\\
        $4\times 6 \neq 2\times 11$ donc $\dfrac{4}{11}\neq\dfrac{2}{6}$
    \end{multicols}
    }
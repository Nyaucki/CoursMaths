\section{Vers une nouvelle définition des fractions}

\dfnt{fractions}
{Pour $a$ un nombre et $b\neq 0$, $\dfrac{a}{b}$ est le nombre qui multiplié par $b$ donne $a$.}
%
\exmpl{
    \begin{multicols}{3}
    \noindent$$5\times\dfrac{3}{5}=3$$
    $$12\times\dfrac{-7}{12}=-7$$
    $$127\times\dfrac{3456}{127}=3456$$
    \end{multicols}}

\rmq{Cette définition est la conséquence de l'association fraction division}

% \rmq{$3\times\dfrac{2}{3}$ correspond à \fracpizza[blue]{2}{3}\fracpizza[green]{2}{3}\fracpizza{2}{3}}


\prop{Dénominateur nul}
{Une fraction ne peut pas avoir pour dénominateur 0.}

% \rmq{En effet, $\dfrac{5}{0}$ serait le nombre qui multiplié par 0 donne 5. Or un tel nombre ne peut exister.}


\section{Premières conséquences}

\prop{Fraction d'une quantité}
{Pour tout nombre $a,b$ et pour $c\neq 0$ :
$$a\times\dfrac{b}{c}=\dfrac{a\times b}{c}$$}

\exmpl{
    \begin{multicols}{3}
        $$
            3\times\dfrac{4}{5}
            =\dfrac{3\times 4}{5}
            =\dfrac{12}{5}
        $$
        $$
            10\times\dfrac{4}{7}
            =\dfrac{10\times 4}{7}
            =\dfrac{40}{7}
        $$
        $$
            6\times\dfrac{5}{2}
            =3\times2\times\dfrac{5}{2}
            =3\times 5
            =15
        $$
    \end{multicols}
}

\rmq{Prendre une fraction d'une quantité revient à multiplier la quantité par cette fraction}

\exmpl{Les trois quarts de 44 se calculent en faisant $44\times \dfrac{3}{4}$. }


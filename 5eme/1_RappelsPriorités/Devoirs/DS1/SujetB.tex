\rfoot{B}

\exo{4}{Calcul}

Effectuer les calculs suivants en détaillant les étapes.

\begin{multicols}{2}
    \cnt\\ 
    $13-2+4\times 2$

    \vspace*{14em}
    \columnbreak
    \cnt\\
    $(8\times (1+3)-2):5$

    \vspace*{14em}
\end{multicols}

\begin{multicols}{2}
    \exo{2}{Cours}
    
    Retrouve le nombre de départs en sachant que : 
    
    Si j'ajoute 2, je multiplie par 3, j'enlève 2 et je multiplie le tout par 8, j'obtiens 80.

    \vspace*{14em}
    \columnbreak
    \exo{2}{Chercher}
    
    Placer les parenthèses au bon endroit pour rendre l'égalité vraie :
    
    $2+3\times 5-1=14$

    \vspace*{14em}
\end{multicols}

\begin{multicols}{2}
    \exo{1}{Modéliser}\\
    Écrire le Calcul permettant de trouver le nombre de bonbons de Zoé : \vspace*{1em} \\   
    Elle en achète 50, elle en donne 3 lots de 5 à son frère qui lui rends 4 bonbons.
    \vspace*{6em}
    
    \columnbreak
    \exo{1}{Communiquer}
    
    Trouve l'erreur :
    \begin{align*}
        &1+(6-2)\times 2\\
        =&1+(6-4)\\
        =&1+2\\
        =&3
    \end{align*}
    \vspace*{6em}
\end{multicols}


\documentclass{/home/nyaucki/Documents/Prof/CoursMaths/mycls/DevoirMaison}
\usepackage{tabularx}
\usepackage{pythontex}
\renewcommand{\arraystretch}{1.5}
\begin{document}
%%%%%%%%%%%%%%%%%%%%%%%%%%%% A compléter pour l'entête %%%%%%%%%%%%%%%%%%%%%%%%%

\newcommand{\classe}{3emeC} % A compléter pour la classe cooncernée

\newcommand{\dateRendu}{07/10/2024} % A compléter pour la date de rendue

\newcommand{\devoirNumero}{11} % A compléter pour la date de rendue


%%%%%%%%%%%%%%%%%%%%%%%%%%%% Autres réglages %%%%%%%%%%%%%%%%%%%%%%%%%

\setlength{\columnseprule}{0.4pt} %Avoir les lignes entre les multicols

\renewcommand{\classe}{5H}





\renewcommand{\nom}{ALVES FERRO Adam} 

\renewcommand{\prenom}{Adam}

\hrulefill
\begin{figure}[H]
\centering
\begin{tabularx}{0.9\textwidth}{p{1.4cm}p{8cm}X}
\classe & \textbf{Devoir Maison \devoirNumero ~- Pour le \dateRendu} & Nom : \nom
\end{tabularx}
\end{figure}
\vspace{-1em}
\hrulefill

\begin{center}
	Si vous voulez aider \prenom , merci de ne pas juste lui donner la solution. 

	En prenant le temps de lui expliquer, vous l'aiderez beaucoup plus.
\end{center}

\medskip

Effectuer les calculs suivants en détaillant les étapes.

\begin{multicols}{2}
	\begin{enumerate}[label=\alph*.]
		\item $8 + 6 \times 6 - 5$  \vspace*{7em}
		\item $(5 +1)\times (14 -5)$ \vspace*{7em}
	\end{enumerate}
\end{multicols}

\begin{multicols}{2}
	\begin{enumerate}[start=3,label=\alph*.]
		\item $(1 \times 1 +11 ) \div 2$  \vspace*{8em}
		\item $(7 + (11 -5)\times 10)\times 4$ \vspace*{8em}
	\end{enumerate}
\end{multicols}



\vfill




\renewcommand{\nom}{AMARAL Kyle} 

\renewcommand{\prenom}{Kyle}

\hrulefill
\begin{figure}[H]
\centering
\begin{tabularx}{0.9\textwidth}{p{1.4cm}p{8cm}X}
\classe & \textbf{Devoir Maison \devoirNumero ~- Pour le \dateRendu} & Nom : \nom
\end{tabularx}
\end{figure}
\vspace{-1em}
\hrulefill

\begin{center}
	Si vous voulez aider \prenom , merci de ne pas juste lui donner la solution. 

	En prenant le temps de lui expliquer, vous l'aiderez beaucoup plus.
\end{center}

\medskip

Effectuer les calculs suivants en détaillant les étapes.

\begin{multicols}{2}
	\begin{enumerate}[label=\alph*.]
		\item $1 + 3 \times 5 - 2$  \vspace*{7em}
		\item $(3 +2)\times (8 -5)$ \vspace*{7em}
	\end{enumerate}
\end{multicols}

\begin{multicols}{2}
	\begin{enumerate}[start=3,label=\alph*.]
		\item $(3 \times 3 +3 ) \div 4$  \vspace*{8em}
		\item $(7 + (8 -4)\times 9)\times 2$ \vspace*{8em}
	\end{enumerate}
\end{multicols}



\newpage





\renewcommand{\nom}{BLANUTA Alexandru} 

\renewcommand{\prenom}{Alexandru}

\hrulefill
\begin{figure}[H]
\centering
\begin{tabularx}{0.9\textwidth}{p{1.4cm}p{8cm}X}
\classe & \textbf{Devoir Maison \devoirNumero ~- Pour le \dateRendu} & Nom : \nom
\end{tabularx}
\end{figure}
\vspace{-1em}
\hrulefill

\begin{center}
	Si vous voulez aider \prenom , merci de ne pas juste lui donner la solution. 

	En prenant le temps de lui expliquer, vous l'aiderez beaucoup plus.
\end{center}

\medskip

Effectuer les calculs suivants en détaillant les étapes.

\begin{multicols}{2}
	\begin{enumerate}[label=\alph*.]
		\item $5 + 4 \times 7 - 1$  \vspace*{7em}
		\item $(2 +1)\times (14 -4)$ \vspace*{7em}
	\end{enumerate}
\end{multicols}

\begin{multicols}{2}
	\begin{enumerate}[start=3,label=\alph*.]
		\item $(1 \times 3 +9 ) \div 2$  \vspace*{8em}
		\item $(3 + (6 -4)\times 3)\times 5$ \vspace*{8em}
	\end{enumerate}
\end{multicols}



\vfill




\renewcommand{\nom}{BOGHIAN Cléopatra} 

\renewcommand{\prenom}{Cléopatra}

\hrulefill
\begin{figure}[H]
\centering
\begin{tabularx}{0.9\textwidth}{p{1.4cm}p{8cm}X}
\classe & \textbf{Devoir Maison \devoirNumero ~- Pour le \dateRendu} & Nom : \nom
\end{tabularx}
\end{figure}
\vspace{-1em}
\hrulefill

\begin{center}
	Si vous voulez aider \prenom , merci de ne pas juste lui donner la solution. 

	En prenant le temps de lui expliquer, vous l'aiderez beaucoup plus.
\end{center}

\medskip

Effectuer les calculs suivants en détaillant les étapes.

\begin{multicols}{2}
	\begin{enumerate}[label=\alph*.]
		\item $8 + 6 \times 4 - 8$  \vspace*{7em}
		\item $(1 +4)\times (12 -5)$ \vspace*{7em}
	\end{enumerate}
\end{multicols}

\begin{multicols}{2}
	\begin{enumerate}[start=3,label=\alph*.]
		\item $(3 \times 2 +6 ) \div 3$  \vspace*{8em}
		\item $(2 + (12 -5)\times 9)\times 5$ \vspace*{8em}
	\end{enumerate}
\end{multicols}



\newpage





\renewcommand{\nom}{BRHANE Hermela} 

\renewcommand{\prenom}{Hermela}

\hrulefill
\begin{figure}[H]
\centering
\begin{tabularx}{0.9\textwidth}{p{1.4cm}p{8cm}X}
\classe & \textbf{Devoir Maison \devoirNumero ~- Pour le \dateRendu} & Nom : \nom
\end{tabularx}
\end{figure}
\vspace{-1em}
\hrulefill

\begin{center}
	Si vous voulez aider \prenom , merci de ne pas juste lui donner la solution. 

	En prenant le temps de lui expliquer, vous l'aiderez beaucoup plus.
\end{center}

\medskip

Effectuer les calculs suivants en détaillant les étapes.

\begin{multicols}{2}
	\begin{enumerate}[label=\alph*.]
		\item $7 + 6 \times 9 - 4$  \vspace*{7em}
		\item $(1 +4)\times (14 -4)$ \vspace*{7em}
	\end{enumerate}
\end{multicols}

\begin{multicols}{2}
	\begin{enumerate}[start=3,label=\alph*.]
		\item $(2 \times 2 +8 ) \div 3$  \vspace*{8em}
		\item $(8 + (8 -5)\times 8)\times 7$ \vspace*{8em}
	\end{enumerate}
\end{multicols}



\vfill




\renewcommand{\nom}{BYTYCI Shpresa} 

\renewcommand{\prenom}{Shpresa}

\hrulefill
\begin{figure}[H]
\centering
\begin{tabularx}{0.9\textwidth}{p{1.4cm}p{8cm}X}
\classe & \textbf{Devoir Maison \devoirNumero ~- Pour le \dateRendu} & Nom : \nom
\end{tabularx}
\end{figure}
\vspace{-1em}
\hrulefill

\begin{center}
	Si vous voulez aider \prenom , merci de ne pas juste lui donner la solution. 

	En prenant le temps de lui expliquer, vous l'aiderez beaucoup plus.
\end{center}

\medskip

Effectuer les calculs suivants en détaillant les étapes.

\begin{multicols}{2}
	\begin{enumerate}[label=\alph*.]
		\item $6 + 4 \times 5 - 4$  \vspace*{7em}
		\item $(5 +3)\times (7 -4)$ \vspace*{7em}
	\end{enumerate}
\end{multicols}

\begin{multicols}{2}
	\begin{enumerate}[start=3,label=\alph*.]
		\item $(3 \times 3 +3 ) \div 3$  \vspace*{8em}
		\item $(8 + (8 -5)\times 4)\times 6$ \vspace*{8em}
	\end{enumerate}
\end{multicols}



\newpage





\renewcommand{\nom}{CIRPACI Timotei} 

\renewcommand{\prenom}{Timotei}

\hrulefill
\begin{figure}[H]
\centering
\begin{tabularx}{0.9\textwidth}{p{1.4cm}p{8cm}X}
\classe & \textbf{Devoir Maison \devoirNumero ~- Pour le \dateRendu} & Nom : \nom
\end{tabularx}
\end{figure}
\vspace{-1em}
\hrulefill

\begin{center}
	Si vous voulez aider \prenom , merci de ne pas juste lui donner la solution. 

	En prenant le temps de lui expliquer, vous l'aiderez beaucoup plus.
\end{center}

\medskip

Effectuer les calculs suivants en détaillant les étapes.

\begin{multicols}{2}
	\begin{enumerate}[label=\alph*.]
		\item $8 + 8 \times 7 - 4$  \vspace*{7em}
		\item $(5 +5)\times (10 -4)$ \vspace*{7em}
	\end{enumerate}
\end{multicols}

\begin{multicols}{2}
	\begin{enumerate}[start=3,label=\alph*.]
		\item $(1 \times 1 +11 ) \div 3$  \vspace*{8em}
		\item $(2 + (6 -5)\times 7)\times 6$ \vspace*{8em}
	\end{enumerate}
\end{multicols}



\vfill




\renewcommand{\nom}{DIEYE Nelson} 

\renewcommand{\prenom}{Nelson}

\hrulefill
\begin{figure}[H]
\centering
\begin{tabularx}{0.9\textwidth}{p{1.4cm}p{8cm}X}
\classe & \textbf{Devoir Maison \devoirNumero ~- Pour le \dateRendu} & Nom : \nom
\end{tabularx}
\end{figure}
\vspace{-1em}
\hrulefill

\begin{center}
	Si vous voulez aider \prenom , merci de ne pas juste lui donner la solution. 

	En prenant le temps de lui expliquer, vous l'aiderez beaucoup plus.
\end{center}

\medskip

Effectuer les calculs suivants en détaillant les étapes.

\begin{multicols}{2}
	\begin{enumerate}[label=\alph*.]
		\item $5 + 3 \times 5 - 1$  \vspace*{7em}
		\item $(1 +3)\times (10 -5)$ \vspace*{7em}
	\end{enumerate}
\end{multicols}

\begin{multicols}{2}
	\begin{enumerate}[start=3,label=\alph*.]
		\item $(2 \times 3 +6 ) \div 3$  \vspace*{8em}
		\item $(5 + (9 -5)\times 10)\times 1$ \vspace*{8em}
	\end{enumerate}
\end{multicols}



\newpage





\renewcommand{\nom}{FILKI Rim} 

\renewcommand{\prenom}{Rim}

\hrulefill
\begin{figure}[H]
\centering
\begin{tabularx}{0.9\textwidth}{p{1.4cm}p{8cm}X}
\classe & \textbf{Devoir Maison \devoirNumero ~- Pour le \dateRendu} & Nom : \nom
\end{tabularx}
\end{figure}
\vspace{-1em}
\hrulefill

\begin{center}
	Si vous voulez aider \prenom , merci de ne pas juste lui donner la solution. 

	En prenant le temps de lui expliquer, vous l'aiderez beaucoup plus.
\end{center}

\medskip

Effectuer les calculs suivants en détaillant les étapes.

\begin{multicols}{2}
	\begin{enumerate}[label=\alph*.]
		\item $8 + 3 \times 8 - 4$  \vspace*{7em}
		\item $(1 +2)\times (13 -5)$ \vspace*{7em}
	\end{enumerate}
\end{multicols}

\begin{multicols}{2}
	\begin{enumerate}[start=3,label=\alph*.]
		\item $(2 \times 3 +6 ) \div 4$  \vspace*{8em}
		\item $(6 + (11 -4)\times 10)\times 7$ \vspace*{8em}
	\end{enumerate}
\end{multicols}



\vfill




\renewcommand{\nom}{FRITZ Rose} 

\renewcommand{\prenom}{Rose}

\hrulefill
\begin{figure}[H]
\centering
\begin{tabularx}{0.9\textwidth}{p{1.4cm}p{8cm}X}
\classe & \textbf{Devoir Maison \devoirNumero ~- Pour le \dateRendu} & Nom : \nom
\end{tabularx}
\end{figure}
\vspace{-1em}
\hrulefill

\begin{center}
	Si vous voulez aider \prenom , merci de ne pas juste lui donner la solution. 

	En prenant le temps de lui expliquer, vous l'aiderez beaucoup plus.
\end{center}

\medskip

Effectuer les calculs suivants en détaillant les étapes.

\begin{multicols}{2}
	\begin{enumerate}[label=\alph*.]
		\item $7 + 6 \times 3 - 9$  \vspace*{7em}
		\item $(2 +5)\times (8 -5)$ \vspace*{7em}
	\end{enumerate}
\end{multicols}

\begin{multicols}{2}
	\begin{enumerate}[start=3,label=\alph*.]
		\item $(2 \times 1 +10 ) \div 2$  \vspace*{8em}
		\item $(3 + (8 -5)\times 1)\times 5$ \vspace*{8em}
	\end{enumerate}
\end{multicols}



\newpage





\renewcommand{\nom}{HALBERT Cassiopée} 

\renewcommand{\prenom}{Cassiopée}

\hrulefill
\begin{figure}[H]
\centering
\begin{tabularx}{0.9\textwidth}{p{1.4cm}p{8cm}X}
\classe & \textbf{Devoir Maison \devoirNumero ~- Pour le \dateRendu} & Nom : \nom
\end{tabularx}
\end{figure}
\vspace{-1em}
\hrulefill

\begin{center}
	Si vous voulez aider \prenom , merci de ne pas juste lui donner la solution. 

	En prenant le temps de lui expliquer, vous l'aiderez beaucoup plus.
\end{center}

\medskip

Effectuer les calculs suivants en détaillant les étapes.

\begin{multicols}{2}
	\begin{enumerate}[label=\alph*.]
		\item $8 + 8 \times 6 - 4$  \vspace*{7em}
		\item $(3 +1)\times (10 -5)$ \vspace*{7em}
	\end{enumerate}
\end{multicols}

\begin{multicols}{2}
	\begin{enumerate}[start=3,label=\alph*.]
		\item $(3 \times 3 +3 ) \div 3$  \vspace*{8em}
		\item $(4 + (8 -2)\times 2)\times 6$ \vspace*{8em}
	\end{enumerate}
\end{multicols}



\vfill




\renewcommand{\nom}{HARKAT Rama} 

\renewcommand{\prenom}{Rama}

\hrulefill
\begin{figure}[H]
\centering
\begin{tabularx}{0.9\textwidth}{p{1.4cm}p{8cm}X}
\classe & \textbf{Devoir Maison \devoirNumero ~- Pour le \dateRendu} & Nom : \nom
\end{tabularx}
\end{figure}
\vspace{-1em}
\hrulefill

\begin{center}
	Si vous voulez aider \prenom , merci de ne pas juste lui donner la solution. 

	En prenant le temps de lui expliquer, vous l'aiderez beaucoup plus.
\end{center}

\medskip

Effectuer les calculs suivants en détaillant les étapes.

\begin{multicols}{2}
	\begin{enumerate}[label=\alph*.]
		\item $3 + 6 \times 7 - 7$  \vspace*{7em}
		\item $(4 +1)\times (11 -4)$ \vspace*{7em}
	\end{enumerate}
\end{multicols}

\begin{multicols}{2}
	\begin{enumerate}[start=3,label=\alph*.]
		\item $(1 \times 2 +10 ) \div 2$  \vspace*{8em}
		\item $(9 + (9 -5)\times 4)\times 3$ \vspace*{8em}
	\end{enumerate}
\end{multicols}



\newpage





\renewcommand{\nom}{HOXHA Ajan} 

\renewcommand{\prenom}{Ajan}

\hrulefill
\begin{figure}[H]
\centering
\begin{tabularx}{0.9\textwidth}{p{1.4cm}p{8cm}X}
\classe & \textbf{Devoir Maison \devoirNumero ~- Pour le \dateRendu} & Nom : \nom
\end{tabularx}
\end{figure}
\vspace{-1em}
\hrulefill

\begin{center}
	Si vous voulez aider \prenom , merci de ne pas juste lui donner la solution. 

	En prenant le temps de lui expliquer, vous l'aiderez beaucoup plus.
\end{center}

\medskip

Effectuer les calculs suivants en détaillant les étapes.

\begin{multicols}{2}
	\begin{enumerate}[label=\alph*.]
		\item $1 + 9 \times 4 - 6$  \vspace*{7em}
		\item $(2 +2)\times (7 -5)$ \vspace*{7em}
	\end{enumerate}
\end{multicols}

\begin{multicols}{2}
	\begin{enumerate}[start=3,label=\alph*.]
		\item $(2 \times 3 +6 ) \div 2$  \vspace*{8em}
		\item $(4 + (9 -4)\times 2)\times 10$ \vspace*{8em}
	\end{enumerate}
\end{multicols}



\vfill




\renewcommand{\nom}{LAPOUJADE Cristal} 

\renewcommand{\prenom}{Cristal}

\hrulefill
\begin{figure}[H]
\centering
\begin{tabularx}{0.9\textwidth}{p{1.4cm}p{8cm}X}
\classe & \textbf{Devoir Maison \devoirNumero ~- Pour le \dateRendu} & Nom : \nom
\end{tabularx}
\end{figure}
\vspace{-1em}
\hrulefill

\begin{center}
	Si vous voulez aider \prenom , merci de ne pas juste lui donner la solution. 

	En prenant le temps de lui expliquer, vous l'aiderez beaucoup plus.
\end{center}

\medskip

Effectuer les calculs suivants en détaillant les étapes.

\begin{multicols}{2}
	\begin{enumerate}[label=\alph*.]
		\item $8 + 8 \times 6 - 2$  \vspace*{7em}
		\item $(4 +4)\times (8 -5)$ \vspace*{7em}
	\end{enumerate}
\end{multicols}

\begin{multicols}{2}
	\begin{enumerate}[start=3,label=\alph*.]
		\item $(3 \times 2 +6 ) \div 2$  \vspace*{8em}
		\item $(5 + (12 -4)\times 6)\times 4$ \vspace*{8em}
	\end{enumerate}
\end{multicols}



\newpage





\renewcommand{\nom}{LECOMTE Gabriel} 

\renewcommand{\prenom}{Gabriel}

\hrulefill
\begin{figure}[H]
\centering
\begin{tabularx}{0.9\textwidth}{p{1.4cm}p{8cm}X}
\classe & \textbf{Devoir Maison \devoirNumero ~- Pour le \dateRendu} & Nom : \nom
\end{tabularx}
\end{figure}
\vspace{-1em}
\hrulefill

\begin{center}
	Si vous voulez aider \prenom , merci de ne pas juste lui donner la solution. 

	En prenant le temps de lui expliquer, vous l'aiderez beaucoup plus.
\end{center}

\medskip

Effectuer les calculs suivants en détaillant les étapes.

\begin{multicols}{2}
	\begin{enumerate}[label=\alph*.]
		\item $4 + 4 \times 3 - 4$  \vspace*{7em}
		\item $(3 +3)\times (12 -5)$ \vspace*{7em}
	\end{enumerate}
\end{multicols}

\begin{multicols}{2}
	\begin{enumerate}[start=3,label=\alph*.]
		\item $(1 \times 3 +9 ) \div 3$  \vspace*{8em}
		\item $(1 + (8 -3)\times 8)\times 7$ \vspace*{8em}
	\end{enumerate}
\end{multicols}



\vfill




\renewcommand{\nom}{LYAZIDI Abderrahman} 

\renewcommand{\prenom}{Abderrahman}

\hrulefill
\begin{figure}[H]
\centering
\begin{tabularx}{0.9\textwidth}{p{1.4cm}p{8cm}X}
\classe & \textbf{Devoir Maison \devoirNumero ~- Pour le \dateRendu} & Nom : \nom
\end{tabularx}
\end{figure}
\vspace{-1em}
\hrulefill

\begin{center}
	Si vous voulez aider \prenom , merci de ne pas juste lui donner la solution. 

	En prenant le temps de lui expliquer, vous l'aiderez beaucoup plus.
\end{center}

\medskip

Effectuer les calculs suivants en détaillant les étapes.

\begin{multicols}{2}
	\begin{enumerate}[label=\alph*.]
		\item $1 + 4 \times 6 - 6$  \vspace*{7em}
		\item $(5 +5)\times (13 -4)$ \vspace*{7em}
	\end{enumerate}
\end{multicols}

\begin{multicols}{2}
	\begin{enumerate}[start=3,label=\alph*.]
		\item $(3 \times 2 +6 ) \div 2$  \vspace*{8em}
		\item $(1 + (6 -3)\times 10)\times 6$ \vspace*{8em}
	\end{enumerate}
\end{multicols}



\newpage





\renewcommand{\nom}{MAKUIKILA Noah} 

\renewcommand{\prenom}{Noah}

\hrulefill
\begin{figure}[H]
\centering
\begin{tabularx}{0.9\textwidth}{p{1.4cm}p{8cm}X}
\classe & \textbf{Devoir Maison \devoirNumero ~- Pour le \dateRendu} & Nom : \nom
\end{tabularx}
\end{figure}
\vspace{-1em}
\hrulefill

\begin{center}
	Si vous voulez aider \prenom , merci de ne pas juste lui donner la solution. 

	En prenant le temps de lui expliquer, vous l'aiderez beaucoup plus.
\end{center}

\medskip

Effectuer les calculs suivants en détaillant les étapes.

\begin{multicols}{2}
	\begin{enumerate}[label=\alph*.]
		\item $4 + 8 \times 9 - 9$  \vspace*{7em}
		\item $(3 +3)\times (12 -5)$ \vspace*{7em}
	\end{enumerate}
\end{multicols}

\begin{multicols}{2}
	\begin{enumerate}[start=3,label=\alph*.]
		\item $(1 \times 3 +9 ) \div 3$  \vspace*{8em}
		\item $(6 + (10 -4)\times 7)\times 4$ \vspace*{8em}
	\end{enumerate}
\end{multicols}



\vfill




\renewcommand{\nom}{MARTET Kiara} 

\renewcommand{\prenom}{Kiara}

\hrulefill
\begin{figure}[H]
\centering
\begin{tabularx}{0.9\textwidth}{p{1.4cm}p{8cm}X}
\classe & \textbf{Devoir Maison \devoirNumero ~- Pour le \dateRendu} & Nom : \nom
\end{tabularx}
\end{figure}
\vspace{-1em}
\hrulefill

\begin{center}
	Si vous voulez aider \prenom , merci de ne pas juste lui donner la solution. 

	En prenant le temps de lui expliquer, vous l'aiderez beaucoup plus.
\end{center}

\medskip

Effectuer les calculs suivants en détaillant les étapes.

\begin{multicols}{2}
	\begin{enumerate}[label=\alph*.]
		\item $10 + 7 \times 9 - 5$  \vspace*{7em}
		\item $(1 +4)\times (6 -5)$ \vspace*{7em}
	\end{enumerate}
\end{multicols}

\begin{multicols}{2}
	\begin{enumerate}[start=3,label=\alph*.]
		\item $(1 \times 2 +10 ) \div 4$  \vspace*{8em}
		\item $(8 + (12 -3)\times 7)\times 5$ \vspace*{8em}
	\end{enumerate}
\end{multicols}



\newpage





\renewcommand{\nom}{PUDRINI Marica} 

\renewcommand{\prenom}{Marica}

\hrulefill
\begin{figure}[H]
\centering
\begin{tabularx}{0.9\textwidth}{p{1.4cm}p{8cm}X}
\classe & \textbf{Devoir Maison \devoirNumero ~- Pour le \dateRendu} & Nom : \nom
\end{tabularx}
\end{figure}
\vspace{-1em}
\hrulefill

\begin{center}
	Si vous voulez aider \prenom , merci de ne pas juste lui donner la solution. 

	En prenant le temps de lui expliquer, vous l'aiderez beaucoup plus.
\end{center}

\medskip

Effectuer les calculs suivants en détaillant les étapes.

\begin{multicols}{2}
	\begin{enumerate}[label=\alph*.]
		\item $1 + 6 \times 5 - 4$  \vspace*{7em}
		\item $(2 +3)\times (8 -5)$ \vspace*{7em}
	\end{enumerate}
\end{multicols}

\begin{multicols}{2}
	\begin{enumerate}[start=3,label=\alph*.]
		\item $(3 \times 1 +9 ) \div 3$  \vspace*{8em}
		\item $(6 + (11 -5)\times 5)\times 4$ \vspace*{8em}
	\end{enumerate}
\end{multicols}



\vfill




\renewcommand{\nom}{SAHUT Margaux} 

\renewcommand{\prenom}{Margaux}

\hrulefill
\begin{figure}[H]
\centering
\begin{tabularx}{0.9\textwidth}{p{1.4cm}p{8cm}X}
\classe & \textbf{Devoir Maison \devoirNumero ~- Pour le \dateRendu} & Nom : \nom
\end{tabularx}
\end{figure}
\vspace{-1em}
\hrulefill

\begin{center}
	Si vous voulez aider \prenom , merci de ne pas juste lui donner la solution. 

	En prenant le temps de lui expliquer, vous l'aiderez beaucoup plus.
\end{center}

\medskip

Effectuer les calculs suivants en détaillant les étapes.

\begin{multicols}{2}
	\begin{enumerate}[label=\alph*.]
		\item $2 + 9 \times 4 - 5$  \vspace*{7em}
		\item $(2 +5)\times (11 -5)$ \vspace*{7em}
	\end{enumerate}
\end{multicols}

\begin{multicols}{2}
	\begin{enumerate}[start=3,label=\alph*.]
		\item $(1 \times 1 +11 ) \div 3$  \vspace*{8em}
		\item $(1 + (11 -2)\times 5)\times 1$ \vspace*{8em}
	\end{enumerate}
\end{multicols}



\newpage





\renewcommand{\nom}{SALEH Warren} 

\renewcommand{\prenom}{Warren}

\hrulefill
\begin{figure}[H]
\centering
\begin{tabularx}{0.9\textwidth}{p{1.4cm}p{8cm}X}
\classe & \textbf{Devoir Maison \devoirNumero ~- Pour le \dateRendu} & Nom : \nom
\end{tabularx}
\end{figure}
\vspace{-1em}
\hrulefill

\begin{center}
	Si vous voulez aider \prenom , merci de ne pas juste lui donner la solution. 

	En prenant le temps de lui expliquer, vous l'aiderez beaucoup plus.
\end{center}

\medskip

Effectuer les calculs suivants en détaillant les étapes.

\begin{multicols}{2}
	\begin{enumerate}[label=\alph*.]
		\item $9 + 3 \times 3 - 3$  \vspace*{7em}
		\item $(2 +1)\times (6 -4)$ \vspace*{7em}
	\end{enumerate}
\end{multicols}

\begin{multicols}{2}
	\begin{enumerate}[start=3,label=\alph*.]
		\item $(2 \times 2 +8 ) \div 2$  \vspace*{8em}
		\item $(9 + (6 -4)\times 4)\times 2$ \vspace*{8em}
	\end{enumerate}
\end{multicols}



\vfill




\renewcommand{\nom}{SOUALHIA Mehdi} 

\renewcommand{\prenom}{Mehdi}

\hrulefill
\begin{figure}[H]
\centering
\begin{tabularx}{0.9\textwidth}{p{1.4cm}p{8cm}X}
\classe & \textbf{Devoir Maison \devoirNumero ~- Pour le \dateRendu} & Nom : \nom
\end{tabularx}
\end{figure}
\vspace{-1em}
\hrulefill

\begin{center}
	Si vous voulez aider \prenom , merci de ne pas juste lui donner la solution. 

	En prenant le temps de lui expliquer, vous l'aiderez beaucoup plus.
\end{center}

\medskip

Effectuer les calculs suivants en détaillant les étapes.

\begin{multicols}{2}
	\begin{enumerate}[label=\alph*.]
		\item $7 + 8 \times 9 - 9$  \vspace*{7em}
		\item $(4 +5)\times (14 -5)$ \vspace*{7em}
	\end{enumerate}
\end{multicols}

\begin{multicols}{2}
	\begin{enumerate}[start=3,label=\alph*.]
		\item $(2 \times 1 +10 ) \div 2$  \vspace*{8em}
		\item $(1 + (11 -5)\times 2)\times 8$ \vspace*{8em}
	\end{enumerate}
\end{multicols}



\newpage





\renewcommand{\nom}{VEZ Dawson} 

\renewcommand{\prenom}{Dawson}

\hrulefill
\begin{figure}[H]
\centering
\begin{tabularx}{0.9\textwidth}{p{1.4cm}p{8cm}X}
\classe & \textbf{Devoir Maison \devoirNumero ~- Pour le \dateRendu} & Nom : \nom
\end{tabularx}
\end{figure}
\vspace{-1em}
\hrulefill

\begin{center}
	Si vous voulez aider \prenom , merci de ne pas juste lui donner la solution. 

	En prenant le temps de lui expliquer, vous l'aiderez beaucoup plus.
\end{center}

\medskip

Effectuer les calculs suivants en détaillant les étapes.

\begin{multicols}{2}
	\begin{enumerate}[label=\alph*.]
		\item $5 + 8 \times 7 - 6$  \vspace*{7em}
		\item $(2 +4)\times (12 -4)$ \vspace*{7em}
	\end{enumerate}
\end{multicols}

\begin{multicols}{2}
	\begin{enumerate}[start=3,label=\alph*.]
		\item $(1 \times 2 +10 ) \div 4$  \vspace*{8em}
		\item $(1 + (11 -4)\times 2)\times 4$ \vspace*{8em}
	\end{enumerate}
\end{multicols}



\vfill\end{document}
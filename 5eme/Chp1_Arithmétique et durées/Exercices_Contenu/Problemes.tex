%! TEX root = ../Exercices.tex
\section{problèmes}

\exo{prop600}

601 est divisble par 1. 602 est divisible par 2. 603 est divisible par 3. Est-ce vrai pour tous les nombres entre 1 et 10 ?
\dotlines{1}

\exo{prop101}

1010 est un multiple de 10. 1111 est un multiple de 11. 1212 est un multiple de 12. Est-ce vrai pour tous les nombres entre 10 et 99 ?
\dotlines{1}

\exo{GOT7}

7 enfants on récupéré des bonbons pour Halloween et sont sur le point de se battre car il n'arrivent pas à partager.
\begin{itemize}
	\item Mark, qui ne veut pas se battre, préfère partir et abandonner sa part. Les 6 restants se rendent compte qu'ils peuvent alors partager équitablement.
	\item Jinyoung réalise que les bonbons sont au caramel et qu'il n'aime pas ça. Heureusement, les 5 restant peuvent encore partager équiteblement.
	\item Jackson est appelé par sa maman et doit rentrer tout de suite ! Le tas de bonbon peut encore être partagé entre les 4 restants.
	\item JB donne un bonbon à Jackson avant qu'il ne parte. Et c'est le drame, il n'est plus possible de partager équitablement en 4.
	\item Bambam recompte les bonbons et annonce que maintenant, il ne sera plus possible de faire de partage équitable.
\end{itemize}

\begin{enumerate}
	\item Quel type de nombre ne peut pas être partagé ? \dotfill
	\item Combien de bonbons y avait il au début ? \dotlines{2}
\end{enumerate}

\newpage

\exo{ExpliqueDecomposition}

\begin{enumerate}
	\item Quelle est la décomposition en produit de nombres premiers de 60 ?
		\dotlines{2}
	\item Donner tous les diviseurde de 60.
		\dotlines{2}
	\item Robin prétends qu'après avoir fait la décomposition en produit de nombres premiers, il est beaucoup plus facile de trouver tous les diviseurs d'un nombre. Pourquoi ?
		\dotlines{2}
	\item Sur le cahier, utiliser cette méthode pour trouver la liste des diviseurs de 72.
\end{enumerate}



\exo{PGCD1}

Un boulanger a préparé 90 croissants et 198 pains au chocolat. Il cherche à faire des lots équitables contenant tous le même nombre de croissants et le même nombre de pains au chocolat.

\begin{enumerate}
	\item Pourquoi ne peut-il pas faire 15 lots ? \dotfill
	\item Quelle est la décomposition produit de nombres premiers de 198 ?
		\dotlines{1}
	\item Donner tous les diviseurs de 90 et de 198.
		\dotlines{2}
	\item Combien de lots peut il faire au maximum? \dotfill
	\item Dans chaque lots, il y aura \fillin croissants et \fillin pains au chocolats.
\end{enumerate}

\exo{PGCD2}

C'est l'assault sur l'île d'Onigashima. Il y a 105 pirates et 140 samouraïs, et il serait judicieux de les répartir en groupe équitables.

\begin{enumerate}
	\item Pourquoi ne peut-on pas faire 21 groupes ? \dotfill
	\item Quelle est la décomposition produit de nombres premiers de 105 ?
		\dotlines{1}
	\item Donner tous les diviseurs de 140 et de 105.
		\dotlines{2}
	\item Combien de groupes peut-on faire au maximum? \dotfill
	\item Dans chaque groupe, il y aura \fillin pirates et \fillin samouraïs.
\end{enumerate}

\exo{Tarot}

Un jeu de tarot contient 78 cartes comprenant 21 atouts, une carte appelée "Excuse" et les cartes restantes sont partagées en 4 couleurs : coeur, carreau, pique et trèfle.

\begin{enumerate}
	\item Combien de carte y a-t-il par couleur ?
		\dotlines{1}

		Le tarot peut se jouer à 3, 4 ou 5. On distribue équitablement les cartes aux joueurs, et ce qui reste est appelé "le chien" et est donné au joueur qui attaque.  
\item Remplir le tableau suivant :
		\begin{tabular}{c*{3}{|c}}
			\renewcommand{\arraystretch}{3.5}
			Nombre de joueurs & 3 & 4 & 5 \\\hline&&&\\[0.5cm]
			Cartes par joueurs & \fillin[1cm] & 18 & 15 \\\hline&&&\\[0.5cm]
			Cartes dans le chien & 6 & \fillin[1cm] & \fillin[1cm]\\[0.1cm] &&&
		\end{tabular}

	\item Lenon dit " Dans tous les cas, le chien est trop gros. On pourrait mettre moins de carte dans le chien et en donner plus au joueurs". A-t-il raison ? Pourquoi ?
		\dotlines{2}
	\item John a perdu 3 cartes dans son jeu, mais veut malgré tout joueur. Il voudrait savoir combien de cartes ditribuer à chaque personne pour que le nombre de carte dans le chien soit le plus proche possible de ce qui est distribué avec un jeu complet :

		\begin{tabular}{c*{3}{|c}}
			\renewcommand{\arraystretch}{3.5}
			Nombre de joueurs & 3 & 4 & 5 \\\hline&&&\\[0.5cm]
			Cartes par joueurs & \fillin[1cm] & \fillin[1cm]& \fillin[1cm]\\\hline&&&\\[0.5cm]
			Cartes dans le chien & \fillin[1cm] & \fillin[1cm] & \fillin[1cm]\\[0.1cm] &&&
		\end{tabular}
	\item Même question, mais pour Bob qui possède un jeu où il manque 4 cartes.

		\begin{tabular}{c*{3}{|c}}
			\renewcommand{\arraystretch}{3.5}
			Nombre de joueurs & 3 & 4 & 5 \\\hline&&&\\[0.5cm]
			Cartes par joueurs & \fillin[1cm] & \fillin[1cm]& \fillin[1cm]\\\hline&&&\\[0.5cm]
			Cartes dans le chien & \fillin[1cm] & \fillin[1cm] & \fillin[1cm]\\[0.1cm] &&&
		\end{tabular}
\end{enumerate}

\newpage

\exo{Celimene}

Pour l'anniversaire de Célimène, ses parents ont préparé 27 fondants au chocolats. Célimène va les partager équitablement avec ses copines, puis comme c'est son anniversaire, elle pourra aussi avoir les fondant restants.

\begin{enumerate}
	\item Si elle invite 4 copines, combien de fondants aura Célimène ?
		\dotlines{1}
	\item Si elle invite 6 copines, combien de fondants aura Célimène ?
		\dotlines{1}
	\item Vaut-il mieux pour Célimène inviter 4 ou 6 copines ?\dotfill
	\item Célimène doit inviter entre 2 et 10 copines. Quel est le nombre de copine qui lui permettra d'avoir le plus de fondants possible ?
		\dotlines{3}
\end{enumerate}

\exo{Impression}

L'imprimante du collège met 3 secondes pour imprimmer 1 page. Monsieur Loizon veut imprimmer un document de 5 pages pour chacun de ses 76 élèves de 5ème. La récrée dure 15 minutes, aura-t-il assez de temps ? 
\dotlines{3}

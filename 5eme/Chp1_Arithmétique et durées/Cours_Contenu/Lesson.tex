%! TEX root = ../Cours..
\section{Lesson}
\subsection{Arithmétique}
\dfnt
{
	\begin{itemize}
		\item Une \cachecour{division euclidienne} est une division avec un quotient entier et un reste.
		\item 4 est un \cachecour{diviseur} de 12 car le reste de la division euclidienne de 12 par 4 est 0.
		\item 22 est un \cachecour{multiple} de 2 car il existe un nombre (11) tel que $22=2\times11$.
		\item Un \cachecour{nombre premier} est un nombre possédant exactement deux diviseurs : 1 et lui même.
	\end{itemize}
}

\prop{Propriété du reste}
{
	Le reste d'une division euclidienne est toujours inférieur au quotient.
}



\prop{Diviseurs universel}
{
	Tous les nombres ont pour diviseurs au moins 1 et eux même.
}



\prop{Lien multiple et diviseur}
{Si un nombre est un diviseur d'un autre nombre, alors ce dernier est un multiple du premier}

\prop{Diviseur de diviseur}
{
	4 est un diviseur de 12 qui est un diviseur de 24. Donc 4 est un diviseur de 24.
	\\
	5 n'est pas un diviseur de 21. Donc les multiples de 5 ne sont pas non plus des diviseurs de 21.
}



\prop{Critères de divisibilité}
{Un nombre est divisible par :
\begin{itemize}
	\item 2 s'il est pair (se termine par 0 ; 2 ; 4 ; 6 ou 8).
	\item 3 si la somme de ses chiffres est divisible par 3.
	\item 5 s'il se termine par 0 ou 5.
	\item 9 si la somme de ses chiffres est divisible par 9.
\end{itemize}
}

\prop{Décomposition en produit de facteurs premiers}
{
	Tout nombre entier peut être écrit comme le produit de nombres premiers
}

\subsection{Durée}

\prop{Conversion des durées}
{
	\begin{itemize}
	\item une minute dure 60 secondes
	\item une heure dure 60 minutes
	\item une journée dure 24 heures
	\end{itemize}
}

